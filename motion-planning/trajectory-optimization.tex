\chapterimage{motion-planning.jpg}{OPERS field at UCSC}

\chapter{Trajectory optimization}
\index{motion profiles!trajectory optimization}

To contextualize some definitions related to trajectory optimization, let's use
the example of a drivetrain in FRC. A \textit{path} is a set of (x, y) points
for the \gls{system} to follow. A \textit{trajectory} is a path that includes
both the states (e.g., position and velocity) and control inputs (e.g., voltage)
of the \gls{system} as functions of time. \textit{Trajectory optimization}
refers to a set of methods for finding the best choice of trajectory (for some
mathematical definition of ``best") by selecting states and inputs of the
\gls{system} as functions of time.

Matthew Kelly has an approachable introduction to trajectory optimization on his
website \cite{bib:intro_to_traj_opt}.
