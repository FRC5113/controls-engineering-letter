\newenvironment{code}%
  {\FloatBarrier
   \begin{snippet}}%
 {\end{snippet}
   \FloatBarrier}

\newenvironment{coderemote}[2]%
  {\FloatBarrier
   \begin{snippet}
     \lstinputlisting[escapechar=, style=custom#1]{#2}}%
 {\end{snippet}
   \FloatBarrier}

\newenvironment{coderemotesubset}[4]%
  {\FloatBarrier
   \begin{snippet}
     \lstinputlisting[escapechar=, style=custom#1, firstline=#3, lastline=#4]{#2}}%
 {\end{snippet}
   \FloatBarrier}

\newenvironment{bookfigure}%
  {\begin{figure}[ht]%
    \begin{center}%
  }%
  {\end{center}%
    \end{figure}}

\newenvironment{nofloatfigure}%
  {\begin{figure}[H]%
    \centering%
  }%
  {\end{figure}}

% #1: column arguments to tabular
\newenvironment{figurekey}%
  {\begin{center}%
    \renewcommand{\arraystretch}{1.3}}
  {\end{center}}

\newenvironment{booktable}%
  {\begin{table}[ht]%
    \renewcommand{\arraystretch}{1.5}%
    \centering%
  }%
  {\end{table}}

% #1: file name without extension
\newenvironment{svg}[1]
  {\begin{figure}%
    \centering%
    \includegraphics[width=\linewidth * 3 / 4]{#1.pdf}}%
  {\end{figure}}

% #1: number of minisvgs in same figure
% #2: file name without extension
\newenvironment{minisvg}[2]
  {\begin{minipage}{\linewidth/#1 - 0.3cm}%
    \begin{center}%
      \includegraphics[width=\textwidth]{#2.pdf}}%
  {\end{center}%
    \end{minipage}}

\newcommand{\mat}[1] {\mathbf{#1}}
\newcommand{\matbar}[1] {\overline{\mathbf{#1}}}
\newcommand{\T} {^\mathsf{T}}

% #1: coordinate frame within which expression is represented (e.g., R)
% #2: expression
\newcommand{\crdfrm}[2] {\prescript{#1}{}{#2}}

% Set up snippet environment
\newcounter{snippet}[chapter]
\renewcommand{\thesnippet}{\thechapter.\arabic{snippet}}
\newenvironment{snippet}{
  \renewcommand{\caption}[1]{
    \refstepcounter{snippet}
    \begin{center}
      \par\noindent{\footnotesize Snippet \thesnippet. ##1}
    \end{center}
  }
}
{\par\noindent}

% #1: xshift
% #2: yshift
% #3: xcoordshift
% #4: ycoordshift
% #5: func to plot
\newcommand{\drawtimeplot}[5] {
  \begin{scope}[xshift=#1-0.5cm,yshift=#2-0.5cm]
    \def\xcoordshift{#3,0}
    \def\ycoordshift{0,#4}

    \draw[fill=headingbg] (0,0) rectangle (1,1);
    \begin{scope}[shift=(\xcoordshift)]
      % Draw Y axis
      \draw[->] (0cm,0.05cm) -- (0cm,0.9375cm) node {};
      \begin{scope}[shift=(\ycoordshift)]
        % Draw X axis
        \draw[->,font=\tiny] (-0.075cm,0cm) -- (0.8125cm,0cm)
          node[label={[below]90:$t$}] {};

        \draw[xscale=1.2,yscale=0.3,domain=0:0.5,smooth,variable=\x,line width=0.5]
          plot ({\x},{#5});
      \end{scope}
    \end{scope}
  \end{scope}}

% #1: xshift
% #2: yshift
\newcommand{\drawpole}[2] {
  \begin{scope}[xshift=#1,yshift=#2]
    \draw[themecolor,rotate=45] (-0.1,-0.0025) rectangle (0.1,0.0025);
    \draw[themecolor,rotate=-45] (-0.1,-0.0025) rectangle (0.1,0.0025);
  \end{scope}}

% Set custom \cleardoublepage
\makeatletter
\newcommand{\setcleardoublepage}{%
  % Removes the header from odd empty pages at the end of chapters
  \renewcommand{\cleardoublepage}{%
    \clearpage
    \ifodd
      \c@page
    \else
      \vspace*{\stretch{1}}
      \begin{center}
        \textit{\large This page intentionally left blank}
      \end{center}
      \vspace*{\stretch{4}}
      \thispagestyle{empty}
      \newpage
    \fi
  }
}
\makeatother

% Make \cleardoublepage do nothing
\newcommand{\unsetcleardoublepage}{%
  \renewcommand{\cleardoublepage}{}
}
