\section{Discretization methods}

\Gls{discretization} is done using a zero-order hold. That is, the \gls{system}
\gls{state} is only updated at discrete intervals and it's held constant between
samples (see figure \ref{fig:zoh}). The exact method of applying this uses the
matrix exponential, but this can be computationally expensive. Instead,
approximations such as the following are used.
\begin{enumerate}
  \item Forward Euler method. This is defined as
    $y_{n+1} = y_n + f(t_n, y_n) \Delta t$.
    \index{discretization!forward Euler method}
  \item Backward Euler method. This is defined as
    $y_{n+1} = y_n + f(t_{n+1}, y_{n+1}) \Delta t$.
    \index{discretization!backward Euler method}
  \item Bilinear transform. The first-order bilinear approximation is
    $s = \frac{2}{T} \frac{1 - z^{-1}}{1 + z^{-1}}$.
    \index{discretization!bilinear transform}
\end{enumerate}

where the function $f(t_n, y_n)$ is the slope of $y$ at $n$ and $T$ is the
sample period for the discrete \gls{system}. Each of these methods is
essentially finding the area underneath a curve. The forward and backward Euler
methods use rectangles to approximate that area while the bilinear transform
uses trapezoids (see figures \ref{fig:discretization_methods_vel} and
\ref{fig:discretization_methods_pos}). Since these are approximations, there is
distortion between the real discrete \gls{system}'s poles and the approximate
poles. This is in addition to the phase loss introduced by discretizing at a
given sample rate in the first place. For fast-changing \glspl{system}, this
distortion can quickly lead to instability.
\begin{bookfigure}
  \begin{minisvg}{2}{build/\chapterpath/zoh}
      \caption{Zero-order hold of a system response}
      \label{fig:zoh}
  \end{minisvg}
  \hfill
  \begin{minisvg}{2}{build/\chapterpath/discretization_methods_vel}
    \caption{Discretization methods applied to velocity data}
    \label{fig:discretization_methods_vel}
  \end{minisvg}
  \hfill
  \begin{minisvg}{2}{build/\chapterpath/discretization_methods_pos}
    \caption{Position plot of discretization methods applied to velocity data}
    \label{fig:discretization_methods_pos}
  \end{minisvg}
\end{bookfigure}

Figures \ref{fig:sampling_simulation_0.1}, \ref{fig:sampling_simulation_0.05},
and \ref{fig:sampling_simulation_0.01} show simulations of the same controller
for different sampling methods and sample rates, which have varying levels of
fidelity to the real \gls{system}.
\begin{bookfigure}
  \begin{minisvg}{2}{build/\chapterpath/sampling_simulation_010}
    \caption{Sampling methods for system simulation with $T = 0.1$ s}
    \label{fig:sampling_simulation_0.1}
  \end{minisvg}
  \hfill
  \begin{minisvg}{2}{build/\chapterpath/sampling_simulation_005}
    \caption{Sampling methods for system simulation with $T = 0.05$ s}
    \label{fig:sampling_simulation_0.05}
  \end{minisvg}
  \hfill
  \begin{minisvg}{2}{build/\chapterpath/sampling_simulation_004}
    \caption{Sampling methods for system simulation with $T = 0.01$ s}
    \label{fig:sampling_simulation_0.01}
  \end{minisvg}
\end{bookfigure}

Forward Euler is numerically unstable for low sample rates. The bilinear
transform is a significant improvement due to it being a second-order
approximation, but zero-order hold performs best due to the matrix exponential
including much higher orders.

Table \ref{tab:disc_approx_scalar} compares the Taylor series expansions of
several common discretization methods (these are found using polynomial
division). The bilinear transform does best with accuracy trailing off after the
third-order term. Forward Euler has no second-order or higher terms, so it
undershoots. Backward Euler has twice the second-order term and overshoots the
remaining higher order terms as well.
\begin{booktable}
  \begin{tabular}{|cll|}
    \hline
    \rowcolor{headingbg}
    \multicolumn{1}{|c}{\textbf{Method}} &
      \multicolumn{1}{c}{\textbf{Conversion to z}} &
      \multicolumn{1}{c|}{\textbf{Taylor series expansion}} \\
    \hline
    Zero-order hold &
      $e^{Ts}$ &
      $1 + Ts + \frac{1}{2}T^2s^2 + \frac{1}{6}T^3s^3 + \ldots$ \\
    Bilinear &
      $\frac{1 + \frac{1}{2}Ts}{1 - \frac{1}{2}Ts}$ &
      $1 + Ts + \frac{1}{2}T^2s^2 + \frac{1}{4}T^3s^3 + \ldots$ \\
    Forward Euler &
      $1 + Ts$ &
      $1 + Ts$ \\
    Reverse Euler &
      $\frac{1}{1 - Ts}$ &
      $1 + Ts + T^2s^2 + T^3s^3 + \ldots$ \\
    \hline
  \end{tabular}
  \caption{Taylor series expansions of discretization methods (scalar case). The
    zero-order hold discretization method is exact.}
  \label{tab:disc_approx_scalar}
\end{booktable}
