\section{Phase loss}

Implementing a discrete control system is easier than implementing a continuous
one, but \gls{discretization} has drawbacks. A microcontroller updates the
system input in discrete intervals of duration $T$; it's held constant between
updates. This introduces an average sample delay of $\frac{T}{2}$ that leads to
phase loss in the controller. Phase loss is the reduction of
\gls{phase margin}\footnote{See section \ref{subsec:gain_phase_margin} for an
explanation of phase margin.} that occurs in digital implementations of feedback
controllers from sampling the continuous \gls{system} at discrete time
intervals. As the sample rate of the controller decreases, the \gls{phase
margin} decreases according to $-\frac{T}{2}\omega$ where $T$ is the sample
period and $\omega$ is the frequency of the \gls{system} dynamics. Instability
occurs if the \gls{phase margin} of the \gls{system} reaches zero. Large amounts
of phase loss can make a stable controller in the continuous domain become
unstable in the discrete domain. Here are a few ways to combat this.
\begin{itemize}
  \item Run the controller with a high sample rate.
  \item Designing the controller in the analog domain with enough
    \gls{phase margin} to compensate for any phase loss that occurs as part of
    \gls{discretization}.
  \item Convert the \gls{plant} to the digital domain and design the controller
    completely in the digital domain.
\end{itemize}
