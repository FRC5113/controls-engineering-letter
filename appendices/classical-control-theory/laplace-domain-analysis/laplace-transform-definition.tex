\subsection{Laplace transform definition}

The Laplace transform of a function $f(t)$ is defined as
\begin{equation*}
  \mathcal{L}\{f(t)\} = F(s) = \int_0^\infty f(t) e^{-st} \,dt
\end{equation*}

We won't be computing any Laplace transforms by hand using this formula
(everyone in the real world looks these up in a table anyway). Common Laplace
transforms (assuming zero initial conditions) are shown in table
\ref{tab:common_laplace_transforms}. Of particular note are the Laplace
transforms for the derivative, unit step\footnote{The unit step $u(t)$ is
defined as $0$ for $t < 0$ and $1$ for $t \ge 0$.}, and exponential decay. We
can see that a derivative is equivalent to multiplying by $s$, and an integral
is equivalent to multiplying by $\frac{1}{s}$.
\begin{booktable}
  \begin{tabular}{|ccc|}
    \hline
    \rowcolor{headingbg}
    & \textbf{Time domain} & \textbf{Laplace domain} \\
    \hline
    Linearity & $a\,f(t) + b\,g(t)$ & $a\,F(s) + b\,G(s)$ \\
    Convolution & $(f * g)(t)$ & $F(s) \,G(s)$ \\
    Derivative & $f'(t)$ & $s \,F(s)$ \\
    $n^{th}$ derivative & $f^{(n)}(t)$ & $s^n \,F(s)$ \\
    Unit step & $u(t)$ & $\frac{1}{s}$ \\
    Ramp & $t \,u(t)$ & $\frac{1}{s^2}$ \\
    Exponential decay & $e^{-\alpha t} u(t)$ & $\frac{1}{s + \alpha}$ \\
    \hline
  \end{tabular}
  \caption{Common Laplace transforms and Laplace transform properties with zero
    initial conditions}
  \label{tab:common_laplace_transforms}
\end{booktable}
