\subsection{Fourier transform}

The Fourier transform decomposes a function of time into its component
frequencies. Each of these frequencies is part of what's called a
\textit{basis}. These basis waveforms can be multiplied by their respective
contribution amount and summed to produce the original signal (this weighted sum
is called a linear combination). In other words, the Fourier transform provides
a way for us to determine, given some signal, what frequencies can we add
together and in what amounts to produce the original signal.

Think of an Fmajor4 chord which has the notes $F_4$ ($349.23$ Hz), $A_4$ ($440$
Hz), and $C_4$ ($261.63$ Hz). The waveform over time looks like figure
\ref{fig:fourier_chord}.
\begin{svg}{build/\sectionpath/fourier_chord}
  \caption{Frequency decomposition of Fmajor4 chord}
  \label{fig:fourier_chord}
\end{svg}

Notice how this complex waveform can be represented just by three frequencies.
They show up as Dirac delta functions\footnote{The Dirac delta function is zero
everywhere except at the origin. The nonzero region has an infinitesimal width
and has a height such that the area within that region is $1$.} in the frequency
domain with the area underneath them equal to their contribution (see figure
\ref{fig:fourier_chord_fft}).
\begin{svg}{build/\sectionpath/fourier_chord_fft}
  \caption{Fourier transform of Fmajor4 chord}
  \label{fig:fourier_chord_fft}
\end{svg}
