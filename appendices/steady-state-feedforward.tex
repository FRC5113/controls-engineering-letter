\chapterimage{appendices.jpg}{Sunset in an airplane over New Mexico}

\chapter{Steady-state feedforward}

Steady-state feedforwards apply the \gls{control effort} required to keep a
\gls{system} at the \gls{reference} if it is no longer moving (i.e., the
\gls{system} is at steady-state). The first steady-state feedforward converts
desired \glspl{output} to desired \glspl{state}.

\begin{equation*}
  \mtx{x}_c = \mtx{N}_x\mtx{y}_c
\end{equation*}

$\mtx{N}_x$ converts desired \glspl{output} $\mtx{y}_c$ to desired \glspl{state}
$\mtx{x}_c$ (also known as $\mtx{r}$). For steady-state, that is

\begin{equation}
  \mtx{x}_{ss} = \mtx{N}_x\mtx{y}_{ss} \label{eq:x_ss}
\end{equation}

The second steady-state feedforward converts the desired \glspl{output}
$\mtx{y}$ to the \gls{control input} required at steady-state.

\begin{equation*}
  \mtx{u}_c = \mtx{N}_u\mtx{y}_c
\end{equation*}

$\mtx{N}_u$ converts the desired \glspl{output} $\mtx{y}$ to the
\gls{control input} $\mtx{u}$ required at steady-state. For steady-state, that
is

\begin{equation}
  \mtx{u}_{ss} = \mtx{N}_u\mtx{y}_{ss} \label{eq:u_ss}
\end{equation}

\section{Continuous case}

To find the \gls{control input} required at steady-state, set equation
\eqref{eq:ss_ctrl_x} to zero.

\begin{align*}
  \dot{\mtx{x}} &= \mtx{A}\mtx{x} + \mtx{B}\mtx{u} \\
  \mtx{y} &= \mtx{C}\mtx{x} + \mtx{D}\mtx{u}
\end{align*}

\begin{align*}
  \mtx{0} &= \mtx{A}\mtx{x}_{ss} + \mtx{B}\mtx{u}_{ss} \\
  \mtx{y}_{ss} &= \mtx{C}\mtx{x}_{ss} + \mtx{D}\mtx{u}_{ss}
\end{align*}

\begin{align*}
  \mtx{0} &= \mtx{A}\mtx{N}_x\mtx{y}_{ss} + \mtx{B}\mtx{N}_u\mtx{y}_{ss} \\
  \mtx{y}_{ss} &= \mtx{C}\mtx{N}_x\mtx{y}_{ss} + \mtx{D}\mtx{N}_u\mtx{y}_{ss}
\end{align*}

\begin{align*}
  \begin{bmatrix}
    \mtx{0} \\
    \mtx{y}_{ss}
  \end{bmatrix} &=
  \begin{bmatrix}
    \mtx{A}\mtx{N}_x + \mtx{B}\mtx{N}_u \\
    \mtx{C}\mtx{N}_x + \mtx{D}\mtx{N}_u
  \end{bmatrix}
  \mtx{y}_{ss} \\
  \begin{bmatrix}
    \mtx{0} \\
    \mtx{1}
  \end{bmatrix} &=
  \begin{bmatrix}
    \mtx{A}\mtx{N}_x + \mtx{B}\mtx{N}_u \\
    \mtx{C}\mtx{N}_x + \mtx{D}\mtx{N}_u
  \end{bmatrix} \\
  \begin{bmatrix}
    \mtx{0} \\
    \mtx{1}
  \end{bmatrix} &=
  \begin{bmatrix}
    \mtx{A} & \mtx{B} \\
    \mtx{C} & \mtx{D}
  \end{bmatrix}
  \begin{bmatrix}
    \mtx{N}_x \\
    \mtx{N}_u
  \end{bmatrix} \\
  \begin{bmatrix}
    \mtx{N}_x \\
    \mtx{N}_u
  \end{bmatrix} &=
  \begin{bmatrix}
    \mtx{A} & \mtx{B} \\
    \mtx{C} & \mtx{D}
  \end{bmatrix}^{\dagger}
  \begin{bmatrix}
    \mtx{0} \\
    \mtx{1}
  \end{bmatrix}
\end{align*}

where $^\dagger$ is the Moore-Penrose pseudoinverse.

\section{Discrete case}

Now, we'll do the same thing for the discrete \gls{system}. To find the
\gls{control input} required at steady-state, set equation \eqref{eq:ssz_ctrl_x}
to zero.

\begin{align*}
  \mtx{x}_{k+1} &= \mtx{A}\mtx{x}_k + \mtx{B}\mtx{u}_k \\
  \mtx{y}_k &= \mtx{C}\mtx{x}_k + \mtx{D}\mtx{u}_k
\end{align*}

\begin{align*}
  \mtx{x}_{ss} &= \mtx{A}\mtx{x}_{ss} + \mtx{B}\mtx{u}_{ss} \\
  \mtx{y}_{ss} &= \mtx{C}\mtx{x}_{ss} + \mtx{D}\mtx{u}_{ss}
\end{align*}

\begin{align*}
  \mtx{0} &= (\mtx{A} - \mtx{I})\mtx{x}_{ss} + \mtx{B}\mtx{u}_{ss} \\
  \mtx{y}_{ss} &= \mtx{C}\mtx{x}_{ss} + \mtx{D}\mtx{u}_{ss}
\end{align*}

\begin{align*}
  \mtx{0} &= (\mtx{A} - \mtx{I})\mtx{N}_x\mtx{y}_{ss} +
    \mtx{B}\mtx{N}_u\mtx{y}_{ss} \\
  \mtx{y}_{ss} &= \mtx{C}\mtx{N}_x\mtx{y}_{ss} + \mtx{D}\mtx{N}_u\mtx{y}_{ss}
\end{align*}

\begin{align*}
  \begin{bmatrix}
    \mtx{0} \\
    \mtx{y}_{ss}
  \end{bmatrix} &=
  \begin{bmatrix}
    (\mtx{A} - \mtx{I})\mtx{N}_x + \mtx{B}\mtx{N}_u \\
    \mtx{C}\mtx{N}_x + \mtx{D}\mtx{N}_u
  \end{bmatrix}
  \mtx{y}_{ss} \\
  \begin{bmatrix}
    \mtx{0} \\
    \mtx{1}
  \end{bmatrix} &=
  \begin{bmatrix}
    (\mtx{A} - \mtx{I})\mtx{N}_x + \mtx{B}\mtx{N}_u \\
    \mtx{C}\mtx{N}_x + \mtx{D}\mtx{N}_u
  \end{bmatrix} \\
  \begin{bmatrix}
    \mtx{0} \\
    \mtx{1}
  \end{bmatrix} &=
  \begin{bmatrix}
    \mtx{A} - \mtx{I} & \mtx{B} \\
    \mtx{C} & \mtx{D}
  \end{bmatrix}
  \begin{bmatrix}
    \mtx{N}_x \\
    \mtx{N}_u
  \end{bmatrix} \\
  \begin{bmatrix}
    \mtx{N}_x \\
    \mtx{N}_u
  \end{bmatrix} &=
  \begin{bmatrix}
    \mtx{A} - \mtx{I} & \mtx{B} \\
    \mtx{C} & \mtx{D}
  \end{bmatrix}^{\dagger}
  \begin{bmatrix}
    \mtx{0} \\
    \mtx{1}
  \end{bmatrix}
\end{align*}

where $^\dagger$ is the Moore-Penrose pseudoinverse.

\section{Deriving steady-state input}

Now, we'll find an expression that uses $\mtx{N}_x$ and $\mtx{N}_u$ to convert
the \gls{reference} $\mtx{r}$ to a \gls{control input} feedforward
$\mtx{u}_{ff}$. Let's start with equation \eqref{eq:x_ss}.

\begin{align*}
  \mtx{x}_{ss} &= \mtx{N}_x \mtx{y}_{ss} \\
  \mtx{N}_x^\dagger \mtx{x}_{ss} &= \mtx{y}_{ss}
\end{align*}

Now substitute this into equation \eqref{eq:u_ss}.

\begin{align*}
  \mtx{u}_{ss} &= \mtx{N}_u \mtx{y}_{ss} \\
  \mtx{u}_{ss} &= \mtx{N}_u (\mtx{N}_x^\dagger \mtx{x}_{ss}) \\
  \mtx{u}_{ss} &= \mtx{N}_u \mtx{N}_x^\dagger \mtx{x}_{ss}
\end{align*}

$\mtx{u}_{ss}$ and $\mtx{x}_{ss}$ are also known as $\mtx{u}_{ff}$ and $\mtx{r}$
respectively.

\begin{align*}
  \mtx{u}_{ff} = \mtx{N}_u \mtx{N}_x^\dagger \mtx{r}
\end{align*}

So all together, we get theorem \ref{thm:steady-state_ff}.

\index{feedforward!steady-state feedforward}
\begin{theorem}[Steady-state feedforward]
  \label{thm:steady-state_ff}

  \begin{align}
    &\text{Continuous:} \nonumber \\
    &\begin{bmatrix}
      \mtx{N}_x \\
      \mtx{N}_u
    \end{bmatrix} =
    \begin{bmatrix}
      \mtx{A} & \mtx{B} \\
      \mtx{C} & \mtx{D}
    \end{bmatrix}^{\dagger}
    \begin{bmatrix}
      \mtx{0} \\
      \mtx{1}
    \end{bmatrix} \\
    &\text{Discrete:} \nonumber \\
    &\begin{bmatrix}
      \mtx{N}_x \\
      \mtx{N}_u
    \end{bmatrix} =
    \begin{bmatrix}
      \mtx{A} - \mtx{I} & \mtx{B} \\
      \mtx{C} & \mtx{D}
    \end{bmatrix}^{\dagger}
    \begin{bmatrix}
      \mtx{0} \\
      \mtx{1}
    \end{bmatrix}
  \end{align}

  \begin{equation}
    \mtx{u}_{ff} = \mtx{N}_u \mtx{N}_x^\dagger \mtx{r}
  \end{equation}

  In the augmented matrix, $\mtx{B}$ should contain one column corresponding to
  an actuator and $\mtx{C}$ should contain one row whose \gls{output} will be
  driven by that actuator. More than one actuator or output can be included in
  the computation at once, but the result won't be the same as if they were
  computed independently and summed afterward.

  After computing the feedforward for each actuator-output pair, the respective
  collections of $\mtx{N}_x$ and $\mtx{N}_u$ matrices can summed to produce the
  combined feedforward.
\end{theorem}

If the augmented matrix in theorem \ref{thm:steady-state_ff} is square (number
of \glspl{input} = number of \glspl{output}), the normal matrix inverse can be
used instead.

\subsection{Case study: second-order CIM motor model}

Each feedforward implementation has advantages. The steady-state feedforward
allows using specific actuators to maintain the \gls{reference} at steady-state.
\Gls{plant} inversion doesn't support this, but can be used for \gls{reference}
\gls{tracking} as well with the same tuning parameters as LQR design. Figure
\ref{fig:case_study_ss_ff1} shows both types of feedforwards applied to a
second-order CIM motor model.

\begin{bookfigure}
  \begin{minisvg}{2}{build/\partpath/case_study_ss_ff1}
    \caption{Second-order CIM motor response with various feedforwards}
    \label{fig:case_study_ss_ff1}
  \end{minisvg}
  \hfill
  \begin{minisvg}{2}{build/\partpath/case_study_ss_ff2}
    \caption{Second-order CIM motor response with plant inversions}
    \label{fig:case_study_ss_ff2}
  \end{minisvg}
\end{bookfigure}

\Gls{plant} inversion isn't as effective in figure \ref{fig:case_study_ss_ff1}
because the $\mtx{R}$ matrix penalized \gls{control effort}. If the $\mtx{R}$
cost matrix is removed from the \gls{plant} inversion calculation, the
\gls{reference} \gls{tracking} is much better (see figure
\ref{fig:case_study_ss_ff2}).
