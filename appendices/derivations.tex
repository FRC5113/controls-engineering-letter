\chapterimage{appendices.jpg}{Sunset in an airplane over New Mexico}

\chapter{Derivations}

\section{Transfer function in feedback}
\label{sec:deriv_tf_feedback}

Given the feedback network in figure \ref{fig:closed_loop_deriv}, find an
expression for $Y(s)$.

\begin{bookfigure}
  \begin{tikzpicture}[auto, >=latex']
    % Place the blocks
    \node [name=input] {$X(s)$};
    \node [sum, right=of input] (sum) {};
    \node [block, right=of sum] (G) {$G(s)$};
    \node [right=of G] (output) {$Y(s)$};
    \node [block, below=of G] (measurements) {$H(s)$};

    % Connect the nodes
    \draw [arrow] (input) -- node[pos=0.85] {$+$} (sum);
    \draw [arrow] (sum) -- node {$Z(s)$} (G);
    \draw [arrow] (G) -- node [name=y] {} (output);
    \draw [arrow] (y) |- (measurements);
    \draw [arrow] (measurements) -| node[pos=0.99, right] {$-$} (sum);
  \end{tikzpicture}

  \caption{Closed-loop block diagram}
  \label{fig:closed_loop_deriv}
\end{bookfigure}

\begin{align}
  Y(s) &= Z(s) G(s) \nonumber \\
  Z(s) &= X(s) - Y(s) H(s) \nonumber \\
  X(s) &= Z(s) + Y(s) H(s) \nonumber \\
  X(s) &= Z(s) + Z(s) G(s) H(s) \nonumber \\
  \frac{Y(s)}{X(s)} &= \frac{Z(s) G(s)}{Z(s) + Z(s) G(s) H(s)} \nonumber \\
  \frac{Y(s)}{X(s)} &= \frac{G(s)}{1 + G(s) H(s)}
\end{align}

A more general form is

\begin{equation}
  \frac{Y(s)}{X(s)} = \frac{G(s)}{1 \mp G(s) H(s)}
\end{equation}

where positive feedback uses the top sign and negative feedback uses the bottom
sign.

\section{Linear quadratic regulator}
\label{sec:deriv_lqr}

Let a continuous time linear \gls{system} be defined as

\begin{equation}
  \dot{\mtx{x}} = \mtx{A}\mtx{x} + \mtx{B}\mtx{u}
\end{equation}

with the cost function

\begin{equation*}
  J = \int\limits_0^\infty
    \begin{bmatrix}
      \mtx{x} \\
      \mtx{u}
    \end{bmatrix}^T
    \begin{bmatrix}
      \mtx{Q} & \mtx{N} \\
      \mtx{N}^T & \mtx{R}
    \end{bmatrix}
    \begin{bmatrix}
      \mtx{x} \\
      \mtx{u}
    \end{bmatrix} dt
\end{equation*}

where $J$ represents a trade-off between \gls{state} excursion and
\gls{control effort} with the weighting factors $\mtx{Q}$, $\mtx{R}$, and
$\mtx{N}$. $\mtx{Q}$ is the weighting factor for \gls{error}, $\mtx{R}$ is the
weighting factor for \gls{control effort}, and $\mtx{N}$ is a cross term between
\gls{error} and \gls{control effort}.

\begin{align*}
  J &= \int\limits_0^\infty
    \begin{bmatrix}
      \mtx{x} \\
      \mtx{u}
    \end{bmatrix}^T
    \begin{bmatrix}
      \mtx{Q}\mtx{x} + \mtx{N}\mtx{u} \\
      \mtx{N}^T\mtx{x} + \mtx{R}\mtx{u}
    \end{bmatrix} dt \\
  J &= \int\limits_0^\infty
    \begin{bmatrix}
      \mtx{x}^T & \mtx{u}^T
    \end{bmatrix}
    \begin{bmatrix}
      \mtx{Q}\mtx{x} + \mtx{N}\mtx{u} \\
      \mtx{N}^T\mtx{x} + \mtx{R}\mtx{u}
    \end{bmatrix} dt \\
  J &= \int\limits_0^\infty
    \left(\mtx{x}^T\left(\mtx{Q}\mtx{x} + \mtx{N}\mtx{u}\right) +
      \mtx{u}^T\left(\mtx{N}^T\mtx{x} + \mtx{R}\mtx{u}\right)\right) dt \\
  J &= \int\limits_0^\infty
    \left(\mtx{x}^T\mtx{Q}\mtx{x} + \mtx{x}^T\mtx{N}\mtx{u} +
      \mtx{u}^T\mtx{N}^T\mtx{x} + \mtx{u}^T\mtx{R}\mtx{u}\right) dt \\
  J &= \int\limits_0^\infty
    \left(\mtx{x}^T\mtx{Q}\mtx{x} + \mtx{x}^T\mtx{N}\mtx{u} +
      \left(\mtx{x}^T\mtx{N}\mtx{u}\right)^T + \mtx{u}^T\mtx{R}\mtx{u}\right)
    dt \\
  J &= \int\limits_0^\infty
    \left(\mtx{x}^T\mtx{Q}\mtx{x} + 2\mtx{x}^T\mtx{N}\mtx{u} +
      \mtx{u}^T\mtx{R}\mtx{u}\right) dt \\
  J &= \int\limits_0^\infty
    \left(\mtx{x}^T\mtx{Q}\mtx{x} + \mtx{u}^T\mtx{R}\mtx{u} +
      2\mtx{x}^T\mtx{N}\mtx{u}\right) dt
\end{align*}

One application for $\mtx{N}$ is \textit{implicit model following}. That is,
making one system behave like some other desired system. This is used on the
Blackhawk helicopter at NASA Ames research center when they want to make it fly
like experimental aircraft (within the limits of the helicopter's actuators, of
course).

The feedback \gls{control law} which minimizes $J$ subject to the constraint
$\dot{\mtx{x}} = \mtx{A}\mtx{x} + \mtx{B}\mtx{u}$ is

\begin{equation*}
  \mtx{u} = -\mtx{K}\mtx{x}
\end{equation*}

where $\mtx{K}$ is given by

\begin{equation*}
  \mtx{K} = \mtx{R}^{-1} \left(\mtx{B}^T\mtx{P} + \mtx{N}^T\right)
\end{equation*}

and $\mtx{P}$ is found by solving the continuous time algebraic Riccati equation
defined as

\begin{equation*}
  \mtx{A}^T\mtx{P} + \mtx{P}\mtx{A} - \left(\mtx{P}\mtx{B} +
    \mtx{N}\right) \mtx{R}^{-1} \left(\mtx{B}^T\mtx{P} + \mtx{N}^T\right) +
    \mtx{Q} = 0
\end{equation*}

or alternatively

\begin{equation*}
  \mathscrbf{A}^T\mtx{P} + \mtx{P}\mathscrbf{A} -
    \mtx{P}\mtx{B}\mtx{R}^{-1}\mtx{B}^T\mtx{P} + \mathscrbf{Q} = 0
\end{equation*}

with

\begin{align*}
  \mathscrbf{A} &= \mtx{A} - \mtx{B}\mtx{R}^{-1}\mtx{N}^T \\
  \mathscrbf{Q} &= \mtx{Q} - \mtx{N}\mtx{R}^{-1}\mtx{N}^T
\end{align*}

If there is no cross-correlation between \gls{error} and \gls{control effort},
$\mtx{N}$ is a zero matrix and the cost function simplifies to

\begin{equation*}
  J = \int\limits_0^\infty
    \left(\mtx{x}^T\mtx{Q}\mtx{x} + \mtx{u}^T\mtx{R}\mtx{u}\right) dt
\end{equation*}

The feedback \gls{control law} which minimizes this $J$ subject to
$\dot{\mtx{x}} = \mtx{A}\mtx{x} + \mtx{B}\mtx{u}$ is

\begin{equation*}
  \mtx{u} = -\mtx{K}\mtx{x}
\end{equation*}

where $\mtx{K}$ is given by

\begin{equation*}
  \mtx{K} = \mtx{R}^{-1}\mtx{B}^T\mtx{P}
\end{equation*}

and $\mtx{P}$ is found by solving the continuous time algebraic Riccati equation
defined as

\begin{equation*}
  \mtx{A}^T\mtx{P} + \mtx{P}\mtx{A} -
    \mtx{P}\mtx{B}\mtx{R}^{-1}\mtx{B}^T\mtx{P} + \mtx{Q} = 0
\end{equation*}

The discrete time LQR \gls{controller} is computed via a slightly different cost
function, constraint, and resulting algebraic Ricatti equation. Snippet
\ref{lst:lqr} computes the optimal infinite horizon, discrete time LQR
\gls{controller}.

\begin{code}{Python}{build/frccontrol/frccontrol/lqr.py}
  \caption{Infinite horizon, discrete time LQR computation in Python}
  \label{lst:lqr}
\end{code}

Other formulations of LQR for finite horizon and discrete time can be seen on
Wikipedia \cite{bib:wiki_lqr}.

MIT OpenCourseWare has a rigorous proof of the results shown above
\cite{bib:lqr_derivs}.

\section{Zero-order hold for state-space}
\label{sec:deriv_zoh_ss}

Starting with the continuous \gls{model}

\begin{equation*}
  \dot{\mtx{x}}(t) = \mtx{A}\mtx{x}(t) + \mtx{B}\mtx{u}(t)
\end{equation*}

by premultiplying the \gls{model} by $e^{-\mtx{A}t}$, we get

\begin{align*}
  e^{-\mtx{A}t}\dot{\mtx{x}}(t) &= e^{-\mtx{A}t}\mtx{A}\mtx{x}(t) +
    e^{-\mtx{A}t}\mtx{B}\mtx{u}(t) \\
  e^{-\mtx{A}t}\dot{\mtx{x}}(t) - e^{-\mtx{A}t}\mtx{A}\mtx{x}(t) &=
    e^{-\mtx{A}t}\mtx{B}\mtx{u}(t)
\end{align*}

The derivative of the matrix exponential is

\begin{equation*}
  \frac{d}{dt}e^{\mtx{A}t} = \mtx{A}e^{\mtx{A}t} = e^{\mtx{A}t}\mtx{A}
\end{equation*}

so we recognize the previous equation as

\begin{equation*}
  \frac{d}{dt}\left(e^{-\mtx{A}t}\mtx{x}(t)\right) =
    e^{-\mtx{A}t}\mtx{B}\mtx{u}(t)
\end{equation*}

By integrating this equation, we get

\begin{align*}
  e^{-\mtx{A}t}\mtx{x}(t) - e^0\mtx{x}(0) &=
    \int_0^t e^{-\mtx{A}\tau}\mtx{B}\mtx{u}(\tau) \,d\tau \\
  \mtx{x}(t) &= e^{\mtx{A}t}\mtx{x}(0) +
    \int_0^t e^{\mtx{A}(t - \tau)}\mtx{B}\mtx{u}(\tau) \,d\tau
\end{align*}

which is an analytical solution to the continuous \gls{model}. Now we want to
\glslink{discretization}{discretize} it.

\begin{align*}
  \mtx{x}_k &\stackrel{def}{=} \mtx{x}(kT) \\
  \mtx{x}_k &= e^{\mtx{A}kT}\mtx{x}(0) +
    \int_0^{kT} e^{\mtx{A}(kT - \tau)}\mtx{B}\mtx{u}(\tau) \,d\tau \\
  \mtx{x}_{k+1} &= e^{\mtx{A}(k + 1)T}\mtx{x}(0) +
    \int_0^{(k + 1)T} e^{\mtx{A}((k + 1)T - \tau)}\mtx{B}\mtx{u}(\tau) \,d\tau
    \\
  \mtx{x}_{k+1} &= e^{\mtx{A}(k + 1)T}\mtx{x}(0) +
    \int_0^{kT} e^{\mtx{A}((k + 1)T - \tau)}\mtx{B}\mtx{u}(\tau) \,d\tau +
    \int_{kT}^{(k + 1)T} e^{\mtx{A}((k + 1)T - \tau)}\mtx{B}\mtx{u}(\tau)
    \,d\tau \\
  \mtx{x}_{k+1} &= e^{\mtx{A}(k + 1)T}\mtx{x}(0) +
    \int_0^{kT} e^{\mtx{A}((k + 1)T - \tau)}\mtx{B}\mtx{u}(\tau) \,d\tau +
    \int_{kT}^{(k + 1)T} e^{\mtx{A}(kT + T - \tau)}\mtx{B}\mtx{u}(\tau) \,d\tau
    \\
  \mtx{x}_{k+1} &= e^{\mtx{A}T} \underbrace{\left(e^{\mtx{A}kT}\mtx{x}(0) +
    \int_0^{kT} e^{\mtx{A}(kT - \tau)}\mtx{B}\mtx{u}(\tau)
    \,d\tau\right)}_{\mtx{x}_k} +
    \int_{kT}^{(k + 1)T} e^{\mtx{A}(kT + T - \tau)}\mtx{B}\mtx{u}(\tau) \,d\tau
\end{align*}

We assume that $\mtx{u}$ is constant during each timestep, so it can be pulled
out of the integral.

\begin{equation*}
  \mtx{x}_{k+1} = e^{\mtx{A}T}\mtx{x}_k +
    \left(\int_{kT}^{(k + 1)T} e^{\mtx{A}(kT + T - \tau)} \,d\tau\right)
    \mtx{B}\mtx{u}_k
\end{equation*}

The second term can be simplified by substituting it with the function
$v(\tau) = kT + T - \tau$. Note that $d\tau = -dv$.

\begin{align*}
  \mtx{x}_{k+1} &= e^{\mtx{A}T}\mtx{x}_k -
    \left(\int_{v(kT)}^{v((k + 1)T)} e^{\mtx{A}v} \,dv\right)
    \mtx{B}\mtx{u}_k \\
  \mtx{x}_{k+1} &= e^{\mtx{A}T}\mtx{x}_k -
    \left(\int_T^0 e^{\mtx{A}v} \,dv\right) \mtx{B}\mtx{u}_k \\
  \mtx{x}_{k+1} &= e^{\mtx{A}T}\mtx{x}_k +
    \left(\int_0^T e^{\mtx{A}v} \,dv\right) \mtx{B}\mtx{u}_k \\
  \mtx{x}_{k+1} &= e^{\mtx{A}T}\mtx{x}_k +
    \mtx{A}^{-1}e^{\mtx{A}v} \rvert_0^T \mtx{B}\mtx{u}_k \\
  \mtx{x}_{k+1} &= e^{\mtx{A}T}\mtx{x}_k +
    \mtx{A}^{-1}(e^{\mtx{A}T} - e^{\mtx{A}0}) \mtx{B}\mtx{u}_k \\
  \mtx{x}_{k+1} &= e^{\mtx{A}T}\mtx{x}_k +
    \mtx{A}^{-1}(e^{\mtx{A}T} - \mtx{I}) \mtx{B}\mtx{u}_k
\end{align*}

which is an exact solution to the \gls{discretization} problem.

\section{Kalman filter as Luenberger observer}
\label{sec:deriv_kalman_luenberger}

A Luenberger \gls{observer} is defined as

\begin{align}
  \hat{\mtx{x}}_{k+1}^+ &= \mtx{A}\hat{\mtx{x}}_k^- + \mtx{B}\mtx{u}_k + \mtx{L}
    (\mtx{y}_k - \hat{\mtx{y}}_k) \label{eq:luenberger1} \\
  \hat{\mtx{y}}_k &= \mtx{C} \hat{\mtx{x}}_k^- \label{eq:luenberger2}
\end{align}

where a superscript of minus denotes \textit{a priori} and plus denotes
\textit{a posteriori} estimate. Combining equation (\ref{eq:luenberger1}) and
equation (\ref{eq:luenberger2}) gives

\begin{equation} \label{eq:luenberger}
  \hat{\mtx{x}}_{k+1}^+ = \mtx{A}\hat{\mtx{x}}_k^- + \mtx{B}\mtx{u}_k + \mtx{L}
    (\mtx{y}_k - \mtx{C}\hat{\mtx{x}}_k^-)
\end{equation}

The following is a Kalman filter that considers the current update step and the
next predict step together rather than the current predict step and current
update step.

\begin{align}
  \text{Update step} \nonumber \\
  \mtx{K}_k &= \mtx{P}_k^- \mtx{H}^T (\mtx{H}\mtx{P}_k^- \mtx{H}^T +
    \mtx{R})^{-1} \\
  \hat{\mtx{x}}_k^+ &= \hat{\mtx{x}}_k^- + \mtx{K}_k(\mtx{y}_k -
    \mtx{H}\hat{\mtx{x}}_k^-) \label{eq:post2_x} \\
  \mtx{P}_k^+ &= (\mtx{I} - \mtx{K}_k\mtx{H})\mtx{P}_k^- \\
  \text{Predict step} \nonumber \\
  \hat{\mtx{x}}_{k+1}^+ &= \mtx{A}\hat{\mtx{x}}_k^+ + \mtx{B}\mtx{u}_k
    \label{eq:pre2_x} \\
  \mtx{P}_{k+1}^- &= \mtx{A} \mtx{P}_k^+ \mtx{A}^T +
    \mtx{\Gamma}\mtx{Q}\mtx{\Gamma}^T
\end{align}

Substitute equation (\ref{eq:post2_x}) into equation (\ref{eq:pre2_x}).

\begin{align*}
  \hat{\mtx{x}}_{k+1}^+ &= \mtx{A}(\hat{\mtx{x}}_k^- + \mtx{K}_k(\mtx{y}_k -
    \mtx{H}\hat{\mtx{x}}_k^-)) + \mtx{B}\mtx{u}_k \\
  \hat{\mtx{x}}_{k+1}^+ &= \mtx{A}\mtx{x}_k^- + \mtx{A}\mtx{K}_k(\mtx{y}_k -
    \mtx{H}\hat{\mtx{x}}_k^-) + \mtx{B}\mtx{u}_k \\
  \hat{\mtx{x}}_{k+1}^+ &= \mtx{A}\hat{\mtx{x}}_k^- + \mtx{B}\mtx{u}_k +
    \mtx{A}\mtx{K}_k(\mtx{y}_k - \mtx{H}\hat{\mtx{x}}_k^-)
\end{align*}

Let $\mtx{C} = \mtx{H}$ and $\mtx{L} = \mtx{A} \mtx{K}_k$.

\begin{equation} \label{eq:app_kalman_leunberger}
  \hat{\mtx{x}}_{k+1}^+ = \mtx{A}\hat{\mtx{x}}_k^- + \mtx{B}\mtx{u}_k + \mtx{L}
    (\mtx{y}_k - \mtx{C}\hat{\mtx{x}}_k^-)
\end{equation}

which matches equation (\ref{eq:luenberger}). Therefore, the eigenvalues of the
Kalman filter \gls{observer} can be obtained by

\begin{align}
  &\eig(\mtx{A} - \mtx{L}\mtx{C}) \nonumber \\
  &\eig(\mtx{A} - (\mtx{A}\mtx{K}_k)(\mtx{H})) \nonumber \\
  &\eig(\mtx{A}(\mtx{I} - \mtx{K}_k\mtx{H}))
\end{align}

\subsection{Luenberger observer with separate prediction and update}
\label{subsec:deriv_luenberger_separate}

To run a Luenberger \gls{observer} with separate prediction and update steps,
substitute the relationship between the Luenberger \gls{observer} and Kalman
filter matrices derived above into the Kalman filter equations.

Appendix \ref{sec:deriv_kalman_luenberger} shows that $\mtx{C} = \mtx{H}$ and
$\mtx{L} = \mtx{A}\mtx{K}_k$. Since $\mtx{L}$ and $\mtx{A}$ are constant, one
must assume $\mtx{K}_k$ has reached steady-state. Then,
$\mtx{K} = \mtx{A}^{-1}\mtx{L}$. Substitute this and $\mtx{C} = \mtx{H}$ into
the Kalman filter update equation.

\begin{align*}
  \hat{\mtx{x}}_{k+1}^+ &= \hat{\mtx{x}}_{k+1}^- + \mtx{K}(\mtx{y}_{k+1} -
    \mtx{H}\hat{\mtx{x}}_{k+1}^-) \\
  \hat{\mtx{x}}_{k+1}^+ &= \hat{\mtx{x}}_{k+1}^- + \mtx{A}^{-1}\mtx{L}
    (\mtx{y}_{k+1} - \mtx{C}\hat{\mtx{x}}_{k+1}^-)
\end{align*}

Substitute in equation (\ref{eq:z_obsv_y}).

\begin{equation*}
  \hat{\mtx{x}}_{k+1}^+ = \hat{\mtx{x}}_{k+1}^- + \mtx{A}^{-1}\mtx{L}
    (\mtx{y}_{k+1} - \hat{\mtx{y}}_{k+1})
\end{equation*}

The predict step is the same as the Kalman filter's. Therefore, a Luenberger
\gls{observer} run with prediction and update steps is written as follows.

\begin{align}
  \text{Predict step} \nonumber \\
  \hat{\mtx{x}}_{k+1}^- &= \mtx{A}\hat{\mtx{x}}_k^- + \mtx{B}\mtx{u}_k \\
  \text{Update step} \nonumber \\
  \hat{\mtx{x}}_{k+1}^+ &= \hat{\mtx{x}}_{k+1}^- + \mtx{A}^{-1}\mtx{L}
    (\mtx{y}_{k+1} - \hat{\mtx{y}}_{k+1}) \\
  \hat{\mtx{y}}_{k+1} &= \mtx{C} \hat{\mtx{x}}_{k+1}^-
\end{align}

\section{Trapezoidal motion profile}
\label{sec:deriv_trapezoid_profile}

\section{S-curve motion profile}
\label{sec:deriv_s-curve_profile}
