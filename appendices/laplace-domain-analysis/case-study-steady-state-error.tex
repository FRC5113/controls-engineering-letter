\section{Case study: steady-state error}
\index{steady-state error}

To demonstrate the problem of \gls{steady-state error}, we will use a DC brushed
motor controlled by a velocity PID controller. A DC brushed motor has a transfer
function from voltage ($V$) to angular velocity ($\dot{\theta}$) of

\begin{equation}
  G(s) = \frac{\dot{\Theta}(s)}{V(s)} = \frac{K}{(Js+b)(Ls+R)+K^2}
\end{equation}

First, we'll try controlling it with a P controller defined as

\begin{equation*}
  K(s) = K_p
\end{equation*}

When these are in unity feedback, the transfer function from the input voltage
to the error is

\begin{align*}
  \frac{E(s)}{V(s)} &= \frac{1}{1 + K(s)G(s)} \\
  E(s) &= \frac{1}{1 + K(s)G(s)} V(s) \\
  E(s) &= \frac{1}{1 + (K_p) \left(\frac{K}{(Js+b)(Ls+R)+K^2}\right)} V(s) \\
  E(s) &= \frac{1}{1 + \frac{K_p K}{(Js+b)(Ls+R)+K^2}} V(s)
\end{align*}

The steady-state of a transfer function can be found via

\begin{equation}
  \lim_{s\to0} sH(s)
\end{equation}

since steady-state has an input frequency of zero.

\begin{align}
  e_{ss} &= \lim_{s\to0} sE(s) \nonumber \\
  e_{ss} &= \lim_{s\to0} s \frac{1}{1 + \frac{K_p K}{(Js+b)(Ls+R)+K^2}} V(s)
    \nonumber \\
  e_{ss} &= \lim_{s\to0} s \frac{1}{1 + \frac{K_p K}{(Js+b)(Ls+R)+K^2}}
    \frac{1}{s} \nonumber \\
  e_{ss} &= \lim_{s\to0} \frac{1}{1 + \frac{K_p K}{(Js+b)(Ls+R)+K^2}}
    \nonumber \\
  e_{ss} &= \frac{1}{1 + \frac{K_p K}{(J(0)+b)(L(0)+R)+K^2}} \nonumber \\
  e_{ss} &= \frac{1}{1 + \frac{K_p K}{bR+K^2}} \label{eq:ss_nonzero}
\end{align}

Notice that the \gls{steady-state error} is nonzero. To fix this, an integrator
must be included in the controller.

\begin{equation*}
  K(s) = K_p + \frac{K_i}{s}
\end{equation*}

The same steady-state calculations are performed as before with the new
controller.

\begin{align*}
  \frac{E(s)}{V(s)} &= \frac{1}{1 + K(s)G(s)} \\
  E(s) &= \frac{1}{1 + K(s)G(s)} V(s) \\
  E(s) &= \frac{1}{1 + \left(K_p + \frac{K_i}{s}\right)
    \left(\frac{K}{(Js+b)(Ls+R)+K^2}\right)} \left(\frac{1}{s}\right) \\
  e_{ss} &= \lim_{s\to0} s \frac{1}{1 + \left(K_p + \frac{K_i}{s}\right)
    \left(\frac{K}{(Js+b)(Ls+R)+K^2}\right)} \left(\frac{1}{s}\right) \\
  e_{ss} &= \lim_{s\to0} \frac{1}{1 + \left(K_p + \frac{K_i}{s}\right)
    \left(\frac{K}{(Js+b)(Ls+R)+K^2}\right)} \\
  e_{ss} &= \lim_{s\to0} \frac{1}{1 + \left(K_p + \frac{K_i}{s}\right)
    \left(\frac{K}{(Js+b)(Ls+R)+K^2}\right)} \frac{s}{s} \\
  e_{ss} &= \lim_{s\to0} \frac{s}{s + \left(K_p s + K_i\right)
    \left(\frac{K}{(Js+b)(Ls+R)+K^2}\right)} \\
  e_{ss} &= \frac{0}{0 + (K_p (0) + K_i)
    \left(\frac{K}{(J(0)+b)(L(0)+R)+K^2}\right)} \\
  e_{ss} &= \frac{0}{K_i \frac{K}{bR+K^2}}
\end{align*}

The denominator is nonzero, so $e_{ss} = 0$. Therefore, an integrator is
required to eliminate \gls{steady-state error} in all cases for this
\gls{model}.

It should be noted that $e_{ss}$ in equation \eqref{eq:ss_nonzero} approaches
zero for $K_p = \infty$. This is known as a bang-bang controller. In practice,
an infinite switching frequency cannot be achieved, but it may be close enough
for some performance specifications.
