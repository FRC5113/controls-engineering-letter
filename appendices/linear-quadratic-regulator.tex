\chapterimage{appendices.jpg}{Sunset in an airplane over New Mexico}

\chapter{Linear-quadratic regulator}
\label{ch:deriv_lqr}

This appendix will go into more detail on the linear-quadratic regulator's
derivation and interesting applications.

\section{Derivation}

Let a continuous time linear \gls{system} be defined as

\begin{equation}
  \dot{\mtx{x}} = \mtx{A}\mtx{x} + \mtx{B}\mtx{u}
\end{equation}

with the cost function

\begin{equation*}
  J = \int\limits_0^\infty
    \begin{bmatrix}
      \mtx{x} \\
      \mtx{u}
    \end{bmatrix}^T
    \begin{bmatrix}
      \mtx{Q} & \mtx{N} \\
      \mtx{N}^T & \mtx{R}
    \end{bmatrix}
    \begin{bmatrix}
      \mtx{x} \\
      \mtx{u}
    \end{bmatrix} dt
\end{equation*}

where $J$ represents a trade-off between \gls{state} excursion and
\gls{control effort} with the weighting factors $\mtx{Q}$, $\mtx{R}$, and
$\mtx{N}$. $\mtx{Q}$ is the weight matrix for \gls{error}, $\mtx{R}$ is the
weight matrix for \gls{control effort}, and $\mtx{N}$ is a cross weight matrix
between \gls{error} and \gls{control effort}. $\mtx{N}$ is commonly utilized
when penalizing the output in addition to the state and input.

\begin{align*}
  J &= \int\limits_0^\infty
    \begin{bmatrix}
      \mtx{x} \\
      \mtx{u}
    \end{bmatrix}^T
    \begin{bmatrix}
      \mtx{Q}\mtx{x} + \mtx{N}\mtx{u} \\
      \mtx{N}^T\mtx{x} + \mtx{R}\mtx{u}
    \end{bmatrix} dt \\
  J &= \int\limits_0^\infty
    \begin{bmatrix}
      \mtx{x}^T & \mtx{u}^T
    \end{bmatrix}
    \begin{bmatrix}
      \mtx{Q}\mtx{x} + \mtx{N}\mtx{u} \\
      \mtx{N}^T\mtx{x} + \mtx{R}\mtx{u}
    \end{bmatrix} dt \\
  J &= \int\limits_0^\infty
    \left(\mtx{x}^T\left(\mtx{Q}\mtx{x} + \mtx{N}\mtx{u}\right) +
      \mtx{u}^T\left(\mtx{N}^T\mtx{x} + \mtx{R}\mtx{u}\right)\right) dt \\
  J &= \int\limits_0^\infty
    \left(\mtx{x}^T\mtx{Q}\mtx{x} + \mtx{x}^T\mtx{N}\mtx{u} +
      \mtx{u}^T\mtx{N}^T\mtx{x} + \mtx{u}^T\mtx{R}\mtx{u}\right) dt \\
  J &= \int\limits_0^\infty
    \left(\mtx{x}^T\mtx{Q}\mtx{x} + \mtx{x}^T\mtx{N}\mtx{u} +
      \left(\mtx{x}^T\mtx{N}\mtx{u}\right)^T + \mtx{u}^T\mtx{R}\mtx{u}\right)
    dt \\
  J &= \int\limits_0^\infty
    \left(\mtx{x}^T\mtx{Q}\mtx{x} + 2\mtx{x}^T\mtx{N}\mtx{u} +
      \mtx{u}^T\mtx{R}\mtx{u}\right) dt \\
  J &= \int\limits_0^\infty
    \left(\mtx{x}^T\mtx{Q}\mtx{x} + \mtx{u}^T\mtx{R}\mtx{u} +
      2\mtx{x}^T\mtx{N}\mtx{u}\right) dt
\end{align*}

The feedback \gls{control law} which minimizes $J$ subject to the constraint
$\dot{\mtx{x}} = \mtx{A}\mtx{x} + \mtx{B}\mtx{u}$ is

\begin{equation*}
  \mtx{u} = -\mtx{K}\mtx{x}
\end{equation*}

where $\mtx{K}$ is given by

\begin{equation*}
  \mtx{K} = \mtx{R}^{-1} \left(\mtx{B}^T\mtx{S} + \mtx{N}^T\right)
\end{equation*}

and $\mtx{S}$ is found by solving the continuous time algebraic Riccati equation
defined as

\begin{equation*}
  \mtx{A}^T\mtx{S} + \mtx{S}\mtx{A} - \left(\mtx{S}\mtx{B} +
    \mtx{N}\right) \mtx{R}^{-1} \left(\mtx{B}^T\mtx{S} + \mtx{N}^T\right) +
    \mtx{Q} = 0
\end{equation*}

or alternatively

\begin{equation*}
  \mathscrbf{A}^T\mtx{S} + \mtx{S}\mathscrbf{A} -
    \mtx{S}\mtx{B}\mtx{R}^{-1}\mtx{B}^T\mtx{S} + \mathscrbf{Q} = 0
\end{equation*}

with

\begin{align*}
  \mathscrbf{A} &= \mtx{A} - \mtx{B}\mtx{R}^{-1}\mtx{N}^T \\
  \mathscrbf{Q} &= \mtx{Q} - \mtx{N}\mtx{R}^{-1}\mtx{N}^T
\end{align*}

If there is no cross-correlation between \gls{error} and \gls{control effort},
$\mtx{N}$ is a zero matrix and the cost function simplifies to

\begin{equation*}
  J = \int\limits_0^\infty
    \left(\mtx{x}^T\mtx{Q}\mtx{x} + \mtx{u}^T\mtx{R}\mtx{u}\right) dt
\end{equation*}

The feedback \gls{control law} which minimizes this $J$ subject to
$\dot{\mtx{x}} = \mtx{A}\mtx{x} + \mtx{B}\mtx{u}$ is

\begin{equation*}
  \mtx{u} = -\mtx{K}\mtx{x}
\end{equation*}

where $\mtx{K}$ is given by

\begin{equation*}
  \mtx{K} = \mtx{R}^{-1}\mtx{B}^T\mtx{S}
\end{equation*}

and $\mtx{S}$ is found by solving the continuous time algebraic Riccati equation
defined as

\begin{equation*}
  \mtx{A}^T\mtx{S} + \mtx{S}\mtx{A} -
    \mtx{S}\mtx{B}\mtx{R}^{-1}\mtx{B}^T\mtx{S} + \mtx{Q} = 0
\end{equation*}

The discrete time LQR \gls{controller} is computed via a slightly different cost
function, constraint, and resulting algebraic Riccati equation. Snippet
\ref{lst:lqr} computes the optimal infinite horizon, discrete time LQR
\gls{controller}.

\begin{code}{Python}{build/frccontrol/frccontrol/lqr.py}
  \caption{Infinite horizon, discrete time LQR computation in Python}
  \label{lst:lqr}
\end{code}

Other formulations of LQR for finite horizon and discrete time can be seen on
Wikipedia \cite{bib:wiki_lqr}.

MIT OpenCourseWare has a rigorous proof of the results shown above
\cite{bib:lqr_derivs}.

\section{Implicit model following}

If we want to design a feedback controller that erases the dynamics of our
system and makes it behave like some other system, we can use \textit{implicit
model following}. This is used on the Blackhawk helicopter at NASA Ames research
center when they want to make it fly like experimental aircraft (within the
limits of the helicopter's actuators, of course).
