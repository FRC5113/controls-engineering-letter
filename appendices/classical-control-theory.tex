\chapterimage{appendices.jpg}{Sunset in an airplane over New Mexico}

\chapter{Classical control theory}

This appendix is provided for those who are curious about a lower-level
interpretation of control systems. It describes what a transfer function is and
shows how they can be used to analyze dynamical systems. Emphasis is placed on
the geometric intuition of this analysis rather than the frequency domain math.
Many tools exclusive to classical control theory (root locus, Bode plots,
Nyquist plots, etc.) aren't useful for or relevant to the examples presented in
the main chapters, so they would serve only to complicate the learning process.

Classical control theory is interesting in that one can perform stability and
robustness analyses and design reasonable controllers for systems on the back of
a napkin. It's also useful for controlling systems which don't have a model. One
can generate a Bode plot of a system by feeding in sine waves of increasing
frequency and recording the amplitude of the output oscillations. This data can
be used to create a transfer function or lead and lag compensators can be
applied directly based on the Bode plot. However, computing power is much more
plentiful nowadays; we should take advantage of this in the design phase and use
the more modern tools that enables when it makes sense.

\renewcommand*{\chapterpath}{\partpath/classical-control-theory}
\section{Classical vs modern control theory}

State-space notation provides a more convenient and compact way to model and
analyze \glspl{system} with multiple \glspl{input} and \glspl{output}. For a
\gls{system} with $p$ \glspl{input} and $q$ \glspl{output}, we would have to
write $q \times p$ transfer functions to represent it. Not only is the resulting
algebra unwieldy, but it only works for linear \glspl{system}. Including nonzero
initial conditions complicates the algebra even more. State-space representation
uses the time domain instead of the Laplace domain, so it can model nonlinear
\glspl{system}\footnote{This book primarily focuses on analysis and control of
linear \glspl{system}. See chapter \ref{ch:nonlinear_control} for more on
nonlinear control.} and trivially supports nonzero initial conditions.

If modern control theory is so great and classical control theory isn't needed
to use it, why learn classical control theory at all? We teach classical control
theory because it provides a framework within which to understand results from
the mathematical machinery of modern control as well as vocabulary with which to
communicate that understanding. For example, faster poles (poles moved to the
left in the s-plane) mean faster decay, and oscillation means there is at least
one pair of complex conjugate poles. Not only can you describe what happened
succinctly, but you know why it happened from a theoretical perspective.

This book uses LQR and modern control over, say, loop shaping with Bode and
Nyquist plots because we have accurate dynamical models to leverage, and LQR
allows directly expressing what the author is concerned with optimizing: state
excursion relative to control effort. Applying lead and lag compensators, while
effective for robust controller design, doesn't provide the same expressive
power.

\chapterimage{appendices.jpg}{Sunset in an airplane over New Mexico}

\chapter{Transfer functions}

This chapter is intended to provide a framework within which to understand
results from the mathematical machinery of modern control as well as vocabulary
to communicate that understanding. We will briefly discuss what transfer
functions are, how the locations of poles and zeroes affect
\gls{system response} and stability, and how controllers affect pole locations.

\renewcommand*{\chapterpath}{\partpath/transfer-functions}
\subsection{Laplace transform}

The Laplace domain is a generalization of the frequency domain that has the
frequency ($j\omega$) on the imaginary y-axis and a real number on the x-axis,
yielding a two-dimensional coordinate system. We represent coordinates in this
space as a complex number $s = \sigma + j\omega$. The real part $\sigma$
corresponds to the x-axis and the imaginary part $j\omega$ corresponds to the
y-axis (see figure \ref{fig:laplace_domain}).
\begin{bookfigure}
  \begin{tikzpicture}[auto, >=latex']
    %\draw [help lines] (-4,-2) grid (4,4);

    % Draw main axes
    \draw[<->] (-4,0) -- (4,0) node[below] {\small Re($\sigma$)};
    \draw[<->] (0,-3) -- (0,3) node[right] {\small Im($j\omega$)};
  \end{tikzpicture}

  \caption{Laplace domain}
  \label{fig:laplace_domain}
\end{bookfigure}

To extend our analogy of each coordinate being represented by some basis, we now
have the y coordinate representing the oscillation frequency of the
\gls{system response} (the frequency domain) and also the x coordinate
representing the speed at which that oscillation decays and the \gls{system}
converges to zero (i.e., a decaying exponential). Figure
\ref{fig:impulse_response_poles} shows this for various points.

If we move the component frequencies in the Fmajor4 chord example parallel to
the real axis to $\sigma = -25$, the resulting time domain response attenuates
according to the decaying exponential $e^{-25t}$ (see figure
\ref{fig:laplace_chord_attenuating}).
\begin{svg}{build/\sectionpath/laplace_chord_attenuating}
  \caption{Fmajor4 chord at $\sigma = 0$ and $\sigma = -25$}
  \label{fig:laplace_chord_attenuating}
\end{svg}

Note that this explanation as a basis isn't exact because the Laplace basis
isn't orthogonal (that is, the x and y coordinates affect each other and have
cross-talk). In the frequency domain, we had a basis of sine waves that we
represented as delta functions in the frequency domain. Each frequency
contribution was independent of the others. In the Laplace domain, this is not
the case; a pure exponential is $\frac{1}{s - a}$ (a rational function where $a$
is a real number) instead of a delta function. This function is nonzero at
points that aren't actually frequencies present in the time domain. Figure
\ref{fig:laplace_chord_3d} demonstrates this, which shows the Laplace transform
of the Fmajor4 chord plotted in 3D.
\begin{svg}{build/\sectionpath/laplace_chord_3d}
  \caption{Laplace transform of Fmajor4 chord plotted in 3D}
  \label{fig:laplace_chord_3d}
\end{svg}

Notice how the values of the function around each component frequency decrease
according to $\frac{1}{\sqrt{x^2 + y^2}}$ in the $x$ and $y$ directions (in just
the $x$ direction, it would be $\frac{1}{x}$).

\section{Parts of a transfer function}

A transfer function maps an input coordinate to an output coordinate in the
Laplace domain. These can be obtained by applying the Laplace transform to a
differential equation and rearranging the terms to obtain a ratio of the output
variable to the input variable. Equation (\ref{eq:transfer_func}) is an example
of a transfer function.

\begin{equation} \label{eq:transfer_func}
  H(s) = \frac{\overbrace{(s-9+9i)(s-9-9i)}^{zeroes}}
    {\underbrace{s(s+10)}_{poles}}
\end{equation}

\subsection{Poles and zeroes}

The roots of factors in the numerator of a transfer function are called
\textit{zeroes} because they make the transfer function approach zero. Likewise,
the roots of factors in the denominator of a transfer function are called
\textit{poles} because they make the transfer function approach infinity; on a
3D graph, these look like the poles of a circus tent (see figure
\ref{fig:tf_3d}).

When the factors of the denominator are broken apart using partial fraction
expansion into something like $\frac{A}{s + a} + \frac{B}{s + b}$, the constants
$A$ and $B$ are called residues, which determine how much each pole contributes
to the \gls{system response}.

The factors representing poles are each the Laplace transform of a decaying
exponential\footnote{We are handwaving Laplace transform derivations because
they are complicated and neither relevant nor useful.}. That means the time
domain responses of \glspl{system} comprise decaying exponentials (e.g.,
$y = e^{-t}$).

\begin{svg}{build/code/tf_3d}
  \caption{Equation \ref{eq:transfer_func} plotted in 3D}
  \label{fig:tf_3d}
\end{svg}

\begin{remark}
  Imaginary poles and zeroes always come in complex conjugate pairs (e.g.,
  $-2 + 3i$, $-2 - 3i$).
\end{remark}

\index{Stability!poles and zeroes}
The locations of the closed-loop poles in the complex plane determine the
stability of the \gls{system}. Each pole represents a frequency mode of the
\gls{system}, and their location determines how much of each response is induced
for a given input frequency. Figure \ref{fig:impulse_response_poles} shows the
\glspl{impulse response} in the time domain for transfer functions with various
pole locations. They all have an initial condition of $1$.

\begin{bookfigure}
  \begin{tikzpicture}[auto, >=latex']
    % \draw [help lines] (-4,-2) grid (4,4);

    % Draw main axes
    \draw[->] (-4.2,0) -- (4.2,0) node[below] {\small Re($\sigma$)};
    \draw[->] (0,-2) -- (0,4.2) node[right] {\small Im($j\omega$)};

    % Stable: e^-1.75t * cos(1.75wt) (80/3*w for readability)
    \drawtimeplot{-2.125cm}{2.5cm}{0.125cm}{0.44375cm}{
      exp(-1.75 * \x) * cos(80/3 * 1.75 * deg(\x))}
    \drawpole{-1.75cm}{1.75cm}

    % Stable: e^-2.5t
    \drawtimeplot{-2.25cm}{0.75cm}{0.125cm}{0.125cm}{exp(-2 * \x)}
    \drawpole{-2cm}{0cm}

    % Stable: e^-t
    \drawtimeplot{-1.125cm}{-0.75cm}{0.125cm}{0.125cm}{exp(-\x)}
    \drawpole{-1cm}{0cm}

    % Marginally stable: cos(wt) (80/3*w for readability)
    \drawtimeplot{-0.75cm}{1.125cm}{0.125cm}{0.44375cm}{cos(80/3 * deg(\x))}
    \drawpole{0cm}{1cm}

    % Marginally stable cos(2wt) (80/3*w for readability)
    \drawtimeplot{0cm}{2.75cm}{0.125cm}{0.44375cm}{cos(80/3 * 2 * deg(\x))}
    \drawpole{0cm}{2cm}

    % Integrator
    \drawtimeplot{0.25cm}{-0.75cm}{0.125cm}{0.125cm}{1}
    \drawpole{0cm}{0cm}

    % Unstable: e^t
    \drawtimeplot{1.125cm}{0.75cm}{0.125cm}{0.125cm}{exp(\x)}
    \drawpole{1cm}{0cm}

    % Unstable: e^2t
    \drawtimeplot{2.25cm}{-0.75cm}{0.125cm}{0.125cm}{exp(2 * \x)}
    \drawpole{2cm}{0cm}

    % Unstable: e^0.75t * cos(1.75wt) (80/3*w for readability)
    \drawtimeplot{1.5cm}{2.25cm}{0.125cm}{0.44375cm}{
      exp(0.75 * \x) * cos(80/3 * 1.75 * deg(\x))}
    \drawpole{0.75cm}{1.75cm}

    % LHP and RHP labels
    \draw (-3.5,1.5) node {LHP};
    \draw (3.5,1.5) node {RHP};

    % Stable and unstable labels
    \draw (-2.5,3.5) node {\small Stable};
    \draw (2.5,3.5) node {\small Unstable};
  \end{tikzpicture}

  \caption{Impulse response vs pole location}
  \label{fig:impulse_response_poles}
\end{bookfigure}

\begin{booktable}
  \begin{tabular}{|ll|}
    \hline
    \rowcolor{headingbg}
    \textbf{Location} & \textbf{Stability} \\
    \hline
    Left Half-plane (LHP) & Stable \\
    Imaginary axis & Marginally stable \\
    Right Half-plane (RHP) & Unstable \\
    \hline
  \end{tabular}

  \caption{Pole location and stability}
  \label{tab:pole_locations}
\end{booktable}

When a \gls{system} is stable, its output may oscillate but it converges to
steady-state. When a \gls{system} is marginally stable, its output oscillates at
a constant amplitude forever. When a \gls{system} is unstable, its output grows
without bound.

\subsection{Nonminimum phase zeroes}

While poles in the RHP are unstable, the same is not true for zeroes. They can
be characterized by the \gls{system} initially moving in the wrong direction
before heading toward the \gls{reference}. Since the poles always move toward
the zeroes, zeroes impose a ``speed limit" on the \gls{system response} because
it takes a finite amount of time to move the wrong direction, then change
directions.

One example of this type of \gls{system} is bicycle steering. Try riding a
bicycle without holding the handle bars, then poke the right handle; the bicycle
turns right. Furthermore, if one is holding the handlebars and wants to turn
left, rotating the handlebars counterclockwise will make the bicycle fall toward
the right. The rider has to lean into the turn and overpower the nonminimum
phase dynamics to go the desired direction.

Another example is a Segway. To move forward by some distance, the Segway must
first roll backward to rotate the Segway forward. Once the Segway starts falling
in that direction, it begins rolling forward to avoid falling over until
it reaches the target distance. At that point, the Segway increases its forward
speed to pitch backward and slow itself down. To come to a stop, the Segway
rolls backward again to level itself out.

\subsection{Pole-zero cancellation}
\label{subsec:pole-zero_cancellation}

Pole-zero cancellation occurs when a pole and zero are located at the same place
in the s-plane. This effectively eliminates the contribution of each to the
\gls{system} dynamics. By placing poles and zeroes at various locations (this is
done by placing transfer functions in series), we can eliminate undesired
\gls{system} dynamics. While this may appear to be a useful design tool at
first, there are major caveats. Most of these are due to \gls{model} uncertainty
resulting in poles which aren't in the locations the controls designer expected.

Notch filters are typically used to dampen a specific range of frequencies in
the \gls{system response}. If its band is made too narrow, it can still leave the
undesirable dynamics, but now you can no longer measure them in the response.
They are still happening, but they are what's called \textit{unobservable}.

Never pole-zero cancel unstable or nonminimum phase dynamics. If the \gls{model}
doesn't quite reflect reality, an attempted pole cancellation by placing a
nonminimum phase zero results in the pole still moving to the zero placed next
to it. You have the same dynamics as before, but the pole is also stuck where it
is no matter how much \gls{feedback gain} is applied. For an attempted
nonminimum phase zero cancellation, you have effectively placed an unstable pole
that's unobservable. This means the \gls{system} will be going unstable and
blowing up, but you won't be able to detect this and react to it.

Keep in mind when making design decisions that the \gls{model} likely isn't
perfect. The whole point of feedback control is to be robust to this kind of
uncertainty.

\section{Transfer functions in feedback}

For \glspl{controller} to \glslink{regulator}{regulate} a \gls{system} or
\glslink{tracking}{track} a reference, they must be placed in positive or
negative feedback with the \gls{plant} (whether to use positive or negative
depends on the \gls{plant} in question). Stable feedback loops attempt to make
the \gls{output} equal the \gls{reference}.

\begin{bookfigure}
  \begin{tikzpicture}[auto, >=latex']
    % Place the blocks
    \node [name=input] {$X(s)$};
    \node [sum, right=of input] (sum) {};
    \node [block, right=of sum] (K) {$K$};
    \node [block, right=of K] (G) {$G$};
    \node [right=of G] (output) {$Y(s)$};
    \node [block, below=of $(K)!0.5!(G)$] (H) {$H$};

    % Connect the nodes
    \draw [arrow] (input) -- node[pos=0.85] {$+$} (sum);
    \draw [arrow] (sum) -- node {} (K);
    \draw [arrow] (K) -- node {} (G);
    \draw [arrow] (G) -- node[name=y] {} (output);
    \draw [arrow] (y) |- (H);
    \draw [arrow] (H) -| node[pos=0.97, right] {$-$} (sum);
  \end{tikzpicture}

  \caption{Feedback controller block diagram}
  \label{fig:feedback_controller_block_diagram}

  \begin{figurekey}
    \begin{tabular}{llll}
      $X(s)$ & input & $H$ & measurement transfer function \\
      $K$ & controller gain & $Y(s)$ & output \\
      $G$ & plant transfer function & & \\
    \end{tabular}
  \end{figurekey}
\end{bookfigure}

The transfer function of figure \ref{fig:feedback_controller_block_diagram}, a
\gls{control system} diagram with feedback, from input to output is

\begin{equation}
  G_{cl}(s) = \frac{Y(s)}{X(s)} = \frac{KG}{1 + KGH}
\end{equation}

The numerator is the \gls{open-loop gain} and the denominator is one plus the
gain around the feedback loop, which may include parts of the
\gls{open-loop gain} (see appendix \ref{sec:deriv_tf_feedback} for a
derivation). As another example, the transfer function from the input to the
\gls{error} is

\begin{equation}
  G_{cl}(s) = \frac{E(s)}{X(s)} = \frac{1}{1 + KGH}
\end{equation}

The roots of the denominator of $G_{cl}(s)$ are different from those of the
open-loop transfer function $KG(s)$. These are called the closed-loop poles.

\section{Classical vs modern control}

State-space notation provides a more convenient and compact way to model and
analyze \glspl{system} with multiple \glspl{input} and \glspl{output}. For a
\gls{system} with $p$ \glspl{input} and $q$ \glspl{output}, we would have to
write $q \times p$ transfer functions to represent it. Not only is the resulting
algebra unwieldy, but it only works for linear \glspl{system}. Including nonzero
initial conditions complicates the algebra even more. State-space representation
uses the time domain instead of the Laplace domain, so it can model nonlinear
\glspl{system}\footnote{This book focuses on analysis and control of linear
\glspl{system}. See chapter \ref{ch:nonlinear_control} for more on nonlinear
control.} and trivially supports nonzero initial conditions.

If modern control theory is so great and classical control theory isn't needed
to use it, why learn classical control theory at all? We teach classical control
theory because it provides a framework within which to understand results from
the mathematical machinery of modern control as well as vocabulary with which to
communicate that understanding. For example, faster poles (poles moved to the
left in the s-plane) mean faster decay, and oscillation means there is at least
one pair of complex conjugate poles. Not only can you describe what happened
succinctly, but you know why it happened from a theoretical perspective.


\chapterimage{laplace-domain-analysis.jpg}{Treeline by Crown/Merril bus stop at UCSC}

\chapter{Laplace domain analysis}

This chapter briefly discusses what transfer functions are, how the locations of
poles and zeroes affects \gls{system response} and stability, and how
controllers affect pole locations. The case studies cover various aspects of PID
control using the algebraic approach of transfer functions.

\renewcommand*{\chapterpath}{\partpath/laplace-domain-analysis}
\section{Fourier transform}

The Fourier transform decomposes a function of time into its component
frequencies. Each of these frequencies is part of what's called a
\textit{basis}. These basis waveforms can be multiplied by their respective
contribution amount and summed to produce the original signal (this weighted sum
is called a linear combination). In other words, the Fourier transform provides
a way for us to determine, given some signal, what frequencies can we add
together and in what amounts to produce the original signal.

Think of an Fmajor4 chord which has the notes $F_4$ ($349.23$ Hz), $A_4$ ($440$
Hz), and $C_4$ ($261.63$ Hz). The waveform over time looks like figure
\ref{fig:fourier_chord}.
\begin{svg}{build/\chapterpath/fourier_chord}
  \caption{Frequency decomposition of Fmajor4 chord}
  \label{fig:fourier_chord}
\end{svg}

Notice how this complex waveform can be represented just by three frequencies.
They show up as Dirac delta functions\footnote{The Dirac delta function is zero
everywhere except at the origin. The nonzero region has an infinitesimal width
and has a height such that the area within that region is $1$.} in the frequency
domain with the area underneath them equal to their contribution (see figure
\ref{fig:fourier_chord_fft}).
\begin{svg}{build/\chapterpath/fourier_chord_fft}
  \caption{Fourier transform of Fmajor4 chord}
  \label{fig:fourier_chord_fft}
\end{svg}

\subsection{Laplace transform}

The Laplace domain is a generalization of the frequency domain that has the
frequency ($j\omega$) on the imaginary y-axis and a real number on the x-axis,
yielding a two-dimensional coordinate system. We represent coordinates in this
space as a complex number $s = \sigma + j\omega$. The real part $\sigma$
corresponds to the x-axis and the imaginary part $j\omega$ corresponds to the
y-axis (see figure \ref{fig:laplace_domain}).
\begin{bookfigure}
  \begin{tikzpicture}[auto, >=latex']
    %\draw [help lines] (-4,-2) grid (4,4);

    % Draw main axes
    \draw[<->] (-4,0) -- (4,0) node[below] {\small Re($\sigma$)};
    \draw[<->] (0,-3) -- (0,3) node[right] {\small Im($j\omega$)};
  \end{tikzpicture}

  \caption{Laplace domain}
  \label{fig:laplace_domain}
\end{bookfigure}

To extend our analogy of each coordinate being represented by some basis, we now
have the y coordinate representing the oscillation frequency of the
\gls{system response} (the frequency domain) and also the x coordinate
representing the speed at which that oscillation decays and the \gls{system}
converges to zero (i.e., a decaying exponential). Figure
\ref{fig:impulse_response_poles} shows this for various points.

If we move the component frequencies in the Fmajor4 chord example parallel to
the real axis to $\sigma = -25$, the resulting time domain response attenuates
according to the decaying exponential $e^{-25t}$ (see figure
\ref{fig:laplace_chord_attenuating}).
\begin{svg}{build/\sectionpath/laplace_chord_attenuating}
  \caption{Fmajor4 chord at $\sigma = 0$ and $\sigma = -25$}
  \label{fig:laplace_chord_attenuating}
\end{svg}

Note that this explanation as a basis isn't exact because the Laplace basis
isn't orthogonal (that is, the x and y coordinates affect each other and have
cross-talk). In the frequency domain, we had a basis of sine waves that we
represented as delta functions in the frequency domain. Each frequency
contribution was independent of the others. In the Laplace domain, this is not
the case; a pure exponential is $\frac{1}{s - a}$ (a rational function where $a$
is a real number) instead of a delta function. This function is nonzero at
points that aren't actually frequencies present in the time domain. Figure
\ref{fig:laplace_chord_3d} demonstrates this, which shows the Laplace transform
of the Fmajor4 chord plotted in 3D.
\begin{svg}{build/\sectionpath/laplace_chord_3d}
  \caption{Laplace transform of Fmajor4 chord plotted in 3D}
  \label{fig:laplace_chord_3d}
\end{svg}

Notice how the values of the function around each component frequency decrease
according to $\frac{1}{\sqrt{x^2 + y^2}}$ in the $x$ and $y$ directions (in just
the $x$ direction, it would be $\frac{1}{x}$).

\chapterimage{appendices.jpg}{Sunset in an airplane over New Mexico}

\chapter{Transfer functions}

This chapter is intended to provide a framework within which to understand
results from the mathematical machinery of modern control as well as vocabulary
to communicate that understanding. We will briefly discuss what transfer
functions are, how the locations of poles and zeroes affect
\gls{system response} and stability, and how controllers affect pole locations.

\renewcommand*{\chapterpath}{\partpath/transfer-functions}
\subsection{Laplace transform}

The Laplace domain is a generalization of the frequency domain that has the
frequency ($j\omega$) on the imaginary y-axis and a real number on the x-axis,
yielding a two-dimensional coordinate system. We represent coordinates in this
space as a complex number $s = \sigma + j\omega$. The real part $\sigma$
corresponds to the x-axis and the imaginary part $j\omega$ corresponds to the
y-axis (see figure \ref{fig:laplace_domain}).
\begin{bookfigure}
  \begin{tikzpicture}[auto, >=latex']
    %\draw [help lines] (-4,-2) grid (4,4);

    % Draw main axes
    \draw[<->] (-4,0) -- (4,0) node[below] {\small Re($\sigma$)};
    \draw[<->] (0,-3) -- (0,3) node[right] {\small Im($j\omega$)};
  \end{tikzpicture}

  \caption{Laplace domain}
  \label{fig:laplace_domain}
\end{bookfigure}

To extend our analogy of each coordinate being represented by some basis, we now
have the y coordinate representing the oscillation frequency of the
\gls{system response} (the frequency domain) and also the x coordinate
representing the speed at which that oscillation decays and the \gls{system}
converges to zero (i.e., a decaying exponential). Figure
\ref{fig:impulse_response_poles} shows this for various points.

If we move the component frequencies in the Fmajor4 chord example parallel to
the real axis to $\sigma = -25$, the resulting time domain response attenuates
according to the decaying exponential $e^{-25t}$ (see figure
\ref{fig:laplace_chord_attenuating}).
\begin{svg}{build/\sectionpath/laplace_chord_attenuating}
  \caption{Fmajor4 chord at $\sigma = 0$ and $\sigma = -25$}
  \label{fig:laplace_chord_attenuating}
\end{svg}

Note that this explanation as a basis isn't exact because the Laplace basis
isn't orthogonal (that is, the x and y coordinates affect each other and have
cross-talk). In the frequency domain, we had a basis of sine waves that we
represented as delta functions in the frequency domain. Each frequency
contribution was independent of the others. In the Laplace domain, this is not
the case; a pure exponential is $\frac{1}{s - a}$ (a rational function where $a$
is a real number) instead of a delta function. This function is nonzero at
points that aren't actually frequencies present in the time domain. Figure
\ref{fig:laplace_chord_3d} demonstrates this, which shows the Laplace transform
of the Fmajor4 chord plotted in 3D.
\begin{svg}{build/\sectionpath/laplace_chord_3d}
  \caption{Laplace transform of Fmajor4 chord plotted in 3D}
  \label{fig:laplace_chord_3d}
\end{svg}

Notice how the values of the function around each component frequency decrease
according to $\frac{1}{\sqrt{x^2 + y^2}}$ in the $x$ and $y$ directions (in just
the $x$ direction, it would be $\frac{1}{x}$).

\section{Parts of a transfer function}

A transfer function maps an input coordinate to an output coordinate in the
Laplace domain. These can be obtained by applying the Laplace transform to a
differential equation and rearranging the terms to obtain a ratio of the output
variable to the input variable. Equation (\ref{eq:transfer_func}) is an example
of a transfer function.

\begin{equation} \label{eq:transfer_func}
  H(s) = \frac{\overbrace{(s-9+9i)(s-9-9i)}^{zeroes}}
    {\underbrace{s(s+10)}_{poles}}
\end{equation}

\subsection{Poles and zeroes}

The roots of factors in the numerator of a transfer function are called
\textit{zeroes} because they make the transfer function approach zero. Likewise,
the roots of factors in the denominator of a transfer function are called
\textit{poles} because they make the transfer function approach infinity; on a
3D graph, these look like the poles of a circus tent (see figure
\ref{fig:tf_3d}).

When the factors of the denominator are broken apart using partial fraction
expansion into something like $\frac{A}{s + a} + \frac{B}{s + b}$, the constants
$A$ and $B$ are called residues, which determine how much each pole contributes
to the \gls{system response}.

The factors representing poles are each the Laplace transform of a decaying
exponential\footnote{We are handwaving Laplace transform derivations because
they are complicated and neither relevant nor useful.}. That means the time
domain responses of \glspl{system} comprise decaying exponentials (e.g.,
$y = e^{-t}$).

\begin{svg}{build/code/tf_3d}
  \caption{Equation \ref{eq:transfer_func} plotted in 3D}
  \label{fig:tf_3d}
\end{svg}

\begin{remark}
  Imaginary poles and zeroes always come in complex conjugate pairs (e.g.,
  $-2 + 3i$, $-2 - 3i$).
\end{remark}

\index{Stability!poles and zeroes}
The locations of the closed-loop poles in the complex plane determine the
stability of the \gls{system}. Each pole represents a frequency mode of the
\gls{system}, and their location determines how much of each response is induced
for a given input frequency. Figure \ref{fig:impulse_response_poles} shows the
\glspl{impulse response} in the time domain for transfer functions with various
pole locations. They all have an initial condition of $1$.

\begin{bookfigure}
  \begin{tikzpicture}[auto, >=latex']
    % \draw [help lines] (-4,-2) grid (4,4);

    % Draw main axes
    \draw[->] (-4.2,0) -- (4.2,0) node[below] {\small Re($\sigma$)};
    \draw[->] (0,-2) -- (0,4.2) node[right] {\small Im($j\omega$)};

    % Stable: e^-1.75t * cos(1.75wt) (80/3*w for readability)
    \drawtimeplot{-2.125cm}{2.5cm}{0.125cm}{0.44375cm}{
      exp(-1.75 * \x) * cos(80/3 * 1.75 * deg(\x))}
    \drawpole{-1.75cm}{1.75cm}

    % Stable: e^-2.5t
    \drawtimeplot{-2.25cm}{0.75cm}{0.125cm}{0.125cm}{exp(-2 * \x)}
    \drawpole{-2cm}{0cm}

    % Stable: e^-t
    \drawtimeplot{-1.125cm}{-0.75cm}{0.125cm}{0.125cm}{exp(-\x)}
    \drawpole{-1cm}{0cm}

    % Marginally stable: cos(wt) (80/3*w for readability)
    \drawtimeplot{-0.75cm}{1.125cm}{0.125cm}{0.44375cm}{cos(80/3 * deg(\x))}
    \drawpole{0cm}{1cm}

    % Marginally stable cos(2wt) (80/3*w for readability)
    \drawtimeplot{0cm}{2.75cm}{0.125cm}{0.44375cm}{cos(80/3 * 2 * deg(\x))}
    \drawpole{0cm}{2cm}

    % Integrator
    \drawtimeplot{0.25cm}{-0.75cm}{0.125cm}{0.125cm}{1}
    \drawpole{0cm}{0cm}

    % Unstable: e^t
    \drawtimeplot{1.125cm}{0.75cm}{0.125cm}{0.125cm}{exp(\x)}
    \drawpole{1cm}{0cm}

    % Unstable: e^2t
    \drawtimeplot{2.25cm}{-0.75cm}{0.125cm}{0.125cm}{exp(2 * \x)}
    \drawpole{2cm}{0cm}

    % Unstable: e^0.75t * cos(1.75wt) (80/3*w for readability)
    \drawtimeplot{1.5cm}{2.25cm}{0.125cm}{0.44375cm}{
      exp(0.75 * \x) * cos(80/3 * 1.75 * deg(\x))}
    \drawpole{0.75cm}{1.75cm}

    % LHP and RHP labels
    \draw (-3.5,1.5) node {LHP};
    \draw (3.5,1.5) node {RHP};

    % Stable and unstable labels
    \draw (-2.5,3.5) node {\small Stable};
    \draw (2.5,3.5) node {\small Unstable};
  \end{tikzpicture}

  \caption{Impulse response vs pole location}
  \label{fig:impulse_response_poles}
\end{bookfigure}

\begin{booktable}
  \begin{tabular}{|ll|}
    \hline
    \rowcolor{headingbg}
    \textbf{Location} & \textbf{Stability} \\
    \hline
    Left Half-plane (LHP) & Stable \\
    Imaginary axis & Marginally stable \\
    Right Half-plane (RHP) & Unstable \\
    \hline
  \end{tabular}

  \caption{Pole location and stability}
  \label{tab:pole_locations}
\end{booktable}

When a \gls{system} is stable, its output may oscillate but it converges to
steady-state. When a \gls{system} is marginally stable, its output oscillates at
a constant amplitude forever. When a \gls{system} is unstable, its output grows
without bound.

\subsection{Nonminimum phase zeroes}

While poles in the RHP are unstable, the same is not true for zeroes. They can
be characterized by the \gls{system} initially moving in the wrong direction
before heading toward the \gls{reference}. Since the poles always move toward
the zeroes, zeroes impose a ``speed limit" on the \gls{system response} because
it takes a finite amount of time to move the wrong direction, then change
directions.

One example of this type of \gls{system} is bicycle steering. Try riding a
bicycle without holding the handle bars, then poke the right handle; the bicycle
turns right. Furthermore, if one is holding the handlebars and wants to turn
left, rotating the handlebars counterclockwise will make the bicycle fall toward
the right. The rider has to lean into the turn and overpower the nonminimum
phase dynamics to go the desired direction.

Another example is a Segway. To move forward by some distance, the Segway must
first roll backward to rotate the Segway forward. Once the Segway starts falling
in that direction, it begins rolling forward to avoid falling over until
it reaches the target distance. At that point, the Segway increases its forward
speed to pitch backward and slow itself down. To come to a stop, the Segway
rolls backward again to level itself out.

\subsection{Pole-zero cancellation}
\label{subsec:pole-zero_cancellation}

Pole-zero cancellation occurs when a pole and zero are located at the same place
in the s-plane. This effectively eliminates the contribution of each to the
\gls{system} dynamics. By placing poles and zeroes at various locations (this is
done by placing transfer functions in series), we can eliminate undesired
\gls{system} dynamics. While this may appear to be a useful design tool at
first, there are major caveats. Most of these are due to \gls{model} uncertainty
resulting in poles which aren't in the locations the controls designer expected.

Notch filters are typically used to dampen a specific range of frequencies in
the \gls{system response}. If its band is made too narrow, it can still leave the
undesirable dynamics, but now you can no longer measure them in the response.
They are still happening, but they are what's called \textit{unobservable}.

Never pole-zero cancel unstable or nonminimum phase dynamics. If the \gls{model}
doesn't quite reflect reality, an attempted pole cancellation by placing a
nonminimum phase zero results in the pole still moving to the zero placed next
to it. You have the same dynamics as before, but the pole is also stuck where it
is no matter how much \gls{feedback gain} is applied. For an attempted
nonminimum phase zero cancellation, you have effectively placed an unstable pole
that's unobservable. This means the \gls{system} will be going unstable and
blowing up, but you won't be able to detect this and react to it.

Keep in mind when making design decisions that the \gls{model} likely isn't
perfect. The whole point of feedback control is to be robust to this kind of
uncertainty.

\section{Transfer functions in feedback}

For \glspl{controller} to \glslink{regulator}{regulate} a \gls{system} or
\glslink{tracking}{track} a reference, they must be placed in positive or
negative feedback with the \gls{plant} (whether to use positive or negative
depends on the \gls{plant} in question). Stable feedback loops attempt to make
the \gls{output} equal the \gls{reference}.

\begin{bookfigure}
  \begin{tikzpicture}[auto, >=latex']
    % Place the blocks
    \node [name=input] {$X(s)$};
    \node [sum, right=of input] (sum) {};
    \node [block, right=of sum] (K) {$K$};
    \node [block, right=of K] (G) {$G$};
    \node [right=of G] (output) {$Y(s)$};
    \node [block, below=of $(K)!0.5!(G)$] (H) {$H$};

    % Connect the nodes
    \draw [arrow] (input) -- node[pos=0.85] {$+$} (sum);
    \draw [arrow] (sum) -- node {} (K);
    \draw [arrow] (K) -- node {} (G);
    \draw [arrow] (G) -- node[name=y] {} (output);
    \draw [arrow] (y) |- (H);
    \draw [arrow] (H) -| node[pos=0.97, right] {$-$} (sum);
  \end{tikzpicture}

  \caption{Feedback controller block diagram}
  \label{fig:feedback_controller_block_diagram}

  \begin{figurekey}
    \begin{tabular}{llll}
      $X(s)$ & input & $H$ & measurement transfer function \\
      $K$ & controller gain & $Y(s)$ & output \\
      $G$ & plant transfer function & & \\
    \end{tabular}
  \end{figurekey}
\end{bookfigure}

The transfer function of figure \ref{fig:feedback_controller_block_diagram}, a
\gls{control system} diagram with feedback, from input to output is

\begin{equation}
  G_{cl}(s) = \frac{Y(s)}{X(s)} = \frac{KG}{1 + KGH}
\end{equation}

The numerator is the \gls{open-loop gain} and the denominator is one plus the
gain around the feedback loop, which may include parts of the
\gls{open-loop gain} (see appendix \ref{sec:deriv_tf_feedback} for a
derivation). As another example, the transfer function from the input to the
\gls{error} is

\begin{equation}
  G_{cl}(s) = \frac{E(s)}{X(s)} = \frac{1}{1 + KGH}
\end{equation}

The roots of the denominator of $G_{cl}(s)$ are different from those of the
open-loop transfer function $KG(s)$. These are called the closed-loop poles.

\section{Classical vs modern control}

State-space notation provides a more convenient and compact way to model and
analyze \glspl{system} with multiple \glspl{input} and \glspl{output}. For a
\gls{system} with $p$ \glspl{input} and $q$ \glspl{output}, we would have to
write $q \times p$ transfer functions to represent it. Not only is the resulting
algebra unwieldy, but it only works for linear \glspl{system}. Including nonzero
initial conditions complicates the algebra even more. State-space representation
uses the time domain instead of the Laplace domain, so it can model nonlinear
\glspl{system}\footnote{This book focuses on analysis and control of linear
\glspl{system}. See chapter \ref{ch:nonlinear_control} for more on nonlinear
control.} and trivially supports nonzero initial conditions.

If modern control theory is so great and classical control theory isn't needed
to use it, why learn classical control theory at all? We teach classical control
theory because it provides a framework within which to understand results from
the mathematical machinery of modern control as well as vocabulary with which to
communicate that understanding. For example, faster poles (poles moved to the
left in the s-plane) mean faster decay, and oscillation means there is at least
one pair of complex conjugate poles. Not only can you describe what happened
succinctly, but you know why it happened from a theoretical perspective.


\section{Root locus} \label{sec:root_locus}
\index{Stability!root locus}

In closed-loop, the poles can be moved around by adjusting the controller gain,
but the zeroes stay put. The root locus shows where the poles will go as the
gain for a P controller is increased and tells us for what range of gains the
controller will be stable. As the controller gain is increased, poles can move
toward negative infinity (figure \ref{fig:rlocus_infty}), move toward each other
then split toward asymptotes (figure \ref{fig:rlocus_asymptotes}), or move
toward zeroes (figure \ref{fig:rlocus_zeroes}). The \gls{system} in figure
\ref{fig:rlocus_zeroes} becomes unstable as the gain is increased.

\begin{bookfigure}
  \begin{minisvg}{build/code/rlocus_infty}
    \caption{Root locus showing pole moving toward negative infinity}
    \label{fig:rlocus_infty}
  \end{minisvg}
  \hfill
  \begin{minisvg}{build/code/rlocus_asymptotes}
    \caption{Root locus showing poles moving toward asymptotes}
    \label{fig:rlocus_asymptotes}
  \end{minisvg}
  \hfill
  \begin{minisvg}{build/code/rlocus_zeroes}
    \caption{Root locus of equation (\ref{eq:transfer_func}) showing poles
      moving toward zeroes.}
    \label{fig:rlocus_zeroes}
  \end{minisvg}
\end{bookfigure}

We won't be using root locus plots for any of our control systems analysis
later, but it does help provide an intuition for what \glspl{controller}
actually do to a \gls{system}.

If poles are much farther left in the LHP than the typical \gls{system} dynamics
exhibit, they can be considered negligible. Every \gls{system} has some form of
unmodeled high frequency, nonlinear dynamics, but they can be safely ignored
depending on the operating regime.

To demonstrate this, consider the transfer function for a second-order DC
brushed motor from voltage to position

\begin{equation*}
  G(s) = \frac{K}{s((Js + b)(Ls + R) + K^2)}
\end{equation*}

where $J = 3.2284 \times 10^{-6}$ $kg$-$m^2$, $b = 3.5077 \times 10^{-6}$
$N$-$m$-$s$, $K_e = K_t = 0.0274 \,V/rad/s$, $R = 4 \,\Omega$, and
$L = 2.75 \times 10^{-6} \,H$.

This \gls{plant} has the root locus shown in figure
\ref{fig:highfreq_unstable_rlocus}. In proportional feedback, the \gls{plant} is
unstable for large values of $K$. However, if we remove the unstable pole by
setting $L$ in the transfer function to zero, we get the root locus in figure
\ref{fig:highfreq_stable_rlocus}. For small values of $K$, both \glspl{system}
are stable and have nearly indistinguishable \glspl{step response} due to the
exceedingly small contribution from the fast pole (see figures
\ref{fig:highfreq_unstable_step} and \ref{fig:highfreq_stable_step}). The high
frequency dynamics only cause instability for large values of $K$ that induce
fast \glspl{system response}. In other words, the \glspl{system response} of the
second-order model and its first-order approximation are similar for low
frequency operating regimes.

\begin{bookfigure}
  \begin{minisvg}{build/code/highfreq_unstable_rlocus}
    \caption{Root locus of second-order DC brushed motor plant}
    \label{fig:highfreq_unstable_rlocus}
  \end{minisvg}
  \hfill
  \begin{minisvg}{build/code/highfreq_stable_rlocus}
    \caption{Root locus of first-order DC brushed motor plant}
    \label{fig:highfreq_stable_rlocus}
  \end{minisvg}
\end{bookfigure}

\begin{bookfigure}
  \begin{minisvg}{build/code/highfreq_unstable_step}
    \caption{Step response of second-order DC brushed motor plant}
    \label{fig:highfreq_unstable_step}
  \end{minisvg}
  \hfill
  \begin{minisvg}{build/code/highfreq_stable_step}
    \caption{Step response of first-order DC brushed motor plant}
    \label{fig:highfreq_stable_step}
  \end{minisvg}
\end{bookfigure}

Why can't unstable poles close to the origin be ignored in the same way? The
response of high frequency stable poles decays rapidly. Unstable poles, on the
other hand, represent unstable dynamics which cause the \gls{system}
\gls{output} to grow to infinity. Regardless of how slow these unstable dynamics
are, they will eventually dominate the response.

\section{Case studies of Laplace domain analysis}

These case studies cover various aspects of PID control using the algebraic
approach of transfer functions. For this, we'll be using equation
(\ref{eq:pid_tf}), the transfer function for a PID controller.

\begin{equation}
  K(s) = K_p + \frac{K_i}{s} + K_ds \label{eq:pid_tf}
\end{equation}

Remember, multiplication by $\frac{1}{s}$ corresponds to an integral in the
Laplace domain and multiplication by $s$ corresponds to a derivative.

\subsection{Steady-state error}
\index{Steady-state error}

To demonstrate the problem of \gls{steady-state error}, we will use a DC brushed
motor controlled by a velocity PID controller. A DC brushed motor has a transfer
function from voltage ($V$) to angular velocity ($\dot{\theta}$) of

\begin{equation}
  G(s) = \frac{\dot{\Theta}(s)}{V(s)} = \frac{K}{(Js+b)(Ls+R)+K^2}
\end{equation}

First, we'll try controlling it with a P controller defined as

\begin{equation*}
  K(s) = K_p
\end{equation*}

When these are in unity feedback, the transfer function from the input voltage
to the error is

\begin{align*}
  \frac{E(s)}{V(s)} &= \frac{1}{1 + K(s)G(s)} \\
  E(s) &= \frac{1}{1 + K(s)G(s)} V(s) \\
  E(s) &= \frac{1}{1 + (K_p) \left(\frac{K}{(Js+b)(Ls+R)+K^2}\right)} V(s) \\
  E(s) &= \frac{1}{1 + \frac{K_p K}{(Js+b)(Ls+R)+K^2}} V(s)
\end{align*}

The steady-state of a transfer function can be found via

\begin{equation}
  \lim_{s\to0} sH(s)
\end{equation}

since steady-state has an input frequency of zero.

\begin{align}
  e_{ss} &= \lim_{s\to0} sE(s) \nonumber \\
  e_{ss} &= \lim_{s\to0} s \frac{1}{1 + \frac{K_p K}{(Js+b)(Ls+R)+K^2}} V(s)
    \nonumber \\
  e_{ss} &= \lim_{s\to0} s \frac{1}{1 + \frac{K_p K}{(Js+b)(Ls+R)+K^2}}
    \frac{1}{s} \nonumber \\
  e_{ss} &= \lim_{s\to0} \frac{1}{1 + \frac{K_p K}{(Js+b)(Ls+R)+K^2}}
    \nonumber \\
  e_{ss} &= \frac{1}{1 + \frac{K_p K}{(J(0)+b)(L(0)+R)+K^2}} \nonumber \\
  e_{ss} &= \frac{1}{1 + \frac{K_p K}{bR+K^2}} \label{eq:ss_nonzero}
\end{align}

Notice that the \gls{steady-state error} is nonzero. To fix this, an integrator
must be included in the controller.

\begin{equation*}
  K(s) = K_p + \frac{K_i}{s}
\end{equation*}

The same steady-state calculations are performed as before with the new
controller.

\begin{align*}
  \frac{E(s)}{V(s)} &= \frac{1}{1 + K(s)G(s)} \\
  E(s) &= \frac{1}{1 + K(s)G(s)} V(s) \\
  E(s) &= \frac{1}{1 + \left(K_p + \frac{K_i}{s}\right)
    \left(\frac{K}{(Js+b)(Ls+R)+K^2}\right)} \left(\frac{1}{s}\right) \\
  e_{ss} &= \lim_{s\to0} s \frac{1}{1 + \left(K_p + \frac{K_i}{s}\right)
    \left(\frac{K}{(Js+b)(Ls+R)+K^2}\right)} \left(\frac{1}{s}\right) \\
  e_{ss} &= \lim_{s\to0} \frac{1}{1 + \left(K_p + \frac{K_i}{s}\right)
    \left(\frac{K}{(Js+b)(Ls+R)+K^2}\right)} \\
  e_{ss} &= \lim_{s\to0} \frac{1}{1 + \left(K_p + \frac{K_i}{s}\right)
    \left(\frac{K}{(Js+b)(Ls+R)+K^2}\right)} \frac{s}{s} \\
  e_{ss} &= \lim_{s\to0} \frac{s}{s + \left(K_p s + K_i\right)
    \left(\frac{K}{(Js+b)(Ls+R)+K^2}\right)} \\
  e_{ss} &= \frac{0}{0 + (K_p (0) + K_i)
    \left(\frac{K}{(J(0)+b)(L(0)+R)+K^2}\right)} \\
  e_{ss} &= \frac{0}{K_i \frac{K}{bR+K^2}}
\end{align*}

The denominator is nonzero, so $e_{ss} = 0$. Therefore, an integrator is
required to eliminate \gls{steady-state error} in all cases for this
\gls{model}.

It should be noted that $e_{ss}$ in equation (\ref{eq:ss_nonzero}) approaches
zero for $K_p = \infty$. This is known as a bang-bang controller. In practice,
an infinite switching frequency cannot be achieved, but it may be close enough
for some performance specifications.

\subsection{Flywheel PID control}
\label{subsec:flywheel_pid_control}
\index{PID control}

PID controllers typically control voltage to a motor in FRC independent of the
equations of motion of that motor. For position PID control, large values of
$K_p$ can lead to overshoot and $K_d$ is commonly used to reduce overshoots.
Let's consider a flywheel controlled with a standard PID controller. Why
wouldn't $K_d$ provide damping for velocity overshoots in this case?

PID control is designed to control second-order and first-order \glspl{system}
well. It can be used to control a lot of things, but struggles when given higher
order \glspl{system}. It has three degrees of freedom. Two are used to place the
two poles of the \gls{system}, and the third is used to remove steady-state
error. With higher order \glspl{system} like a one input, seven \gls{state}
\gls{system}, there aren't enough degrees of freedom to place the \gls{system}'s
poles in desired locations. This will result in poor control.

The math for PID doesn't assume voltage, a motor, etc. It defines an output
based on derivatives and integrals of its input. We happen to use it for motors
because it actually works pretty well for it because motors are second-order
\glspl{system}.

The following math will be in continuous time, but the same ideas apply to
discrete time. This is all assuming a velocity controller.

Our simple motor model hooked up to a mass is

\begin{align}
  V &= IR + \frac{\omega}{K_v} \label{eq:cs_flywheel_1} \\
  \tau &= I K_t \label{eq:cs_flywheel_2} \\
  \tau &= J \frac{d\omega}{dt} \label{eq:cs_flywheel_3}
\end{align}

For an explanation of where these equations come from, read section
\ref{sec:dc_brushed_motor}.

First, we'll solve for $\frac{d\omega}{dt}$ in terms of $V$.

Substitute equation (\ref{eq:cs_flywheel_2}) into equation
(\ref{eq:cs_flywheel_1}).

\begin{align*}
  V &= IR + \frac{\omega}{K_v} \\
  V &= \left(\frac{\tau}{K_t}\right) R + \frac{\omega}{K_v}
\end{align*}

Substitute in equation (\ref{eq:cs_flywheel_3}).

\begin{align*}
  V &= \frac{\left(J \frac{d\omega}{dt}\right)}{K_t} R + \frac{\omega}{K_v} \\
\end{align*}

Solve for $\frac{d\omega}{dt}$.

\begin{align*}
  V &= \frac{J \frac{d\omega}{dt}}{K_t} R + \frac{\omega}{K_v} \\
  V - \frac{\omega}{K_v} &= \frac{J \frac{d\omega}{dt}}{K_t} R \\
  \frac{d\omega}{dt} &= \frac{K_t}{JR} \left(V - \frac{\omega}{K_v}\right) \\
  \frac{d\omega}{dt} &= -\frac{K_t}{JRK_v} \omega + \frac{K_t}{JR} V
\end{align*}

Now take the Laplace transform. Because the Laplace transform is a linear
operator, we can take the Laplace transform of each term individually. Based on
table \ref{tab:common_laplace_transforms}, $\frac{d\omega}{dt}$ becomes
$s\omega$ and $\omega(t)$ and $V(t)$ become $\omega(s)$ and $V(s)$ respectively
(the parenthetical notation has been dropped for clarity).

\begin{equation}
  s \omega = -\frac{K_t}{JRK_v} \omega + \frac{K_t}{JR} V
  \label{eq:cs_motor_tf}
\end{equation}

Solve for the transfer function $H(s) = \frac{\omega}{V}$.

\begin{align*}
  s \omega &= -\frac{K_t}{JRK_v} \omega + \frac{K_t}{JR} V \\
  \left(s + \frac{K_t}{JRK_v}\right) \omega &= \frac{K_t}{JR} V \\
  \frac{\omega}{V} &= \frac{\frac{K_t}{JR}}{s + \frac{K_t}{JRK_v}} \\
\end{align*}

That gives us a pole at $-\frac{K_t}{JRK_v}$, which is actually stable. Notice
that there is only one pole.

First, we'll use a simple P loop.

\begin{equation*}
  V = K_p (\omega_{goal} - \omega)
\end{equation*}

Substitute this controller into equation (\ref{eq:cs_motor_tf}).

\begin{equation*}
  s \omega = -\frac{K_t}{JRK_v} \omega + \frac{K_t}{JR} K_p (\omega_{goal} -
    \omega)
\end{equation*}

Solve for the transfer function $H(s) = \frac{\omega}{\omega_{goal}}$.

\begin{align*}
  s \omega &= -\frac{K_t}{JRK_v} \omega + \frac{K_t K_p}{JR} (\omega_{goal} -
    \omega) \\
  s \omega &= -\frac{K_t}{JRK_v} \omega + \frac{K_t K_p}{JR} \omega_{goal} -
    \frac{K_t K_p}{JR} \omega \\
  \left(s + \frac{K_t}{JRK_v} + \frac{K_t K_p}{JR}\right) \omega &=
    \frac{K_t K_p}{JR} \omega_{goal} \\
  \frac{\omega}{\omega_{goal}} &= \frac{\frac{K_t K_p}{JR}}
    {\left(s + \frac{K_t}{JRK_v} + \frac{K_t K_p}{JR}\right)} \\
\end{align*}

This has a pole at $-\left(\frac{K_t}{JRK_v} + \frac{K_t K_p}{JR}\right)$.
Assuming that that quantity is negative (i.e., we are stable), that pole
corresponds to a time constant of
$\frac{1}{\frac{K_t}{JRK_v} + \frac{K_t K_p}{JR}}$.

As can be seen above, a flywheel has a single pole. It therefore only needs a
single pole controller to place all of its poles anywhere.

\begin{remark}
  This analysis assumes that the motor is well coupled to the mass and that the
  time constant of the inductor is small enough that it doesn't factor into the
  motor equations. In Austin Schuh's experience with 971's robots, these are
  pretty good assumptions.
\end{remark}

Next, we'll try a PD loop. (This will use a perfect derivative, but anyone
following along closely already knows that we can't really take a derivative
here, so the math will need to be updated at some point. We could switch to
discrete time and pick a differentiation method, or pick some other way of
modeling the derivative.)

\begin{equation*}
  V = K_p (\omega_{goal} - \omega) + K_d s (\omega_{goal} - \omega)
\end{equation*}

Substitute this controller into equation (\ref{eq:cs_motor_tf}).

\begin{align*}
  s \omega &= -\frac{K_t}{JRK_v} \omega + \frac{K_t}{JR}
    \left(K_p (\omega_{goal} - \omega) + K_d s (\omega_{goal} - \omega)\right)
    \\
  s \omega &= -\frac{K_t}{JRK_v} \omega + \frac{K_t K_p}{JR}
    (\omega_{goal} - \omega) + \frac{K_t K_d s}{JR} (\omega_{goal} - \omega) \\
  s \omega &= -\frac{K_t}{JRK_v} \omega + \frac{K_t K_p}{JR} \omega_{goal} -
    \frac{K_t K_p}{JR} \omega + \frac{K_t K_d s}{JR} \omega_{goal} -
    \frac{K_t K_d s}{JR} \omega \\
\end{align*}

Collect the common terms on separate sides and refactor.

\begin{align*}
  s \omega + \frac{K_t K_d s}{JR} \omega + \frac{K_t}{JRK_v} \omega +
    \frac{K_t K_p}{JR} \omega &= \frac{K_t K_p}{JR} \omega_{goal} +
    \frac{K_t K_d s}{JR} \omega_{goal} \\
  \left(s \left(1 + \frac{K_t K_d}{JR}\right) + \frac{K_t}{JRK_v} +
    \frac{K_t K_p}{JR}\right) \omega &= \frac{K_t}{JR}
    \left(K_p + K_d s\right) \omega_{goal} \\
  \frac{\omega}{\omega_{goal}} &= \frac{\frac{K_t}{JR}
    \left(K_p + K_d s\right)}{\left(s \left(1 + \frac{K_t K_d}{JR}\right) +
    \frac{K_t}{JRK_v} + \frac{K_t K_p}{JR}\right)} \\
\end{align*}

So, we added a zero at $-\frac{K_p}{K_d}$ and moved our pole to
$-\frac{\frac{K_t}{JRK_v} + \frac{K_t K_p}{JR}}{1 + \frac{K_t K_d}{JR}}$. This
isn't progress. We've added more complexity to our \gls{system} and, practically
speaking, gotten nothing good in return. Zeroes should be avoided if at all
possible because they amplify unwanted high frequency modes of the \gls{system}
and are noisier the faster the \gls{system} is sampled. At least this is a stable
zero, but it's still undesirable.

In summary, derivative doesn't help on an ideal flywheel. $K_d$ may compensate
for unmodeled dynamics such as accelerating projectiles slowing the flywheel
down, but that effect may also increase recovery time; $K_d$ drives the
acceleration to zero in the undesired case of negative acceleration as well as
well as the actually desired case of positive acceleration.

We'll cover a superior compensation method much later in subsection
\ref{subsec:u_error_estimation} that avoids zeroes in the \gls{controller},
doesn't act against the desired control action, and facilitates better
\gls{tracking}.

\section{Gain margin and phase margin}
\label{sec:gain_phase_margin}
\index{stability!gain margin}
\index{stability!phase margin}

One generally needs to learn about Bode plots and Nyquist plots to truly
understand gain and phase margin and their origins, but those plots are large
topics unto themselves. Since we won't be using either of them for controller
design, we'll just cover what gain and phase margin are in a general sense and
how they are used.

Gain margin and phase margin are two metrics for measuring a \gls{system}'s
relative stability. Gain and phase margin are the amounts by which the
closed-loop gain and phase can be varied respectively before the \gls{system}
becomes unstable. In a sense, they are safety margins for when unmodeled
dynamics affect the \gls{system response}.

For a more thorough explanation of gain and phase margin, watch Brian Douglas's
video on them \cite{bib:gain_phase_margin}. He has other videos too on classical
control methods like Bode and Nyquist plots that we recommend.


\section{s-plane to z-plane}
\label{sec:s-plane_to_z-plane}

Transfer functions are converted to impulse responses using the Z-transform. The
s-plane's LHP maps to the inside of a unit circle in the z-plane. Table
\ref{tab:s2z_mapping} contains a few common points and figure
\ref{fig:s2z_mapping} shows the mapping visually.

\begin{booktable}
  \begin{tabular}{|cc|}
    \hline
    \rowcolor{headingbg}
    \textbf{s-plane} & \textbf{z-plane} \\
    \hline
    $(0, 0)$ & $(1, 0)$ \\
    imaginary axis & edge of unit circle \\
    $(-\infty, 0)$ & $(0, 0)$ \\
    \hline
  \end{tabular}
  \caption{Mapping from s-plane to z-plane}
  \label{tab:s2z_mapping}
\end{booktable}

\begin{bookfigure}
  \begin{minisvg}{2}{build/\chapterpath/s_plane}
  \end{minisvg}
  \hfill
  \begin{minisvg}{2}{build/\chapterpath/z_plane}
  \end{minisvg}
  \caption{Mapping of axes from s-plane (left) to z-plane (right)}
  \label{fig:s2z_mapping}
\end{bookfigure}

\subsection{z-plane stability}

Eigenvalues of a \gls{system} that are within the unit circle are stable, but
why is that? Let's consider a scalar equation $x_{k + 1} = ax_k$. $a < 1$ makes
$x_{k + 1}$ converge to zero. The same applies to a complex number like
$z = x + yi$ for $x_{k + 1} = zx_k$. If the magnitude of the complex number $z$
is less than one, $x_{k+1}$ will converge to zero. Values with a magnitude of
$1$ oscillate forever because $x_{k+1}$ never decays.

\subsection{z-plane behavior}

As $\omega$ increases in $s = j\omega$, a pole in the z-plane moves around the
perimeter of the unit circle. Once it hits $\frac{\omega_s}{2}$ (half the
sampling frequency) at $(-1, 0)$, the pole wraps around. This is due to poles
faster than the sample frequency folding down to below the sample frequency
(that is, higher frequency signals \textit{alias} to lower frequency ones).

You may notice that poles can be placed at $(0, 0)$ in the z-plane. This is
known as a deadbeat controller. An $\rm N^{th}$-order deadbeat controller decays
to the \gls{reference} in N timesteps. While this sounds great, there are other
considerations like \gls{control effort}, \gls{robustness}, and
\gls{noise immunity}. These will be discussed in more detail with LQR and LQE.

If poles from $(1, 0)$ to $(0, 0)$ on the x-axis approach infinity, then what do
poles from $(-1, 0)$ to $(0, 0)$ represent? Them being faster than infinity
doesn't make sense. Poles in this location exhibit oscillatory behavior similar
to complex conjugate pairs. See figures \ref{fig:z_oscillations_1p} and
\ref{fig:z_oscillations_2p}. The jaggedness of these signals is due to the
frequency of the \gls{system} dynamics being above the Nyquist frequency. The
\glslink{discretization}{discretized} signal doesn't have enough samples to
reconstruct the continuous \gls{system}'s dynamics.

\begin{bookfigure}
  \begin{minisvg}{2}{build/\chapterpath/z_oscillations_1p}
    \caption{Single poles in various locations in z-plane}
    \label{fig:z_oscillations_1p}
  \end{minisvg}
  \hfill
  \begin{minisvg}{2}{build/\chapterpath/z_oscillations_2p}
    \caption{Complex conjugate poles in various locations in z-plane}
    \label{fig:z_oscillations_2p}
  \end{minisvg}
\end{bookfigure}

\subsection{Nyquist frequency}
\index{digital signal processing!Nyquist frequency}
\index{digital signal processing!aliasing}

To completely reconstruct a signal, the Nyquist-Shannon sampling theorem states
that it must be sampled at a frequency at least twice the maximum frequency it
contains. The highest frequency a given sample rate can capture is called the
Nyquist frequency, which is half the sample frequency. This is why recorded
audio is sampled at $44.1k\,Hz$. The maximum frequency a typical human can hear
is about $20\,kHz$, so the Nyquist frequency is $40\,kHz$. ($44.1kHz$ in
particular was chosen for unrelated historical reasons.)

Frequencies above the Nyquist frequency are folded down across it. The higher
frequency and the folded down lower frequency are said to alias each
other\footnote{The aliases of a frequency $f$ can be expressed as
$f_{alias}(N) \stackrel{def}{=} |f - Nf_s|$. For example, if a $200\,Hz$ sine
wave is sampled at $150\,Hz$, the \gls{observer} will see a $50\,Hz$ signal
instead of a $200\,Hz$ one.}. Figure \ref{fig:nyquist} provides a demonstration
of aliasing.

\begin{svg}{build/\chapterpath/nyquist}
  \caption{The samples of two sine waves can be identical when at least one of
    them is at a frequency above half the sample rate. In this case, the $2\,Hz$
    sine wave is above the Nyquist frequency $1.5\,Hz$.}
  \label{fig:nyquist}
\end{svg}

The effect of these high-frequency aliases can be reduced with a low-pass filter
(called an anti-aliasing filter in this application).

\section{Phase loss}

Implementing a discrete control system is easier than implementing a continuous
one, but \gls{discretization} has drawbacks. A microcontroller updates the
system input in discrete intervals of duration $T$; it's held constant between
updates. This introduces an average sample delay of $\frac{T}{2}$ that leads to
phase loss in the controller. Phase loss is the reduction of \gls{phase margin}
that occurs in digital implementations of feedback controllers from sampling the
continuous \gls{system} at discrete time intervals. As the sample rate of the
controller decreases, the \gls{phase margin} decreases according to
$-\frac{T}{2}\omega$ where $T$ is the sample period and $\omega$ is the
frequency of the \gls{system} dynamics. Instability occurs if the \gls{phase
margin} of the \gls{system} reaches zero. Large amounts of phase loss can make a
stable controller in the continuous domain become unstable in the discrete
domain. Here are a few ways to combat this.
\begin{itemize}
  \item Run the controller with a high sample rate.
  \item Designing the controller in the analog domain with enough
    \gls{phase margin} to compensate for any phase loss that occurs as part of
    \gls{discretization}.
  \item Convert the \gls{plant} to the digital domain and design the controller
    completely in the digital domain.
\end{itemize}

