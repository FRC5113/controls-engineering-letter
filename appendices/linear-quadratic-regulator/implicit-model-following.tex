\section{Implicit model following}
\label{sec:implicit_model_following}
\index{controller design!linear-quadratic regulator!implicit model following}

If we want to design a feedback controller that erases the dynamics of our
system and makes it behave like some other system, we can use \textit{implicit
model following}. This is used on the Blackhawk helicopter at NASA Ames research
center when they want to make it fly like experimental aircraft (within the
limits of the helicopter's actuators, of course).

\subsection{Following reference system matrix}

Let the original system dynamics be
\begin{align*}
  \mat{x}_{k+1} &= \mat{A}\mat{x}_k + \mat{B}\mat{u}_k \\
  \mat{y}_k &= \mat{C}\mat{x}_k
\end{align*}

and the desired system dynamics be $\mat{z}_{k+1} = \mat{A}_{ref}\mat{z}_k$.
\begin{align*}
  \mat{y}_{k+1} &= \mat{C}\mat{x}_{k+1} \\
  \mat{y}_{k+1} &= \mat{C}(\mat{A}\mat{x}_k + \mat{B}\mat{u}_k) \\
  \mat{y}_{k+1} &= \mat{C}\mat{A}\mat{x}_k + \mat{C}\mat{B}\mat{u}_k
\end{align*}

We want to minimize the following cost functional.
\begin{equation*}
  J = \sum_{k=0}^\infty \left((\mat{y}_{k+1} - \mat{z}_{k+1})\T \mat{Q}
    (\mat{y}_{k+1} - \mat{z}_{k+1}) + \mat{u}_k\T\mat{R}\mat{u}_k\right)
\end{equation*}

We'll be measuring the desired system's state, so let $\mat{y} = \mat{z}$.
\begin{align*}
  \mat{z}_{k+1} &= \mat{A}_{ref}\mat{y}_k \\
  \mat{z}_{k+1} &= \mat{A}_{ref}\mat{C}\mat{x}_k
\end{align*}

Therefore,
\begin{align*}
  \mat{y}_{k+1} - \mat{z}_{k+1} &=
    \mat{C}\mat{A}\mat{x}_k + \mat{C}\mat{B}\mat{u}_k -
    (\mat{A}_{ref}\mat{C}\mat{x}_k) \\
  \mat{y}_{k+1} - \mat{z}_{k+1} &=
    (\mat{C}\mat{A} - \mat{A}_{ref}\mat{C})\mat{x}_k + \mat{C}\mat{B}\mat{u}_k
\end{align*}

Substitute this into the cost functional.
\begin{align*}
  J &= \sum_{k=0}^\infty \left((\mat{y}_{k+1} - \mat{z}_{k+1})\T \mat{Q}
    (\mat{y}_{k+1} - \mat{z}_{k+1}) + \mat{u}_k\T\mat{R}\mat{u}_k\right) \\
  J &= \sum_{k=0}^\infty \left(
    ((\mat{C}\mat{A} - \mat{A}_{ref}\mat{C})\mat{x}_k + \mat{C}\mat{B}\mat{u}_k)\T
    \mat{Q}
    ((\mat{C}\mat{A} - \mat{A}_{ref}\mat{C})\mat{x}_k + \mat{C}\mat{B}\mat{u}_k) +
    \mat{u}_k\T\mat{R}\mat{u}_k\right)
\end{align*}

Apply the transpose to the left-hand side of the $\mat{Q}$ term.
\begin{equation*}
  J = \sum_{k=0}^\infty \left(
    (\mat{x}_k\T(\mat{C}\mat{A} - \mat{A}_{ref}\mat{C})\T + \mat{u}_k\T(\mat{C}\mat{B})\T)
    \mat{Q}
    ((\mat{C}\mat{A} - \mat{A}_{ref}\mat{C})\mat{x}_k + \mat{C}\mat{B}\mat{u}_k) +
    \mat{u}_k\T\mat{R}\mat{u}_k\right)
\end{equation*}

Factor out $\begin{bmatrix}\mat{x}_k \\ \mat{u}_k\end{bmatrix}\T$ from the left
side and $\begin{bmatrix}\mat{x}_k \\ \mat{u}_k\end{bmatrix}$ from the right
side of each term.
\begin{align*}
  J &= \sum_{k=0}^\infty \left(
    \begin{bmatrix}
      \mat{x}_k \\
      \mat{u}_k
    \end{bmatrix}\T
    \begin{bmatrix}
      (\mat{C}\mat{A} - \mat{A}_{ref}\mat{C})\T \\
      (\mat{C}\mat{B})\T
    \end{bmatrix}
    \mat{Q}
    \begin{bmatrix}
      \mat{C}\mat{A} - \mat{A}_{ref}\mat{C} &
      \mat{C}\mat{B}
    \end{bmatrix}
    \begin{bmatrix}
      \mat{x}_k \\
      \mat{u}_k
    \end{bmatrix} \right. \\
  &\qquad \left. +
    \begin{bmatrix}
      \mat{x}_k \\
      \mat{u}_k
    \end{bmatrix}\T
    \begin{bmatrix}
      \mat{0} & \mat{0} \\
      \mat{0} & \mat{R}
    \end{bmatrix}
    \begin{bmatrix}
      \mat{x}_k \\
      \mat{u}_k
    \end{bmatrix}
    \right) \\
  J &= \sum_{k=0}^\infty \left(
    \begin{bmatrix}
      \mat{x}_k \\
      \mat{u}_k
    \end{bmatrix}\T
    \left(
    \begin{bmatrix}
      (\mat{C}\mat{A} - \mat{A}_{ref}\mat{C})\T \\
      (\mat{C}\mat{B})\T
    \end{bmatrix}
    \mat{Q}
    \begin{bmatrix}
      \mat{C}\mat{A} - \mat{A}_{ref}\mat{C} &
      \mat{C}\mat{B}
    \end{bmatrix} +
    \begin{bmatrix}
      \mat{0} & \mat{0} \\
      \mat{0} & \mat{R}
    \end{bmatrix}
    \right)
    \begin{bmatrix}
      \mat{x}_k \\
      \mat{u}_k
    \end{bmatrix}
    \right)
\end{align*}

Multiply in $\mat{Q}$.
\begin{equation*}
  J = \sum_{k=0}^\infty \left(
    \begin{bmatrix}
      \mat{x}_k \\
      \mat{u}_k
    \end{bmatrix}\T
    \left(
    \begin{bmatrix}
      (\mat{C}\mat{A} - \mat{A}_{ref}\mat{C})\T\mat{Q} \\
      (\mat{C}\mat{B})\T\mat{Q}
    \end{bmatrix}
    \begin{bmatrix}
      \mat{C}\mat{A} - \mat{A}_{ref}\mat{C} &
      \mat{C}\mat{B}
    \end{bmatrix} +
    \begin{bmatrix}
      \mat{0} & \mat{0} \\
      \mat{0} & \mat{R}
    \end{bmatrix}
    \right)
    \begin{bmatrix}
      \mat{x}_k \\
      \mat{u}_k
    \end{bmatrix}
    \right)
\end{equation*}

Multiply matrices in the left term together.
\begin{align*}
  J &= \sum_{k=0}^\infty \left(
    \begin{bmatrix}
      \mat{x}_k \\
      \mat{u}_k
    \end{bmatrix}\T
    \left(
    \begin{bmatrix}
      (\mat{C}\mat{A} - \mat{A}_{ref}\mat{C})\T\mat{Q}(\mat{C}\mat{A} - \mat{A}_{ref}\mat{C}) &
      (\mat{C}\mat{A} - \mat{A}_{ref}\mat{C})\T\mat{Q}(\mat{C}\mat{B}) \\
      (\mat{C}\mat{B})\T\mat{Q}(\mat{C}\mat{A} - \mat{A}_{ref}\mat{C}) &
      (\mat{C}\mat{B})\T\mat{Q}(\mat{C}\mat{B})
    \end{bmatrix} \right.\right. \\
  &\qquad \left.\left. +
    \begin{bmatrix}
      \mat{0} & \mat{0} \\
      \mat{0} & \mat{R}
    \end{bmatrix}
    \right)
    \begin{bmatrix}
      \mat{x}_k \\
      \mat{u}_k
    \end{bmatrix}
    \right)
\end{align*}

Add the terms together.
\begin{equation}
  J = \sum_{k=0}^\infty
  \begin{bmatrix}
    \mat{x}_k \\
    \mat{u}_k
  \end{bmatrix}\T
  \begin{bsmallmatrix}
    \underbrace{(\mat{C}\mat{A} - \mat{A}_{ref}\mat{C})\T\mat{Q}
      (\mat{C}\mat{A} - \mat{A}_{ref}\mat{C})}_{\mat{Q}} &
    \underbrace{(\mat{C}\mat{A} - \mat{A}_{ref}\mat{C})\T\mat{Q}
      (\mat{C}\mat{B})}_{\mat{N}} \\
    \underbrace{(\mat{C}\mat{B})\T\mat{Q}
      (\mat{C}\mat{A} - \mat{A}_{ref}\mat{C})}_{\mat{N}\T} &
    \underbrace{(\mat{C}\mat{B})\T\mat{Q}(\mat{C}\mat{B}) + \mat{R}}_{\mat{R}}
  \end{bsmallmatrix}
  \begin{bmatrix}
    \mat{x}_k \\
    \mat{u}_k
  \end{bmatrix}
\end{equation}

Thus, implicit model following can be defined as the following optimization
problem.
\begin{theorem}[Implicit model following]
  \begin{align}
    \mat{u}_k^* = \argmin_{\mat{u}_k} &\sum_{k=0}^\infty
    \begin{bmatrix}
      \mat{x}_k \\
      \mat{u}_k
    \end{bmatrix}\T
    \begin{bmatrix}
      \mat{Q}_{imf} & \mat{N}_{imf} \\
      \mat{N}_{imf}\T & \mat{R}_{imf}
    \end{bmatrix}
    \begin{bmatrix}
      \mat{x}_k \\
      \mat{u}_k
    \end{bmatrix}
    \nonumber \\
    \text{subject to } &\mat{x}_{k+1} = \mat{A}\mat{x}_k + \mat{B}\mat{u}_k
  \end{align}
  where
  \begin{align*}
    \mat{Q}_{imf} &= (\mat{C}\mat{A} - \mat{A}_{ref}\mat{C})\T\mat{Q}
      (\mat{C}\mat{A} - \mat{A}_{ref}\mat{C}) \\
    \mat{N}_{imf} &= (\mat{C}\mat{A} - \mat{A}_{ref}\mat{C})\T\mat{Q}
      (\mat{C}\mat{B}) \\
    \mat{R}_{imf} &= (\mat{C}\mat{B})\T\mat{Q}(\mat{C}\mat{B}) + \mat{R}
  \end{align*}

  The optimal control policy $\mat{u}_k^*$ is $-\mat{K}\mat{x}_k$. $\mat{K}$ can
  be computed via the typical LQR equations based on the algebraic Ricatti
  equation.
\end{theorem}

The control law $\mat{u}_{imf,k} = -\mat{K}\mat{x}_k$ makes
$\mat{x}_{k+1} = \mat{A}\mat{x}_k + \mat{B}\mat{u}_{imf,k}$ match the open-loop
response of $\mat{z}_{k+1} = \mat{A}_{ref}\mat{z}_k$.

If the original and desired system have the same states, then
$\mat{C} = \mat{I}$ and the cost functional simplifies to
\begin{equation}
  J = \sum_{k=0}^\infty
  \begin{bmatrix}
    \mat{x}_k \\
    \mat{u}_k
  \end{bmatrix}\T
  \begin{bmatrix}
    \underbrace{(\mat{A} - \mat{A}_{ref})\T\mat{Q}
      (\mat{A} - \mat{A}_{ref})}_{\mat{Q}} &
    \underbrace{(\mat{A} - \mat{A}_{ref})\T\mat{Q}\mat{B}}_{\mat{N}} \\
    \underbrace{\mat{B}\T\mat{Q}(\mat{A} - \mat{A}_{ref})}_{\mat{N}\T} &
    \underbrace{\mat{B}\T\mat{Q}\mat{B} + \mat{R}}_{\mat{R}}
  \end{bmatrix}
  \begin{bmatrix}
    \mat{x}_k \\
    \mat{u}_k
  \end{bmatrix}
\end{equation}

\subsection{Following reference input matrix}

The feedback control law above makes the open-loop system behave like
$\mat{A}_{ref}$, but the input dynamics are still that of the original system.
Here's how to make the input dynamics behave like $\mat{B}_{ref}$. We want to
find the $\mat{u}_{imf,k}$ that makes the real model follow the reference model.
\begin{align*}
  \mat{x}_{k+1} &= \mat{A}\mat{x}_k + \mat{B}\mat{u}_{imf,k} \\
  \mat{z}_{k+1} &= \mat{A}_{ref}\mat{z}_k + \mat{B}_{ref}\mat{u}_k
\end{align*}

Let $\mat{x} = \mat{z}$.
\begin{align*}
  \mat{x}_{k+1} &= \mat{z}_{k+1} \\
  \mat{A}\mat{x}_k + \mat{B}\mat{u}_{imf,k} &= \mat{A}_{ref}\mat{x}_k +
    \mat{B}_{ref}\mat{u}_k \\
  \mat{B}\mat{u}_{imf,k} &= \mat{A}_{ref}\mat{x}_k - \mat{A}\mat{x}_k +
    \mat{B}_{ref}\mat{u}_k \\
  \mat{B}\mat{u}_{imf,k} &= (\mat{A}_{ref} - \mat{A})\mat{x}_k +
    \mat{B}_{ref}\mat{u}_k \\
  \mat{u}_{imf,k} &= \mat{B}^+ ((\mat{A}_{ref} - \mat{A})\mat{x}_k +
    \mat{B}_{ref}\mat{u}_k) \\
  \mat{u}_{imf,k} &= -\mat{B}^+ (\mat{A} - \mat{A}_{ref})\mat{x}_k +
    \mat{B}^+ \mat{B}_{ref}\mat{u}_k
\end{align*}

The first term makes the open-loop poles match that of the reference model, and
the second term makes the input behave like that of the reference model.
