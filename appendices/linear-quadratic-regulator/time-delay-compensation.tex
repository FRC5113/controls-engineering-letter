\section{Time delay compensation}
\index{controller design!linear-quadratic regulator!time delay compensation}

Linear-Quadratic regulator controller gains tend to be aggressive. If sensor
measurements are time-delayed too long, the LQR may be unstable (see figure
\ref{fig:elevator_time_delay_no_comp}). However, if we know the amount of delay,
we can compute the control based on where the system will be after the time
delay.

We can compensate for the time delay if we know the control law we're applying
in future timesteps ($\mat{u} = -\mat{K}\mat{x}$) and the duration of the time
delay. To get the true state at the current time for control purposes, we
project our delayed state forward by the time delay using our model and the
aforementioned control law. Figure \ref{fig:elevator_time_delay_comp} shows
improved control with the predicted state.\footnote{Input delay and output delay
have the same effect on the system, so the time delay can be simulated with
either an input delay buffer or a measurement delay buffer.}
\begin{bookfigure}
  \begin{minisvg}{2}{build/\chapterpath/elevator_time_delay_no_comp}
    \caption{Elevator response at 5ms sample period with 50ms of output lag}
    \label{fig:elevator_time_delay_no_comp}
  \end{minisvg}
  \hfill
  \begin{minisvg}{2}{build/\chapterpath/elevator_time_delay_comp}
    \caption{Elevator response at 5ms sample period with 50ms of output lag
      (delay-compensated)}
    \label{fig:elevator_time_delay_comp}
  \end{minisvg}
\end{bookfigure}

For steady-state controller gains, this method of delay compensation seems to
work better for second-order systems than first-order systems. Figures
\ref{fig:drivetrain_time_delay_no_comp} and
\ref{fig:flywheel_time_delay_no_comp} show time delay for a drivetrain velocity
system and flywheel system respectively. Figures
\ref{fig:drivetrain_time_delay_comp} and \ref{fig:flywheel_time_delay_comp} show
that compensating the controller gain significantly reduces the feedback gain.
For systems with fast dynamics and a long time delay, the delay-compensated
controller has an almost open-loop response because only feedforward has a
significant effect; this has poor disturbance rejection. Fixing the source of
the time delay is always preferred for these systems.
\begin{bookfigure}
  \begin{minisvg}{2}{build/\chapterpath/drivetrain_time_delay_no_comp}
    \caption{Drivetrain velocity response at 1ms sample period with 40ms of
      output lag}
    \label{fig:drivetrain_time_delay_no_comp}
  \end{minisvg}
  \hfill
  \begin{minisvg}{2}{build/\chapterpath/drivetrain_time_delay_comp}
    \caption{Drivetrain velocity response at 1ms sample period with 40ms of
      output lag (delay-compensated)}
    \label{fig:drivetrain_time_delay_comp}
  \end{minisvg}
  \begin{minisvg}{2}{build/\chapterpath/flywheel_time_delay_no_comp}
    \caption{Flywheel response at 1ms sample period with 100ms of output lag}
    \label{fig:flywheel_time_delay_no_comp}
  \end{minisvg}
  \hfill
  \begin{minisvg}{2}{build/\chapterpath/flywheel_time_delay_comp}
    \caption{Flywheel response at 1ms sample period with 100ms of output lag
      (delay-compensated)}
    \label{fig:flywheel_time_delay_comp}
  \end{minisvg}
\end{bookfigure}

Since we are computing the control based on future states and the state
exponentially converges to zero over time, the control action we apply at the
current timestep also converges to zero for longer time delays. During startup,
the inputs we use to predict the future state are zero because there's initially
no input history. This means the initial inputs are larger to give the system a
kick in the right direction. As the input delay buffer fills up, the controller
gain converges to a smaller steady-state value. If one uses the steady-state
controller gain during startup, the transient response may be slow.

All figures shown here use the steady-state control law (the second case in
equation \eqref{eq:discrete_delay_comp_control_law}).

We'll show how to derive this controller gain compensation for continuous and
discrete systems.

\subsection{Continuous case}

The undelayed continuous linear system is defined as
$\dot{\mat{x}} = \mat{A}\mat{x}(t) + \mat{B}\mat{u}(t)$ with the controller
$\mat{u}(t) = -\mat{K}\mat{x}(t)$. Let $L$ be the amount of time delay in
seconds. We can avoid the time delay if we compute the control based on the
plant $L$ seconds in the future.
\begin{equation*}
  \mat{u}(t) = -\mat{K}\mat{x}(t + L)
\end{equation*}

We need to find $\mat{x}(t + L)$ given $\mat{x}(t)$. Since we know
$\mat{u}(t) = -\mat{K}\mat{x}(t)$ will be applied over the time interval
$[t, t + L)$, substitute it into the continuous model.
\begin{align*}
  \dot{\mat{x}} &= \mat{A}\mat{x}(t) + \mat{B}\mat{u}(t) \\
  \dot{\mat{x}} &= \mat{A}\mat{x}(t) + \mat{B}(-\mat{K}\mat{x}(t)) \\
  \dot{\mat{x}} &= \mat{A}\mat{x}(t) - \mat{B}\mat{K}\mat{x}(t) \\
  \dot{\mat{x}} &= (\mat{A} - \mat{B}\mat{K}) \mat{x}(t)
\end{align*}

We now have a differential equation for the closed-loop system dynamics. Take
the matrix exponential from the current time $t$ to $L$ in the future to obtain
$\mat{x}(t + L)$.
\begin{equation}
  \mat{x}(t + L) = e^{(\mat{A} - \mat{B}\mat{K})L} \mat{x}(t)
    \label{eq:continuous_advance_state_by_delay_post}
\end{equation}

This works for $t \geq L$, but if $t < L$, we have no input history for the time
interval $[t, L)$. If we assume the inputs for $[t, L)$ are zero, the state
prediction for that interval is
\begin{equation*}
  \mat{x}(L) = e^{\mat{A}(L - t)} \mat{x}(t)
\end{equation*}

The time interval $[0, t)$ has nonzero inputs since it's in the past and was
using the normal control law.
\begin{align}
  \mat{x}(t + L) &= e^{(\mat{A} - \mat{B}\mat{K})t} \mat{x}(L) \nonumber \\
  \mat{x}(t + L) &= e^{(\mat{A} - \mat{B}\mat{K})t} e^{\mat{A}(L - t)}
    \mat{x}(t) \label{eq:continuous_advance_state_by_delay_pre}
\end{align}

Therefore, equations \eqref{eq:continuous_advance_state_by_delay_post} and
\eqref{eq:continuous_advance_state_by_delay_pre} give the latency-compensated
control law for all $t \geq 0$.
\begin{align}
  \mat{u}(t) &= -\mat{K} \mat{x}(t + L) \nonumber \\
  \mat{u}(t) &=
  \begin{cases}
    -\mat{K} e^{(\mat{A} - \mat{B}\mat{K})t} e^{\mat{A}(L - t)} \mat{x}(t) &
      \text{if } 0 \leq t < L \\
    -\mat{K} e^{(\mat{A} - \mat{B}\mat{K})L} \mat{x}(t) & \text{if } t \geq L
  \end{cases}
\end{align}

\subsection{Discrete case}

The undelayed discrete linear system is defined as
$\mat{x}_{k+1} = \mat{A}\mat{x}_k + \mat{B}\mat{u}_k$ with the controller
$\mat{u}_k = -\mat{K}\mat{x}_k$. Let $L$ be the amount of time delay in seconds.
We can avoid the time delay if we compute the control based on the plant $L$
seconds in the future.
\begin{equation*}
  \mat{u}_k = -\mat{K}\mat{x}_{k + L/T}
\end{equation*}

We need to find $\mat{x}_{k + L/T}$ given $\mat{x}_k$. Since we know
$\mat{u}_k = -\mat{K}\mat{x}_k$ will be applied for the timesteps $k$ through
$k + \frac{L}{T}$, substitute it into the discrete model.
\begin{align*}
  \mat{x}_{k+1} &= \mat{A}\mat{x}_k + \mat{B}\mat{u}_k \\
  \mat{x}_{k+1} &= \mat{A}\mat{x}_k + \mat{B}(-\mat{K}\mat{x}_k) \\
  \mat{x}_{k+1} &= \mat{A}\mat{x}_k - \mat{B}\mat{K}\mat{x}_k \\
  \mat{x}_{k+1} &= (\mat{A} - \mat{B}\mat{K}) \mat{x}_k
\end{align*}

Let $T$ be the duration between timesteps in seconds and $L$ be the amount of
time delay in seconds. $\frac{L}{T}$ gives the number of timesteps represented
by $L$.
\begin{equation}
  \mat{x}_{k + L/T} = (\mat{A} - \mat{B}\mat{K})^\frac{L}{T} \mat{x}_k
    \label{eq:discrete_advance_state_by_delay_post}
\end{equation}

This works for $kT \geq L$ where $kT$ is the current time, but if $kT < L$, we
have no input history for the time interval $[kT, L)$. If we assume the inputs
for $[kT, L)$ are zero, the state prediction for that interval is
\begin{align*}
  \mat{x}_{L/T} &= \mat{A}^\frac{L - kT}{T} \mat{x}_k \\
  \mat{x}_{L/T} &= \mat{A}^{\frac{L}{T} - k} \mat{x}_k
\end{align*}

The time interval $[0, kT)$ has nonzero inputs since it's in the past and was
using the normal control law.
\begin{align}
  \mat{x}_{k + L/T} &= (\mat{A} - \mat{B}\mat{K})^k \mat{x}_{L/T} \nonumber \\
  \mat{x}_{k + L/T} &= (\mat{A} - \mat{B}\mat{K})^k
    \mat{A}^{\frac{L}{T} - k} \mat{x}_k
    \label{eq:discrete_advance_state_by_delay_pre}
\end{align}

Therefore, equations \eqref{eq:discrete_advance_state_by_delay_post} and
\eqref{eq:discrete_advance_state_by_delay_pre} give the latency-compensated
control law for all $t \geq 0$.
\begin{align}
  \mat{u}_k &= -\mat{K} \mat{x}_{k + L/T} \nonumber \\
  \mat{u}_k &=
  \begin{cases}
    -\mat{K} (\mat{A} - \mat{B}\mat{K})^k \mat{A}^{\frac{L}{T} - k} \mat{x}_k &
      \text{if } 0 \leq k < \frac{L}{T} \\
    -\mat{K} (\mat{A} - \mat{B}\mat{K})^\frac{L}{T} \mat{x}_k &
      \text{if } k \geq \frac{L}{T}
  \end{cases}
  \label{eq:discrete_delay_comp_control_law}
\end{align}

If the delay $L$ isn't a multiple of the sample period $T$ in equation
\eqref{eq:discrete_delay_comp_control_law}, we have to evaluate fractional
matrix powers, which can be tricky. Eigen (a C++ library) supports fractional
powers with the \texttt{pow()} member function provided by
\texttt{<unsupported/Eigen/MatrixFunctions>}. scipy (a Python library) supports
fractional powers with the free function
\texttt{scipy.linalg.fractional\_matrix\_power()}. If the language you're using
doesn't provide such a function, you can try the following approach instead.

Let there be a matrix $\mat{M}$ raised to a fractional power $n$. If $\mat{M}$
is diagonalizable, we can obtain an exact answer for $\mat{M}^n$ by decomposing
$\mat{M}$ into $\mat{P}\mat{D}\mat{P}^{-1}$ where $\mat{D}$ is a diagonal
matrix, computing $\mat{D}^n$ as each diagonal element raised to $n$, then
recomposing $\mat{M}^n$ as $\mat{P}\mat{D}^n\mat{P}^{-1}$.

If a matrix raised to a fractional power in equation
\eqref{eq:discrete_delay_comp_control_law} isn't diagonalizable, we have to
approximate by rounding $\frac{L}{T}$ to the nearest integer. This approximation
gets worse as $L \bmod T$ approaches $\frac{T}{2}$.
