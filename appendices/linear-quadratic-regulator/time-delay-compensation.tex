\section{Time delay compensation}
\index{controller design!linear-quadratic regulator!time delay compensation}

Linear-Quadratic regulator controller gains tend to be aggressive. If sensor
measurements are time-delayed too long, the LQR may be unstable (see figure
\ref{fig:elevator_time_delay_no_comp}). However, if we know the amount of delay,
we can compute the control based on where the system will be after the time
delay.

We can compensate for the time delay if we know the control law we're applying
in future timesteps ($\mtx{u} = -\mtx{K}\mtx{x}$) and the duration of the time
delay. To get the true state at the current time for control purposes, we
project our delayed state forward by the time delay using our model and the
aforementioned control law. Figure \ref{fig:elevator_time_delay_comp} shows
improved control with the predicted state.\footnote{Input delay and output delay
have the same effect on the system, so the time delay can be simulated with
either an input delay buffer or a measurement delay buffer.}
\begin{bookfigure}
  \begin{minisvg}{2}{build/\chapterpath/elevator_time_delay_no_comp}
    \caption{Elevator response at 5ms sample period with 50ms of output lag}
    \label{fig:elevator_time_delay_no_comp}
  \end{minisvg}
  \hfill
  \begin{minisvg}{2}{build/\chapterpath/elevator_time_delay_comp}
    \caption{Elevator response at 5ms sample period with 50ms of output lag
      (delay-compensated)}
    \label{fig:elevator_time_delay_comp}
  \end{minisvg}
\end{bookfigure}

For steady-state controller gains, this method of delay compensation seems to
work better for second-order systems than first-order systems. Figures
\ref{fig:drivetrain_time_delay_no_comp} and
\ref{fig:flywheel_time_delay_no_comp} show time delay for a drivetrain velocity
system and flywheel system respectively. Figures
\ref{fig:drivetrain_time_delay_comp} and \ref{fig:flywheel_time_delay_comp} show
that compensating the controller gain significantly reduces the feedback gain.
For systems with fast dynamics and a long time delay, the delay-compensated
controller has an almost open-loop response because only feedforward has a
significant effect; this has poor disturbance rejection. Fixing the source of
the time delay is always preferred for these systems.
\begin{bookfigure}
  \begin{minisvg}{2}{build/\chapterpath/drivetrain_time_delay_no_comp}
    \caption{Drivetrain velocity response at 1ms sample period with 40ms of
      output lag}
    \label{fig:drivetrain_time_delay_no_comp}
  \end{minisvg}
  \hfill
  \begin{minisvg}{2}{build/\chapterpath/drivetrain_time_delay_comp}
    \caption{Drivetrain velocity response at 1ms sample period with 40ms of
      output lag (delay-compensated)}
    \label{fig:drivetrain_time_delay_comp}
  \end{minisvg}
  \begin{minisvg}{2}{build/\chapterpath/flywheel_time_delay_no_comp}
    \caption{Flywheel response at 1ms sample period with 100ms of output lag}
    \label{fig:flywheel_time_delay_no_comp}
  \end{minisvg}
  \hfill
  \begin{minisvg}{2}{build/\chapterpath/flywheel_time_delay_comp}
    \caption{Flywheel response at 1ms sample period with 100ms of output lag
      (delay-compensated)}
    \label{fig:flywheel_time_delay_comp}
  \end{minisvg}
\end{bookfigure}

Since we are computing the control based on future states and the state
exponentially converges to zero over time, the control action we apply at the
current timestep also converges to zero for longer time delays. During startup,
the inputs we use to predict the future state are zero because there's initially
no input history. This means the initial inputs are larger to give the system a
kick in the right direction. As the input delay buffer fills up, the controller
gain converges to a smaller steady-state value. If one uses the steady-state
controller gain during startup, the transient response may be slow.

All figures shown here use the steady-state control law (the second case in
equation \eqref{eq:discrete_delay_comp_control_law}).

We'll show how to derive this controller gain compensation for continuous and
discrete systems.

\subsection{Continuous case}

We can avoid the time delay if we compute the control based on the plant $L$
seconds in the future. Therefore, we need to derive an equation for the plant's
state $L$ seconds in the future given the current state.

The continuous linear system is defined as
\begin{equation*}
  \dot{\mtx{x}} = \mtx{A}\mtx{x}(t) + \mtx{B}\mtx{u}(t)
\end{equation*}

Let the controller for this system be
\begin{equation*}
  \mtx{u}(t) = -\mtx{K}\mtx{x}(t)
\end{equation*}

Substitute this into the continuous model.
\begin{align*}
  \dot{\mtx{x}} &= \mtx{A}\mtx{x}(t) + \mtx{B}\mtx{u}(t) \\
  \dot{\mtx{x}} &= \mtx{A}\mtx{x}(t) + \mtx{B}(-\mtx{K}\mtx{x}(t)) \\
  \dot{\mtx{x}} &= \mtx{A}\mtx{x}(t) - \mtx{B}\mtx{K}\mtx{x}(t) \\
  \dot{\mtx{x}} &= (\mtx{A} - \mtx{B}\mtx{K}) \mtx{x}(t)
\end{align*}

Let $L$ be the amount of time delay in seconds. Take the matrix exponential from
the current time $t$ to $L$ in the future.
\begin{equation}
  \mtx{x}(t + L) = e^{(\mtx{A} - \mtx{B}\mtx{K})L} \mtx{x}(t)
    \label{eq:continuous_advance_state_by_delay_post}
\end{equation}

This works when $t > L$, but when $0 \leq t < L$, the inputs in the time
interval $[t, L)$ are zero because there's initially no input history. The state
prediction for that time interval is thus
\begin{equation*}
  \mtx{x}(L) = e^{\mtx{A}(L - t)} \mtx{x}(t)
\end{equation*}

The time interval $[0, t)$ has nonzero inputs, so it uses the normal control
law.
\begin{align}
  \mtx{x}(t + L) &= e^{(\mtx{A} - \mtx{B}\mtx{K})t} \mtx{x}(L) \nonumber \\
  \mtx{x}(t + L) &= e^{(\mtx{A} - \mtx{B}\mtx{K})t} e^{\mtx{A}(L - t)}
    \mtx{x}(t) \label{eq:continuous_advance_state_by_delay_pre}
\end{align}

Therefore, equations \eqref{eq:continuous_advance_state_by_delay_post} and
\eqref{eq:continuous_advance_state_by_delay_pre} give the latency-compensated
control law for all $t \geq 0$.
\begin{align}
  \mtx{u}(t) &= -\mtx{K} \mtx{x}(t + L) \nonumber \\
  \mtx{u}(t) &=
  \begin{cases}
    -\mtx{K} e^{(\mtx{A} - \mtx{B}\mtx{K})t} e^{\mtx{A}(L - t)} \mtx{x}(t) &
      \text{if } 0 \leq t < L \\
    -\mtx{K} e^{(\mtx{A} - \mtx{B}\mtx{K})L} \mtx{x}(t) & \text{if } t \geq L
  \end{cases}
\end{align}

\subsection{Discrete case}

We can avoid the time delay if we compute the control based on the plant $L$
seconds in the future. Therefore, we need to derive an equation for the plant's
state $L$ seconds in the future given the current state.

The discrete linear system is defined as
\begin{equation*}
  \mtx{x}_{k+1} = \mtx{A}\mtx{x}_k + \mtx{B}\mtx{u}_k
\end{equation*}

Let the controller for this system be
\begin{equation*}
  \mtx{u}_k = -\mtx{K}\mtx{x}_k
\end{equation*}

Substitute this into the discrete model.
\begin{align*}
  \mtx{x}_{k+1} &= \mtx{A}\mtx{x}_k + \mtx{B}\mtx{u}_k \\
  \mtx{x}_{k+1} &= \mtx{A}\mtx{x}_k + \mtx{B}(-\mtx{K}\mtx{x}_k) \\
  \mtx{x}_{k+1} &= \mtx{A}\mtx{x}_k - \mtx{B}\mtx{K}\mtx{x}_k \\
  \mtx{x}_{k+1} &= (\mtx{A} - \mtx{B}\mtx{K}) \mtx{x}_k
\end{align*}

Let $T$ be the duration between timesteps in seconds and $L$ be the amount of
time delay in seconds. $\frac{L}{T}$ gives the number of timesteps represented
by $L$.
\begin{equation}
  \mtx{x}_{k+L} = (\mtx{A} - \mtx{B}\mtx{K})^\frac{L}{T} \mtx{x}_k
    \label{eq:discrete_advance_state_by_delay_post}
\end{equation}

This works when the current time $kT$ is greater than or equal to $L$, but when
$0 \leq kT < L$, the inputs in the time interval $[kT, L)$ are zero because
there's initially no input history. The state prediction for that time interval
is thus
\begin{equation*}
  \mtx{x}_L = \mtx{A}^\frac{L - kT}{T} \mtx{x}_k
\end{equation*}

The time interval $[0, kT)$ has nonzero inputs, so it uses the normal control
law.
\begin{align}
  \mtx{x}_{k + L} &= (\mtx{A} - \mtx{B}\mtx{K})^\frac{L}{T} \mtx{x}_L
    \nonumber \\
  \mtx{x}_{k + L} &= (\mtx{A} - \mtx{B}\mtx{K})^\frac{L}{T}
    \mtx{A}^\frac{L - kT}{T} \mtx{x}_k
    \label{eq:discrete_advance_state_by_delay_pre}
\end{align}

Therefore, equations \eqref{eq:discrete_advance_state_by_delay_post} and
\eqref{eq:discrete_advance_state_by_delay_pre} give the latency-compensated
control law for all $t \geq 0$.
\begin{align}
  \mtx{u}_k &= -\mtx{K} \mtx{x}_{k + L} \nonumber \\
  \mtx{u}_k &=
  \begin{cases}
    -\mtx{K} (\mtx{A} - \mtx{B}\mtx{K})^\frac{L}{T} \mtx{A}^\frac{L - kT}{T}
      \mtx{x}_k & \text{if } 0 \leq kT < L \\
    -\mtx{K} (\mtx{A} - \mtx{B}\mtx{K})^\frac{L}{T} \mtx{x}_k &
      \text{if } kT \geq L
  \end{cases}
  \label{eq:discrete_delay_comp_control_law}
\end{align}

If the delay $L$ isn't a multiple of the sample period $T$ in equation
\eqref{eq:discrete_delay_comp_control_law}, we have to evaluate a fractional
matrix power, which can be tricky. If $\mtx{A} - \mtx{B}\mtx{K}$ is
diagonalizable, we can obtain an exact answer for
$(\mtx{A} - \mtx{B}\mtx{K})^\frac{L}{T}$ by decomposing
$\mtx{A} - \mtx{B}\mtx{K}$ into $\mtx{P}\mtx{D}\mtx{P}^{-1}$ where $\mtx{D}$ is
a diagonal matrix, computing $\mtx{D}^\frac{L}{T}$ as each diagonal element
raised to $\frac{L}{T}$, then recomposing
$\mtx{P}\mtx{D}^\frac{L}{T}\mtx{P}^{-1}$. If $\mtx{A} - \mtx{B}\mtx{K}$ isn't
diagonalizable, we'll have to approximate the matrix power by rounding
$\frac{L}{T}$ to the nearest integer. This approximation gets worse as
$L \bmod T$ approaches $\frac{T}{2}$.
