\section{Time delay compensation}
\index{controller design!linear-quadratic regulator!time delay compensation}

Linear-Quadratic regulator controller gains tend to be aggressive, and in the
presence of sensor delay, they may be unstable (see figure
\ref{fig:elevator_time_delay_no_comp}). However, if we know the amount of delay,
we can design the controller for a projected version of the undelayed model.
That is, we can compute the control based on where the system will be after the
time delay (see figure \ref{fig:elevator_time_delay_comp}).
\begin{bookfigure}
  \begin{minisvg}{2}{build/\chapterpath/elevator_time_delay_no_comp}
    \caption{Elevator response at 5ms sample period with 50ms of output lag
      (uncompensated controller gains)}
    \label{fig:elevator_time_delay_no_comp}
  \end{minisvg}
  \hfill
  \begin{minisvg}{2}{build/\chapterpath/elevator_time_delay_comp}
    \caption{Elevator response at 5ms sample period with 50ms of output lag
      (compensated controller gains)}
    \label{fig:elevator_time_delay_comp}
  \end{minisvg}
\end{bookfigure}

This method of delay compensation seems to work best for second-order systems.
Figure \ref{fig:drivetrain_time_delay_no_comp} shows time delay for a drivetrain
velocity system. Figure \ref{fig:drivetrain_time_delay_comp} shows that
compensating the controller gain significantly reduces the feedback gain. Mainly
feedforward is acting for a predominantly open-loop response, which has poor
disturbance rejection. Fixing the source of the time delay is always preferred
in general, but especially in cases like these.
\begin{bookfigure}
  \begin{minisvg}{2}{build/\chapterpath/drivetrain_time_delay_no_comp}
    \caption{Drivetrain response at 1ms sample period with 40ms of output lag
      (uncompensated controller gain)}
    \label{fig:drivetrain_time_delay_no_comp}
  \end{minisvg}
  \hfill
  \begin{minisvg}{2}{build/\chapterpath/drivetrain_time_delay_comp}
    \caption{Drivetrain response at 1ms sample period with 40ms of output lag
      (compensated controller gain)}
    \label{fig:drivetrain_time_delay_comp}
  \end{minisvg}
\end{bookfigure}

We'll show how to derive this controller gain compensation for continuous and
discrete systems. For more details on the method involved and controller startup
considerations, read the original paper \cite{bib:lqr_time_delay}.

\subsection{Continuous case}

The continuous linear system is defined as
\begin{equation*}
  \dot{\mtx{x}} = \mtx{A}\mtx{x}(t) + \mtx{B}\mtx{u}(t)
\end{equation*}

Let the controller for this system be
\begin{equation*}
  \mtx{u}(t) = -\mtx{K}\mtx{x}(t)
\end{equation*}

Substitute this into the continuous model.
\begin{align*}
  \dot{\mtx{x}} &= \mtx{A}\mtx{x}(t) + \mtx{B}\mtx{u}(t) \\
  \dot{\mtx{x}} &= \mtx{A}\mtx{x}(t) + \mtx{B}(-\mtx{K}\mtx{x}(t)) \\
  \dot{\mtx{x}} &= \mtx{A}\mtx{x}(t) - \mtx{B}\mtx{K}\mtx{x}(t) \\
  \dot{\mtx{x}} &= (\mtx{A} - \mtx{B}\mtx{K}) \mtx{x}(t)
\end{align*}

Let $L$ be the amount of time delay in seconds. Take the matrix exponential from
the current time $t$ to $L$ in the future.
\begin{equation}
  \mtx{x}(t + L) = e^{(\mtx{A} - \mtx{B}\mtx{K})L} \mtx{x}(t)
    \label{eq:continuous_advance_state_by_delay}
\end{equation}

We can avoid the time delay if we compute the control based on the plant $L$
seconds in the future using equation
\eqref{eq:continuous_advance_state_by_delay}. Therefore, the latency-compensated
controller is
\begin{align}
  \mtx{u}(t) &= -\mtx{K}\mtx{x}(t + L) \nonumber \\
  \mtx{u}(t) &= -\mtx{K} e^{(\mtx{A} - \mtx{B}\mtx{K})L} \mtx{x}(t)
\end{align}

\subsection{Discrete case}

The discrete linear system is defined as
\begin{equation*}
  \mtx{x}_{k+1} = \mtx{A}\mtx{x}_k + \mtx{B}\mtx{u}_k
\end{equation*}

Let the controller for this system be
\begin{equation*}
  \mtx{u}_k = -\mtx{K}\mtx{x}_k
\end{equation*}

Substitute this into the discrete model.
\begin{align*}
  \mtx{x}_{k+1} &= \mtx{A}\mtx{x}_k + \mtx{B}\mtx{u}_k \\
  \mtx{x}_{k+1} &= \mtx{A}\mtx{x}_k + \mtx{B}(-\mtx{K}\mtx{x}_k) \\
  \mtx{x}_{k+1} &= \mtx{A}\mtx{x}_k - \mtx{B}\mtx{K}\mtx{x}_k \\
  \mtx{x}_{k+1} &= (\mtx{A} - \mtx{B}\mtx{K}) \mtx{x}_k
\end{align*}

Let $T$ be the duration between timesteps in seconds and $L$ be the amount of
time delay in seconds. $\frac{L}{T}$ gives the number of timesteps represented
by $L$.
\begin{equation}
  \mtx{x}_{k+L} = (\mtx{A} - \mtx{B}\mtx{K})^\frac{L}{T} \mtx{x}_k
    \label{eq:discrete_advance_state_by_delay}
\end{equation}

We can avoid the time delay if we compute the control based on the plant $L$
seconds in the future using equation \eqref{eq:discrete_advance_state_by_delay}.
Therefore, the latency-compensated controller is
\begin{align}
  \mtx{u}_k &= -\mtx{K}\mtx{x}_{k+L} \nonumber \\
  \mtx{u}_k &= -\mtx{K} (\mtx{A} - \mtx{B}\mtx{K})^\frac{L}{T} \mtx{x}_k
    \label{eq:discrete_delay_comp_control_law}
\end{align}

If the delay $L$ isn't a multiple of the sample period $T$ in equation
\eqref{eq:discrete_delay_comp_control_law}, we have to evaluate a fractional
matrix power, which can be tricky. If $\mtx{A} - \mtx{B}\mtx{K}$ is
diagonalizable, we can obtain an exact answer for
$(\mtx{A} - \mtx{B}\mtx{K})^\frac{L}{T}$ by decomposing
$\mtx{A} - \mtx{B}\mtx{K}$ into $\mtx{P}\mtx{D}\mtx{P}^{-1}$ where $\mtx{D}$ is
a diagonal matrix, computing $\mtx{D}^\frac{L}{T}$ as each diagonal element
raised to $\frac{L}{T}$, then recomposing
$\mtx{P}\mtx{D}^\frac{L}{T}\mtx{P}^{-1}$. If $\mtx{A} - \mtx{B}\mtx{K}$ isn't
diagonalizable, we'll have to approximate the matrix power by rounding
$\frac{L}{T}$ to the nearest integer. This approximation gets worse as
$L \bmod T$ approaches $\frac{T}{2}$.
