\chapterimage{appendices.jpg}{Sunset in an airplane over New Mexico}

\chapter{Installing Python packages} \label{ch:installing_python_packages}

\section{Windows instructions}

To install Python, download the installer for Python 3.5 or higher from
\url{https://www.python.org/downloads/} and run it.

To install Python packages, run \texttt{py -3 -m pip install pkg} via cmd.exe or
Powershell where \texttt{pkg} should be the name of the package. Packages can be
upgraded with \texttt{py -3 -m pip install --user --upgrade pkg}.

\section{Linux instructions}

To install Python, install the appropriate packages from table
\ref{tab:required_system_packages} using your system's package manager.

\begin{booktable}
  \begin{tabular}{|ll|}
    \hline
    \rowcolor{headingbg}
    \textbf{Debian/Ubuntu} & \textbf{Arch Linux} \\
    \hline
    python3 & python \\
    python3-pip & python-pip \\
    \hline
  \end{tabular}
  \caption{Required system packages}
  \label{tab:required_system_packages}
\end{booktable}

To install Python packages, run \texttt{pip3 install --user pkg} where
\texttt{pkg} should be the name of the package. Using \texttt{--user} makes
installation not require root privileges. Packages can be upgraded with
\texttt{pip3 install --user --upgrade pkg}.
