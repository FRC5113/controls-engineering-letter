\section{Ethos of this book}

This book is intended as both a tutorial for new students and as a reference
manual for more experienced readers who need to review a thing or two. While it
isn't comprehensive, the reader will hopefully learn enough to either implement
the concepts presented themselves or know where to look for more information.

Some parts are mathematically rigorous, but I believe in giving students a solid
theoretical foundation with emphasis on intuition so they can apply it to new
problems. To achieve deep understanding of the topics in this book, math is
unavoidable. With that said, I try to provide practical and intuitive
explanations whenever possible.

Most teaching resources separate linear and nonlinear control with the latter
being reserved for a different course. Here, they are introduced together
because the concepts of nonlinear control apply often, and it isn't that much of
a leap (if Lyapunov stability isn't included). The control and estimation
chapters cover relevant tools for dealing with nonlinearities like linearization
when appropriate.
