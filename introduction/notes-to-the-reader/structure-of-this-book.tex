\section{Structure of this book}

This book consists of five parts and a collection of appendices that address the
four tasks a controls engineer carries out: derive a model of the system
(kinematics), design a controller for the model (control theory), design an
observer to estimate the current state of the model (localization), and plan how
the controller is going to drive the model to a desired state (motion planning).

Part \ref{part:fundamentals_of_control_theory},
``\nameref{part:fundamentals_of_control_theory}," introduces the basics of
control theory, teaches the fundamentals of PID controller design, describes
what a transfer function is, and shows how they can be used to analyze dynamical
systems. Emphasis is placed on the geometric intuition of this analysis rather
than the math.

Part \ref{part:modern_control_theory}, ``\nameref{part:modern_control_theory},"
first provides a crash course in the geometric intuition behind linear algebra
and covers enough of the mechanics of evaluating matrix algebra for the reader
to follow along in later chapters. It covers state-space representation,
controllability, and observability. The intuition gained in part
\ref{part:fundamentals_of_control_theory} and the notation of linear algebra are
used to model and control linear multiple-input, multiple-output (MIMO) systems
and covers discretization, LQR controller design, LQE observer design, and
feedforwards. Then, these concepts are applied to design and implement
controllers for real systems. The examples from part \ref{part:system_modeling}
are converted to state-space representation, implemented, and tested with a
discrete controller.

Part \ref{part:modern_control_theory} also introduces the basics of nonlinear
control system analysis with Lyapunov functions. It presents an example of a
nonlinear controller for a unicycle-like vehicle as well as how to apply it to a
two-wheeled vehicle. Since nonlinear control isn't the focus of this book, we
mention other books and resources for further reading.

Part \ref{part:estimation_and_localization},
``\nameref{part:estimation_and_localization}," introduces the field of
stochastic control theory. The Luenberger observer and the probability theory
behind the Kalman filter is taught with several examples of creative
applications of Kalman filter theory.

Part \ref{part:system_modeling}, ``\nameref{part:system_modeling}," introduces
the basic calculus and physics concepts required to derive the models used in
the previous chapters. It walks through the derivations for several common FRC
subsystems. Then, methods for system identification are discussed for
empirically measuring model parameters.

Part \ref{part:motion_planning}, ``\nameref{part:motion_planning}," covers
planning how the robot will get from its current state to some desired state in
a manner achievable by its dynamics. It introduces motion profiles with one
degree of freedom for simple maneuvers. Trajectory optimization methods are
presented for generating profiles with higher degrees of freedom.

The appendices provide further enrichment that isn't required for a passing
understanding of the material. This includes derivations for many of the results
presented and used in the mainmatter of the book.

The Python scripts used to generate the plots in the case studies double as
reference implementations of the techniques discussed in their respective
chapters. They are available in this book's Git repository. Its location is
listed on the copyright page.
