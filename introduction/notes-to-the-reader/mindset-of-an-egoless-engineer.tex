\section{Mindset of an egoless engineer}
\label{sec:mindset_of_an_egoless_engineer}

Engineering has a mindset, not just a skillset. Engineers have a unique way of
approaching problems, and the following maxim summarizes what I hope to teach my
robotics students (with examples drawn from controls engineering).
\begin{quote}
  ``Engineer based on requirements, not an ideology."
\end{quote}

Engineering is filled with trade-offs. The tools should fit the job, and not
every problem is a nail waiting to be struck by a hammer. Instead, assess the
minimum requirements (min specs) for a solution to the task at hand and do only
enough work to satisfy them; exceeding your specifications is a waste of time
and money. If you require performance or maintainability above the min specs,
your min specs were chosen incorrectly by definition.

Controls engineering is pragmatic in a similar respect:
\textit{solve. the. problem}. For control of nonlinear systems,
\href{https://faculty.washington.edu/devasia/Inversion.html}{plant inversion}
is elegant on paper but doesn't work with an inaccurate model, yet using a
theoretically incorrect solution like linear approximations of the nonlinear
system works well enough to be used industry-wide. There are more sophisticated
controllers than PID, but we use PID anyway for its versatility and simplicity.
Sometimes the inferior solutions are more effective or have a more desirable
cost-benefit ratio than what the control system designer considers ideal or
clean. Choose the tool that is most effective.

Solutions need to be good enough, but do not need to be perfect. We want to
avoid integrators as they introduce instability, but we use them anyway because
they work well for meeting tracking specifications. One should not blindly
defend a design or follow an ideology, because there is always a case where its
antithesis is a better option. The engineer should be able to determine when
this is the case, set aside their ego, and do what will meet the specifications
of their client (e.g., system response characteristics, maintainability,
usability). Preferring one solution over another for pragmatic or technical
reasons is fine, but the engineer should not care on a personal level which
sufficient solution is chosen.
