\section{What is gain?}
\index{Gain}

\Gls{gain} is a proportional value that shows the relationship between the
magnitude of an input signal to the magnitude of an output signal at
steady-state. Many \glspl{system} contain a method by which the gain can be
altered, providing more or less ``power" to the \gls{system}.

Figure \ref{fig:input_output_gain} shows a \gls{system} with a hypothetical
input and output. Since the output is twice the amplitude of the input, the
\gls{system} has a gain of $2$.

\begin{bookfigure}
  \begin{tikzpicture}[auto, >=latex']
    % \draw [help lines] (-4,-2) grid (4,2);

    % Input
    \drawtimeplot{-2.5cm}{0cm}{0.125cm}{0.44375cm}{0.6 * cos(40 * deg(\x))}
    \draw (-2.5,1) node {\small Input};

    \node [block] (sys) {K};
    \draw (0,1) node {\small System};

    % Output
    \drawtimeplot{2.5cm}{0cm}{0.125cm}{0.44375cm}{1.2 * cos(40* deg(\x))}
    \draw (2.5,1) node {\small Output};

    % Arrows between input/output and system
    \draw[->] (-2,0) -- (sys);
    \draw[->] (sys) -- (2,0);
  \end{tikzpicture}

  \caption{Demonstration of system with a gain of $K = 2$}
  \label{fig:input_output_gain}
\end{bookfigure}
