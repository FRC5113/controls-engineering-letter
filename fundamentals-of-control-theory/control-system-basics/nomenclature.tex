\section{Nomenclature}

Most resources for advanced engineering topics assume a level of knowledge well
above that which is necessary. Part of the problem is the use of jargon. While
it efficiently communicates ideas to those within the field, new people who
aren't familiar with it are lost. See the glossary for a list of words and
phrases commonly used in control theory, their origins, and their meaning. Links
to the glossary are provided for certain words throughout the book and will use
\textcolor{glscolor}{this color}.

The \gls{system} or collection of actuators being controlled by a
\gls{control system} is called the \gls{plant}. A \gls{controller} is used to
drive the \gls{plant} from its current state to some desired state (the
\gls{reference}). \Glspl{controller} which don't include information measured
from the \gls{plant}'s \gls{output} are called open-loop \glspl{controller}.
\Glspl{controller} which incorporate information fed back from the \gls{plant}'s
\gls{output} are called closed-loop \glspl{controller} or feedback
\glspl{controller}. Figure \ref{fig:system_nomenclature} shows a \gls{plant} in
feedback with a \gls{controller}.

\begin{bookfigure}
  \begin{tikzpicture}[auto, >=latex']
    \fontsize{9pt}{10pt}

    % Place the blocks
    \node [name=input] {$r(t)$};
    \node [sum, right=0.5cm of input] (errorsum) {};
    \node [block, right=1.0cm of errorsum] (controller) {Controller};
    \node [block, right=1.0cm of controller] (plant) {Plant};
    \node [right=1.0cm of plant] (output) {};
    \node [coordinate, below=1.0cm of controller] (measurements) {};

    % Connect the nodes
    \draw [arrow] (input) -- node[pos=0.9] {$+$} (errorsum);
    \draw [-] (errorsum) -- node {$e(t)$} (controller);
    \draw [arrow] (errorsum) -- (controller);
    \draw [arrow] (controller) -- node {$u(t)$} (plant);
    \draw [arrow] (plant) -- node [name=y] {$y(t)$} (output);
    \draw [-] (y) |- (measurements);
    \draw [arrow] (measurements) -| node[pos=0.99, right] {$-$} (errorsum);
  \end{tikzpicture}

  \caption{Control system nomenclature for a closed-loop system}
  \label{fig:system_nomenclature}

  \begin{figurekey}
    \begin{tabular}{llll}
      $r(t)$ & \gls{reference} & $u(t)$ & \gls{control input} \\
      $e(t)$ & \gls{error} & $y(t)$ & \gls{output} \\
    \end{tabular}
  \end{figurekey}
\end{bookfigure}

Note that the \gls{input} and \gls{output} of a \gls{system} are defined from
the \gls{plant}'s point of view. The negative feedback \gls{controller} shown is
driving the difference between the \gls{reference} and \gls{output}, also known
as the \gls{error}, to zero.
