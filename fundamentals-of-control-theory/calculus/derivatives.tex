\section{Derivatives}

Derivatives are expressions for the slope of a curve at arbitrary points. Common
notations for this operation on a function like $f(x)$ include
\begin{booktable}
  \begin{tabular}{|cc|}
    \hline
    \rowcolor{headingbg}
    \multicolumn{1}{|c}{\textbf{Leibniz notation}} &
      \multicolumn{1}{c|}{\textbf{Lagrange notation}} \\
    \hline
    $\frac{d}{dx} f(x)$ & $f'(x)$ \\
    $\frac{d^2}{dx^2} f(x)$ & $f''(x)$ \\
    $\frac{d^3}{dx^3} f(x)$ & $f'''(x)$ \\
    $\frac{d^4}{dx^4} f(x)$ & $f^{(4)}(x)$ \\
    $\frac{d^n}{dx^n} f(x)$ & $f^{(n)}(x)$ \\
    \hline
  \end{tabular}
  \caption{Notation for derivatives of $f(x)$}
\end{booktable}

Lagrange notation is usually voiced as ``f prime of x", ``f double-prime of x",
etc.

\subsection{Power rule}
\begin{align*}
  f(x) &= x^n \\
  f'(x) &= nx^{n - 1}
\end{align*}

\subsection{Product rule}

This is for taking the derivative of the product of two expressions.
\begin{align*}
  h(x) &= f(x)g(x) \\
  h'(x) &= f'(x)g(x) + f(x)g'(x)
\end{align*}

\subsection{Chain rule}

This is for taking the derivative of nested expressions.
\begin{align*}
  h(x) &= f(g(x)) \\
  h'(x) &= f'(g(x)) \cdot g'(x)
  \intertext{For example,}
  h(x) &= \left(3x + 2\right)^5 \\
  h'(x) &= 5\left(3x + 2\right)^4 \cdot \left(3\right) \\
  h'(x) &= 15\left(3x + 2\right)^4
\end{align*}

\subsection{Partial derivatives}
\label{subsec:partial_derivatives}

A partial derivative of a function of several variables is its derivative with
respect to one of those variables, with the others held constant (as opposed to
the total derivative, in which all variables are allowed to vary). Partial
derivatives use $\partial$ instead of $d$ in Leibniz notation.

For example, let $h(x, y) = 3xy + 2x$. For the partial derivative with respect
to $x$, $y$ is treated as a constant.
\begin{equation*}
  \frac{\partial h(x, y)}{\partial x} = 3y + 2
\end{equation*}

For the partial derivative with respect to $y$, $x$ is treated as a constant, so
the second term becomes zero.
\begin{equation*}
  \frac{\partial h(x, y)}{\partial y} = 3x
\end{equation*}
