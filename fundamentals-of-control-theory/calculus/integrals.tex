\section{Integrals}

The integral is the inverse operation of the derivative and calculates the area
under a curve. Here is an example of one based on table
\ref{tab:common_derivatives_and_integrals}.
\begin{align*}
  \int e^{at} \,dt \\
  \frac{1}{a}e^{at} + C
\end{align*}

The arbitrary constant $C$ is needed because when you take a derivative,
constants are discarded because vertical offsets don't affect the slope. When
performing the inverse operation, we don't have enough information to determine
the constant.

However, we can provide bounds for the integration.
\begin{align*}
  &\int_0^t e^{at} \,dt \\
  &\left.\left(\frac{1}{a}e^{at} + C\right)\right\vert_0^t \\
  &\left(\frac{1}{a}e^{at} + C\right) -
    \left(\frac{1}{a}e^{a \cdot 0} + C\right) \\
  &\left(\frac{1}{a}e^{at} + C\right) - \left(\frac{1}{a} + C\right) \\
  &\frac{1}{a}e^{at} + C - \frac{1}{a} - C \\
  &\frac{1}{a}e^{at} - \frac{1}{a}
\end{align*}

When we do this, the constant cancels out.

\subsection{Change of variables}
\label{subsec:calculus_change_of_vars}

Change of variables substitutes an expression with a single variable to make the
calculation more straightforward. Here's an example of integration which
utilizes it.
\begin{equation*}
  \int \cos\omega t \,dt
\end{equation*}

Let $u = \omega t$.
\begin{align*}
  du &= \omega \,dt \\
  dt &= \frac{1}{\omega} \,du
\end{align*}

Now substitute the expressions for $u$ and $dt$ in.
\begin{align*}
  &\int \cos u \,\frac{1}{\omega} \,du \\
  &\frac{1}{\omega} \int \cos u \,du \\
  &\frac{1}{\omega} \sin u + C \\
  &\frac{1}{\omega} \sin\omega t + C
\end{align*}
