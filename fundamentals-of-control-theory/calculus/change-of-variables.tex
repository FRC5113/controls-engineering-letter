\section{Change of variables}
\label{sec:calculus_change_of_variables}

Change of variables is a technique for simlifying problems in which expressions
are replaced with new variables to make the problem more tractible. This can
mean either the problem is more straightforward or it matches a common form for
which tools for finding solutions are readily available. Here's an example of
integration which utilizes it.
\begin{equation*}
  \int \cos\omega t \,dt
\end{equation*}

Let $u = \omega t$.
\begin{align*}
  du &= \omega \,dt \\
  dt &= \frac{1}{\omega} \,du
\end{align*}

Now substitute the expressions for $u$ and $dt$ in.
\begin{align*}
  &\int \cos u \,\frac{1}{\omega} \,du \\
  &\frac{1}{\omega} \int \cos u \,du \\
  &\frac{1}{\omega} \sin u + C \\
  &\frac{1}{\omega} \sin\omega t + C
\end{align*}

Another example, which will be relevant when we actually cover state-space
notation ($\dot{\mat{x}} = \mat{A}\mat{x} + \mat{B}\mat{u}$), is a closed-loop
state-space system.
\begin{align*}
  \dot{\mat{x}} &= (\mat{A} - \mat{B}\mat{K})\mat{x} + \mat{B}\mat{K}\mat{r} \\
  \dot{\mat{x}} &= \mat{A}_{cl}\mat{x} + \mat{B}_{cl}\mat{u}_{cl}
\end{align*}

where $\mat{A}_{cl} = \mat{A} - \mat{B}\mat{K}$, $\mat{B}_{cl} = \mat{B}\mat{K}$,
and $\mat{u}_{cl} = \mat{r}$. Since it matches the form of the open-loop system,
all the same analysis tools will work with it.
