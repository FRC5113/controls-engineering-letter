\section{Transfer functions in feedback}

For \glspl{controller} to \glslink{regulator}{regulate} a \gls{system} or
\glslink{tracking}{track} a reference, they must be placed in positive or
negative feedback with the \gls{plant} (whether to use positive or negative
depends on the \gls{plant} in question). Stable feedback loops attempt to make
the \gls{output} equal the \gls{reference}.

\begin{bookfigure}
  \begin{tikzpicture}[auto, >=latex']
    % Place the blocks
    \node [name=input] {$X(s)$};
    \node [sum, right=of input] (sum) {};
    \node [block, right=of sum] (K) {$K$};
    \node [block, right=of K] (G) {$G$};
    \node [right=of G] (output) {$Y(s)$};
    \node [block, below=of $(K)!0.5!(G)$] (H) {$H$};

    % Connect the nodes
    \draw [arrow] (input) -- node[pos=0.85] {$+$} (sum);
    \draw [arrow] (sum) -- node {} (K);
    \draw [arrow] (K) -- node {} (G);
    \draw [arrow] (G) -- node[name=y] {} (output);
    \draw [arrow] (y) |- (H);
    \draw [arrow] (H) -| node[pos=0.97, right] {$-$} (sum);
  \end{tikzpicture}

  \caption{Feedback controller block diagram}
  \label{fig:feedback_controller_block_diagram}

  \begin{figurekey}
    \begin{tabular}{llll}
      $X(s)$ & input & $H$ & measurement transfer function \\
      $K$ & controller gain & $Y(s)$ & output \\
      $G$ & plant transfer function & & \\
    \end{tabular}
  \end{figurekey}
\end{bookfigure}

The transfer function of figure \ref{fig:feedback_controller_block_diagram}, a
\gls{control system} diagram with feedback, from input to output is

\begin{equation}
  G_{cl}(s) = \frac{Y(s)}{X(s)} = \frac{KG}{1 + KGH}
\end{equation}

The numerator is the \gls{open-loop gain} and the denominator is one plus the
gain around the feedback loop, which may include parts of the
\gls{open-loop gain} (see appendix \ref{sec:deriv_tf_feedback} for a
derivation). As another example, the transfer function from the input to the
\gls{error} is

\begin{equation}
  G_{cl}(s) = \frac{E(s)}{X(s)} = \frac{1}{1 + KGH}
\end{equation}

The roots of the denominator of $G_{cl}(s)$ are different from those of the
open-loop transfer function $KG(s)$. These are called the closed-loop poles.

\subsection{Root locus}
\label{subsec:root_locus}
\index{Stability!root locus}

In closed-loop, the poles can be moved around by adjusting the controller gain,
but the zeroes stay put. The root locus shows where the poles will go as the
gain for a P controller is increased and tells us for what range of gains the
controller will be stable. As the controller gain is increased, poles can move
toward negative infinity (figure \ref{fig:rlocus_infty}), move toward each other
then split toward asymptotes (figure \ref{fig:rlocus_asymptotes}), or move
toward zeroes (figure \ref{fig:rlocus_zeroes}). The \gls{system} in figure
\ref{fig:rlocus_zeroes} becomes unstable as the gain is increased.

\begin{bookfigure}
  \begin{minisvg}{2}{build/code/rlocus_infty}
    \caption{Root locus showing pole moving toward negative infinity}
    \label{fig:rlocus_infty}
  \end{minisvg}
  \hfill
  \begin{minisvg}{2}{build/code/rlocus_asymptotes}
    \caption{Root locus showing poles moving toward asymptotes}
    \label{fig:rlocus_asymptotes}
  \end{minisvg}
  \hfill
  \begin{minisvg}{2}{build/code/rlocus_zeroes}
    \caption{Root locus of equation \eqref{eq:transfer_func} showing poles
      moving toward zeroes.}
    \label{fig:rlocus_zeroes}
  \end{minisvg}
\end{bookfigure}

We won't be using root locus plots for any of our control systems analysis
later, but it does help provide an intuition for what \glspl{controller}
actually do to a \gls{system}.

If poles are much farther left in the LHP than the typical \gls{system} dynamics
exhibit, they can be considered negligible. Every \gls{system} has some form of
unmodeled high frequency, nonlinear dynamics, but they can be safely ignored
depending on the operating regime.

To demonstrate this, consider the transfer function for a second-order DC
brushed motor (a CIM motor) from voltage to position

\begin{equation*}
  G(s) = \frac{K}{s((Js + b)(Ls + R) + K^2)}
\end{equation*}

where $J = 3.2284 \times 10^{-6}$ $kg$-$m^2$, $b = 3.5077 \times 10^{-6}$
$N$-$m$-$s$, $K_e = K_t = 0.0181 \,V/rad/s$, $R = 0.0902 \,\Omega$, and
$L = 230 \times 10^{-6} \,H$.

This \gls{plant} has the root locus shown in figure
\ref{fig:highfreq_unstable_rlocus}. In proportional feedback, the \gls{plant} is
unstable for large values of $K$. However, if we remove the unstable pole by
setting $L$ in the transfer function to zero, we get the root locus in figure
\ref{fig:highfreq_stable_rlocus}. For small values of $K$, both \glspl{system}
are stable and have nearly indistinguishable \glspl{step response} due to the
exceedingly small contribution from the fast pole (see figures
\ref{fig:highfreq_unstable_step} and \ref{fig:highfreq_stable_step}). The high
frequency dynamics only cause instability for large values of $K$ that induce
fast \glspl{system response}. In other words, the \glspl{system response} of the
second-order model and its first-order approximation are similar for low
frequency operating regimes.

\begin{bookfigure}
  \begin{minisvg}{2}{build/code/highfreq_unstable_rlocus}
    \caption{Root locus of second-order DC brushed motor plant}
    \label{fig:highfreq_unstable_rlocus}
  \end{minisvg}
  \hfill
  \begin{minisvg}{2}{build/code/highfreq_stable_rlocus}
    \caption{Root locus of first-order DC brushed motor plant}
    \label{fig:highfreq_stable_rlocus}
  \end{minisvg}
\end{bookfigure}

\begin{bookfigure}
  \begin{minisvg}{2}{build/code/highfreq_unstable_step}
    \caption{Step response of second-order DC brushed motor plant}
    \label{fig:highfreq_unstable_step}
  \end{minisvg}
  \hfill
  \begin{minisvg}{2}{build/code/highfreq_stable_step}
    \caption{Step response of first-order DC brushed motor plant}
    \label{fig:highfreq_stable_step}
  \end{minisvg}
\end{bookfigure}

Why can't unstable poles close to the origin be ignored in the same way? The
response of high frequency stable poles decays rapidly. Unstable poles, on the
other hand, represent unstable dynamics which cause the \gls{system}
\gls{output} to grow to infinity. Regardless of how slow these unstable dynamics
are, they will eventually dominate the response.
