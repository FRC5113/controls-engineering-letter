\chapterimage{pid-controllers.jpg}{On trail between McHenry Library and Media Theater at UCSC}

\chapter{PID controllers}
\index{PID control}

The PID controller is a commonly used feedback controller consisting of
proportional, integral, and derivative terms, hence the name. This chapter will
build up the definition of a PID controller term by term while trying to provide
intuition for how each of them behaves.

First, we'll get some nomenclature for PID controllers out of the way. The
\gls{reference} is called the \gls{setpoint} (the desired position) and the
\gls{output} is called the \gls{process variable} (the measured position). Below
are some common variable naming conventions for relevant quantities.

\begin{figurekey}
  \begin{tabular}{llll}
    $r(t)$ & \gls{setpoint} & $u(t)$ & \gls{control input} \\
    $e(t)$ & \gls{error} & $y(t)$ & \gls{output}
  \end{tabular}
\end{figurekey}

The \gls{error} $e(t)$ is $r(t) - y(t)$.

For those already familiar with PID control, this book's interpretation won't be
consistent with the classical intuition of "past", "present", and "future"
error. We will be approaching it from the viewpoint of modern control theory
with proportional controllers applied to different physical quantities we care
about. This will provide a more complete explanation of the derivative term's
behavior for constant and moving \glspl{setpoint}, and this intuition will carry
over to the modern control methods covered later in this book.

The proportional term drives the position error to zero, the derivative term
drives the velocity error to zero, and the integral term accumulates the area
between the \gls{setpoint} and \gls{output} plots over time (the integral of
position \gls{error}) and adds the current total to the \gls{control input}.
We'll go into more detail on each of these.

\renewcommand*{\chapterpath}{\partpath/pid-controllers}
\section{Proportional term}

The \textit{Proportional} term drives the position error to zero.

\begin{definition}[Proportional controller]
  \begin{equation}
    u(t) = K_p e(t)
  \end{equation}

  where $K_p$ is the proportional gain and $e(t)$ is the error at the current
  time $t$.
\end{definition}

Figure \ref{fig:p_ctrl_diag} shows a block diagram for a \gls{system}
controlled by a P controller.

\begin{bookfigure}
  \begin{tikzpicture}[auto, >=latex']
    \fontsize{9pt}{10pt}

    % Place the blocks
    \node [name=input] {$r(t)$};
    \node [sum, right=0.5cm of input] (errorsum) {};
    \node [coordinate, right=0.75cm of errorsum] (branch) {};
    \node [block, right=0.5cm of branch] (I) { $K_p e(t)$ };
    \node [coordinate, right=0.5cm of I] (ctrlsum) {};
    \node [block, right=0.75cm of ctrlsum] (plant) {Plant};
    \node [right=0.75cm of plant] (output) {};
    \node [coordinate, below=0.5cm of I] (measurements) {};

    % Connect the nodes
    \draw [arrow] (input) -- node[pos=0.9] {$+$} (errorsum);
    \draw [-] (errorsum) -- node {$e(t)$} (branch);
    \draw [arrow] (branch) -- (I);
    \draw [arrow] (I) -- node {$u(t)$} (plant);
    \draw [arrow] (plant) -- node [name=y] {$y(t)$} (output);
    \draw [-] (y) |- (measurements);
    \draw [arrow] (measurements) -| node[pos=0.99, right] {$-$} (errorsum);
  \end{tikzpicture}

  \caption{P controller block diagram}
  \label{fig:p_ctrl_diag}
\end{bookfigure}

Proportional gains act like ``software-defined springs" that pull the
\gls{system} toward the desired position. Recall from physics that we model
springs as $F = -kx$ where $F$ is the force applied, $k$ is a proportional
constant, and $x$ is the displacement from the equilibrium point. This can be
written another way as $F = k(0 - x)$ where $0$ is the equilibrium point.
If we let the equilibrium point be our feedback controller's \gls{setpoint}, the
equations have a one-to-one correspondence.

\begin{align*}
  F &= k(r - x) \\
  u(t) &= K_p e(t) = K_p(r(t) - y(t))
\end{align*}

so the ``force" with which the proportional controller pulls the \gls{system}'s
\gls{output} toward the \gls{setpoint} is proportional to the \gls{error}, just
like a spring.

\section{Derivative term}

The \textit{Derivative} term drives the velocity error to zero.

\begin{definition}[PD controller]
  \begin{equation}
    u(t) = K_p e(t) + K_d \frac{de}{dt}
  \end{equation}

  where $K_p$ is the proportional gain, $K_d$ is the derivative gain, and $e(t)$
  is the error at the current time $t$.
\end{definition}

Figure \ref{fig:pd_ctrl_diag} shows a block diagram for a \gls{system}
controlled by a PD controller.

\begin{bookfigure}
  \begin{tikzpicture}[auto, >=latex']
    \fontsize{9pt}{10pt}

    % Place the blocks
    \node [name=input] {$r(t)$};
    \node [sum, right=0.5cm of input] (errorsum) {};
    \node [coordinate, right=0.75cm of errorsum] (branch) {};
    \node [coordinate, right=1.0cm of branch] (I) {};
    \node [block, above=0.25cm of I] (P) { $K_p e(t)$ };
    \node [block, below=0.25cm of I] (D) { $K_d \frac{de(t)}{dt}$ };
    \node [sum, right=1.0cm of I] (ctrlsum) {};
    \node [block, right=0.75cm of ctrlsum] (plant) {Plant};
    \node [right=0.75cm of plant] (output) {};
    \node [coordinate, below=0.5cm of D] (measurements) {};

    % Connect the nodes
    \draw [arrow] (input) -- node[pos=0.9] {$+$} (errorsum);
    \draw [-] (errorsum) -- node {$e(t)$} (branch);
    \draw [arrow] (branch) |- (P);
    \draw [arrow] (branch) |- (D);
    \draw [arrow] (P) -| node[pos=0.95, left] {$+$} (ctrlsum);
    \draw [arrow] (D) -| node[pos=0.95, right] {$+$} (ctrlsum);
    \draw [arrow] (ctrlsum) -- node {$u(t)$} (plant);
    \draw [arrow] (plant) -- node [name=y] {$y(t)$} (output);
    \draw [-] (y) |- (measurements);
    \draw [arrow] (measurements) -| node[pos=0.99, right] {$-$} (errorsum);
  \end{tikzpicture}

  \caption{PD controller block diagram}
  \label{fig:pd_ctrl_diag}
\end{bookfigure}

A PD controller has a proportional controller for position ($K_p$) and a
proportional controller for velocity ($K_d$). The velocity \gls{setpoint} is
implicitly provided by how the position \gls{setpoint} changes over time. To
prove this, we will rearrange the equation for a PD controller.

\begin{equation*}
  u_k = K_p e_k + K_d \frac{e_k - e_{k-1}}{dt}
\end{equation*}

where $u_k$ is the \gls{control input} at timestep $k$ and $e_k$ is the
\gls{error} at timestep $k$. $e_k$ is defined as $e_k = r_k - x_k$ where $r_k$
is the \gls{setpoint} and $x_k$ is the current \gls{state} at timestep $k$.

\begin{align*}
  u_k &= K_p (r_k - x_k) + K_d \frac{(r_k - x_k) - (r_{k-1} - x_{k-1})}{dt} \\
  u_k &= K_p (r_k - x_k) + K_d \frac{r_k - x_k - r_{k-1} + x_{k-1}}{dt} \\
  u_k &= K_p (r_k - x_k) + K_d \frac{r_k - r_{k-1} - x_k + x_{k-1}}{dt} \\
  u_k &= K_p (r_k - x_k) + K_d \frac{(r_k - r_{k-1}) - (x_k - x_{k-1})}{dt} \\
  u_k &= K_p (r_k - x_k) + K_d \left(\frac{r_k - r_{k-1}}{dt} -
    \frac{x_k - x_{k-1}}{dt}\right)
\end{align*}

Notice how $\frac{r_k - r_{k-1}}{dt}$ is the velocity of the \gls{setpoint}. By
the same reasoning, $\frac{x_k - x_{k-1}}{dt}$ is the \gls{system}'s velocity at
a given timestep. That means the $K_d$ term of the PD controller is driving the
estimated velocity to the \gls{setpoint} velocity.

If the \gls{setpoint} is constant, the implicit velocity \gls{setpoint} is zero,
so the $K_d$ term slows the \gls{system} down if it's moving. This acts like a
``software-defined damper". These are commonly seen on door closers, and their
damping force increases linearly with velocity.

\section{Integral term}

The \textit{Integral} term accumulates the area between the \gls{setpoint}
and \gls{output} plots over time (i.e., the integral of position \gls{error})
and adds the current total to the \gls{control input}. Accumulating the area
between two curves is called integration.

\begin{definition}[PI controller]
  \begin{equation}
    u(t) = K_p e(t) + K_i \int_0^t e(\tau) \,d\tau
  \end{equation}

  where $K_p$ is the proportional gain, $K_i$ is the integral gain, $e(t)$ is
  the error at the current time $t$, and $\tau$ is the integration variable.

  The integral integrates from time $0$ to the current time $t$. We use $\tau$
  for the integration because we need a variable to take on multiple values
  throughout the integral, but we can't use $t$ because we already defined that
  as the current time.
\end{definition}

Figure \ref{fig:pi_ctrl_diag} shows a block diagram for a \gls{system}
controlled by a PI controller.

\begin{bookfigure}
  \begin{tikzpicture}[auto, >=latex']
    \fontsize{9pt}{10pt}

    % Place the blocks
    \node [name=input] {$r(t)$};
    \node [sum, right=0.5cm of input] (errorsum) {};
    \node [coordinate, right=0.75cm of errorsum] (branch) {};
    \node [coordinate, right=1.25cm of branch] (I) {};
    \node [block, above=0.25cm of I] (P) { $K_p e(t)$ };
    \node [block, below=0.25cm of I] (D) { $K_i \int_0^t e(\tau) \,d\tau$ };
    \node [sum, right=1.25cm of I] (ctrlsum) {};
    \node [block, right=0.75cm of ctrlsum] (plant) {Plant};
    \node [right=0.75cm of plant] (output) {};
    \node [coordinate, below=0.5cm of D] (measurements) {};

    % Connect the nodes
    \draw [arrow] (input) -- node[pos=0.9] {$+$} (errorsum);
    \draw [-] (errorsum) -- node {$e(t)$} (branch);
    \draw [arrow] (branch) |- (P);
    \draw [arrow] (branch) |- (D);
    \draw [arrow] (P) -| node[pos=0.95, left] {$+$} (ctrlsum);
    \draw [arrow] (D) -| node[pos=0.95, right] {$+$} (ctrlsum);
    \draw [arrow] (ctrlsum) -- node {$u(t)$} (plant);
    \draw [arrow] (plant) -- node [name=y] {$y(t)$} (output);
    \draw [-] (y) |- (measurements);
    \draw [arrow] (measurements) -| node[pos=0.99, right] {$-$} (errorsum);
  \end{tikzpicture}

  \caption{PI controller block diagram}
  \label{fig:pi_ctrl_diag}
\end{bookfigure}

When the \gls{system} is close to the \gls{setpoint} in steady-state, the
proportional term may be too small to pull the \gls{output} all the way to the
\gls{setpoint}, and the derivative term is zero. This can result in
\gls{steady-state error}, as shown in figure
\ref{fig:p_controller_steady-state_error}.

\begin{svg}{build/code/p_controller_ss_error}
  \caption{P controller with steady-state error}
  \label{fig:p_controller_steady-state_error}
\end{svg}

A common way of eliminating \gls{steady-state error} is to integrate the
\gls{error} and add it to the \gls{control input}. This increases the
\gls{control effort} until the \gls{system} converges. Figure
\ref{fig:p_controller_steady-state_error} shows an example of
\gls{steady-state error} for a flywheel, and figure
\ref{fig:pi_controller_steady-state_error} shows how an integrator added to the
flywheel controller eliminates it. However, too high of an integral gain can
lead to overshoot, as shown in figure
\ref{fig:pi_controller_steady-state_error_overshoot}.

\begin{bookfigure}
  \begin{minisvg}{build/code/pi_controller_ss_error}
    \caption{PI controller without steady-state error}
    \label{fig:pi_controller_steady-state_error}
  \end{minisvg}
  \hfill
  \begin{minisvg}{build/code/pi_controller_ss_error_overshoot}
    \caption{PI controller with overshoot from large $K_i$ gain}
    \label{fig:pi_controller_steady-state_error_overshoot}
  \end{minisvg}
\end{bookfigure}

There are better approaches to fix \gls{steady-state error} like using
feedforwards or constraining when the integral control acts using other
knowledge of the \gls{system}. We will discuss these in more detail when we get
to modern control theory.

\section{PID controller definition}

When these three terms are combined, one gets the typical definition for a PID
controller.

\begin{definition}[PID controller]
  \begin{equation}
    u(t) = K_p e(t) + K_i \int_0^t e(\tau) \,d\tau + K_d \frac{de}{dt}
  \end{equation}

  where $K_p$ is the proportional gain, $K_i$ is the integral gain, $K_d$ is the
  derivative gain, $e(t)$ is the error at the current time $t$, and $\tau$ is
  the integration variable.
\end{definition}

Figure \ref{fig:pid_ctrl_diag} shows a block diagram for a \gls{system}
controlled by a PID controller.

\begin{bookfigure}
  \begin{tikzpicture}[auto, >=latex']
    \fontsize{9pt}{10pt}

    % Place the blocks
    \node [name=input] {$r(t)$};
    \node [sum, right=0.5cm of input] (errorsum) {};
    \node [coordinate, right=0.75cm of errorsum] (branch) {};
    \node [block, right=0.5cm of branch] (I) { $K_i \int_0^t e(\tau) \,d\tau$ };
    \node [block, above=0.5cm of I] (P) { $K_p e(t)$ };
    \node [block, below=0.5cm of I] (D) { $K_d \frac{de(t)}{dt}$ };
    \node [sum, right=0.5cm of I] (ctrlsum) {};
    \node [block, right=0.75cm of ctrlsum] (plant) {Plant};
    \node [right=0.75cm of plant] (output) {};
    \node [coordinate, below=0.5cm of D] (measurements) {};

    % Connect the nodes
    \draw [arrow] (input) -- node[pos=0.9] {$+$} (errorsum);
    \draw [-] (errorsum) -- node {$e(t)$} (branch);
    \draw [arrow] (branch) |- (P);
    \draw [arrow] (branch) -- (I);
    \draw [arrow] (branch) |- (D);
    \draw [arrow] (P) -| node[pos=0.95, left] {$+$} (ctrlsum);
    \draw [arrow] (I) -- node[pos=0.9, below] {$+$} (ctrlsum);
    \draw [arrow] (D) -| node[pos=0.95, right] {$+$} (ctrlsum);
    \draw [arrow] (ctrlsum) -- node {$u(t)$} (plant);
    \draw [arrow] (plant) -- node [name=y] {$y(t)$} (output);
    \draw [-] (y) |- (measurements);
    \draw [arrow] (measurements) -| node[pos=0.99, right] {$-$} (errorsum);
  \end{tikzpicture}

  \caption{PID controller block diagram}
  \label{fig:pid_ctrl_diag}
\end{bookfigure}

\section{Response types}

A \gls{system} driven by a PID controller generally has three types of
responses: underdamped, overdamped, and critically damped. These are shown in
figure \ref{fig:pid_responses}.

\begin{svg}{build/code/pid_responses}
  \caption{PID controller response types}
  \label{fig:pid_responses}
\end{svg}

For the \glspl{step response} in figure \ref{fig:pid_responses}, \gls{rise time}
is the time the \gls{system} takes to initially reach the \gls{reference} after
applying the \gls{step input}. \Gls{settling time} is the time the \gls{system}
takes to settle at the \gls{reference} after the \gls{step input} is applied.

An \textit{underdamped} response oscillates around the \gls{reference} before
settling. An \textit{overdamped} response is slow to rise and does not overshoot
the \gls{reference}. A \textit{critically damped} response has the fastest
\gls{rise time} without overshooting the \gls{reference}.

\section{Manual tuning}

These steps apply to position PID controllers. Velocity PID controllers
typically don't need $K_d$.
\begin{enumerate}
  \item Set $K_p$, $K_i$, and $K_d$ to zero.
  \item Increase $K_p$ until the \gls{output} starts to oscillate around the
    \gls{setpoint}.
  \item Increase $K_d$ as much as possible without introducing jittering in the
    \gls{system response}.
\end{enumerate}

If the \gls{setpoint} follows a trapezoid profile (see section
\ref{sec:1_dof_motion_profiles}), tuning becomes a lot easier. Plot the position
\gls{setpoint}, velocity \gls{setpoint}, measured position, and measured
velocity. The velocity \gls{setpoint} can be obtained via numerical
differentiation of the position \gls{setpoint} (i.e.,
$v_{desired,k} = \frac{r_k - r_{k-1}}{\Delta t}$). Increase $K_p$ until the
position tracks well, then increase $K_d$ until the velocity tracks well.

If the \gls{controller} settles at an \gls{output} above or below the
\gls{setpoint}, one can increase $K_i$ such that the \gls{controller} reaches
the \gls{setpoint} in a reasonable amount of time. However, a steady-state
feedforward is strongly preferred over integral control (especially for velocity
PID control).
\begin{remark}
  \textit{Note:} Adding an integral gain to the \gls{controller} is an incorrect
  way to eliminate \gls{steady-state error}. A better approach would be to tune
  it with an integrator added to the \gls{plant}, but this requires a
  \gls{model}. Since we are doing output-based rather than model-based control,
  our only option is to add an integrator to the \gls{controller}.
\end{remark}

Beware that if $K_i$ is too large, integral windup can occur. Following a large
change in \gls{setpoint}, the integral term can accumulate an error larger than
the maximal \gls{control input}. As a result, the system overshoots and
continues to increase until this accumulated error is unwound.

\section{Actuator saturation}
\index{controller design!actuator saturation}

A controller calculates its output based on the error between the
\gls{reference} and the current \gls{state}. \Gls{plant} in the real world don't
have unlimited control authority available for the controller to apply. When the
actuator limits are reached, the controller acts as if the gain has been
temporarily reduced.

We'll try to explain this through a bit of math. Let's say we have a controller
$u = k(r - x)$ where $u$ is the \gls{control effort}, $k$ is the gain, $r$ is
the \gls{reference}, and $x$ is the current \gls{state}. Let $u_{max}$ be the
limit of the actuator's output which is less than the uncapped value of $u$ and
$k_{max}$ be the associated maximum gain. We will now compare the capped and
uncapped controllers for the same \gls{reference} and current \gls{state}.
\begin{align*}
  u_{max} &< u \\
  k_{max}(r - x) &< k(r - x) \\
  k_{max} &< k
\end{align*}

For the inequality to hold, $k_{max}$ must be less than the original value for
$k$. This reduced gain is evident in a \gls{system response} when there is a
linear change in state instead of an exponential one as it approaches the
\gls{reference}. This is due to the \gls{control effort} no longer following a
decaying exponential plot. Once the \gls{system} is closer to the
\gls{reference}, the controller will stop saturating and produce realistic
controller values again.

\section{Limitations}

PID's heuristic method of tuning is a reasonable choice when there is no
\textit{a priori} knowledge of the \gls{system} dynamics. However, controllers
with much better response can be developed if a \glslink{model}{dynamical model}
of the \gls{system} is known. Furthermore, PID only applies to single-input,
single-output (SISO) \glspl{system}; we'll cover methods for multiple-input,
multiple-output (MIMO) control in part \ref{part:modern_control_theory} of this
book.

