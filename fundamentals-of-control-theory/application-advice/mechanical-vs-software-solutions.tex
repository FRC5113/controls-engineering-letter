\section{Mechanical vs software solutions}

While this book focuses on controls engineering applications in FRC, controls
isn't necessary to have a successful robot; FRC is primarily a mechanical
competition, not a software competition. We spend over six weeks designing and
building a robot, but we often allocate just two days for software testing.
Despite this, teams still field competitive robots, which is a testament to how
simple and easy to use the FRC software ecosystem is nowadays. Focus on a
reliabile mechanical design first because a good mechanical design can succeed
with simple software, but the robot can't succeed with unreliable hardware.

The solution to a design problem may be a tradeoff between mechanical and
software complexity. For example, for a mechanism that only needs two positions,
a solenoid connected to a pneumatic actuator may take less effort and be more
reliable than a motor, rotary encoder, and software feedback control. If one can
get the software solution working though, the robot may not need the added space
and weight of a compressor and air tanks.

My rule of thumb for evaluating designs is to prefer elegant mechanical
solutions over comprehensive software solutions because it's easier to make
mechanical solutions reliable in competition. Well-placed sensors can also
drastically improve robot performance and reduce driver cognitive load. An
example would be a limit switch and match timer for automatically deploying
minibots in the 2011 FRC game as soon as the endgame starts. In many cases,
manual processes can be automated later and given a manual fallback if the
associated software or sensor fails. Be cautious with designs that require
closed-loop control to function.

When should problems be solved in hardware instead of software with clever
controls? Controls can handle disturbances like battery voltage drop or
measurement noise, but there are limits. For example, there's nothing software
can do to work around a drivetrain gearbox seizing up or throwing a chain.
Sometimes, you're better off just fixing the root cause in hardware.

Design robot mechanisms for controllability. FRC team 971's
seminar\footnote{\url{https://www.youtube.com/watch?v=VNfFn-gcfFI}} goes into
more detail. Two of the important takeaways from it are:
\begin{itemize}
  \item Reduce gearbox backlash
  \item Choose motors and gear reductions that provide adequate control
        authority
\end{itemize}

Remember, ``fix it in software" isn't always the answer. The remaining chapters
of this book assume you've done the engineering analysis and concluded that your
chosen design would benefit from more sophisticated controls, and you have the
time or expertise to make it work.
