\section{Integral term}

The \textit{Integral} term accumulates the area between the \gls{setpoint}
and \gls{output} plots over time (i.e., the integral of position \gls{error})
and adds the current total to the \gls{control input}. Accumulating the area
between two curves is called integration.

\begin{definition}[PI controller]
  \begin{equation}
    u(t) = K_p e(t) + K_i \int_0^t e(\tau) \,d\tau
  \end{equation}

  where $K_p$ is the proportional gain, $K_i$ is the integral gain, $e(t)$ is
  the error at the current time $t$, and $\tau$ is the integration variable.

  The integral integrates from time $0$ to the current time $t$. We use $\tau$
  for the integration because we need a variable to take on multiple values
  throughout the integral, but we can't use $t$ because we already defined that
  as the current time.
\end{definition}

Figure \ref{fig:pi_ctrl_diag} shows a block diagram for a \gls{system}
controlled by a PI controller.

\begin{bookfigure}
  \begin{tikzpicture}[auto, >=latex']
    \fontsize{9pt}{10pt}

    % Place the blocks
    \node [name=input] {$r(t)$};
    \node [sum, right=0.5cm of input] (errorsum) {};
    \node [coordinate, right=0.75cm of errorsum] (branch) {};
    \node [coordinate, right=1.25cm of branch] (I) {};
    \node [block, above=0.25cm of I] (P) { $K_p e(t)$ };
    \node [block, below=0.25cm of I] (D) { $K_i \int_0^t e(\tau) \,d\tau$ };
    \node [sum, right=1.25cm of I] (ctrlsum) {};
    \node [block, right=0.75cm of ctrlsum] (plant) {Plant};
    \node [right=0.75cm of plant] (output) {};
    \node [coordinate, below=0.5cm of D] (measurements) {};

    % Connect the nodes
    \draw [arrow] (input) -- node[pos=0.9] {$+$} (errorsum);
    \draw [-] (errorsum) -- node {$e(t)$} (branch);
    \draw [arrow] (branch) |- (P);
    \draw [arrow] (branch) |- (D);
    \draw [arrow] (P) -| node[pos=0.95, left] {$+$} (ctrlsum);
    \draw [arrow] (D) -| node[pos=0.95, right] {$+$} (ctrlsum);
    \draw [arrow] (ctrlsum) -- node {$u(t)$} (plant);
    \draw [arrow] (plant) -- node [name=y] {$y(t)$} (output);
    \draw [-] (y) |- (measurements);
    \draw [arrow] (measurements) -| node[pos=0.99, right] {$-$} (errorsum);
  \end{tikzpicture}

  \caption{PI controller block diagram}
  \label{fig:pi_ctrl_diag}
\end{bookfigure}

When the \gls{system} is close to the \gls{setpoint} in steady-state, the
proportional term may be too small to pull the \gls{output} all the way to the
\gls{setpoint}, and the derivative term is zero. This can result in
\gls{steady-state error}, as shown in figure
\ref{fig:p_controller_steady-state_error}.

\begin{svg}{build/code/p_controller_ss_error}
  \caption{P controller with steady-state error}
  \label{fig:p_controller_steady-state_error}
\end{svg}

A common way of eliminating \gls{steady-state error} is to integrate the
\gls{error} and add it to the \gls{control input}. This increases the
\gls{control effort} until the \gls{system} converges. Figure
\ref{fig:p_controller_steady-state_error} shows an example of
\gls{steady-state error} for a flywheel, and figure
\ref{fig:pi_controller_steady-state_error} shows how an integrator added to the
flywheel controller eliminates it. However, too high of an integral gain can
lead to overshoot, as shown in figure
\ref{fig:pi_controller_steady-state_error_overshoot}.

\begin{bookfigure}
  \begin{minisvg}{2}{build/code/pi_controller_ss_error}
    \caption{PI controller without steady-state error}
    \label{fig:pi_controller_steady-state_error}
  \end{minisvg}
  \hfill
  \begin{minisvg}{2}{build/code/pi_controller_ss_error_overshoot}
    \caption{PI controller with overshoot from large $K_i$ gain}
    \label{fig:pi_controller_steady-state_error_overshoot}
  \end{minisvg}
\end{bookfigure}

\begin{remark}
  There are better approaches to fix \gls{steady-state error} like using
  feedforwards or constraining when the integral control acts using other
  knowledge of the \gls{system}. We will discuss these in more detail when we
  get to modern control theory.
\end{remark}
