\chapterimage{control-system-basics.jpg}{Road near walking trail off of Rice Ranch Road in Santa Maria, CA}

\chapter{Control system basics}

Control systems are all around us and we interact with them daily. A small list
of ones you may have seen includes heaters and air conditioners with
thermostats, cruise control and the anti-lock braking system (ABS) on
automobiles, and fan speed modulation on modern laptops. \Glspl{control system}
monitor or control the behavior of \glspl{system} like these and may consist of
humans controlling them directly (manual control), or of only machines
(automatic control).

How can we prove closed-loop \glspl{controller} on an autonomous car, for
example, will behave safely and meet the desired performance specifications in
the presence of uncertainty? Control theory is an application of algebra and
geometry used to analyze and predict the behavior of \glspl{system}, make them
respond how we want them to, and make them \glslink{robustness}{robust} to
\glspl{disturbance} and uncertainty.

Controls engineering is, put simply, the engineering process applied to control
theory. As such, it's more than just applied math. While control theory has some
beautiful math behind it, controls engineering is an engineering discipline like
any other that is filled with trade-offs. The solutions control theory gives
should always be sanity checked and informed by our performance specifications.
We don't need to be perfect; we just need to be good enough to meet our
specifications (see section \ref{sec:mindset_of_an_egoless_engineer} for more on
engineering).
\begin{remark}
  Most resources for advanced engineering topics assume a level of knowledge
  well above that which is necessary. Part of the problem is the use of jargon.
  While it efficiently communicates ideas to those within the field, new people
  who aren't familiar with it are lost. Therefore, it's important to define
  terms before using them. See the glossary for a list of words and phrases
  commonly used in control theory, their origins, and their meaning. Links to
  the glossary are provided for certain words throughout the book and will use
  \textcolor{glscolor}{this color}.
\end{remark}

\renewcommand*{\chapterpath}{\partpath/control-system-basics}
\section{What is gain?}
\index{gain}

\Gls{gain} is a proportional value that shows the relationship between the
magnitude of an input signal to the magnitude of an output signal at
steady-state. Many \glspl{system} contain a method by which the gain can be
altered, providing more or less ``power" to the \gls{system}.

Figure \ref{fig:input_output_gain} shows a \gls{system} with a hypothetical
input and output. Since the output is twice the amplitude of the input, the
\gls{system} has a gain of $2$.
\begin{bookfigure}
  \begin{tikzpicture}[auto, >=latex']
    % \draw [help lines] (-4,-2) grid (4,2);

    % Input
    \drawtimeplot{-2.5cm}{0cm}{0.125cm}{0.44375cm}{0.6 * cos(40 * deg(\x))}
    \draw (-2.5,1) node {\small Input};

    \node [block] (sys) {K};
    \draw (0,1) node {\small System};

    % Output
    \drawtimeplot{2.5cm}{0cm}{0.125cm}{0.44375cm}{1.2 * cos(40* deg(\x))}
    \draw (2.5,1) node {\small Output};

    % Arrows between input/output and system
    \draw[->] (-2,0) -- (sys);
    \draw[->] (sys) -- (2,0);
  \end{tikzpicture}

  \caption{Demonstration of system with a gain of $K = 2$}
  \label{fig:input_output_gain}
\end{bookfigure}

\section{Block diagrams}
\index{block diagrams}

When designing or analyzing a \gls{control system}, it is useful to model it
graphically. Block diagrams are used for this purpose. They can be manipulated
and simplified systematically (see appendix
\ref{ch:simplifying_block_diagrams}). Figure \ref{fig:gain_nomenclature} is an
example of one.
\begin{bookfigure}
  \begin{tikzpicture}[auto, >=latex']
    % Place the blocks
    \node [name=input] {input};
    \node [sum, right=of input] (sum) {};
    \node [block, right=of sum] (P1) {open-loop};
    \node [right=of P1] (output) {output};
    \node [block, below=of P1] (P2) {feedback};

    % Connect the nodes
    \draw [arrow] (input) -- node[pos=0.85] {$+$} (sum);
    \draw [arrow] (sum) -- node {} (P1);
    \draw [arrow] (P1) -- node[name=y] {} (output);
    \draw [arrow] (y) |- (P2);
    \draw [arrow] (P2) -| node[pos=0.97, right] {$\mp$} (sum);
  \end{tikzpicture}

  \caption{Block diagram with nomenclature}
  \label{fig:gain_nomenclature}
\end{bookfigure}

The \gls{open-loop gain} is the total gain from the sum node at the input (the
circle) to the output branch. This would be the \gls{system}'s gain if the
feedback loop was disconnected. The \gls{feedback gain} is the total gain from
the output back to the input sum node. A sum node's output is the sum of its
inputs.

Figure \ref{fig:feedback_block_diagram} is a block diagram with more formal
notation in a feedback configuration.
\begin{bookfigure}
  \begin{tikzpicture}[auto, >=latex']
    % Place the blocks
    \node [name=input] {$X(s)$};
    \node [sum, right=of input] (sum) {};
    \node [block, right=of sum] (P1) {$P_1$};
    \node [right=of P1] (output) {$Y(s)$};
    \node [block, below=of P1] (P2) {$P_2$};

    % Connect the nodes
    \draw [arrow] (input) -- node[pos=0.85] {$+$} (sum);
    \draw [arrow] (sum) -- node {} (P1);
    \draw [arrow] (P1) -- node[name=y] {} (output);
    \draw [arrow] (y) |- (P2);
    \draw [arrow] (P2) -| node[pos=0.97, right] {$\mp$} (sum);
  \end{tikzpicture}

  \caption{Feedback block diagram}
  \label{fig:feedback_block_diagram}
\end{bookfigure}

$\mp$ means ``minus or plus" where a minus represents negative feedback.

\section{Open-loop and closed-loop systems}

The \gls{system} or collection of actuators being controlled by a
\gls{control system} is called the \gls{plant}. A \gls{controller} is used to
drive the plant from its current state to some desired state (the
\gls{reference}). We'll be using the following notation for relevant quantities
in block diagrams.
\begin{figurekey}
  \begin{tabular}{llll}
    $r(t)$ & \gls{reference} & $u(t)$ & \gls{control input} \\
    $e(t)$ & \gls{error} & $y(t)$ & \gls{output} \\
  \end{tabular}
\end{figurekey}

Controllers which don't include information measured from the plant's
\gls{output} are called \textit{open-loop} or \textit{feedforward} controllers.
Figure \ref{fig:open-loop_control_system} shows a plant with a feedforward
controller.
\begin{bookfigure}
  \begin{tikzpicture}[auto, >=latex']
    \fontsize{9pt}{10pt}

    % Place the blocks
    \node [name=input] {$r(t)$};
    \node [block, right=1.0cm of input, align=center] (controller)
      {Feedforward\\controller};
    \node [block, right=1.0cm of controller] (plant) {Plant};
    \node [right=1.0cm of plant] (output) {$y(t)$};
    \node [coordinate, below=1.0cm of controller] (measurements) {};

    % Connect the nodes
    \draw [arrow] (input) -- (controller);
    \draw [arrow] (controller) -- node {$u(t)$} (plant);
    \draw [arrow] (plant) -- (output);
  \end{tikzpicture}

  \caption{Open-loop control system}
  \label{fig:open-loop_control_system}
\end{bookfigure}

Controllers which incorporate information fed back from the plant's output are
called \textit{closed-loop} or \textit{feedback} controllers. Figure
\ref{fig:closed-loop_control_system} shows a plant with a feedback controller.
\begin{bookfigure}
  \begin{tikzpicture}[auto, >=latex']
    \fontsize{9pt}{10pt}

    % Place the blocks
    \node [name=input] {$r(t)$};
    \node [sum, right=0.5cm of input] (errorsum) {};
    \node [block, right=1.0cm of errorsum, align=center] (controller)
      {Feedback\\controller};
    \node [block, right=1.0cm of controller] (plant) {Plant};
    \node [right=1.0cm of plant] (output) {};
    \node [coordinate, below=1.0cm of controller] (measurements) {};

    % Connect the nodes
    \draw [arrow] (input) -- node[pos=0.9] {$+$} (errorsum);
    \draw [-] (errorsum) -- node {$e(t)$} (controller);
    \draw [arrow] (errorsum) -- (controller);
    \draw [arrow] (controller) -- node {$u(t)$} (plant);
    \draw [arrow] (plant) -- node [name=y] {$y(t)$} (output);
    \draw [-] (y) |- (measurements);
    \draw [arrow] (measurements) -| node[pos=0.99, right] {$-$} (errorsum);
  \end{tikzpicture}

  \caption{Closed-loop control system}
  \label{fig:closed-loop_control_system}
\end{bookfigure}

Note that the \gls{input} and \gls{output} of a \gls{system} are defined from
the plant's point of view. The negative feedback controller shown is driving the
difference between the \gls{reference} and \gls{output}, also known as the
\gls{error}, to zero.

Figure \ref{fig:feedforward_and_feedback_control_system} shows a plant with
feedforward and feedback controllers.
\begin{bookfigure}
  \begin{tikzpicture}[auto, >=latex']
    \fontsize{9pt}{10pt}

    % Place the blocks
    \node [name=input] {$r(t)$};
    \node [coordinate, right=0.5cm of input] (ffsplit) {};
    \node [sum, right=0.5cm of ffsplit] (errorsum) {};
    \node [block, right=1.0cm of errorsum, align=center] (feedback)
      {Feedback\\controller};
    \node [block, above=1.0cm of feedback, align=center] (feedforward)
      {Feedforward\\controller};
    \node [sum, right=1.0cm of feedback] (controllersum) {};
    \node [block, right=1.0cm of controllersum] (plant) {Plant};
    \node [right=1.0cm of plant] (output) {};
    \node [coordinate, below=1.0cm of feedback] (measurements) {};

    % Connect the nodes
    \draw [arrow] (input) -- node[pos=0.9] {$+$} (errorsum);
    \draw [arrow] (ffsplit) |- (feedforward);
    \draw [-] (errorsum) -- node {$e(t)$} (feedback);
    \draw [arrow] (errorsum) -- (feedback);
    \draw [arrow] (feedforward) -| node[pos=0.95, left] {$+$} (controllersum);
    \draw [arrow] (feedback) -- node[pos=0.9, below] {$+$} (controllersum);
    \draw [arrow] (controllersum) -- node {$u(t)$} (plant);
    \draw [arrow] (plant) -- node [name=y] {$y(t)$} (output);
    \draw [-] (y) |- (measurements);
    \draw [arrow] (measurements) -| node[pos=0.99, right] {$-$} (errorsum);
  \end{tikzpicture}

  \caption{Control system with feedforward and feedback}
  \label{fig:feedforward_and_feedback_control_system}
\end{bookfigure}

\section{Why feedback control?}

Let's say we are controlling a DC brushed motor. With just a
\glslink{model}{mathematical model} and knowledge of all current \glspl{state}
of the \gls{system} (i.e., angular velocity), we can predict all future
\glspl{state} given the future voltage \glspl{input}. Why then do we need
feedback control? If the \gls{system} is \glslink{disturbance}{disturbed} in any
way that isn't modeled by our equations, like a load was applied to the
armature, or voltage sag in the rest of the circuit caused the commanded voltage
to not match the actual applied voltage, the angular velocity of the motor will
deviate from the \gls{model} over time.

To combat this, we can take measurements of the \gls{system} and the environment
to detect this deviation and account for it. For example, we could measure the
current position and estimate an angular velocity from it. We can then give the
motor corrective commands as well as steer our \gls{model} back to reality. This
feedback allows us to account for uncertainty and be
\glslink{robustness}{robust} to it.

