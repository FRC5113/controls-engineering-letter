\chapterimage{control-system-basics.jpg}{Road near walking trail off of Rice Ranch Road in Santa Maria, CA}

\chapter{Control system basics}

Control systems are all around us and we interact with them daily. A small list
of ones you may have seen includes heaters and air conditioners with
thermostats, cruise control and the anti-lock braking system (ABS) on
automobiles, and fan speed modulation on modern laptops. \Glspl{control system}
monitor or control the behavior of \glspl{system} like these and may consist of
humans controlling them directly (manual control), or of only machines
(automatic control).

How can we prove closed-loop \glspl{controller} on an autonomous car, for
example, will behave safely and meet the desired performance specifications in
the presence of uncertainty? Control theory is an application of algebra and
geometry used to analyze and predict the behavior of \glspl{system}, make them
respond how we want them to, and make them \glslink{robustness}{robust} to
\glspl{disturbance} and uncertainty.

Controls engineering is, put simply, the engineering process applied to control
theory. As such, it's more than just applied math. While control theory has some
beautiful math behind it, controls engineering is an engineering discipline like
any other that is filled with trade-offs. The solutions control theory gives
should always be sanity checked and informed by our performance specifications.
We don't need to be perfect; we just need to be good enough to meet our
specifications.\footnote{See section \ref{sec:mindset_of_an_egoless_engineer}
for more on engineering.}
\begin{remark}
  Most resources for advanced engineering topics assume a level of knowledge
  well above that which is necessary. Part of the problem is the use of jargon.
  While it efficiently communicates ideas to those within the field, new people
  who aren't familiar with it are lost. Therefore, it's important to define
  terms before using them. See the glossary for a list of words and phrases
  commonly used in control theory, their origins, and their meaning. Links to
  the glossary are provided for certain words throughout the book and will use
  \textcolor{glscolor}{this color}.
\end{remark}

\renewcommand*{\chapterpath}{\partpath/control-system-basics}
\section{What is gain?}
\index{gain}

\Gls{gain} is a proportional value that shows the relationship between the
magnitude of an input signal to the magnitude of an output signal at
steady-state. Many \glspl{system} contain a method by which the gain can be
altered, providing more or less ``power" to the \gls{system}.

Figure \ref{fig:input_output_gain} shows a \gls{system} with a hypothetical
input and output. Since the output is twice the amplitude of the input, the
\gls{system} has a gain of $2$.
\begin{bookfigure}
  \begin{tikzpicture}[auto, >=latex']
    % \draw [help lines] (-4,-2) grid (4,2);

    % Input
    \drawtimeplot{-2.5cm}{0cm}{0.125cm}{0.44375cm}{0.6 * cos(40 * deg(\x))}
    \draw (-2.5,1) node {\small Input};

    \node [block] (sys) {K};
    \draw (0,1) node {\small System};

    % Output
    \drawtimeplot{2.5cm}{0cm}{0.125cm}{0.44375cm}{1.2 * cos(40* deg(\x))}
    \draw (2.5,1) node {\small Output};

    % Arrows between input/output and system
    \draw[->] (-2,0) -- (sys);
    \draw[->] (sys) -- (2,0);
  \end{tikzpicture}

  \caption{Demonstration of system with a gain of $K = 2$}
  \label{fig:input_output_gain}
\end{bookfigure}

\section{Block diagrams}
\index{block diagrams}

When designing or analyzing a \gls{control system}, it is useful to model it
graphically. Block diagrams are used for this purpose. They can be manipulated
and simplified systematically (see appendix
\ref{ch:simplifying_block_diagrams}). Figure \ref{fig:gain_nomenclature} is an
example of one.
\begin{bookfigure}
  \begin{tikzpicture}[auto, >=latex']
    % Place the blocks
    \node [name=input] {input};
    \node [sum, right=of input] (sum) {};
    \node [block, right=of sum] (P1) {open-loop};
    \node [right=of P1] (output) {output};
    \node [block, below=of P1] (P2) {feedback};

    % Connect the nodes
    \draw [arrow] (input) -- node[pos=0.85] {$+$} (sum);
    \draw [arrow] (sum) -- node {} (P1);
    \draw [arrow] (P1) -- node[name=y] {} (output);
    \draw [arrow] (y) |- (P2);
    \draw [arrow] (P2) -| node[pos=0.97, right] {$\mp$} (sum);
  \end{tikzpicture}

  \caption{Block diagram with nomenclature}
  \label{fig:gain_nomenclature}
\end{bookfigure}

The \gls{open-loop gain} is the total gain from the sum node at the input (the
circle) to the output branch. This would be the \gls{system}'s gain if the
feedback loop was disconnected. The \gls{feedback gain} is the total gain from
the output back to the input sum node. A sum node's output is the sum of its
inputs.

Figure \ref{fig:feedback_block_diagram} is a block diagram with more formal
notation in a feedback configuration.
\begin{bookfigure}
  \begin{tikzpicture}[auto, >=latex']
    % Place the blocks
    \node [name=input] {$X(s)$};
    \node [sum, right=of input] (sum) {};
    \node [block, right=of sum] (P1) {$P_1$};
    \node [right=of P1] (output) {$Y(s)$};
    \node [block, below=of P1] (P2) {$P_2$};

    % Connect the nodes
    \draw [arrow] (input) -- node[pos=0.85] {$+$} (sum);
    \draw [arrow] (sum) -- node {} (P1);
    \draw [arrow] (P1) -- node[name=y] {} (output);
    \draw [arrow] (y) |- (P2);
    \draw [arrow] (P2) -| node[pos=0.97, right] {$\mp$} (sum);
  \end{tikzpicture}

  \caption{Feedback block diagram}
  \label{fig:feedback_block_diagram}
\end{bookfigure}

$\mp$ means ``minus or plus" where a minus represents negative feedback.

\section{Open-loop and closed-loop systems}

The \gls{system} or collection of actuators being controlled by a
\gls{control system} is called the \gls{plant}. A \gls{controller} is used to
drive the plant from its current state to some desired state (the
\gls{reference}). We'll be using the following notation for relevant quantities
in block diagrams.
\begin{figurekey}
  \begin{tabular}{llll}
    $r(t)$ & \gls{reference} & $u(t)$ & \gls{control input} \\
    $e(t)$ & \gls{error} & $y(t)$ & \gls{output} \\
  \end{tabular}
\end{figurekey}

Controllers which don't include information measured from the plant's
\gls{output} are called \textit{open-loop} or \textit{feedforward} controllers.
Figure \ref{fig:open-loop_control_system} shows a plant with a feedforward
controller.
\begin{bookfigure}
  \begin{tikzpicture}[auto, >=latex']
    \fontsize{9pt}{10pt}

    % Place the blocks
    \node [name=input] {$r(t)$};
    \node [block, right=1.0cm of input, align=center] (controller)
      {Feedforward\\controller};
    \node [block, right=1.0cm of controller] (plant) {Plant};
    \node [right=1.0cm of plant] (output) {$y(t)$};
    \node [coordinate, below=1.0cm of controller] (measurements) {};

    % Connect the nodes
    \draw [arrow] (input) -- (controller);
    \draw [arrow] (controller) -- node {$u(t)$} (plant);
    \draw [arrow] (plant) -- (output);
  \end{tikzpicture}

  \caption{Open-loop control system}
  \label{fig:open-loop_control_system}
\end{bookfigure}

Controllers which incorporate information fed back from the plant's output are
called \textit{closed-loop} or \textit{feedback} controllers. Figure
\ref{fig:closed-loop_control_system} shows a plant with a feedback controller.
\begin{bookfigure}
  \begin{tikzpicture}[auto, >=latex']
    \fontsize{9pt}{10pt}

    % Place the blocks
    \node [name=input] {$r(t)$};
    \node [sum, right=0.5cm of input] (errorsum) {};
    \node [block, right=1.0cm of errorsum, align=center] (controller)
      {Feedback\\controller};
    \node [block, right=1.0cm of controller] (plant) {Plant};
    \node [right=1.0cm of plant] (output) {};
    \node [coordinate, below=1.0cm of controller] (measurements) {};

    % Connect the nodes
    \draw [arrow] (input) -- node[pos=0.9] {$+$} (errorsum);
    \draw [-] (errorsum) -- node {$e(t)$} (controller);
    \draw [arrow] (errorsum) -- (controller);
    \draw [arrow] (controller) -- node {$u(t)$} (plant);
    \draw [arrow] (plant) -- node [name=y] {$y(t)$} (output);
    \draw [-] (y) |- (measurements);
    \draw [arrow] (measurements) -| node[pos=0.99, right] {$-$} (errorsum);
  \end{tikzpicture}

  \caption{Closed-loop control system}
  \label{fig:closed-loop_control_system}
\end{bookfigure}

Note that the \gls{input} and \gls{output} of a \gls{system} are defined from
the plant's point of view. The negative feedback controller shown is driving the
difference between the \gls{reference} and \gls{output}, also known as the
\gls{error}, to zero.

Figure \ref{fig:feedforward_and_feedback_control_system} shows a plant with
feedforward and feedback controllers.
\begin{bookfigure}
  \begin{tikzpicture}[auto, >=latex']
    \fontsize{9pt}{10pt}

    % Place the blocks
    \node [name=input] {$r(t)$};
    \node [coordinate, right=0.5cm of input] (ffsplit) {};
    \node [sum, right=0.5cm of ffsplit] (errorsum) {};
    \node [block, right=1.0cm of errorsum, align=center] (feedback)
      {Feedback\\controller};
    \node [block, above=1.0cm of feedback, align=center] (feedforward)
      {Feedforward\\controller};
    \node [sum, right=1.0cm of feedback] (controllersum) {};
    \node [block, right=1.0cm of controllersum] (plant) {Plant};
    \node [right=1.0cm of plant] (output) {};
    \node [coordinate, below=1.0cm of feedback] (measurements) {};

    % Connect the nodes
    \draw [arrow] (input) -- node[pos=0.9] {$+$} (errorsum);
    \draw [arrow] (ffsplit) |- (feedforward);
    \draw [-] (errorsum) -- node {$e(t)$} (feedback);
    \draw [arrow] (errorsum) -- (feedback);
    \draw [arrow] (feedforward) -| node[pos=0.95, left] {$+$} (controllersum);
    \draw [arrow] (feedback) -- node[pos=0.9, below] {$+$} (controllersum);
    \draw [arrow] (controllersum) -- node {$u(t)$} (plant);
    \draw [arrow] (plant) -- node [name=y] {$y(t)$} (output);
    \draw [-] (y) |- (measurements);
    \draw [arrow] (measurements) -| node[pos=0.99, right] {$-$} (errorsum);
  \end{tikzpicture}

  \caption{Control system with feedforward and feedback}
  \label{fig:feedforward_and_feedback_control_system}
\end{bookfigure}

\section{Feedforward}

So far, we've used feedback control for \gls{reference} \gls{tracking} (making a
\gls{system}'s output follow a desired \gls{reference} signal). While this is
effective, it's a reactionary measure; the \gls{system} won't start applying
\gls{control effort} until the \gls{system} is already behind. If we could tell
the \gls{controller} about the desired movement and required input beforehand,
the \gls{system} could react quicker and the feedback \gls{controller} could do
less work. A \gls{controller} that feeds information forward into the
\gls{plant} like this is called a \gls{feedforward controller}.

A \gls{feedforward controller} injects information about the \gls{system}'s
dynamics (like a \gls{model} does) or the desired movement. The feedforward
handles parts of the control actions we already know must be applied to make a
\gls{system} track a \gls{reference}, then feedback compensates for what we do
not or cannot know about the \gls{system}'s behavior at runtime.

There are two types of feedforwards: model-based feedforward and feedforward for
unmodeled dynamics. The first solves a mathematical model of the system for the
inputs required to meet desired velocities and accelerations. The second
compensates for unmodeled forces or behaviors directly so the feedback
controller doesn't have to. Both types can facilitate simpler feedback
controllers; we'll cover examples of each.

\subsection{Plant inversion}
\label{subsec:plant_inversion}

\Gls{plant} inversion is a method of model-based feedforward for \gls{state}
feedback. It solves the \gls{plant} for the input that will make the \gls{plant}
track a desired state. This is called inversion because in a block diagram, the
inverted \gls{plant} feedforward and \gls{plant} cancel out to produce a unity
system from input to output.

While it can be an effective tool, the following should be kept in mind.
\begin{enumerate}
  \item Don't invert an unstable \gls{plant}. If the expected \gls{plant}
    doesn't match the real \gls{plant} exactly, the \gls{plant} inversion will
    still result in an unstable \gls{system}. Stabilize the \gls{plant} first
    with feedback, then inject an inversion.
  \item Don't invert a nonminimum phase system. The advice for pole-zero
    cancellation in subsection \ref{subsec:pole-zero_cancellation} applies here.
\end{enumerate}

\subsubsection{Necessary theorems}

The following theorem will be needed to derive the linear plant inversion
equation.
\begin{theorem}
  \label{thm:partial_xax}

  $\frac{\partial \mtx{x}^T\mtx{A}\mtx{x}}{\partial\mtx{x}} =
    2\mtx{A}\mtx{x}$ where $\mtx{A}$ is symmetric.
\end{theorem}

\subsubsection{Setup}

Let's start with the equation for the \gls{reference} dynamics
\begin{equation*}
  \mtx{r}_{k+1} = \mtx{A}\mtx{r}_k + \mtx{B}\mtx{u}_k
\end{equation*}

where $\mtx{u}_k$ is the feedforward input. Note that this feedforward equation
does not and should not take into account any feedback terms. We want to find
the optimal $\mtx{u}_k$ such that we minimize the \gls{tracking} error between
$\mtx{r}_{k+1}$ and $\mtx{r}_k$.
\begin{equation*}
  \mtx{r}_{k+1} - \mtx{A}\mtx{r}_k = \mtx{B}\mtx{u}_k
\end{equation*}

To solve for $\mtx{u}_k$, we need to take the inverse of the nonsquare matrix
$\mtx{B}$. This isn't possible, but we can find the pseudoinverse given some
constraints on the \gls{state} \gls{tracking} error and \gls{control effort}. To
find the optimal solution for these sorts of trade-offs, one can define a cost
function and attempt to minimize it. To do this, we'll first solve the
expression for $\mtx{0}$.
\begin{equation*}
  \mtx{0} = \mtx{B}\mtx{u}_k - (\mtx{r}_{k+1} - \mtx{A}\mtx{r}_k)
\end{equation*}

This expression will be the \gls{state} \gls{tracking} cost we use in the
following cost function as an $H_2$ norm.
\begin{equation*}
  \mtx{J} = (\mtx{B}\mtx{u}_k - (\mtx{r}_{k+1} - \mtx{A}\mtx{r}_k))^T
    (\mtx{B}\mtx{u}_k - (\mtx{r}_{k+1} - \mtx{A}\mtx{r}_k))
\end{equation*}

\subsubsection{Minimization}

Given theorem \ref{thm:partial_xax}, find the minimum of $\mtx{J}$ by taking the
partial derivative with respect to $\mtx{u}_k$ and setting the result to
$\mtx{0}$.
\begin{align*}
  \frac{\partial\mtx{J}}{\partial\mtx{u}_k} &= 2\mtx{B}^T
    (\mtx{B}\mtx{u}_k - (\mtx{r}_{k+1} - \mtx{A}\mtx{r}_k)) \\
  \mtx{0} &= 2\mtx{B}^T
    (\mtx{B}\mtx{u}_k - (\mtx{r}_{k+1} - \mtx{A}\mtx{r}_k)) \\
  \mtx{0} &= 2\mtx{B}^T\mtx{B}\mtx{u}_k -
    2\mtx{B}^T(\mtx{r}_{k+1} - \mtx{A}\mtx{r}_k) \\
  2\mtx{B}^T\mtx{B}\mtx{u}_k &=
    2\mtx{B}^T(\mtx{r}_{k+1} - \mtx{A}\mtx{r}_k) \\
  \mtx{B}^T\mtx{B}\mtx{u}_k &=
    \mtx{B}^T(\mtx{r}_{k+1} - \mtx{A}\mtx{r}_k) \\
  \mtx{u}_k &=
    (\mtx{B}^T\mtx{B})^{-1} \mtx{B}^T(\mtx{r}_{k+1} - \mtx{A}\mtx{r}_k)
\end{align*}

$(\mtx{B}^T\mtx{B})^{-1} \mtx{B}^T$ is the Moore-Penrose pseudoinverse of
$\mtx{B}$ denoted by $\mtx{B}^\dagger$.
\begin{theorem}[Linear plant inversion]
  \label{thm:linear_plant_inversion}

  Given the discrete model
  $\mtx{x}_{k+1} = \mtx{A}\mtx{x}_k + \mtx{B}\mtx{u}_k$, the plant inversion
  feedforward is
  \begin{equation}
    \mtx{u}_k = \mtx{B}^\dagger (\mtx{r}_{k+1} - \mtx{A}\mtx{r}_k)
  \end{equation}

  where $\mtx{B}^\dagger$ is the Moore-Penrose pseudoinverse of $\mtx{B}$,
  $\mtx{r}_{k+1}$ is the reference at the next timestep, and $\mtx{r}_k$ is the
  reference at the current timestep.
\end{theorem}
\index{feedforward!linear plant inversion}
\index{optimal control!linear plant inversion}

\subsubsection{Discussion}

Linear \gls{plant} inversion in theorem \ref{thm:linear_plant_inversion}
compensates for \gls{reference} dynamics that don't follow how the \gls{model}
inherently behaves. If they do follow the \gls{model}, the feedforward has
nothing to do as the \gls{model} already behaves in the desired manner. When
this occurs, $\mtx{r}_{k+1} - \mtx{A}\mtx{r}_k$ will return a zero vector.

For example, a constant \gls{reference} requires a feedforward that opposes
\gls{system} dynamics that would change the \gls{state} over time. If the
\gls{system} has no dynamics, then $\mtx{A} = \mtx{I}$ and thus
\begin{align*}
  \mtx{u}_k &= \mtx{B}_\dagger (\mtx{r}_{k+1} - \mtx{I}\mtx{r}_k) \\
  \mtx{u}_k &= \mtx{B}_\dagger (\mtx{r}_{k+1} - \mtx{r}_k)
\end{align*}

For a constant \gls{reference}, $\mtx{r}_{k+1} = \mtx{r}_k$.
\begin{align*}
  \mtx{u}_k &= \mtx{B}_\dagger (\mtx{r}_k - \mtx{r}_k) \\
  \mtx{u}_k &= \mtx{B}_\dagger (\mtx{0}) \\
  \mtx{u}_k &= \mtx{0}
\end{align*}

so no feedforward is required to hold a \gls{system} with no dynamics at a
constant \gls{reference}, as expected.

Figure \ref{fig:case_study_ff} shows \gls{plant} inversion applied to a
second-order CIM motor model. \Gls{plant} inversion accounts for the motor
back-EMF and eliminates steady-state error.
\begin{svg}{build/\chapterpath/case_study_ff}
  \caption{Second-order CIM motor response with plant inversion}
  \label{fig:case_study_ff}
\end{svg}

\subsection{Unmodeled dynamics}

In addition to \gls{plant} inversion, one can include feedforwards for unmodeled
dynamics. Consider an elevator model which doesn't include gravity. A constant
voltage offset can be used compensate for this. The feedforward takes the form
of a voltage constant because voltage is proportional to force applied, and the
force is acting in only one direction at all times.
\begin{equation}
  u_k = V_{app}
\end{equation}

where $V_{app}$ is a constant. Another feedforward holds a single-jointed arm
steady in the presence of gravity. It has the following form.
\begin{equation}
  u_k = V_{app} \cos\theta
\end{equation}

where $V_{app}$ is the voltage required to keep the single-jointed arm level
with the ground, and $\theta$ is the angle of the arm relative to the ground.
Therefore, the force applied is greatest when the arm is parallel with the
ground and zero when the arm is perpendicular to the ground (at that point, the
joint supports all the weight).

Note that the elevator model could be augmented easily enough to include gravity
and still be linear, but this wouldn't work for the single-jointed arm since a
trigonometric function is required to model the gravitational force in the arm's
rotating reference frame\footnote{While the applied torque of the motor is
constant throughout the arm's range of motion, the torque caused by gravity in
the opposite direction varies according to the arm's angle.}.

\section{Why feedback control?}

Let's say we are controlling a DC brushed motor. With just a
\glslink{model}{mathematical model} and knowledge of all current \glspl{state}
of the \gls{system} (i.e., angular velocity), we can predict all future
\glspl{state} given the future voltage \glspl{input}. Why then do we need
feedback control? If the \gls{system} is \glslink{disturbance}{disturbed} in any
way that isn't modeled by our equations, like a load was applied to the
armature, or voltage sag in the rest of the circuit caused the commanded voltage
to not match the actual applied voltage, the angular velocity of the motor will
deviate from the \gls{model} over time.

To combat this, we can take measurements of the \gls{system} and the environment
to detect this deviation and account for it. For example, we could measure the
current position and estimate an angular velocity from it. We can then give the
motor corrective commands as well as steer our \gls{model} back to reality. This
feedback allows us to account for uncertainty and be
\glslink{robustness}{robust} to it.

