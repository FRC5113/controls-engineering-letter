\section{Single-jointed arm}
\label{sec:ss_model_single-jointed_arm}

This single-jointed arm consists of a DC brushed motor attached to a pulley that
spins a straight bar in pitch.
\begin{bookfigure}
  \begin{tikzpicture}[auto, >=latex', circuit ee IEC,
                    set resistor graphic=var resistor IEC graphic]
  % \draw [help lines] (-1,-3) grid (7,4);

  % Electrical equivalent circuit
  \draw (0,2) to [voltage source={direction info'={->}, info'=$V$}] (0,0);
  \draw (0,2) to [current direction={info=$I$}] (0,3);
  \draw (0,3) -- (0.5,3);
  \draw (0.5,3) to [resistor={info={$R$}}] (2,3);

  \draw (2,3) -- (2.5,3);
  \draw (2.5,3) to [voltage source={direction info'={->}, info'=$V_{emf}$}]
    (2.5,0);
  \draw (0,0) -- (2.5,0);

  % Motor
  \begin{scope}[xshift=2.4cm,yshift=1.05cm]
    \draw[fill=black] (0,0) rectangle (0.2,0.9);
    \draw[fill=white] (0.1,0.45) ellipse (0.3 and 0.3);
  \end{scope}

  % Transmission gear one
  \begin{scope}[xshift=3.75cm,yshift=1.17cm]
    \draw[fill=black!50] (0.2,0.33) ellipse (0.08 and 0.33);
    \draw[fill=black!50, color=black!50] (0,0) rectangle (0.2,0.66);
    \draw[fill=white] (0,0.33) ellipse (0.08 and 0.33);
    \draw (0,0.66) -- (0.2,0.66);
    \draw (0,0) -- (0.2,0) node[pos=0.5,below] {$G$};
  \end{scope}

  % Output shaft of motor
  \begin{scope}[xshift=2.8cm,yshift=1.45cm]
    \draw[fill=black!50] (0,0) rectangle (0.95,0.1);
  \end{scope}

  % Angular velocity arrow of drive -> transmission
  \draw[line width=0.7pt,<-] (3.2,1) arc (-30:30:1) node[above] {$\omega_m$};

  % Transmission gear two
  \begin{scope}[xshift=3.75cm,yshift=1.83cm]
    \draw[fill=black!50] (0.2,0.68) ellipse (0.13 and 0.67);
    \draw[fill=black!50, color=black!50] (0,0) rectangle (0.2,1.35);
    \draw[fill=white] (0,0.68) ellipse (0.13 and 0.67);
    \draw (0,1.35) -- (0.2,1.35);
    \draw (0,0) -- (0.2,0);
  \end{scope}
  \begin{scope}[xshift=5.075cm,yshift=2.4cm]
    % Single-jointed arm
    \draw[fill=white] (0,0) -- (0.1,-0.05) -- (0.35,1.45) -- (0.25,1.5)
      -- cycle;
    \draw[fill=black!50] (0.1,-0.05) -- (0.3,-0.05) -- (0.55,1.45) --
      (0.35,1.45) -- cycle;
    \draw[fill=white] (0.25,1.5) -- (0.35,1.45) -- (0.55,1.45) -- (0.45,1.5)
      -- cycle;

    % Arm length arrow
    \draw[line width=0.7pt,<->] (0.55,-0.05) -- node[right] {$l$} (0.8,1.45);

    % Mass label
    \draw (-0.05,1.2) node {$m$};
  \end{scope}

  % Transmission shaft from gear two to arm
  \begin{scope}[xshift=4.09cm,yshift=2.42cm]
    \draw[fill=black!50] (0,0) rectangle (1.06,0.1);
  \end{scope}

  % Angular velocity arrow between transmission and arm
  \draw[line width=0.7pt,->] (4.54,1.97) arc (-30:30:1) node[above]
    {$\omega_{arm}$};

  % Descriptions of subsystems under graphic
  \begin{scope}[xshift=-0.5cm,yshift=-0.28cm]
    \draw[decorate,decoration={brace,amplitude=10pt}]
      (3.5,0) -- (0,0) node[midway,yshift=-20pt] {circuit};
    \draw[decorate,decoration={brace,amplitude=10pt}]
      (6.55,0) -- (3.75,0) node[midway,yshift=-20pt] {mechanics};
  \end{scope}
\end{tikzpicture}

  \caption{Single-jointed arm system diagram}
\end{bookfigure}

\subsection{Continuous state-space model}
\index{FRC models!single-jointed arm equations}

Using equation \eqref{eq:dot_omega_arm}, the angle and angular rate derivatives
of the arm can be written as
\begin{align}
  \dot{\theta}_{arm} &= \omega_{arm} \\
  \dot{\omega}_{arm} &= -\frac{G^2 K_t}{K_v RJ} \omega_{arm} + \frac{G K_t}{RJ} V
\end{align}

Factor out $\omega_{arm}$ and $V$ into column vectors.
\begin{align*}
  \dot{\begin{bmatrix}
    \omega_{arm}
  \end{bmatrix}} &=
  \begin{bmatrix}
    -\frac{G^2 K_t}{K_v RJ}
  \end{bmatrix}
  \begin{bmatrix}
    \omega_{arm}
  \end{bmatrix} +
  \begin{bmatrix}
    \frac{GK_t}{RJ}
  \end{bmatrix}
  \begin{bmatrix}
    V
  \end{bmatrix}
  \intertext{Augment the matrix equation with the angle state $\theta_{arm}$,
    which has the model equation $\dot{\theta}_{arm} = \omega_{arm}$. The matrix
    elements corresponding to $\omega_{arm}$ will be $1$, and the others will be
    $0$ since they don't appear, so
    $\dot{\theta}_{arm} = 0\theta_{arm} + 1\omega_{arm} + 0V$. The existing rows
    will have zeroes inserted where $\theta_{arm}$ is multiplied in.}
  \dot{\begin{bmatrix}
    \theta_{arm} \\
    \omega_{arm}
  \end{bmatrix}} &=
  \begin{bmatrix}
    0 & 1 \\
    0 & -\frac{G^2 K_t}{K_v RJ}
  \end{bmatrix}
  \begin{bmatrix}
    \theta_{arm} \\
    \omega_{arm}
  \end{bmatrix} +
  \begin{bmatrix}
    0 \\
    \frac{GK_t}{RJ}
  \end{bmatrix}
  \begin{bmatrix}
    V
  \end{bmatrix}
\end{align*}
\begin{theorem}[Single-jointed arm state-space model]
  \begin{align*}
    \dot{\mat{x}} &= \mat{A} \mat{x} + \mat{B} \mat{u} \\
    \mat{y} &= \mat{C} \mat{x} + \mat{D} \mat{u}
  \end{align*}
  \begin{equation*}
    \mat{x} =
    \begin{bmatrix}
      \theta_{arm} \\
      \omega_{arm}
    \end{bmatrix}
    \quad
    \mat{y} = \theta_{arm}
    \quad
    \mat{u} = V
  \end{equation*}
  \begin{align}
    \mat{A} &=
    \begin{bmatrix}
      0 & 1 \\
      0 & -\frac{G^2 K_t}{K_v RJ}
    \end{bmatrix} \\
    \mat{B} &=
    \begin{bmatrix}
      0 \\
      \frac{G K_t}{RJ}
    \end{bmatrix} \\
    \mat{C} &=
    \begin{bmatrix}
      1 & 0
    \end{bmatrix} \\
    \mat{D} &= 0
  \end{align}
\end{theorem}

\subsection{Model augmentation}

As per subsection \ref{subsec:input_error_estimation}, we will now augment the
\gls{model} so a $u_{error}$ state is added to the \gls{control input}.

The \gls{plant} and \gls{observer} augmentations should be performed before the
\gls{model} is \glslink{discretization}{discretized}. After the \gls{controller}
gain is computed with the unaugmented discrete \gls{model}, the controller may
be augmented. Therefore, the \gls{plant} and \gls{observer} augmentations assume
a continuous \gls{model} and the \gls{controller} augmentation assumes a
discrete \gls{controller}.
\begin{equation*}
  \mat{x}_{aug} =
  \begin{bmatrix}
    \mat{x} \\
    u_{error}
  \end{bmatrix}
  \quad
  \mat{y} = \theta_{arm}
  \quad
  \mat{u} = V
\end{equation*}
\begin{equation}
  \mat{A}_{aug} =
  \begin{bmatrix}
    \mat{A} & \mat{B} \\
    \mat{0}_{1 \times 2} & 0
  \end{bmatrix}
  \quad
  \mat{B}_{aug} =
  \begin{bmatrix}
    \mat{B} \\
    0
  \end{bmatrix}
  \quad
  \mat{C}_{aug} =
  \begin{bmatrix}
    \mat{C} & 0
  \end{bmatrix}
  \quad
  \mat{D}_{aug} = \mat{D}
\end{equation}
\begin{equation}
  \mat{K}_{aug} = \begin{bmatrix}
    \mat{K} & 1
  \end{bmatrix}
  \quad
  \mat{r}_{aug} = \begin{bmatrix}
    \mat{r} \\
    0
  \end{bmatrix}
\end{equation}

This will compensate for unmodeled dynamics such as gravity or other external
loading from lifted objects. However, if only gravity compensation is desired,
a feedforward of the form $u_{ff} = V_{gravity} \cos\theta$ is preferred where
$V_{gravity}$ is the voltage required to hold the arm level with the ground and
$\theta$ is the angle of the arm with the ground.

\subsection{Gravity feedforward}

Input voltage is proportional to torque and gravity is a constant force, but the
torque applied against the motor varies according to the arm's angle. We'll use
sum of torques to find a compensating torque.

We'll model gravity as a disturbance described by $-mg$ where $m$ is the arm's
mass. To compensate for it, we want to find a torque that is equal and opposite
to the torque applied to the arm by gravity. The bottom row of the continuous
elevator model contains the angular acceleration terms, so $Bu_{ff}$ is angular
acceleration caused by the motor; $JBu_{ff}$ is the torque.
\begin{align*}
  J Bu_{ff} &= -(\mat{r}\times\mat{F}) \\
  J Bu_{ff} &= -(rF\cos\theta)
\end{align*}

Torque is usually written as $rF\sin\theta$ where $\theta$ is the angle between
the $\mat{r}$ and $\mat{F}$ vectors, but $\theta$ in this case is being measured
from the horizontal axis rather than the vertical one, so the force vector is
$\frac{\pi}{4}$ radians out of phase. Thus, an angle of $0$ results in the
maximum torque from gravity being applied rather than the minimum.

The force of gravity $mg$ is applied at the center of the arm's mass. For a
uniform beam, this is halfway down its length, or $\frac{L}{2}$ where $L$ is the
length of the arm.
\begin{align*}
  J Bu_{ff} &= -\left(\left(\frac{L}{2}\right)(-mg)\cos\theta\right) \\
  J Bu_{ff} &= mg \frac{L}{2}\cos\theta
  \intertext{$B = \frac{GK_t}{RJ}$, so}
  J \frac{GK_t}{RJ} u_{ff} &= mg \frac{L}{2}\cos\theta \\
  u_{ff} &= \frac{RJ}{JGK_t} mg \frac{L}{2}\cos\theta \\
  u_{ff} &= \frac{L}{2} \frac{Rmg}{GK_t}\cos\theta
\end{align*}

$\frac{L}{2}$ can be adjusted according to the location of the arm's center of
mass.

\subsection{Simulation}

Python Control will be used to \glslink{discretization}{discretize} the
\gls{model} and simulate it. One of the frccontrol
examples\footnote{\url{https://github.com/calcmogul/frccontrol/blob/main/examples/single_jointed_arm.py}}
creates and tests a controller for it. Figure
\ref{fig:single_jointed_arm_response} shows the closed-loop \gls{system}
response.
\begin{svg}{build/\chapterpath/single_jointed_arm_response}
  \caption{Single-jointed arm response}
  \label{fig:single_jointed_arm_response}
\end{svg}

\subsection{Implementation}

C++ and Java implementations of this single-jointed arm controller are available
online.\footnote{\url{https://github.com/wpilibsuite/allwpilib/blob/main/wpilibcExamples/src/main/cpp/examples/StateSpaceArm/cpp/Robot.cpp}}
\footnote{\url{https://github.com/wpilibsuite/allwpilib/blob/main/wpilibjExamples/src/main/java/edu/wpi/first/wpilibj/examples/statespacearm/Robot.java}}
