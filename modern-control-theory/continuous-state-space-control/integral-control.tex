\section{Integral control}
\label{sec:integral_control}

A common way of implementing integral control is to add an additional
\gls{state} that is the integral of the \gls{error} of the variable intended to
have zero \gls{steady-state error}.

There are two drawbacks to this method. First, there is integral windup on a
unit \gls{step input}. That is, the integrator accumulates even if the
\gls{system} is \gls{tracking} the \gls{model} correctly. The second is
demonstrated by an example from Jared Russell of FRC team 254. Say there is a
position/velocity trajectory for some \gls{plant} to follow. Without integral
control, one can calculate a desired $\mat{K}\mat{x}$ to use as the
\gls{control input}. As a result of using both desired position and velocity,
\gls{reference} \gls{tracking} is good. With integral control added, the
\gls{reference} is always the desired position, but there is no way to tell the
controller the desired velocity.

Consider carefully whether integral control is necessary. One can get relatively
close without integral control, and integral adds all the issues listed above.
Below, it is assumed that the controls designer has determined that integral
control will be worth the inconvenience.

We'll present two methods:
\begin{enumerate}
  \item Augment the \gls{plant} as described earlier. For an arm, one would add
    an ``integral of position" state.
  \item Estimate the ``error" in the \gls{control input} (the difference between
    what was applied versus what was observed to happen) via the \gls{observer}
    and compensate for it. We'll call this ``input error estimation".
\end{enumerate}

\subsection{Plant augmentation}
\index{integral control!plant augmentation}

We want to augment the \gls{system} with an integral term that integrates the
\gls{error} $\mat{e} = \mat{r} - \mat{y} = \mat{r} - \mat{C}\mat{x}$.
\begin{align*}
  \mat{x}_I &= \int \mat{e} \,dt \\
  \dot{\mat{x}}_I &= \mat{e} = \mat{r} - \mat{C}\mat{x}
\end{align*}

The \gls{plant} is augmented as
\begin{align*}
  \dot{\begin{bmatrix}
    \mat{x} \\
    \mat{x}_I
  \end{bmatrix}} &=
  \begin{bmatrix}
    \mat{A} & \mat{0} \\
    -\mat{C} & \mat{0}
  \end{bmatrix}
  \begin{bmatrix}
    \mat{x} \\
    \mat{x}_I
  \end{bmatrix} +
  \begin{bmatrix}
    \mat{B} \\
    \mat{0}
  \end{bmatrix}
  \mat{u} +
  \begin{bmatrix}
    \mat{0} \\
    \mat{I}
  \end{bmatrix}
  \mat{r} \\
  \dot{\begin{bmatrix}
    \mat{x} \\
    \mat{x}_I
  \end{bmatrix}} &=
  \begin{bmatrix}
    \mat{A} & \mat{0} \\
    -\mat{C} & \mat{0}
  \end{bmatrix}
  \begin{bmatrix}
    \mat{x} \\
    \mat{x}_I
  \end{bmatrix} +
  \begin{bmatrix}
    \mat{B} & \mat{0} \\
    \mat{0} & \mat{I}
  \end{bmatrix}
  \begin{bmatrix}
    \mat{u} \\
    \mat{r}
  \end{bmatrix}
\end{align*}

The controller is augmented as
\begin{align*}
  \mat{u} &= \mat{K} (\mat{r} - \mat{x}) - \mat{K}_I\mat{x}_I \\
  \mat{u} &=
  \begin{bmatrix}
    \mat{K} & \mat{K}_I
  \end{bmatrix}
  \left(\begin{bmatrix}
    \mat{r} \\
    \mat{0}
  \end{bmatrix} -
  \begin{bmatrix}
    \mat{x} \\
    \mat{x}_I
  \end{bmatrix}\right)
\end{align*}

\subsection{Input error estimation}
\label{subsec:input_error_estimation}
\index{integral control!input error estimation}

Given the desired \gls{input} produced by a \gls{controller}, unmodeled
\glspl{disturbance} may cause the observed behavior of a \gls{system} to deviate
from its \gls{model}. Input error estimation uses a state observer to estimate
the difference between the desired \gls{input} and a hypothetical \gls{input}
that makes the \gls{model} match the observed behavior. This value can be added
to the \gls{control input} to make the \gls{controller} compensate for unmodeled
\glspl{disturbance} and make the \gls{model} better predict the \gls{system}'s
future behavior.

First, we'll consider the one-dimensional case. Let $u_{error}$ be the
difference between the \gls{input} actually applied to a \gls{system} and the
desired \gls{input}. The $u_{error}$ term is then added to the \gls{system} as
follows.
\begin{align*}
  \dot{x} &= Ax + B\left(u + u_{error}\right)
  \intertext{$u + u_{error}$ is the hypothetical \gls{input} actually applied to
    the \gls{system}.}
  \dot{x} &= Ax + Bu + Bu_{error}
  \intertext{The following equation generalizes this to a multiple-input
    \gls{system}.}
  \dot{\mat{x}} &= \mat{A}\mat{x} + \mat{B}\mat{u} + \mat{B}_{error}u_{error}
\end{align*}

where $\mat{B}_{error}$ is a column vector that maps $u_{error}$ to changes in
the rest of the \gls{state} the same way $\mat{B}$ does for $\mat{u}$.
$\mat{B}_{error}$ is only a column of $\mat{B}$ if $u_{error}$ corresponds to an
existing \gls{input} within $\mat{u}$.

Given the above equation, we'll augment the \gls{plant} as
\begin{align*}
  \dot{\begin{bmatrix}
    \mat{x} \\
    u_{error}
  \end{bmatrix}} &=
  \begin{bmatrix}
    \mat{A} & \mat{B}_{error} \\
    \mat{0} & \mat{0}
  \end{bmatrix}
  \begin{bmatrix}
    \mat{x} \\
    u_{error}
  \end{bmatrix} +
  \begin{bmatrix}
    \mat{B} \\
    \mat{0}
  \end{bmatrix}
  \mat{u} \\
  \mat{y} &= \begin{bmatrix}
    \mat{C} & 0
  \end{bmatrix} \begin{bmatrix}
    \mat{x} \\
    u_{error}
  \end{bmatrix} + \mat{D}\mat{u}
\end{align*}

Notice how the \gls{state} is augmented with $u_{error}$. With this \gls{model},
the \gls{observer} will estimate both the \gls{state} and the $u_{error}$ term.
The controller is augmented similarly. $\mat{r}$ is augmented with a zero for
the goal $u_{error}$ term.
\begin{align*}
  \mat{u} &= \mat{K} \left(\mat{r} - \mat{x}\right) - \mat{k}_{error}u_{error}
    \\
  \mat{u} &=
  \begin{bmatrix}
    \mat{K} & \mat{k}_{error}
  \end{bmatrix}
  \left(\begin{bmatrix}
    \mat{r} \\
    0
  \end{bmatrix} -
  \begin{bmatrix}
    \mat{x} \\
    u_{error}
  \end{bmatrix}\right)
\end{align*}

where $\mat{k}_{error}$ is a column vector with a $1$ in a given row if
$u_{error}$ should be applied to that \gls{input} or a $0$ otherwise.

This process can be repeated for an arbitrary \gls{error} which can be corrected
via some linear combination of the \glspl{input}.
