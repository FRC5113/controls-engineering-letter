\section{Case study: flywheel PID control}
\index{PID control!flywheel (modern control)}

PID controllers typically control voltage to a motor in FRC independent of the
equations of motion of that motor. For position PID control, large values of
$K_p$ can lead to overshoot and $K_d$ is commonly used to reduce overshoots.
Let's consider a flywheel controlled with a standard PID controller. Why
wouldn't $K_d$ provide damping for velocity overshoots in this case?

PID control is designed to control second-order and first-order \glspl{system}
well. It can be used to control a lot of things, but struggles when given higher
order \glspl{system}. It has three degrees of freedom. Two are used to place the
two poles of the \gls{system}, and the third is used to remove steady-state
error. With higher order \glspl{system} like a one input, seven \gls{state}
\gls{system}, there aren't enough degrees of freedom to place the \gls{system}'s
poles in desired locations. This will result in poor control.

The math for PID doesn't assume voltage, a motor, etc. It defines an output
based on derivatives and integrals of its input. We happen to use it for motors
because it actually works pretty well for it because motors are second-order
\glspl{system}.

The following math will be in continuous time, but the same ideas apply to
discrete time. This is all assuming a velocity controller.

Our simple motor model hooked up to a mass is
\begin{align}
  V &= IR + \frac{\omega}{K_v} \label{eq:steady-state_error_ss_flywheel_1} \\
  \tau &= I K_t \label{eq:steady-state_error_ss_flywheel_2} \\
  \tau &= J \frac{d\omega}{dt} \label{eq:steady-state_error_ss_flywheel_3}
\end{align}

For an explanation of where these equations come from, read section
\ref{sec:dc_brushed_motor}.

First, we'll solve for $\frac{d\omega}{dt}$ in terms of $V$.

Substitute equation \eqref{eq:steady-state_error_ss_flywheel_2} into equation
\eqref{eq:steady-state_error_ss_flywheel_1}.
\begin{align}
  V &= IR + \frac{\omega}{K_v} \nonumber \\
  V &= \left(\frac{\tau}{K_t}\right) R + \frac{\omega}{K_v} \nonumber
  \intertext{Substitute in equation
    \eqref{eq:steady-state_error_ss_flywheel_3}.}
  V &= \frac{\left(J \frac{d\omega}{dt}\right)}{K_t} R + \frac{\omega}{K_v}
    \nonumber \\
  \intertext{Solve for $\frac{d\omega}{dt}$.}
  V &= \frac{J \frac{d\omega}{dt}}{K_t} R + \frac{\omega}{K_v} \nonumber \\
  V - \frac{\omega}{K_v} &= \frac{J \frac{d\omega}{dt}}{K_t} R \nonumber \\
  \frac{d\omega}{dt} &= \frac{K_t}{JR} \left(V - \frac{\omega}{K_v}\right)
    \nonumber \\
  \underbrace{\frac{d\omega}{dt}}_{\dot{\mat{x}}} &=
    \underbrace{-\frac{K_t}{JRK_v}}_{\mat{A}} \underbrace{\omega}_{\mat{x}} +
    \underbrace{\frac{K_t}{JR}}_{\mat{B}} \underbrace{V}_{\mat{u}}
\end{align}

There's one stable open-loop pole at $-\frac{K_t}{JRK_v}$. Let's try a simple P
controller.
\begin{align*}
  \mat{u} &= \mat{K} (\mat{r} - \mat{x}) \\
  V &= K_p (\omega_{goal} - \omega)
\end{align*}

Closed-loop models have the form
$\dot{\mat{x}} = (\mat{A} - \mat{B}\mat{K})\mat{x} + \mat{B}\mat{K}\mat{r}$.
Therefore, the closed-loop poles are the eigenvalues of
$\mat{A} - \mat{B}\mat{K}$.
\begin{align*}
  \dot{\mat{x}} &= (\mat{A} - \mat{B}\mat{K})\mat{x} + \mat{B}\mat{K}\mat{r}
    \\
  \dot{\omega} &= \left(\left(-\frac{K_t}{JRK_v}\right) -
    \left(\frac{K_t}{JR}\right)(K_p)\right)\omega +
    \left(\frac{K_t}{JR}\right)(K_p)(\omega_{goal}) \\
  \dot{\omega} &= -\left(\frac{K_t}{JRK_v} + \frac{K_t K_p}{JR}\right)\omega +
    \frac{K_t K_p}{JR}\omega_{goal}
\end{align*}

This closed-loop flywheel model has one pole at
$-\left(\frac{K_t}{JRK_v} + \frac{K_t K_p}{JR}\right)$. It therefore only needs
one P controller to place that pole anywhere on the real axis. A derivative
term is unnecessary on an ideal flywheel. It may compensate for unmodeled
dynamics such as accelerating projectiles slowing the flywheel down, but that
effect may also increase recovery time; $K_d$ drives the acceleration to zero in
the undesired case of negative acceleration as well as well as the actually
desired case of positive acceleration.

This analysis assumes that the motor is well coupled to the mass and that the
time constant of the inductor is small enough that it doesn't factor into the
motor equations. The latter is a pretty good assumption, as shown by the slight
wiggle in figure \ref{fig:cs_ss_highfreq_unstable_step} compared to figure
\ref{fig:cs_ss_highfreq_stable_step}. If more mass is added to the motor
armature, the response timescales increase and the inductance matters even less.
\begin{bookfigure}
  \begin{minisvg}{2}{build/figs/highfreq_unstable_step}
    \caption{Step response of second-order DC brushed motor plant augmented with
      position ($L = 230$ μH)}
    \label{fig:cs_ss_highfreq_unstable_step}
  \end{minisvg}
  \hfill
  \begin{minisvg}{2}{build/figs/highfreq_stable_step}
    \caption{Step response of first-order DC brushed motor plant augmented with
      position}
    \label{fig:cs_ss_highfreq_stable_step}
  \end{minisvg}
\end{bookfigure}

Subsection \ref{subsec:input_error_estimation} covers a superior compensation
method that avoids zeroes in the \gls{controller}, doesn't act against the
desired control action, and facilitates better \gls{tracking}.
