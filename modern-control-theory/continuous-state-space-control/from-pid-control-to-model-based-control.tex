\section{From PID control to model-based control}
\index{PID control}

As mentioned before, controls engineers have a more general framework to
describe control theory than just PID control. PID controller designers are
focused on fiddling with controller parameters relating to the current, past,
and future \gls{error} rather than the underlying system \glspl{state}. Integral
control is a commonly used tool, and some people use integral action as the
majority of the control action. While this approach works in a lot of
situations, it is an incomplete view of the world.

Model-based control has a completely different mindset. Controls designers using
model-based control care about developing an accurate \gls{model} of the
\gls{system}, then driving the \glspl{state} they care about to zero (or to a
\gls{reference}). Integral control is added with $u_{error}$ estimation if
needed to handle \gls{model} uncertainty, but we prefer not to use it because
its response is hard to tune and some of its destabilizing dynamics aren't
visible during simulation.

Why use model-based control in FRC? Poor build season schedule management often
leads to the software team:
\begin{enumerate}
  \item Not getting enough time to verify basic functionality and test/tune
    feedback controllers.
  \item Spending dedicated software testing time troubleshooting
    mechanical/electrical issues within recently integrated subsystems instead.
\end{enumerate}

Model-based control (one of the focuses of this book) avoids both problems
because it lets software teams test basic functionality in simulation much
earlier in the build season and tune their feedback controllers automatically.
