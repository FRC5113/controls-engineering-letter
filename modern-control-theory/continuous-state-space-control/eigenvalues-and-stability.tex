\section{Eigenvalues and stability}

If a system is stable, its output will tend toward equilibrium (steady-state)
over time. For a general system $\dot{\mat{x}} = f(\mat{x}, \mat{u})$,
equilibrium points are points where $\dot{\mat{x}} = \mat{0}$. If we let
$\mat{x} = \mat{0}$ and $\mat{u} = \mat{0}$ in
$\dot{\mat{x}} = \mat{A}\mat{x} + \mat{B}\mat{u}$, we can see that
$\dot{\mat{x}} = \mat{0}$, so $\mat{x} = \mat{0}$ is an equilibrium point.

We'd like to know whether all possible unforced system trajectories
($\mat{u} = \mat{0}$) move toward or away from the equilibrium point. If we
solve the system of linear differential equations
$\dot{\mat{x}} = \mat{A}\mat{x}$, we get
$\mat{x}(t) = e^{\mat{A}t} \mat{x}_0$.\footnote{See appendix
\ref{sec:deriv_zoh_ss} for a derivation.} $e^{\mat{A}t}$ is the superposition of
$e^{\lambda_j t}$ terms where $\{\lambda_j\}$ is the set of $\mat{A}$'s
eigenvalues.\footnote{We're handwaving why this is the case, but it's a
consequence of $e^{\mat{A}t}$ being diagonalizable.}

For now, let's consider when all the eigenvalues are real numbers.
\begin{equation*}
  \begin{cases}
    \lambda_j < 0, & e^{\lambda_j t} \text{ decays to zero (stable)}
      \\
    \lambda_j = 0, & e^{\lambda_j t} = 1 \text{ (marginally stable)} \\
    \lambda_j > 0, & e^{\lambda_j t} \text{ grows to infinity (unstable)}
  \end{cases}
\end{equation*}

So the system tends toward the equilibrium point (i.e., it's stable) if
$\lambda_j < 0$ for all $j$.

Now let's consider when the eigenvalues are complex numbers. What does that mean
for the system response? Let $\lambda_j = a_j + b_j i$. Each of the exponential
terms in the solution can be written as
\begin{equation*}
  e^{\lambda_j t} = e^{(a_j + b_j i)t} = e^{a_j t} e^{i b_j t}
\end{equation*}

The complex exponential can be rewritten using Euler's formula.\footnote{Euler's
formula may seem surprising at first, but it's rooted in the fact that complex
exponentials are rotations in the complex plane around the origin. If you can
imagine walking around the unit circle traced by that rotation, you'll notice
the real part of your position oscillates between $-1$ and $1$ over time. That
is, complex exponentials manifest as oscillations in real life. Watch the ``What
is Euler's formula actually saying? | Ep. 4" video from 3Blue1Brown's
\textit{Lockdown live math} series \cite{bib:eulers_formula} for a more in-depth
explanation of Euler's formula.}
\begin{equation*}
  e^{i b_j t} = \cos(b_j t) + i \sin(b_j t)
\end{equation*}

Therefore,
\begin{equation*}
  e^{\lambda_j t} = e^{a_j t} (\cos(b_j t) + i \sin(b_j t))
\end{equation*}

When the eigenvalue's imaginary part $b_j \neq 0$, it contributes oscillation to
the real part's response.

\index{stability!poles}
The eigenvalues of $\mat{A}$ are called \textit{poles}.\footnote{This name comes
from classical control theory. See subsection \ref{subsec:poles_and_zeroes} for
more.} Figure \ref{fig:impulse_response_eig} shows the \glspl{impulse response}
in the time domain for \glspl{system} with various pole locations in the complex
plane (real numbers on the x-axis and imaginary numbers on the y-axis). Each
response has an initial condition of $1$.
\begin{bookfigure}
  \begin{tikzpicture}[auto, >=latex']
  % \draw [help lines] (-4,-2) grid (4,4);

  % Draw main axes
  \draw[->] (-4,0) -- (4,0) node[below] {\small Re($\sigma$)};
  \draw[->] (0,-2) -- (0,4) node[right] {\small Im($j\omega$)};

  % Stable: e^-1.75t * cos(1.75wt) (80/3*w for readability)
  \drawtimeplot{-2.125cm}{2.5cm}{0.125cm}{0.44375cm}{
    exp(-1.75 * \x) * cos(80/3 * 1.75 * deg(\x))}
  \drawpole{-1.75cm}{1.75cm}

  % Stable: e^-2.5t
  \drawtimeplot{-2.25cm}{0.75cm}{0.125cm}{0.125cm}{exp(-2 * \x)}
  \drawpole{-2cm}{0cm}

  % Stable: e^-t
  \drawtimeplot{-1.125cm}{-0.75cm}{0.125cm}{0.125cm}{exp(-\x)}
  \drawpole{-1cm}{0cm}

  % Marginally stable: cos(wt) (80/3*w for readability)
  \drawtimeplot{-0.75cm}{1.125cm}{0.125cm}{0.44375cm}{cos(80/3 * deg(\x))}
  \drawpole{0cm}{1cm}

  % Marginally stable cos(2wt) (80/3*w for readability)
  \drawtimeplot{0cm}{2.75cm}{0.125cm}{0.44375cm}{cos(80/3 * 2 * deg(\x))}
  \drawpole{0cm}{2cm}

  % Integrator
  \drawtimeplot{0.25cm}{-0.75cm}{0.125cm}{0.125cm}{1}
  \drawpole{0cm}{0cm}

  % Unstable: e^t
  \drawtimeplot{1.125cm}{0.75cm}{0.125cm}{0.125cm}{exp(\x)}
  \drawpole{1cm}{0cm}

  % Unstable: e^2t
  \drawtimeplot{2.25cm}{-0.75cm}{0.125cm}{0.125cm}{exp(2 * \x)}
  \drawpole{2cm}{0cm}

  % Unstable: e^0.75t * cos(1.75wt) (80/3*w for readability)
  \drawtimeplot{1.5cm}{2.25cm}{0.125cm}{0.44375cm}{
    exp(0.75 * \x) * cos(80/3 * 1.75 * deg(\x))}
  \drawpole{0.75cm}{1.75cm}

  % LHP and RHP labels
  \draw (-3.5,1.5) node {LHP};
  \draw (3.5,1.5) node {RHP};

  % Stable and unstable labels
  \draw (-2.5,3.5) node {\small Stable};
  \draw (2.5,3.5) node {\small Unstable};
\end{tikzpicture}

  \caption{Impulse response vs pole location}
  \label{fig:impulse_response_eig}
\end{bookfigure}

Poles in the left half-plane (LHP) are stable; the \gls{system}'s output may
oscillate but it converges to steady-state. Poles on the imaginary axis are
marginally stable; the \gls{system}'s output oscillates at a constant amplitude
forever. Poles in the right half-plane (RHP) are unstable; the \gls{system}'s
output grows without bound.
\begin{remark}
  Imaginary poles always come in complex conjugate pairs (e.g., $-2 + 3i$,
  $-2 - 3i$).
\end{remark}
