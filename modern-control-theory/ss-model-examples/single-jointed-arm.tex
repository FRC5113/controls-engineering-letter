\section{Single-jointed arm}
\label{sec:ss_model_single-jointed_arm}

\subsection{Continuous state-space model}
\index{FRC models!single-jointed arm equations}

The angle and angular rate derivatives of the arm can be written as

\begin{align}
  \dot{\theta}_{arm} &= \omega_{arm} \label{eq:arm_cont_ss_pos} \\
  \dot{\omega}_{arm} &= \dot{\omega}_{arm} \label{eq:arm_cont_ss_vel}
\end{align}

By equation \eqref{eq:dot_omega_arm}

\begin{equation*}
  \dot{\omega}_{arm} = -\frac{G^2 K_t}{K_v RJ} \omega_{arm} + \frac{G K_t}{RJ} V
\end{equation*}

Factor out $\omega_{arm}$ and $V$ into column vectors.

\begin{align*}
  \dot{\begin{bmatrix}
    \omega_{arm}
  \end{bmatrix}} &=
  \begin{bmatrix}
    -\frac{G^2 K_t}{K_v RJ}
  \end{bmatrix}
  \begin{bmatrix}
    \omega_{arm}
  \end{bmatrix} +
  \begin{bmatrix}
    \frac{GK_t}{RJ}
  \end{bmatrix}
  \begin{bmatrix}
    V
  \end{bmatrix}
\end{align*}

Augment the matrix equation with the angle state $\theta_{arm}$, which has the
model equation $\dot{\theta}_{arm} = \omega_{arm}$. The matrix elements
corresponding to $\omega_{arm}$ will be $1$, and the others will be $0$ since
they don't appear, so $\dot{\theta}_{arm} = 0\theta_{arm} + 1\omega_{arm} + 0V$.
The existing rows will have zeroes inserted where $\theta_{arm}$ is multiplied
in.

\begin{align*}
  \dot{\begin{bmatrix}
    \theta_{arm} \\
    \omega_{arm}
  \end{bmatrix}} &=
  \begin{bmatrix}
    0 & 1 \\
    0 & -\frac{G^2 K_t}{K_v RJ}
  \end{bmatrix}
  \begin{bmatrix}
    \theta_{arm} \\
    \omega_{arm}
  \end{bmatrix} +
  \begin{bmatrix}
    0 \\
    \frac{GK_t}{RJ}
  \end{bmatrix}
  \begin{bmatrix}
    V
  \end{bmatrix}
\end{align*}

\begin{theorem}[Single-jointed arm state-space model]
  \begin{align*}
    \dot{\mtx{x}} &= \mtx{A} \mtx{x} + \mtx{B} \mtx{u} \\
    \mtx{y} &= \mtx{C} \mtx{x} + \mtx{D} \mtx{u}
  \end{align*}
  \begin{equation*}
    \begin{array}{ccc}
      \mtx{x} =
      \begin{bmatrix}
        \theta_{arm} \\
        \omega_{arm}
      \end{bmatrix} &
      \mtx{y} = \theta_{arm} &
      \mtx{u} = V
    \end{array}
  \end{equation*}
  \begin{equation}
    \begin{array}{cccc}
      \mtx{A} =
      \begin{bmatrix}
        0 & 1 \\
        0 & -\frac{G^2 K_t}{K_v RJ}
      \end{bmatrix} &
      \mtx{B} =
      \begin{bmatrix}
        0 \\
        \frac{G K_t}{RJ}
      \end{bmatrix} &
      \mtx{C} =
      \begin{bmatrix}
        1 & 0
      \end{bmatrix} &
      \mtx{D} = 0
    \end{array}
  \end{equation}
\end{theorem}

\subsection{Model augmentation}

As per subsection \ref{subsec:u_error_estimation}, we will now augment the
\gls{model} so a $u_{error}$ term is added to the \gls{control input}.

The \gls{plant} and \gls{observer} augmentations should be performed before the
\gls{model} is \glslink{discretization}{discretized}. After the \gls{controller}
gain is computed with the unaugmented discrete \gls{model}, the controller may
be augmented. Therefore, the \gls{plant} and \gls{observer} augmentations assume
a continuous \gls{model} and the \gls{controller} augmentation assumes a
discrete \gls{controller}.

\begin{equation*}
  \begin{array}{ccc}
    \mtx{x}_{aug} =
    \begin{bmatrix}
      \mtx{x} \\
      u_{error}
    \end{bmatrix} &
    \mtx{y} = \theta_{arm} &
    \mtx{u} = V
  \end{array}
\end{equation*}

\begin{equation}
  \begin{array}{cccc}
    \mtx{A}_{aug} =
    \begin{bmatrix}
      \mtx{A} & \mtx{B} \\
      \mtx{0}_{1 \times 2} & 0
    \end{bmatrix} &
    \mtx{B}_{aug} =
    \begin{bmatrix}
      \mtx{B} \\
      0
    \end{bmatrix} &
    \mtx{C}_{aug} =
    \begin{bmatrix}
      \mtx{C} & 0
    \end{bmatrix} &
    \mtx{D}_{aug} = \mtx{D}
  \end{array}
\end{equation}

\begin{equation}
  \begin{array}{cc}
    \mtx{K}_{aug} = \begin{bmatrix}
      \mtx{K} & 1
    \end{bmatrix} &
    \mtx{r}_{aug} = \begin{bmatrix}
      \mtx{r} \\
      0
    \end{bmatrix}
  \end{array}
\end{equation}

This will compensate for unmodeled dynamics such as gravity or other external
loading from lifted objects. However, if only gravity compensation is desired,
a feedforward of the form $u_{ff} = V_{gravity} \cos\theta$ is preferred where
$V_{gravity}$ is the voltage required to hold the arm level with the ground and
$\theta$ is the angle of the arm with the ground.

\subsection{Gravity feedforward}

Input voltage is proportional to torque and gravity is a constant force, but the
torque applied against the motor varies according to the arm's angle. We'll use
sum of torques to find a compensating torque.

We'll model gravity as a disturbance described by $-mg$ where $m$ is the arm's
mass. To compensate for it, we want to find a torque that is equal and opposite
to the torque applied to the arm by gravity. The bottom row of the continuous
elevator model contains the angular acceleration terms, so $Bu_{ff}$ is angular
acceleration caused by the motor; $JBu_{ff}$ is the torque.

\begin{align*}
  J Bu_{ff} &= -(\mtx{r}\times\mtx{F}) \\
  J Bu_{ff} &= -(rF\cos\theta)
\end{align*}

Torque is usually written as $rF\sin\theta$ where $\theta$ is the angle between
the $\mtx{r}$ and $\mtx{F}$ vectors, but $\theta$ in this case is being measured
from the horizontal axis rather than the vertical one, so the force vector is
$\frac{\pi}{4}$ radians out of phase. Thus, an angle of $0$ results in the
maximum torque from gravity being applied rather than the minimum.

The force of gravity $mg$ is applied at the center of the arm's mass. For a
uniform beam, this is halfway down its length, or $\frac{L}{2}$ where $L$ is the
length of the arm.

\begin{align*}
  J Bu_{ff} &= -\left(\left(\frac{L}{2}\right)(-mg)\cos\theta\right) \\
  J Bu_{ff} &= mg \frac{L}{2}\cos\theta
\end{align*}

$B = \frac{GK_t}{RJ}$, so

\begin{align*}
  J \frac{GK_t}{RJ} u_{ff} &= mg \frac{L}{2}\cos\theta \\
  u_{ff} &= \frac{RJ}{JGK_t} mg \frac{L}{2}\cos\theta \\
  u_{ff} &= \frac{L}{2} \frac{Rmg}{GK_t}\cos\theta
\end{align*}

$\frac{L}{2}$ can be adjusted according to the location of the arm's center of
mass.

\subsection{Simulation}

Python Control will be used to \glslink{discretization}{discretize} the
\gls{model} and simulate it. One of the frccontrol
examples\footnote{\url{https://github.com/calcmogul/frccontrol/blob/master/examples/single_jointed_arm.py}}
creates and tests a controller for it.

\begin{remark}
  Python Control currently doesn't support finding the transmission zeroes of
  MIMO \glspl{system} with a different number of \glspl{input} than
  \glspl{output}, so \texttt{control.zero()} and
  \texttt{frccontrol.System.plot\_pzmaps()} fail with an error if Slycot isn't
  installed.
\end{remark}

Figures \ref{fig:single_jointed_arm_pzmap_open-loop} through
\ref{fig:single_jointed_arm_pzmap_observer} show the pole-zero maps for the
open-loop \gls{system}, closed-loop \gls{system}, and \gls{observer}. Figure
\ref{fig:single_jointed_arm_response} shows the \gls{system} response with them.

\begin{bookfigure}
  \begin{minisvg}{3}{build/frccontrol/examples/single_jointed_arm_pzmap_open-loop}
    \caption{Single-jointed arm open-loop pole-zero map}
    \label{fig:single_jointed_arm_pzmap_open-loop}
  \end{minisvg}
  \hfill
  \begin{minisvg}{3}{build/frccontrol/examples/single_jointed_arm_pzmap_closed-loop}
    \caption{Single-jointed arm closed-loop pole-zero map}
    \label{fig:single_jointed_arm_pzmap_closed-loop}
  \end{minisvg}
  \hfill
  \begin{minisvg}{3}{build/frccontrol/examples/single_jointed_arm_pzmap_observer}
    \caption{Single-jointed arm observer pole-zero map}
    \label{fig:single_jointed_arm_pzmap_observer}
  \end{minisvg}
\end{bookfigure}

\begin{svg}{build/frccontrol/examples/single_jointed_arm_response}
  \caption{Single-jointed arm response}
  \label{fig:single_jointed_arm_response}
\end{svg}

\subsection{Implementation}

The script linked above also generates two files: SingleJointedArmCoeffs.h and
SingleJointedArmCoeffs.cpp. These can be used with the WPILib StateSpacePlant,
StateSpaceController, and StateSpaceObserver classes in C++ and Java. A C++
implementation of this single-jointed arm controller is available
online\footnote{\url{https://github.com/calcmogul/allwpilib/tree/state-space/wpilibcExamples/src/main/cpp/examples/StateSpaceSingleJointedArm}}.
