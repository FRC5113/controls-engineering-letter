\section{Pendulum}

\subsection{Write model in state-space representation}

Below is the \gls{model} for a pendulum

\begin{equation*}
  \ddot{\theta} = -\frac{g}{l}\sin\theta
\end{equation*}

where $\theta$ is the angle of the pendulum and $l$ is the length of the
pendulum.

Since state-space representation requires that only single derivatives be used,
they should be broken up as separate \glspl{state}. We'll reassign
$\dot{\theta}$ to be $\omega$ so the derivatives are easier to keep straight for
state-space representation.

\begin{equation*}
  \dot{\omega} = -\frac{g}{l}\sin\theta
\end{equation*}

Now separate the \glspl{state}.

\begin{align*}
  \dot{\theta} &= \omega \\
  \dot{\omega} &= -\frac{g}{l} \sin\theta
\end{align*}

Since this \gls{model} is nonlinear, we should
\glslink{linearization}{linearize} \index{Nonlinear control!linearization} it.
We will use the small angle approximation ($\sin\theta = \theta$ for small
values of $\theta$).

\begin{align*}
  \dot{\theta} &= \omega \\
  \dot{\omega} &= -\frac{g}{l} \theta
\end{align*}

Now write the \gls{model} in state-space representation. We'll write out the
system of equations with the zeroed variables included to assist with this.

\begin{align*}
  \dot{\theta} &= \;\;\;\,0 \theta + 1 \omega \\
  \dot{\omega} &= -\frac{g}{l} \theta + 0 \omega
\end{align*}

Factor out $\theta$ and $\omega$ into a column vector.

\begin{align}
  \dot{
  \begin{bmatrix}
    \theta \\
    \omega
  \end{bmatrix}} =
  \begin{bmatrix}
    0 & 1 \\
    -\frac{g}{l} & 0
  \end{bmatrix}
  \begin{bmatrix}
    \theta \\
    \omega
  \end{bmatrix}
\end{align}

\subsection{Add estimator for unmeasured states}

For full \gls{state} feedback, knowledge of all \glspl{state} is required. If
not all \glspl{state} are measured directly, an estimator can be used to
supplement them.

For example, we may only be measuring $\theta$ in the pendulum example, not
$\dot{\theta}$, so we'll need to estimate the latter. The $\mtx{C}$ matrix the
\gls{observer} would use in this case is

\begin{equation*}
  \mtx{C} = \begin{bmatrix}
    1 & 0 \\
  \end{bmatrix}
\end{equation*}

Therefore, the measurement vector is

\begin{align*}
  \mtx{y} &= \mtx{C}\mtx{x} \\
  \mtx{y} &= \begin{bmatrix}
    1 & 0
  \end{bmatrix}
  \begin{bmatrix}
    \theta \\
    \omega
  \end{bmatrix} \\
  \mtx{y} &= 1\theta + 0\omega \\
  \mtx{y} &= \theta
\end{align*}

\subsection{Implement controller}

Use Bryson's rule when making the performance vs \gls{control effort} trade-off.
Optimizing for performance will get you to the \gls{reference} as fast as
possible while optimizing \gls{control effort} will get you to the
\gls{reference} in the most ``fuel-efficient" way possible. The latter, for
example, would potentially avoid voltage drops from motor usage on robots with a
limited power supply, but the result would be slower to reach the
\gls{reference}.

\subsection{Simulate model/controller}

This can be done in any platform supporting numerical computation. Common
choices are MATLAB, v-REP, or Python. Tweak the LQR gains as necessary.

If you're comfortable enough with it, you can use the controller designed by LQR
as a starting point and tweak the pole locations after that with pole placement
to produce the desired response.

\subsubsection{Simulating a closed-loop system}

Recall equation \eqref{eq:s_ref_ctrl_x} where a closed-loop system is written as
$\dot{\mtx{x}} = (\mtx{A} - \mtx{B}\mtx{K})\mtx{x} + \mtx{B}\mtx{K}\mtx{r}$. In
the open-loop \gls{system}, our \gls{control input} $\mtx{u}$ was a vector of
\gls{system} \glspl{input}. In the closed-loop \gls{system}, our
\gls{control input} $\mtx{u}$ is now a vector of \gls{reference} \glspl{state}
$\mtx{r}$. To use this form in simulation, the corresponding state-space
matrices, which we'll denote with apostrophes, would be

\begin{align*}
  \mtx{A}' &= \mtx{A} - \mtx{B}\mtx{K} \\
  \mtx{B}' &= \mtx{B}\mtx{K} \\
  \mtx{C}' &= \mtx{C} - \mtx{D}\mtx{K} \\
  \mtx{D}' &= \mtx{D}\mtx{K}
\end{align*}

\subsection{Verify pole locations}

Check the pole locations as a sanity check and to potentially gain an intuition
for the chosen pole locations.

\subsection{Unit test}

Write unit tests to test the \gls{model} performance and \gls{robustness} with
different initial conditions and \glspl{reference}. For C++, we recommend Google
Test.

\subsection{Test on real system}

Try the controller on a real \gls{system} with low maximum \glspl{control input}
for safety. The \glspl{control input} can be increased after verifying the
sensors function and mechanisms move the correct direction.
