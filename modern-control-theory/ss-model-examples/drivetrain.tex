\section{Drivetrain}

\subsection{Continuous state-space model}
\index{FRC models!drivetrain equations}

The position and velocity of each drivetrain side can be written as

\begin{align}
  \dot{x}_l &= v_l \label{eq:drivetrain_cont_ss_posl} \\
  \dot{v}_l &= \dot{v}_l \label{eq:drivetrain_cont_ss_vell} \\
  \dot{x}_r &= v_r \label{eq:drivetrain_cont_ss_posr} \\
  \dot{v}_r &= \dot{v}_r \label{eq:drivetrain_cont_ss_velr}
\end{align}

By equations (\ref{eq:drivetrain_model_right}) and
(\ref{eq:drivetrain_model_left})

\begin{align*}
  \dot{v}_l &= \left(\frac{1}{m} + \frac{r_b^2}{J}\right)
    \left(C_1 v_l + C_2 V_l\right) +
    \left(\frac{1}{m} - \frac{r_b^2}{J}\right) \left(C_3 v_r + C_4 V_r\right) \\
  \dot{v}_r &= \left(\frac{1}{m} - \frac{r_b^2}{J}\right)
    \left(C_1 v_l + C_2 V_l\right) +
    \left(\frac{1}{m} + \frac{r_b^2}{J}\right) \left(C_3 v_r + C_4 V_r\right)
\end{align*}

\begin{theorem}[Drivetrain state-space model]
  \begin{align*}
    \dot{\mtx{x}} &= \mtx{A} \mtx{x} + \mtx{B} \mtx{u} \\
    \mtx{y} &= \mtx{C} \mtx{x} + \mtx{D} \mtx{u}
  \end{align*}
  \begin{equation*}
    \begin{array}{ccc}
      \mtx{x} =
      \begin{bmatrix}
        x_l \\
        v_l \\
        x_r \\
        v_r
      \end{bmatrix} &
      \mtx{y} =
      \begin{bmatrix}
        x_l \\
        x_r
      \end{bmatrix} &
      \mtx{u} =
      \begin{bmatrix}
        V_l \\
        V_r
      \end{bmatrix}
    \end{array}
  \end{equation*}
  \begin{equation}
    \label{eq:drivetrain_ss_model}
    \begin{array}{ll}
      \mtx{A} =
      \begin{bmatrix}
        0 & 1 & 0 & 0 \\
        0 & \left(\frac{1}{m} + \frac{r_b^2}{J}\right) C_1 & 0 & \left(\frac{1}{m} - \frac{r_b^2}{J}\right) C_3 \\
        0 & 0 & 0 & 1 \\
        0 & \left(\frac{1}{m} - \frac{r_b^2}{J}\right) C_1 & 0 & \left(\frac{1}{m} + \frac{r_b^2}{J}\right) C_3
      \end{bmatrix} &
      \mtx{B} =
      \begin{bmatrix}
        0 & 0 \\
        \left(\frac{1}{m} + \frac{r_b^2}{J}\right) C_2 & \left(\frac{1}{m} - \frac{r_b^2}{J}\right) C_4 \\
        0 & 0 \\
        \left(\frac{1}{m} - \frac{r_b^2}{J}\right) C_2 & \left(\frac{1}{m} + \frac{r_b^2}{J}\right) C_4
      \end{bmatrix} \\
      \mtx{C} =
      \begin{bmatrix}
        1 & 0 & 0 & 0 \\
        0 & 0 & 1 & 0 \\
      \end{bmatrix} &
      \mtx{D} = \mtx{0}_{2 \times 2}
    \end{array}
  \end{equation}

  where $C_1 = -\frac{G_l^2 K_t}{K_v R r^2}$, $C_2 = \frac{G_l K_t}{Rr}$,
  $C_3 = -\frac{G_r^2 K_t}{K_v R r^2}$, and $C_4 = \frac{G_r K_t}{Rr}$.
\end{theorem}

\subsection{Model augmentation}

As per subsection \ref{subsec:u_error_estimation}, we will now augment the
\gls{model} so $u_{error}$ terms are added to the \glspl{control input}.

The \gls{plant} and \gls{observer} augmentations should be performed before the
\gls{model} is \glslink{discretization}{discretized}. After the \gls{controller}
gain is computed with the unaugmented discrete \gls{model}, the controller may
be augmented. Therefore, the \gls{plant} and \gls{observer} augmentations assume
a continuous \gls{model} and the \gls{controller} augmentation assumes a
discrete \gls{controller}.

For this augmented \gls{model}, the left and right wheel positions are filtered
encoder positions and are not adjusted for heading error. The turning velocity
computed from the left and right velocities is adjusted by the gyroscope angular
velocity. The angular velocity $u_{error}$ term is the angular velocity error
between the wheel speeds and the gyroscope measurement.

\begin{equation*}
  \begin{array}{ccc}
    \mtx{x}_{aug} =
    \begin{bmatrix}
      \mtx{x} \\
      u_{error,l} \\
      u_{error,r} \\
      u_{error,angle}
    \end{bmatrix} &
    \mtx{y}_{aug} =
    \begin{bmatrix}
      \mtx{y} \\
      \omega
    \end{bmatrix} &
    \mtx{u} =
    \begin{bmatrix}
      V_l \\
      V_r
    \end{bmatrix}
  \end{array}
\end{equation*}

where $\omega$ is the angular rate of the robot's center of mass measured by a
gyroscope.

\begin{equation*}
  \begin{array}{cc}
    \mtx{B}_{\omega} =
    \begin{bmatrix}
      1 \\
      0 \\
      -1 \\
      0
    \end{bmatrix} &
    \mtx{C}_{\omega} =
    \begin{bmatrix}
      0 & -\frac{1}{2r_b} & 0 & \frac{1}{2r_b}
    \end{bmatrix}
  \end{array}
\end{equation*}

\begin{equation}
  \begin{array}{ll}
    \mtx{A}_{aug} =
    \begin{bmatrix}
      \mtx{A} & \mtx{B} & \mtx{B}_{\omega} \\
      \mtx{0}_{3 \times 4} & \mtx{0}_{3 \times 2} & \mtx{0}_{3 \times 1}
    \end{bmatrix} &
    \mtx{B}_{aug} =
    \begin{bmatrix}
      \mtx{B} \\
      \mtx{0}_{3 \times 2}
    \end{bmatrix} \\
    \mtx{C}_{aug} =
    \begin{bmatrix}
      \mtx{C} & \mtx{0}_{2 \times 3} \\
      \mtx{C}_{\omega} & \mtx{0}_{1 \times 3}
    \end{bmatrix} &
    \mtx{D}_{aug} =
    \begin{bmatrix}
      \mtx{D} \\
      \mtx{0}_{1 \times 2}
    \end{bmatrix}
  \end{array}
\end{equation}

The augmentation of $\mtx{A}$ with $\mtx{B}$ maps the left and right wheel
velocity $u_{error}$ terms to their respective states. The augmentation of
$\mtx{A}$ with $\mtx{B}_{\omega}$ maps the angular velocity $u_{error}$ term to
corresponding changes in the left and right wheel positions.

$\mtx{C}_{\omega}$ maps the left and right wheel velocities to an angular
velocity estimate that the \gls{observer} can compare against the gyroscope
measurement.

\begin{equation}
  \begin{array}{ccc}
    \mtx{K}_{error} =
    \begin{bmatrix}
      1 & 0 & 0 \\
      0 & 1 & 0
    \end{bmatrix} &
    \mtx{K}_{aug} = \begin{bmatrix}
      \mtx{K} & \mtx{K}_{error}
    \end{bmatrix} &
    \mtx{r}_{aug} = \begin{bmatrix}
      \mtx{r} \\
      0 \\
      0 \\
      0
    \end{bmatrix}
  \end{array}
\end{equation}

This will compensate for unmodeled dynamics such as understeer due to the wheel
friction inherent in skid-steer robots.

\begin{remark}
  The process noise for the angular velocity $u_{error}$ term should be the
  encoder \gls{model} uncertainty. The augmented measurement noise term is
  obviously for the gyroscope measurement.
\end{remark}

\subsection{Simulation}

Python Control will be used to \glslink{discretization}{discretize} the
\gls{model} and simulate it. One of the frccontrol
examples\footnote{\url{https://github.com/calcmogul/frccontrol/blob/master/examples/drivetrain.py}}
creates and tests a controller for it.

\begin{remark}
  Python Control currently doesn't support finding the transmission zeroes of
  MIMO \glspl{system} with a different number of \glspl{input} than
  \glspl{output}, so \texttt{control.pzmap()} and
  \texttt{frccontrol.System.plot\_pzmaps()} fail with an error if Slycot isn't
  installed.
\end{remark}

Figure \ref{fig:drivetrain_pzmaps} shows the pole-zero maps for the open-loop
\gls{system}, closed-loop \gls{system}, and \gls{observer}. Figure
\ref{fig:drivetrain_response} shows the \gls{system} response with them.

\begin{svg}{build/frccontrol/examples/drivetrain_pzmaps}
  \caption{Drivetrain pole-zero maps}
  \label{fig:drivetrain_pzmaps}
\end{svg}

\begin{svg}{build/frccontrol/examples/drivetrain_response}
  \caption{Drivetrain response}
  \label{fig:drivetrain_response}
\end{svg}

Given the high inertia in drivetrains, it's better to drive the \gls{reference}
with a motion profile instead of a \gls{step input} for reproducibility.

\subsection{Implementation}

The script linked above also generates two files: DrivetrainCoeffs.h and
DrivetrainCoeffs.cpp. These can be used with the WPILib StateSpacePlant,
StateSpaceController, and StateSpaceObserver classes in C++ and Java. A C++
implementation of this drivetrain controller is available online\footnote{
\url{https://github.com/calcmogul/allwpilib/tree/state-space/wpilibcExamples/src/main/cpp/examples/StateSpaceDrivetrain}}.
