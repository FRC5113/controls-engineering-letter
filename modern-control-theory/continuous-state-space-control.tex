\chapterimage{continuous-state-space-control.jpg}{Night sky above Dufour Street in Santa Cruz, CA}

\chapter{Continuous state-space control}

When we want to command a \gls{system} to a set of \glspl{state}, we design a
controller with certain \glspl{control law} to do it. PID controllers use the
system \glspl{output} with proportional, integral, and derivative
\glspl{control law}. In state-space, we also have knowledge of the system
\glspl{state} so we can do better.

Modern control theory uses state-space representation to model and control
systems. State-space representation models \glspl{system} as a set of
\gls{state}, \gls{input}, and \gls{output} variables related by first-order
differential equations that describe how the \gls{system}'s \gls{state} changes
over time given the current \glspl{state} and \glspl{input}.

\renewcommand*{\chapterpath}{\partpath/continuous-state-space-control}
\section{From PID control to model-based control}
\index{PID control}

As mentioned before, controls engineers have a more general framework to
describe control theory than just PID control. PID controller designers are
focused on fiddling with controller parameters relating to the current, past,
and future \gls{error} rather than the underlying system \glspl{state}. Integral
control is a commonly used tool, and some people use integral action as the
majority of the control action. While this approach works in a lot of
situations, it is an incomplete view of the world.

Model-based control has a completely different mindset. Controls designers using
model-based control care about developing an accurate \gls{model} of the
\gls{system}, then driving the \glspl{state} they care about to zero (or to a
\gls{reference}). Integral control is added with $u_{error}$ estimation if
needed to handle \gls{model} uncertainty, but we prefer not to use it because
its response is hard to tune and some of its destabilizing dynamics aren't
visible during simulation.

\section{What is a dynamical system?}

A dynamical system is a \gls{system} whose motion varies according to a set of
differential equations. A dynamical system is considered \textit{linear} if the
differential equations describing its dynamics consist only of linear operators.
Linear operators are things like constant gain multiplications, derivatives, and
integrals. You can define reasonably accurate linear \glspl{model} for pretty
much everything you'll see in FRC with just those relations.

But let's say you have a DC brushed motor hooked up to a power supply and you
applied a constant voltage to it from rest. The motor approaches a steady-state
angular velocity, but the shape of the angular velocity curve over time isn't a
line. In fact, it's a decaying exponential curve akin to
\begin{equation*}
  \omega = \omega_{max}\left(1 - e^{-t}\right)
\end{equation*}

where $\omega$ is the angular velocity and $\omega_{max}$ is the maximum angular
velocity. If DC brushed motors are said to behave linearly, then why is this?

Linearity refers to a \gls{system}'s equations of motion, not its time domain
response. The equation defining the motor's change in angular velocity over time
looks like
\begin{equation*}
  \dot{\omega} = -a\omega + bV
\end{equation*}

where $\dot{\omega}$ is the derivative of $\omega$ with respect to time, $V$ is
the input voltage, and $a$ and $b$ are constants specific to the motor. This
equation, unlike the one shown before, is actually linear because it only
consists of multiplications and additions relating the \gls{input} $V$ and
current \gls{state} $\omega$.

Also of note is that the relation between the input voltage and the angular
velocity of the output shaft is a linear regression. You'll see why if you model
a DC brushed motor as a voltage source and generator producing back-EMF (in the
equation above, $bV$ corresponds to the voltage source and $-a\omega$
corresponds to the back-EMF). As you increase the input voltage, the back-EMF
increases linearly with the motor's angular velocity. If there was a friction
term that varied with the angular velocity squared (air resistance is one
example), the relation from input to output would be a curve. Friction that
scales with just the angular velocity would result in a lower maximum angular
velocity, but because that term can be lumped into the back-EMF term, the
response is still linear.

\section{Continuous state-space notation}

\subsection{What is state-space?}

Recall from last chapter that 2D space has two axes: $x$ and $y$. We represent
locations within this space as a pair of numbers packaged in a vector, and each
coordinate is a measure of how far to move along the corresponding axis.
State-space is a Cartesian coordinate system with an axis for each \gls{state}
variable, and we represent locations within it the same way we do for 2D space:
with a list of numbers in a vector. Each element in the vector corresponds to a
\gls{state} of the \gls{system}.

In state-space notation, there are also \gls{input} and \gls{output} vectors
with corresponding \gls{input} and \gls{output} spaces. \Glspl{input} drive the
\gls{system} to other points in the state-space, and \glspl{output} are sensor
measurements that have a linear relationship to the \gls{state} and \gls{input}.
Since the mapping from \gls{state} and \gls{input} to change in \gls{state} is a
system of equations, it's natural to write it in matrix form.

\subsection{Definition}

Below is the continuous version of state-space notation.
\begin{definition}[Continuous state-space notation]%
  \index{state-space controllers!continuous open-loop}
  \begin{align}
    \dot{\mat{x}} &= \mat{A}\mat{x} + \mat{B}\mat{u} \label{eq:cont_ss_x} \\
    \mat{y} &= \mat{C}\mat{x} + \mat{D}\mat{u} \label{eq:cont_ss_y}
  \end{align}
  \begin{figurekey}
    \begin{tabular}{llll}
      $\mat{A}$ & system matrix      & $\mat{x}$ & state vector \\
      $\mat{B}$ & input matrix       & $\mat{u}$ & input vector \\
      $\mat{C}$ & output matrix      & $\mat{y}$ & output vector \\
      $\mat{D}$ & feedthrough matrix &  &  \\
    \end{tabular}
  \end{figurekey}
\end{definition}
\begin{booktable}
  \begin{tabular}{|ll|ll|}
    \hline
    \rowcolor{headingbg}
    \textbf{Matrix} & \textbf{Rows $\times$ Columns} &
    \textbf{Matrix} & \textbf{Rows $\times$ Columns} \\
    \hline
    $\mat{A}$ & states $\times$ states & $\mat{x}$ & states $\times$ 1 \\
    $\mat{B}$ & states $\times$ inputs & $\mat{u}$ & inputs $\times$ 1 \\
    $\mat{C}$ & outputs $\times$ states & $\mat{y}$ & outputs $\times$ 1 \\
    $\mat{D}$ & outputs $\times$ inputs &  &  \\
    \hline
  \end{tabular}
  \caption{State-space matrix dimensions}
\end{booktable}

The change in \gls{state} and the \gls{output} are linear combinations of the
\gls{state} vector and the \gls{input} vector. The $\mat{A}$ and $\mat{B}$
matrices are used to map the \gls{state} vector $\mat{x}$ and the \gls{input}
vector $\mat{u}$ to a change in the \gls{state} vector $\dot{\mat{x}}$. The
$\mat{C}$ and $\mat{D}$ matrices are used to map the \gls{state} vector
$\mat{x}$ and the \gls{input} vector $\mat{u}$ to an \gls{output} vector
$\mat{y}$.

\section{Eigenvalues and stability}

If a system is stable, its output will tend toward equilibrium (steady-state)
over time. For a general system $\dot{\mat{x}} = f(\mat{x}, \mat{u})$,
equilibrium points are points where $\dot{\mat{x}} = \mat{0}$. If we let
$\mat{x} = \mat{0}$ and $\mat{u} = \mat{0}$ in
$\dot{\mat{x}} = \mat{A}\mat{x} + \mat{B}\mat{u}$, we can see that
$\dot{\mat{x}} = \mat{0}$, so $\mat{x} = \mat{0}$ is an equilibrium point.

We'd like to know whether all possible unforced system trajectories
($\mat{u} = \mat{0}$) move toward or away from the equilibrium point. If we
solve the system of linear differential equations
$\dot{\mat{x}} = \mat{A}\mat{x}$, we get
$\mat{x}(t) = e^{\mat{A}t} \mat{x}_0$.\footnote{Section
\ref{sec:linear_system_zoh} will explain why the matrix exponential
$e^{\mat{A}t}$ shows up here.} $e^{\mat{A}t}$ is the superposition of
$e^{\lambda_j t}$ terms where $\{\lambda_j\}$ is the set of $\mat{A}$'s
eigenvalues.\footnote{We're handwaving why this is the case, but it's a
consequence of $e^{\mat{A}t}$ being diagonalizable.}

For now, let's consider when all the eigenvalues are real numbers.
\begin{equation*}
  \begin{cases}
    \lambda_j < 0, & e^{\lambda_j t} \text{ decays to zero (stable)}
      \\
    \lambda_j = 0, & e^{\lambda_j t} = 1 \text{ (marginally stable)} \\
    \lambda_j > 0, & e^{\lambda_j t} \text{ grows to infinity (unstable)}
  \end{cases}
\end{equation*}

So the system tends toward the equilibrium point (i.e., it's stable) if
$\lambda_j < 0$ for all $j$.

Now let's consider when the eigenvalues are complex numbers. What does that mean
for the system response? Let $\lambda_j = a_j + b_j i$. Each of the exponential
terms in the solution can be written as
\begin{equation*}
  e^{\lambda_j t} = e^{(a_j + b_j i)t} = e^{a_j t} e^{i b_j t}
\end{equation*}

The complex exponential can be rewritten using Euler's formula.\footnote{Euler's
formula may seem surprising at first, but it's rooted in the fact that complex
exponentials are rotations in the complex plane around the origin. If you can
imagine walking around the unit circle traced by that rotation, you'll notice
the real part of your position oscillates between $-1$ and $1$ over time. That
is, complex exponentials manifest as oscillations in real life.
\begin{center}
  \qrcode{https://www.youtube.com/watch?v=ZxYOEwM6Wbk} \\
  ``What is Euler's formula actually saying? | Ep. 4 Lockdown live math" (51
    minutes) \\
  \footnotesize 3Blue1Brown \\
  \url{https://www.youtube.com/watch?v=ZxYOEwM6Wbk}
\end{center}
}
\begin{equation*}
  e^{i b_j t} = \cos(b_j t) + i \sin(b_j t)
\end{equation*}

Therefore,
\begin{equation*}
  e^{\lambda_j t} = e^{a_j t} (\cos(b_j t) + i \sin(b_j t))
\end{equation*}

When the eigenvalue's imaginary part $b_j \neq 0$, it contributes oscillation to
the real part's response.

\index{stability!poles}
The eigenvalues of $\mat{A}$ are called \textit{poles}.\footnote{This name comes
from classical control theory. See subsection \ref{subsec:poles_and_zeroes} for
more.} Figure \ref{fig:impulse_response_eig} shows the \glspl{impulse response}
in the time domain for \glspl{system} with various pole locations in the complex
plane (real numbers on the x-axis and imaginary numbers on the y-axis). Each
response has an initial condition of $1$.
\begin{bookfigure}
  \begin{tikzpicture}[auto, >=latex']
  % \draw [help lines] (-4,-2) grid (4,4);

  % Draw main axes
  \draw[->] (-4,0) -- (4,0) node[below] {\small Re($\sigma$)};
  \draw[->] (0,-2) -- (0,4) node[right] {\small Im($j\omega$)};

  % Stable: e^-1.75t * cos(1.75wt) (80/3*w for readability)
  \drawtimeplot{-2.125cm}{2.5cm}{0.125cm}{0.44375cm}{
    exp(-1.75 * \x) * cos(80/3 * 1.75 * deg(\x))}
  \drawpole{-1.75cm}{1.75cm}

  % Stable: e^-2.5t
  \drawtimeplot{-2.25cm}{0.75cm}{0.125cm}{0.125cm}{exp(-2 * \x)}
  \drawpole{-2cm}{0cm}

  % Stable: e^-t
  \drawtimeplot{-1.125cm}{-0.75cm}{0.125cm}{0.125cm}{exp(-\x)}
  \drawpole{-1cm}{0cm}

  % Marginally stable: cos(wt) (80/3*w for readability)
  \drawtimeplot{-0.75cm}{1.125cm}{0.125cm}{0.44375cm}{cos(80/3 * deg(\x))}
  \drawpole{0cm}{1cm}

  % Marginally stable cos(2wt) (80/3*w for readability)
  \drawtimeplot{0cm}{2.75cm}{0.125cm}{0.44375cm}{cos(80/3 * 2 * deg(\x))}
  \drawpole{0cm}{2cm}

  % Integrator
  \drawtimeplot{0.25cm}{-0.75cm}{0.125cm}{0.125cm}{1}
  \drawpole{0cm}{0cm}

  % Unstable: e^t
  \drawtimeplot{1.125cm}{0.75cm}{0.125cm}{0.125cm}{exp(\x)}
  \drawpole{1cm}{0cm}

  % Unstable: e^2t
  \drawtimeplot{2.25cm}{-0.75cm}{0.125cm}{0.125cm}{exp(2 * \x)}
  \drawpole{2cm}{0cm}

  % Unstable: e^0.75t * cos(1.75wt) (80/3*w for readability)
  \drawtimeplot{1.5cm}{2.25cm}{0.125cm}{0.44375cm}{
    exp(0.75 * \x) * cos(80/3 * 1.75 * deg(\x))}
  \drawpole{0.75cm}{1.75cm}

  % LHP and RHP labels
  \draw (-3.5,1.5) node {LHP};
  \draw (3.5,1.5) node {RHP};

  % Stable and unstable labels
  \draw (-2.5,3.5) node {\small Stable};
  \draw (2.5,3.5) node {\small Unstable};
\end{tikzpicture}

  \caption{Impulse response vs pole location}
  \label{fig:impulse_response_eig}
\end{bookfigure}

Poles in the left half-plane (LHP) are stable; the \gls{system}'s output may
oscillate but it converges to steady-state. Poles on the imaginary axis are
marginally stable; the \gls{system}'s output oscillates at a constant amplitude
forever. Poles in the right half-plane (RHP) are unstable; the \gls{system}'s
output grows without bound.
\begin{remark}
  Imaginary poles always come in complex conjugate pairs (e.g., $-2 + 3i$,
  $-2 - 3i$).
\end{remark}

\section{Closed-loop controller}

With the \gls{control law} $\mtx{u} = \mtx{K}(\mtx{r} - \mtx{x})$, we can derive
the closed-loop state-space equations. We'll discuss where this
\gls{control law} comes from in subsection \ref{sec:lqr}.

First is the \gls{state} update equation. Substitute the \gls{control law} into
equation (\ref{eq:ss_ctrl_x}).

\begin{align}
  \dot{\mtx{x}} &= \mtx{A}\mtx{x} + \mtx{B}\mtx{K}(\mtx{r} - \mtx{x}) \nonumber
    \\
  \dot{\mtx{x}} &= \mtx{A}\mtx{x} + \mtx{B}\mtx{K}\mtx{r} -
    \mtx{B}\mtx{K}\mtx{x} \nonumber \\
  \dot{\mtx{x}} &= (\mtx{A} - \mtx{B}\mtx{K})\mtx{x} + \mtx{B}\mtx{K}\mtx{r}
\end{align}

Now for the \gls{output} equation. Substitute the \gls{control law} into
equation (\ref{eq:ss_ctrl_y}).

\begin{align}
  \mtx{y} &= \mtx{C}\mtx{x} + \mtx{D}(\mtx{K}(\mtx{r} - \mtx{x})) \nonumber \\
  \mtx{y} &= \mtx{C}\mtx{x} + \mtx{D}\mtx{K}\mtx{r} - \mtx{D}\mtx{K}\mtx{x}
    \nonumber \\
  \mtx{y} &= (\mtx{C} - \mtx{D}\mtx{K})\mtx{x} + \mtx{D}\mtx{K}\mtx{r}
\end{align}

Now, we'll do the same for the discrete \gls{system}. We'd like to know whether
the \gls{system} defined by equation (\ref{eq:ssz_ctrl_x}) operating with the
\gls{control law} $\mtx{u}_k = \mtx{K}(\mtx{r}_k - \mtx{x}_k)$ converges to the
\gls{reference} $\mtx{r}_k$.

\begin{align*}
  \mtx{x}_{k+1} &= \mtx{A}\mtx{x}_k + \mtx{B}\mtx{u}_k \\
  \mtx{x}_{k+1} &= \mtx{A}\mtx{x}_k + \mtx{B}(\mtx{K}(\mtx{r}_k - \mtx{x}_k)) \\
  \mtx{x}_{k+1} &= \mtx{A}\mtx{x}_k + \mtx{B}\mtx{K}\mtx{r}_k -
    \mtx{B}\mtx{K}\mtx{x}_k \\
  \mtx{x}_{k+1} &= \mtx{A}\mtx{x}_k - \mtx{B}\mtx{K}\mtx{x}_k +
    \mtx{B}\mtx{K}\mtx{r}_k \\
  \mtx{x}_{k+1} &= (\mtx{A} - \mtx{B}\mtx{K})\mtx{x}_k + \mtx{B}\mtx{K}\mtx{r}_k
\end{align*}

\begin{theorem}[Closed-loop state-space controller]
  \index{State-space controllers!closed-loop}

  \begin{align}
    \dot{\mtx{x}} &= (\mtx{A} - \mtx{B}\mtx{K})\mtx{x} + \mtx{B}\mtx{K}\mtx{r}
      \label{eq:s_ref_ctrl_x} \\
    \mtx{y} &= (\mtx{C} - \mtx{D}\mtx{K})\mtx{x} + \mtx{D}\mtx{K}\mtx{r}
      \label{eq:s_ref_ctrl_y}
  \end{align}

  \begin{align}
    \mtx{x}_{k+1} &= (\mtx{A} - \mtx{B}\mtx{K})\mtx{x}_k +
      \mtx{B}\mtx{K}\mtx{r}_k \label{eq:z_ref_ctrl_x} \\
    \mtx{y}_k &= (\mtx{C} - \mtx{D}\mtx{K})\mtx{x}_k + \mtx{D}\mtx{K}\mtx{r}_k
      \label{eq:z_ref_ctrl_y}
  \end{align}

  \begin{figurekey}
    \begin{tabular}{llll}
      $\mtx{A}$ & system matrix      & $\mtx{K}$ & controller gain matrix \\
      $\mtx{B}$ & input matrix       & $\mtx{x}$ & state vector \\
      $\mtx{C}$ & output matrix      & $\mtx{r}$ & \gls{reference} vector \\
      $\mtx{D}$ & feedthrough matrix & $\mtx{y}$ & output vector \\
    \end{tabular}
  \end{figurekey}
\end{theorem}

\begin{booktable}
  \begin{tabular}{|ll|ll|}
    \hline
    \rowcolor{headingbg}
    \textbf{Matrix} & \textbf{Rows $\times$ Columns} &
    \textbf{Matrix} & \textbf{Rows $\times$ Columns} \\
    \hline
    $\mtx{A}$ & states $\times$ states & $\mtx{x}$ & states $\times$ 1 \\
    $\mtx{B}$ & states $\times$ inputs & $\mtx{u}$ & inputs $\times$ 1 \\
    $\mtx{C}$ & outputs $\times$ states & $\mtx{y}$ & outputs $\times$ 1 \\
    $\mtx{D}$ & outputs $\times$ inputs & $\mtx{r}$ & states $\times$ 1 \\
    $\mtx{K}$ & inputs $\times$ states &  &  \\
    \hline
  \end{tabular}
  \caption{Controller matrix dimensions}
  \label{tab:ctrl_matrix_dims}
\end{booktable}

\index{Stability!eigenvalues}
Instead of commanding the \gls{system} to a \gls{state} using the vector
$\mtx{u}$ directly, we can now specify a vector of desired \glspl{state} through
$\mtx{r}$ and the \gls{controller} will choose values of $\mtx{u}$ for us over
time to make the \gls{system} converge to the \gls{reference}. For equation
(\ref{eq:s_ref_ctrl_x}) to reach steady-state, the eigenvalues of
$\mtx{A} - \mtx{B}\mtx{K}$ must be in the left-half plane. For equation
(\ref{eq:z_ref_ctrl_x}) to have a bounded output, the eigenvalues of
$\mtx{A} - \mtx{B}\mtx{K}$ must be within the unit circle.

The eigenvalues of $\mtx{A} - \mtx{B}\mtx{K}$ are the poles of the closed-loop
\gls{system}. Therefore, the rate of convergence and stability of the
closed-loop \gls{system} can be changed by moving the poles via the eigenvalues
of $\mtx{A} - \mtx{B}\mtx{K}$. $\mtx{A}$ and $\mtx{B}$ are inherent to the
\gls{system}, but $\mtx{K}$ can be chosen arbitrarily by the controller
designer.

\section{Model augmentation}

This section will teach various tricks for manipulating state-space
\glspl{model} with the goal of demystifying the matrix algebra at play. We will
use the augmentation techniques discussed here in the section on integral
control.

Matrix augmentation is the process of appending rows or columns to a matrix. In
state-space, there are several common types of augmentation used: \gls{plant}
augmentation, controller augmentation, and \gls{observer} augmentation.

\subsection{Plant augmentation}
\index{Model augmentation!of plant}

Plant augmentation is the process of adding a state to a model's state vector
and adding a corresponding row to the $\mtx{A}$ and $\mtx{B}$ matrices.

\subsection{Controller augmentation}
\index{Model augmentation!of controller}

Controller augmentation is the process of adding a column to a controller's
$\mtx{K}$ matrix. This is often done in combination with \gls{plant}
augmentation to add controller dynamics relating to a newly added \gls{state}.

\subsection{Observer augmentation}
\index{Model augmentation!of observer}

Observer augmentation is closely related to \gls{plant} augmentation. In
addition to adding entries to the \gls{observer} matrix $\mtx{L}$, the
\gls{observer} is using this augmented \gls{plant} for estimation purposes. This
is better explained with an example.

By augmenting the \gls{plant} with a bias term with no dynamics (represented by
zeroes in its rows in $\mtx{A}$ and $\mtx{B}$), the \gls{observer} will attempt
to estimate a value for this bias term that makes the \gls{model} best reflect
the measurements taken of the real \gls{system}. Note that we're not collecting
any data on this bias term directly; it's what's known as a hidden \gls{state}.
Rather than our \glspl{input} and other \glspl{state} affecting it directly, the
\gls{observer} determines a value for it based on what is most likely given the
\gls{model} and current measurements. We just tell the \gls{plant} what kind of
dynamics the term has and the \gls{observer} will estimate it for us.

\subsection{Output augmentation}
\index{Model augmentation!of output}

Output augmentation is the process of adding rows to the $\mtx{C}$ matrix. This
is done to help the controls designer visualize the behavior of internal states
or other aspects of the \gls{system} in MATLAB or Python Control. $\mtx{C}$
matrix augmentation doesn't affect \gls{state} feedback, so the designer has a
lot of freedom here. Noting that the \gls{output} is defined as
$\mtx{y} = \mtx{C}\mtx{x} + \mtx{D}\mtx{u}$, The following row augmentations of
$\mtx{C}$ may prove useful. Of course, $\mtx{D}$ needs to be augmented with
zeroes as well in these cases to maintain the correct matrix dimensionality.

Since $\mtx{u} = -\mtx{K}\mtx{x}$, augmenting $\mtx{C}$ with $-\mtx{K}$ makes
the \gls{observer} estimate the \gls{control input} $\mtx{u}$ applied.

\begin{align*}
  \mtx{y} &= \mtx{C}\mtx{x} + \mtx{D}\mtx{u} \\
  \begin{bmatrix}
    \mtx{y} \\
    \mtx{u}
  \end{bmatrix} &=
  \begin{bmatrix}
    \mtx{C} \\
    -\mtx{K}
  \end{bmatrix}
  \mtx{x} +
  \begin{bmatrix}
    \mtx{D} \\
    \mtx{0}
  \end{bmatrix}
  \mtx{u}
\end{align*}

This works because $\mtx{K}$ has the same number of columns as \glspl{state}.

Various \glspl{state} can also be produced in the \gls{output} with $\mtx{I}$
matrix augmentation.

\subsection{Examples}

Snippet \ref{lst:augment_concat} shows how one packs together the following
augmented matrix in Python using concatenation.

\begin{equation*}
  \begin{bmatrix}
    \mtx{A} & \mtx{B} \\
    \mtx{C} & \mtx{D}
  \end{bmatrix}
\end{equation*}

\begin{code}{Python}{code/snippets/augment_concat.py}
  \caption{Matrix augmentation example: concatenation}
  \label{lst:augment_concat}
\end{code}

Snippet \ref{lst:augment_slices} shows how one packs together the same augmented
matrix in Python using array slices.

\begin{code}{Python}{code/snippets/augment_slices.py}
  \caption{Matrix augmentation example: array slices}
  \label{lst:augment_slices}
\end{code}

Section \ref{sec:integral_control} demonstrates \gls{model} augmentation for
different types of integral control.

\section{Integral control}
\label{sec:integral_control}

A common way of implementing integral control is to add an additional
\gls{state} that is the integral of the \gls{error} of the variable intended to
have zero \gls{steady-state error}.

There are two drawbacks to this method. First, there is integral windup on a
unit \gls{step input}. That is, the integrator accumulates even if the
\gls{system} is \gls{tracking} the \gls{model} correctly. The second is
demonstrated by an example from Jared Russell of FRC team 254. Say there is a
position/velocity trajectory for some \gls{plant} to follow. Without integral
control, one can calculate a desired $\mtx{K}\mtx{x}$ to use as the
\gls{control input}. As a result of using both desired position and velocity,
\gls{reference} \gls{tracking} is good. With integral control added, the
\gls{reference} is always the desired position, but there is no way to tell the
controller the desired velocity.

Consider carefully whether integral control is necessary. One can get relatively
close without integral control, and integral adds all the issues listed above.
Below, it is assumed that the controls designer has determined that integral
control will be worth the inconvenience.

There are three methods FRC team 971 has used over the years:

\begin{enumerate}
  \item Augment the \gls{plant} as described earlier. For an arm, one would add
    an ``integral of position" state.
  \item Add an integrator to the output of the controller, then estimate the
    \gls{control effort} being applied. 971 has called this Delta U control. The
    upside is that it doesn't have the windup issue described above; the
    integrator only acts if the \gls{system} isn't behaving like the
    \gls{model}, which was the original intent. The downside is working with it
    is very confusing.
  \item Estimate the ``error" in the \gls{control input} (the difference between
    what was applied versus what was observed to happen) via the \gls{observer}
    and compensate for it.
\end{enumerate}

We'll present the first and third methods since the third is strictly better
than the second.

\subsection{Plant augmentation}
\index{Integral control!plant augmentation}

We want to augment the \gls{system} with an integral term that integrates the
\gls{error} $\mtx{e} = \mtx{r} - \mtx{y} = \mtx{r} - \mtx{C}\mtx{x}$.

\begin{align*}
  \mtx{x}_I &= \int \mtx{e} \,dt \\
  \dot{\mtx{x}}_I &= \mtx{e} = \mtx{r} - \mtx{C}\mtx{x}
\end{align*}

The \gls{plant} is augmented as

\begin{align*}
  \dot{\begin{bmatrix}
    \mtx{x} \\
    \mtx{x}_I
  \end{bmatrix}} &=
  \begin{bmatrix}
    \mtx{A} & \mtx{0} \\
    -\mtx{C} & \mtx{0}
  \end{bmatrix}
  \begin{bmatrix}
    \mtx{x} \\
    \mtx{x}_I
  \end{bmatrix} +
  \begin{bmatrix}
    \mtx{B} \\
    \mtx{0}
  \end{bmatrix}
  \mtx{u} +
  \begin{bmatrix}
    \mtx{0} \\
    \mtx{I}
  \end{bmatrix}
  \mtx{r} \\
  \dot{\begin{bmatrix}
    \mtx{x} \\
    \mtx{x}_I
  \end{bmatrix}} &=
  \begin{bmatrix}
    \mtx{A} & \mtx{0} \\
    -\mtx{C} & \mtx{0}
  \end{bmatrix}
  \begin{bmatrix}
    \mtx{x} \\
    \mtx{x}_I
  \end{bmatrix} +
  \begin{bmatrix}
    \mtx{B} & \mtx{0} \\
    \mtx{0} & \mtx{I}
  \end{bmatrix}
  \begin{bmatrix}
    \mtx{u} \\
    \mtx{r}
  \end{bmatrix}
\end{align*}

The controller is augmented as

\begin{align*}
  \mtx{u} &= \mtx{K} (\mtx{r} - \mtx{x}) - \mtx{K}_I\mtx{x}_I \\
  \mtx{u} &=
  \begin{bmatrix}
    \mtx{K} & \mtx{K}_I
  \end{bmatrix}
  \left(\begin{bmatrix}
    \mtx{r} \\
    \mtx{0}
  \end{bmatrix} -
  \begin{bmatrix}
    \mtx{x} \\
    \mtx{x}_I
  \end{bmatrix}\right)
\end{align*}

\subsection{U error estimation}
\label{subsec:u_error_estimation}
\index{Integral control!U error estimation}

Given the desired \gls{input} produced by a \gls{controller}, unmodeled
\glspl{disturbance} may cause the observed behavior of a \gls{system} to deviate
from its \gls{model}. U error estimation estimates the difference between the
desired \gls{input} and a hypothetical \gls{input} that makes the \gls{model}
match the observed behavior. This value can be added to the \gls{control input}
to make the \gls{controller} compensate for unmodeled \glspl{disturbance} and
make the \gls{model} better predict the \gls{system}'s future behavior.

Let $u_{error}$ be the difference between the \gls{input} actually applied to a
\gls{system} and the desired \gls{input}. The $u_{error}$ term is then added to
the \gls{system} as follows.

\begin{equation*}
  \dot{\mtx{x}} = \mtx{A}\mtx{x} + \mtx{B}\left(\mtx{u} + u_{error}\right)
\end{equation*}

$\mtx{u} + u_{error}$ is the hypothetical \gls{input} actually applied to the
\gls{system}.

\begin{equation*}
  \dot{\mtx{x}} = \mtx{A}\mtx{x} + \mtx{B}\mtx{u} + \mtx{B}u_{error}
\end{equation*}

For a multiple-output \gls{system}, this would be

\begin{equation*}
  \dot{\mtx{x}} = \mtx{A}\mtx{x} + \mtx{B}\mtx{u} + \mtx{B}_{error}u_{error}
\end{equation*}

where $\mtx{B}_{error}$ is the column vector that maps $u_{error}$ to changes in
the rest of the \gls{state} the same way $\mtx{B}$ does for $\mtx{u}$.
$\mtx{B}_{error}$ is only a column of $\mtx{B}$ if $u_{error}$ corresponds to an
existing \gls{input} within $\mtx{u}$.

The \gls{plant} is augmented as

\begin{align*}
  \dot{\begin{bmatrix}
    \mtx{x} \\
    u_{error}
  \end{bmatrix}} &=
  \begin{bmatrix}
    \mtx{A} & \mtx{B}_{error} \\
    \mtx{0} & \mtx{0}
  \end{bmatrix}
  \begin{bmatrix}
    \mtx{x} \\
    u_{error}
  \end{bmatrix} +
  \begin{bmatrix}
    \mtx{B} \\
    \mtx{0}
  \end{bmatrix}
  \mtx{u} \\
  \mtx{y} &= \begin{bmatrix}
    \mtx{C} & 0
  \end{bmatrix} \begin{bmatrix}
    \mtx{x} \\
    u_{error}
  \end{bmatrix} + \mtx{D}\mtx{u}
\end{align*}

With this \gls{model}, the \gls{observer} will estimate both the \gls{state} and
the $u_{error}$ term. The controller is augmented similarly. $\mtx{r}$ is
augmented with a zero for the goal $u_{error}$ term.

\begin{align*}
  \mtx{u} &= \mtx{K} \left(\mtx{r} - \mtx{x}\right) - \mtx{k}_{error}u_{error}
    \\
  \mtx{u} &=
  \begin{bmatrix}
    \mtx{K} & \mtx{k}_{error}
  \end{bmatrix}
  \left(\begin{bmatrix}
    \mtx{r} \\
    0
  \end{bmatrix} -
  \begin{bmatrix}
    \mtx{x} \\
    u_{error}
  \end{bmatrix}\right)
\end{align*}

where $\mtx{k}_{error}$ is a column vector with a $1$ in a given row if
$u_{error}$ should be applied to that \gls{input} or a $0$ otherwise.

This process can be repeated for an arbitrary \gls{error} which can be corrected
via some linear combination of the \glspl{input}.

\section{Double integrator}

The double integrator has two states (position and velocity) and one input
(acceleration). Their relationship can be expressed by the following system of
differential equations, where $x$ is position, $v$ is velocity, and $a$ is
acceleration.
\begin{align*}
  \dot{x} &= v \\
  \dot{v} &= a
\end{align*}

We want to put these into the form
$\dot{\mat{x}} = \mat{A}\mat{x} + \mat{B}\mat{u}$. Let
$\mat{x} = \begin{bmatrix}x & v\end{bmatrix}\T$ and
$\mat{u} = \begin{bmatrix}a\end{bmatrix}\T$. First, add missing terms so that
all equations have the same states and inputs. Then, sort them by states
followed by inputs.
\begin{align*}
  \dot{x} &= 0x + 1v + 0a \\
  \dot{v} &= 0x + 0v + 1a
  \intertext{Now, factor out the constants into matrices.}
  \dot{\begin{bmatrix}
    x \\
    v
  \end{bmatrix}} &=
  \begin{bmatrix}
    0 & 1 \\
    0 & 0
  \end{bmatrix}
  \begin{bmatrix}
    x \\
    v
  \end{bmatrix} +
  \begin{bmatrix}
    0 \\
    1
  \end{bmatrix}
  \begin{bmatrix}
    a
  \end{bmatrix} \\
  \dot{\mat{x}} &= \mat{A}\mat{x} + \mat{B}\mat{u}
\end{align*}

\section{Elevator}
\label{sec:ss_model_elevator}

\subsection{Continuous state-space model}
\index{FRC models!elevator equations}

The position and velocity derivatives of the elevator can be written as

\begin{align}
  \dot{x}_m &= v_m \label{eq:elevator_cont_ss_pos} \\
  \dot{v}_m &= a_m \label{eq:elevator_cont_ss_vel}
\end{align}

where by equation (\ref{eq:elevator_accel}),

\begin{equation*}
  a_m = \frac{GK_t}{Rrm} V - \frac{G^2 K_t}{Rr^2 m K_v} v_m
\end{equation*}

Substitute this into equation (\ref{eq:elevator_cont_ss_vel}).

\begin{align}
  \dot{v}_m &= \frac{GK_t}{Rrm} V - \frac{G^2 K_t}{Rr^2 m K_v} v_m \nonumber \\
  \dot{v}_m &= -\frac{G^2 K_t}{Rr^2 m K_v} v_m + \frac{GK_t}{Rrm} V
\end{align}

Factor out $v_m$ and $V$ into column vectors.

\begin{align*}
  \dot{\begin{bmatrix}
    v_m
  \end{bmatrix}} &=
  \begin{bmatrix}
    -\frac{G^2 K_t}{Rr^2 m K_v}
  \end{bmatrix}
  \begin{bmatrix}
    v_m
  \end{bmatrix} +
  \begin{bmatrix}
    \frac{GK_t}{Rrm}
  \end{bmatrix}
  \begin{bmatrix}
    V
  \end{bmatrix}
\end{align*}

Augment the matrix equation with the position state $x$, which has the model
equation $\dot{x} = v_m$. The matrix elements corresponding to $v_m$ will be
$1$, and the others will be $0$ since they don't appear, so
$\dot{x} = 0x + 1v_m + 0V$. The existing rows will have zeroes inserted where
$x$ is multiplied in.

\begin{align*}
  \dot{\begin{bmatrix}
    x \\
    v_m
  \end{bmatrix}} &=
  \begin{bmatrix}
    0 & 1 \\
    0 & -\frac{G^2 K_t}{Rr^2 m K_v}
  \end{bmatrix}
  \begin{bmatrix}
    x \\
    v_m
  \end{bmatrix} +
  \begin{bmatrix}
    0 \\
    \frac{GK_t}{Rrm}
  \end{bmatrix}
  \begin{bmatrix}
    V
  \end{bmatrix}
\end{align*}

\begin{theorem}[Elevator state-space model]
  \begin{align*}
    \dot{\mtx{x}} &= \mtx{A} \mtx{x} + \mtx{B} \mtx{u} \\
    \mtx{y} &= \mtx{C} \mtx{x} + \mtx{D} \mtx{u}
  \end{align*}
  \begin{equation*}
    \begin{array}{ccc}
      \mtx{x} =
      \begin{bmatrix}
        x \\
        v_m
      \end{bmatrix} &
      \mtx{y} = x &
      \mtx{u} = V
    \end{array}
  \end{equation*}
  \begin{equation}
    \begin{array}{cccc}
      \mtx{A} =
      \begin{bmatrix}
        0 & 1 \\
        0 & -\frac{G^2 K_t}{Rr^2 mK_v}
      \end{bmatrix} &
      \mtx{B} =
      \begin{bmatrix}
        0 \\
        \frac{GK_t}{Rrm}
      \end{bmatrix} &
      \mtx{C} =
      \begin{bmatrix}
        1 & 0
      \end{bmatrix} &
      \mtx{D} = 0
    \end{array}
  \end{equation}
\end{theorem}

\subsection{Model augmentation}

As per subsection \ref{subsec:u_error_estimation}, we will now augment the
\gls{model} so a $u_{error}$ term is added to the \gls{control input}.

The \gls{plant} and \gls{observer} augmentations should be performed before the
\gls{model} is \glslink{discretization}{discretized}. After the \gls{controller}
gain is computed with the unaugmented discrete \gls{model}, the controller may
be augmented. Therefore, the \gls{plant} and \gls{observer} augmentations assume
a continuous \gls{model} and the \gls{controller} augmentation assumes a
discrete \gls{controller}.

\begin{equation*}
  \begin{array}{ccc}
    \mtx{x}_{aug} =
    \begin{bmatrix}
      x \\
      v_m \\
      u_{error}
    \end{bmatrix} &
    \mtx{y} = x &
    \mtx{u} = V
  \end{array}
\end{equation*}

\begin{equation}
  \begin{array}{cccc}
    \mtx{A}_{aug} =
    \begin{bmatrix}
      \mtx{A} & \mtx{B} \\
      \mtx{0}_{1 \times 2} & 0
    \end{bmatrix} &
    \mtx{B}_{aug} =
    \begin{bmatrix}
      \mtx{B} \\
      0
    \end{bmatrix} &
    \mtx{C}_{aug} = \begin{bmatrix}
      \mtx{C} & 0
    \end{bmatrix} &
    \mtx{D}_{aug} = \mtx{D}
  \end{array}
\end{equation}

\begin{equation}
  \begin{array}{cc}
    \mtx{K}_{aug} = \begin{bmatrix}
      \mtx{K} & 1
    \end{bmatrix} &
    \mtx{r}_{aug} = \begin{bmatrix}
      \mtx{r} \\
      0
    \end{bmatrix}
  \end{array}
\end{equation}

This will compensate for unmodeled dynamics such as gravity. However, using a
constant voltage feedforward to counteract gravity is preferred over $u_{error}$
estimation in this case because it results in a simpler controller with similar
performance.

\subsection{Gravity feedforward}

Input voltage is proportional to force and gravity is a constant force, so a
constant voltage feedforward can compensate for gravity. We'll model gravity as
a disturbance described by $-mg$. To compensate for it, we want to find a
voltage that is equal and opposite to it. The bottom row of the continuous
elevator model contains the acceleration terms.

\begin{equation*}
  Bu_{ff} = -(\text{unmodeled dynamics})
\end{equation*}

where $B$ is the motor acceleration term from $\mtx{B}$ and $u_{ff}$ is the
voltage feedforward.

\begin{align*}
  Bu_{ff} &= -(-mg) \\
  Bu_{ff} &= mg \\
  \frac{G K_t}{Rrm} u_{ff} &= mg \\
  u_{ff} &= \frac{Rrm^2 g}{G K_t}
\end{align*}

\subsection{Simulation}

Python Control will be used to \glslink{discretization}{discretize} the
\gls{model} and simulate it. One of the frccontrol
examples\footnote{\url{https://github.com/calcmogul/frccontrol/blob/master/examples/elevator.py}}
creates and tests a controller for it.

\begin{remark}
  Python Control currently doesn't support finding the transmission zeroes of
  MIMO \glspl{system} with a different number of \glspl{input} than
  \glspl{output}, so \texttt{control.pzmap()} and
  \texttt{frccontrol.System.plot\_pzmaps()} fail with an error if Slycot isn't
  installed.
\end{remark}

Figure \ref{fig:elevator_pzmaps} shows the pole-zero maps for the open-loop
\gls{system}, closed-loop \gls{system}, and \gls{observer}. Figure
\ref{fig:elevator_response} shows the \gls{system} response with them.

\begin{svg}{build/frccontrol/examples/elevator_pzmaps}
  \caption{Elevator pole-zero maps}
  \label{fig:elevator_pzmaps}
\end{svg}

\begin{svg}{build/frccontrol/examples/elevator_response}
  \caption{Elevator response}
  \label{fig:elevator_response}
\end{svg}

\subsection{Implementation}

The script linked above also generates two files: ElevatorCoeffs.h and
ElevatorCoeffs.cpp. These can be used with the WPILib StateSpacePlant,
StateSpaceController, and StateSpaceObserver classes in C++ and Java. A C++
implementation of this elevator controller is available online\footnote{
\url{https://github.com/calcmogul/allwpilib/tree/state-space/wpilibcExamples/src/main/cpp/examples/StateSpaceElevator}}.

\section{Flywheel}
\label{sec:ss_model_flywheel}

\subsection{Continuous state-space model}
\index{FRC models!flywheel equations}

By equation \eqref{eq:dot_omega_flywheel}
\begin{equation*}
  \dot{\omega} = -\frac{G^2 K_t}{K_v RJ} \omega + \frac{G K_t}{RJ} V
\end{equation*}

Factor out $\omega$ and $V$ into column vectors.
\begin{align*}
  \dot{\begin{bmatrix}
    \omega
  \end{bmatrix}} &=
  \begin{bmatrix}
    -\frac{G^2 K_t}{K_v RJ}
  \end{bmatrix}
  \begin{bmatrix}
    \omega
  \end{bmatrix} +
  \begin{bmatrix}
    \frac{GK_t}{RJ}
  \end{bmatrix}
  \begin{bmatrix}
    V
  \end{bmatrix}
\end{align*}
\begin{theorem}[Flywheel state-space model]
  \begin{align*}
    \dot{\mat{x}} &= \mat{A} \mat{x} + \mat{B} \mat{u} \\
    \mat{y} &= \mat{C} \mat{x} + \mat{D} \mat{u}
  \end{align*}
  \begin{equation*}
    \begin{array}{ccc}
      \mat{x} = \omega &
      \mat{y} = \omega &
      \mat{u} = V
    \end{array}
  \end{equation*}
  \begin{equation}
    \begin{array}{cccc}
      \mat{A} = -\frac{G^2 K_t}{K_v RJ} &
      \mat{B} = \frac{G K_t}{RJ} &
      \mat{C} = 1 &
      \mat{D} = 0
    \end{array}
  \end{equation}
\end{theorem}

\subsection{Model augmentation}

As per subsection \ref{subsec:input_error_estimation}, we will now augment the
\gls{model} so a $u_{error}$ state is added to the \gls{control input}.

The \gls{plant} and \gls{observer} augmentations should be performed before the
\gls{model} is \glslink{discretization}{discretized}. After the \gls{controller}
gain is computed with the unaugmented discrete \gls{model}, the controller may
be augmented. Therefore, the \gls{plant} and \gls{observer} augmentations assume
a continuous \gls{model} and the \gls{controller} augmentation assumes a
discrete \gls{controller}.
\begin{equation*}
  \begin{array}{ccc}
    \mat{x} =
    \begin{bmatrix}
      \omega \\
      u_{error}
    \end{bmatrix} &
    \mat{y} = \omega &
    \mat{u} = V
  \end{array}
\end{equation*}
\begin{equation}
  \begin{array}{cccc}
    \mat{A}_{aug} =
    \begin{bmatrix}
      \mat{A} & \mat{B} \\
      0 & 0
    \end{bmatrix} &
    \mat{B}_{aug} =
    \begin{bmatrix}
      \mat{B} \\
      0
    \end{bmatrix} &
    \mat{C}_{aug} = \begin{bmatrix}
      \mat{C} & 0
    \end{bmatrix} &
    \mat{D}_{aug} = \mat{D}
  \end{array}
\end{equation}
\begin{equation}
  \begin{array}{cc}
    \mat{K}_{aug} = \begin{bmatrix}
      \mat{K} & 1
    \end{bmatrix} &
    \mat{r}_{aug} = \begin{bmatrix}
      \mat{r} \\
      0
    \end{bmatrix}
  \end{array}
\end{equation}

This will compensate for unmodeled dynamics such as projectiles slowing down the
flywheel.

\subsection{Simulation}

Python Control will be used to \glslink{discretization}{discretize} the
\gls{model} and simulate it. One of the frccontrol
examples\footnote{\url{https://github.com/calcmogul/frccontrol/blob/main/examples/flywheel.py}}
creates and tests a controller for it. Figure \ref{fig:flywheel_response} shows
the closed-loop \gls{system} response.
\begin{svg}{build/frccontrol/examples/flywheel_response}
  \caption{Flywheel response}
  \label{fig:flywheel_response}
\end{svg}

Notice how the \gls{control effort} when the \gls{reference} is reached is
nonzero. This is a plant inversion feedforward compensating for the \gls{system}
dynamics attempting to slow the flywheel down when no voltage is applied.

\subsection{Implementation}

C++ and Java implementations of this flywheel controller are available
online\footnote{\url{https://github.com/wpilibsuite/allwpilib/blob/main/wpilibcExamples/src/main/cpp/examples/StateSpaceFlywheel/cpp/Robot.cpp}}
\footnote{\url{https://github.com/wpilibsuite/allwpilib/blob/main/wpilibjExamples/src/main/java/edu/wpi/first/wpilibj/examples/statespaceflywheel/Robot.java}}.

\subsection{Flywheel model without encoder}

In the FIRST Robotics Competition, we can get the current drawn for specific
channels on the power distribution panel. We can theoretically use this to
estimate the angular velocity of a DC motor without an encoder. We'll start with
the flywheel model derived earlier as equation \eqref{eq:dot_omega_flywheel}.
\begin{align*}
  \dot{\omega} &= \frac{G K_t}{RJ} V - \frac{G^2 K_t}{K_v RJ} \omega \\
  \dot{\omega} &= -\frac{G^2 K_t}{K_v RJ} \omega + \frac{G K_t}{RJ} V
\end{align*}

Next, we'll derive the current $I$ as an output.
\begin{align*}
  V &= IR + \frac{\omega}{K_v} \\
  IR &= V - \frac{\omega}{K_v} \\
  I &= -\frac{1}{K_v R} \omega + \frac{1}{R} V
\end{align*}

Therefore,
\begin{theorem}[Flywheel state-space model without encoder]
  \begin{align*}
    \dot{\mat{x}} &= \mat{A} \mat{x} + \mat{B} \mat{u} \\
    \mat{y} &= \mat{C} \mat{x} + \mat{D} \mat{u}
  \end{align*}
  \begin{equation*}
    \begin{array}{ccc}
      \mat{x} = \omega &
      \mat{y} = I &
      \mat{u} = V
    \end{array}
  \end{equation*}
  \begin{equation}
    \begin{array}{cccc}
      \mat{A} = -\frac{G^2 K_t}{K_v RJ} &
      \mat{B} = \frac{G K_t}{RJ} &
      \mat{C} = -\frac{1}{K_v R} &
      \mat{D} = \frac{1}{R}
    \end{array}
  \end{equation}
\end{theorem}

Notice that in this \gls{model}, the \gls{output} doesn't provide any direct
measurements of the \gls{state}. To estimate the full \gls{state} (also known as
full observability), we only need the \glspl{output} to collectively include
linear combinations of every \gls{state}\footnote{While the flywheel model's
outputs are a linear combination of both the states and inputs, \glspl{input}
don't provide new information about the \glspl{state}. Therefore, they don't
affect whether the system is observable.}. We'll revisit this in chapter
\ref{ch:stochastic_control_theory} with an example that uses range measurements
to estimate an object's orientation.

The effectiveness of this \gls{model}'s \gls{observer} is heavily dependent on
the quality of the current sensor used. If the sensor's noise isn't zero-mean,
the \gls{observer} won't converge to the true \gls{state}.

\subsection{Voltage compensation}

To improve controller \gls{tracking}, one may want to use the voltage
renormalized to the power rail voltage to compensate for voltage drop when
current spikes occur. This can be done as follows.
\begin{equation}
  V = V_{cmd} \frac{V_{nominal}}{V_{rail}}
\end{equation}

where $V$ is the \gls{controller}'s new input voltage, $V_{cmd}$ is the old
input voltage, $V_{nominal}$ is the rail voltage when effects like voltage drop
due to current draw are ignored, and $V_{rail}$ is the real rail voltage.

To drive the \gls{model} with a more accurate voltage that includes voltage
drop, the reciprocal can be used.
\begin{equation}
  V = V_{cmd} \frac{V_{rail}}{V_{nominal}}
\end{equation}

where $V$ is the \gls{model}'s new input voltage. Note that if both the
\gls{controller} compensation and \gls{model} compensation equations are
applied, the original voltage is obtained. The \gls{model} input only drops from
ideal if the compensated \gls{controller} voltage saturates.

\section{Single-jointed arm}

\subsection{Equations of motion}

This single-jointed arm consists of a DC brushed motor attached to a pulley that
spins a straight bar in pitch.
\begin{bookfigure}
  \begin{tikzpicture}[auto, >=latex', circuit ee IEC,
                      set resistor graphic=var resistor IEC graphic]
    % \draw [help lines] (-1,-3) grid (7,4);

    % Electrical equivalent circuit
    \draw (0,2) to [voltage source={direction info'={->}, info'=$V$}] (0,0);
    \draw (0,2) to [current direction={info=$I$}] (0,3);
    \draw (0,3) -- (0.5,3);
    \draw (0.5,3) to [resistor={info={$R$}}] (2,3);

    \draw (2,3) -- (2.5,3);
    \draw (2.5,3) to [voltage source={direction info'={->}, info'=$V_{emf}$}]
      (2.5,0);
    \draw (0,0) -- (2.5,0);

    % Motor
    \begin{scope}[xshift=2.4cm,yshift=1.05cm]
      \draw[fill=black] (0,0) rectangle (0.2,0.9);
      \draw[fill=white] (0.1,0.45) ellipse (0.3 and 0.3);
    \end{scope}

    % Transmission gear one
    \begin{scope}[xshift=3.75cm,yshift=1.17cm]
      \draw[fill=black!50] (0.2,0.33) ellipse (0.08 and 0.33);
      \draw[fill=black!50, color=black!50] (0,0) rectangle (0.2,0.66);
      \draw[fill=white] (0,0.33) ellipse (0.08 and 0.33);
      \draw (0,0.66) -- (0.2,0.66);
      \draw (0,0) -- (0.2,0) node[pos=0.5,below] {$G$};
    \end{scope}

    % Output shaft of motor
    \begin{scope}[xshift=2.8cm,yshift=1.45cm]
      \draw[fill=black!50] (0,0) rectangle (0.95,0.1);
    \end{scope}

    % Angular velocity arrow of drive -> transmission
    \draw[line width=0.7pt,<-] (3.2,1) arc (-30:30:1) node[above] {$\omega_m$};

    % Transmission gear two
    \begin{scope}[xshift=3.75cm,yshift=1.83cm]
      \draw[fill=black!50] (0.2,0.68) ellipse (0.13 and 0.67);
      \draw[fill=black!50, color=black!50] (0,0) rectangle (0.2,1.35);
      \draw[fill=white] (0,0.68) ellipse (0.13 and 0.67);
      \draw (0,1.35) -- (0.2,1.35);
      \draw (0,0) -- (0.2,0);
    \end{scope}
    \begin{scope}[xshift=5.075cm,yshift=2.4cm]
      % Single-jointed arm
      \draw[fill=white] (0,0) -- (0.1,-0.05) -- (0.35,1.45) -- (0.25,1.5)
        -- cycle;
      \draw[fill=black!50] (0.1,-0.05) -- (0.3,-0.05) -- (0.55,1.45) --
        (0.35,1.45) -- cycle;
      \draw[fill=white] (0.25,1.5) -- (0.35,1.45) -- (0.55,1.45) -- (0.45,1.5)
        -- cycle;

      % Arm length arrow
      \draw[line width=0.7pt,<->] (0.55,-0.05) -- node[right] {$l$} (0.8,1.45);

      % Mass label
      \draw (-0.05,1.2) node {$m$};
    \end{scope}

    % Transmission shaft from gear two to arm
    \begin{scope}[xshift=4.09cm,yshift=2.42cm]
      \draw[fill=black!50] (0,0) rectangle (1.06,0.1);
    \end{scope}

    % Angular velocity arrow between transmission and arm
    \draw[line width=0.7pt,->] (4.54,1.97) arc (-30:30:1) node[above]
      {$\omega_{arm}$};

    % Descriptions of subsystems under graphic
    \begin{scope}[xshift=-0.5cm,yshift=-0.28cm]
      \draw[decorate,decoration={brace,amplitude=10pt}]
        (3.5,0) -- (0,0) node[midway,yshift=-20pt] {circuit};
      \draw[decorate,decoration={brace,amplitude=10pt}]
        (6.55,0) -- (3.75,0) node[midway,yshift=-20pt] {mechanics};
    \end{scope}
  \end{tikzpicture}

  \caption{Single-jointed arm system diagram}
  \label{fig:single_jointed_arm}
\end{bookfigure}

Gear ratios are written as output over input, so $G$ is greater than one in
figure \ref{fig:single_jointed_arm}.

We want to derive an equation for the arm angular acceleration
$\dot{\omega}_{arm}$ given an input voltage $V$, which we can integrate to get
arm angular velocity and angle.

We will start with the equation derived earlier for a DC brushed motor, equation
\eqref{eq:motor_tau_V}.
\begin{align}
  V &= \frac{\tau_m}{K_t} R + \frac{\omega_m}{K_v} \nonumber
  \intertext{Solve for the angular acceleration. First, we'll rearrange the
    terms because from inspection, $V$ is the \gls{model} \gls{input},
    $\omega_m$ is the \gls{state}, and $\tau_m$ contains the angular
    acceleration.}
  V &= \frac{R}{K_t} \tau_m + \frac{1}{K_v} \omega_m \nonumber
  \intertext{Solve for $\tau_m$.}
  V &= \frac{R}{K_t} \tau_m + \frac{1}{K_v} \omega_m \nonumber \\
  \frac{R}{K_t} \tau_m &= V - \frac{1}{K_v} \omega_m \nonumber \\
  \tau_m &= \frac{K_t}{R} V - \frac{K_t}{K_v R} \omega_m
  \intertext{Since $\tau_m G = \tau_{arm}$ and $\omega_m = G \omega_{arm}$,}
  \left(\frac{\tau_{arm}}{G}\right) &= \frac{K_t}{R} V -
    \frac{K_t}{K_v R} (G \omega_{arm}) \nonumber \\
  \frac{\tau_{arm}}{G} &= \frac{K_t}{R} V - \frac{G K_t}{K_v R} \omega_{arm}
    \nonumber \\
  \tau_{arm} &= \frac{G K_t}{R} V - \frac{G^2 K_t}{K_v R} \omega_{arm}
    \label{eq:tau_arm}
  \intertext{The torque applied to the arm is defined as}
  \tau_{arm} &= J \dot{\omega}_{arm} \label{eq:tau_arm_def}
  \intertext{where $J$ is the moment of inertia of the arm and
    $\dot{\omega}_{arm}$ is the angular acceleration. Substitute equation
    \eqref{eq:tau_arm_def} into equation \eqref{eq:tau_arm}.}
  (J \dot{\omega}_{arm}) &= \frac{G K_t}{R} V - \frac{G^2 K_t}{K_v R}
    \omega_{arm} \nonumber \\
  \dot{\omega}_{arm} &= -\frac{G^2 K_t}{K_v RJ} \omega_{arm} +
    \frac{G K_t}{RJ} V \nonumber
  \intertext{We'll relabel $\omega_{arm}$ as $\omega$ for convenience.}
  \dot{\omega} &= -\frac{G^2 K_t}{K_v RJ} \omega + \frac{G K_t}{RJ} V
    \label{eq:dot_omega_arm}
\end{align}

This model will be converted to state-space notation in section
\ref{sec:ss_model_single-jointed_arm}.

\subsection{Calculating constants}

\subsubsection{Moment of inertia J}

Given the simplicity of this mechanism, it may be easier to compute this value
theoretically using material properties in CAD. $J$ can also be approximated as
the moment of inertia of a thin rod rotating around one end. Therefore
\begin{equation}
  J = \frac{1}{3}ml^2
\end{equation}

where $m$ is the mass of the arm and $l$ is the length of the arm. Otherwise, a
procedure for measuring it experimentally is presented below.

First, rearrange equation \eqref{eq:dot_omega_arm} into the form $y = mx + b$
such that $J$ is in the numerator of $m$.
\begin{align}
  \dot{\omega} &= -\frac{G^2 K_t}{K_v RJ} \omega + \frac{G K_t}{RJ} V \nonumber
    \\
  J\dot{\omega} &= -\frac{G^2 K_t}{K_v R} \omega + \frac{G K_t}{R} V \nonumber
  \intertext{Multiply by $\frac{K_v R}{G^2 K_t}$ on both sides.}
  \frac{J K_v R}{G^2 K_t} \dot{\omega} &= -\omega + \frac{G K_t}{R} \cdot
    \frac{K_v R}{G^2 K_t} V \nonumber \\
  \frac{J K_v R}{G^2 K_t} \dot{\omega} &= -\omega + \frac{K_v}{G} V \nonumber \\
  \omega &= -\frac{J K_v R}{G^2 K_t} \dot{\omega} + \frac{K_v}{G} V
    \label{eq:arm_J_regression}
\end{align}

The test procedure is as follows.
\begin{enumerate}
  \item Orient the arm such that its axis of rotation is aligned with gravity
    (i.e., the arm is on its side). This avoids gravity affecting the
    measurements.
  \item Run the arm forward at a constant voltage. Record the angular velocity
    over time.
  \item Compute the angular acceleration from the angular velocity data as the
    difference between each sample divided by the time between them.
  \item Perform a linear regression of angular velocity versus angular
    acceleration. The slope of this line has the form $-\frac{J K_v R}{G^2 K_t}$
    as per equation \eqref{eq:arm_J_regression}.
  \item Multiply the slope by $-\frac{G^2 K_t}{K_v R}$ to obtain a least squares
    estimate of $J$.
\end{enumerate}

\section{Controllability and observability}

\subsection{Controllability matrix}

\index{controller design!controllability}
A \gls{system} is controllable if it can be steered from any \gls{state} to any
\gls{state} by a finite sequence of admissible \glspl{input}.

The controllability matrix can be used to determine if a system is controllable.
\begin{theorem}[Controllability]
  A continuous \gls{time-invariant} linear state-space \gls{model} is
  controllable if and only if
  \begin{equation}
    \rank\left(
    \begin{bmatrix}
      \mat{B} & \mat{A}\mat{B} & \cdots & \mat{A}^{n-1}\mat{B}
    \end{bmatrix}
    \right) = n
    \label{eq:ctrl_rank}
  \end{equation}

  where rank is the number of linearly independent rows in a matrix and $n$ is
  the number of \glspl{state}.
\end{theorem}

The controllability matrix in equation \eqref{eq:ctrl_rank} being rank-deficient
means the \glspl{input} cannot apply transforms along all axes in the
state-space; the transformation the matrix represents is collapsed into a lower
dimension.

The condition number of the controllability matrix $\mathcal{C}$ is defined as
$\frac{\sigma_{max}(\mathcal{C})}{\sigma_{min}(\mathcal{C})}$ where
$\sigma_{max}$ is the maximum singular
value\footnote{\label{footn:singular_val}Singular values are a generalization of
eigenvalues for nonsquare matrices.} and $\sigma_{min}$ is the minimum singular
value. As this number approaches infinity, one or more of the \glspl{state}
becomes uncontrollable. This number can also be used to tell us which actuators
are better than others for the given \gls{system}; a lower condition number
means that the actuators have more control authority.

\subsection{Controllability Gramian}
\index{controller design!controllability Gramian}

While the rank of the observability matrix can tell us whether the system is
controllable, it won't tell us which specific states are controllable or how
controllable. The controllability Gramian can be used to determine these things.

If $\mat{A}$ is stable, the controllability Gramian $\mat{W}_c$ is the unique
solution to the following continuous Lyapunov equation.
\begin{equation}
  \mat{A}\mat{W}_c + \mat{W}_c\mat{A}\T + \mat{B}\mat{B}\T = 0
\end{equation}

Alternatively,
\begin{equation}
  \mat{W}_c =
    \int_0^\infty e^{\mat{A}\tau} \mat{B}\mat{B}\T e^{\mat{A}\T\tau} \,d\tau
\end{equation}

If the solution is positive definite, the system is controllable. The
eigenvalues of $\mat{W}_c$ represent how controllable their respective states
are (larger means more controllable).

\subsection{Controllability of specific states}

If you want to know if a specific state is controllable, first find its
corresponding eigenvalue $\lambda$ in $\mat{A}$. Then, that state is
controllable if
\begin{equation}
  \rank\left(
  \begin{bmatrix}
    \mat{A} - \lambda\mat{I} & \mat{B}
  \end{bmatrix}\right) = n
\end{equation}

where $n$ is the number of \glspl{state}.

\subsection{Stabilizability}
\index{controller design!stabilizability}

Stabilizability is a weaker form of controllability. A system is considered
stabilizable if one of the following conditions is true:
\begin{enumerate}
  \item All uncontrollable states can be stabilized
  \item All unstable states are controllable
\end{enumerate}

\subsection{Observability matrix}

\index{observer design!observability}
A \gls{system} is observable if the \gls{state}, whatever it may be, can be
inferred from a finite sequence of \glspl{output}.

Observability and controllability are mathematical duals; controllability proves
that a sequence of \glspl{input} exists that drives the \gls{system} to any
\gls{state}, and observability proves that a sequence of \glspl{output} exists
that drives the \gls{state} estimate to any true \gls{state}.

The observability matrix can be used to determine if a system is observable.
\begin{theorem}[Observability]
  A continuous \gls{time-invariant} linear state-space \gls{model} is observable
  if and only if
  \begin{equation}
    \rank\left(
    \begin{bmatrix}
      \mat{C} \\
      \mat{C}\mat{A} \\
      \vdots \\
      \mat{C}\mat{A}^{n-1}
    \end{bmatrix}\right) = n \label{eq:obsv_rank}
  \end{equation}

  where rank is the number of linearly independent rows in a matrix and $n$ is
  the number of \glspl{state}.
\end{theorem}

The observability matrix in equation \eqref{eq:obsv_rank} being rank-deficient
means the \glspl{output} do not contain contributions from every \gls{state}.
That is, not all \glspl{state} are mapped to a linear combination in the
\gls{output}. Therefore, the \glspl{output} alone are insufficient to estimate
all the \glspl{state}.

The condition number of the observability matrix $\mathcal{O}$ is defined as
$\frac{\sigma_{max}(\mathcal{O})}{\sigma_{min}(\mathcal{O})}$ where
$\sigma_{max}$ is the maximum singular value\footref{footn:singular_val} and
$\sigma_{min}$ is the minimum singular value. As this number approaches
infinity, one or more of the \glspl{state} becomes unobservable. This number can
also be used to tell us which sensors are better than others for the given
\gls{system}; a lower condition number means the \glspl{output} produced by the
sensors are better indicators of the \gls{system} \gls{state}.

\subsection{Observability Gramian}
\index{observer design!observability Gramian}

While the rank of the observability matrix can tell us whether the system is
observable, it won't tell us which specific states are observable or how
observable. The observability Gramian can be used to determine these things.

If $\mat{A}$ is stable, the observability Gramian $\mat{W}_o$ is the unique
solution to the following continuous Lyapunov equation.
\begin{equation}
  \mat{A}\T\mat{W}_o + \mat{W}_o\mat{A} + \mat{C}\T\mat{C} = 0
\end{equation}

Alternatively,
\begin{equation}
  \mat{W}_o =
    \int_0^\infty e^{\mat{A}\T\tau} \mat{C}\T\mat{C} e^{\mat{A}\tau} \,d\tau
\end{equation}

If the solution is positive definite, the system is observable. The eigenvalues
of $\mat{W}_o$ represent how observable their respective states are (larger
means more observable).

\subsection{Observability of specific states}

If you want to know if a specific state is observable, first find its
corresponding eigenvalue $\lambda$ in $\mat{A}$. Then, that state is
observable if
\begin{equation}
  \rank\left(
  \begin{bmatrix}
    \mat{A} - \lambda\mat{I} \\
    \mat{C}
  \end{bmatrix}\right) = n
\end{equation}

where $n$ is the number of \glspl{state}.

\subsection{Detectability}
\index{observer design!detectability}

Detectability is a weaker form of observability. A system is considered
detectable if one of the following conditions is true:
\begin{enumerate}
  \item All unobservable states are stable
  \item All unstable states are observable
\end{enumerate}

