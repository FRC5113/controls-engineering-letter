\chapterimage{ss-controllers.jpg}{Night sky above Dufour Street in Santa Cruz, CA}

\chapter{State-space controllers}

\begin{remark}
  Chapters from here on use the \texttt{frccontrol} Python package to
  demonstrate the concepts discussed and perform the complex math required. See
  appendix \ref{ch:installing_python_packages} for how to install it.
\end{remark}

When we want to command a \gls{system} to a set of \glspl{state}, we design a
controller with certain \glspl{control law} to do it. PID controllers use the
system \glspl{output} with proportional, integral, and derivative
\glspl{control law}. In state-space, we also have knowledge of the system
\glspl{state} so we can do better.

Modern control theory uses state-space representation to model and control
systems. State-space representation models \glspl{system} as a set of
\gls{state}, \gls{input}, and \gls{output} variables related by first-order
differential equations that describe how the \gls{system}'s \gls{state} changes
over time given the current \glspl{state} and \glspl{input}.

\renewcommand*{\chapterpath}{\partpath/ss-controllers}
\section{From PID control to model-based control}
\index{PID control}

As mentioned before, controls engineers have a more general framework to
describe control theory than just PID control. PID controller designers are
focused on fiddling with controller parameters relating to the current, past,
and future \gls{error} rather than the underlying system \glspl{state}. Integral
control is a commonly used tool, and some people use integral action as the
majority of the control action. While this approach works in a lot of
situations, it is an incomplete view of the world.

Model-based control has a completely different mindset. Controls designers using
model-based control care about developing an accurate \gls{model} of the
\gls{system}, then driving the \glspl{state} they care about to zero (or to a
\gls{reference}). Integral control is added with $u_{error}$ estimation if
needed to handle \gls{model} uncertainty, but we prefer not to use it because
its response is hard to tune and some of its destabilizing dynamics aren't
visible during simulation.

\section{What is a dynamical system?}

A dynamical system is a \gls{system} whose motion varies according to a set of
differential equations. A dynamical system is considered \textit{linear} if the
differential equations describing its dynamics consist only of linear operators.
Linear operators are things like constant gain multiplications, derivatives, and
integrals. You can define reasonably accurate linear \glspl{model} for pretty
much everything you'll see in FRC with just those relations.

But let's say you have a DC brushed motor hooked up to a power supply and you
applied a constant voltage to it from rest. The motor approaches a steady-state
angular velocity, but the shape of the angular velocity curve over time isn't a
line. In fact, it's a decaying exponential curve akin to
\begin{equation*}
  \omega = \omega_{max}\left(1 - e^{-t}\right)
\end{equation*}

where $\omega$ is the angular velocity and $\omega_{max}$ is the maximum angular
velocity. If DC brushed motors are said to behave linearly, then why is this?

Linearity refers to a \gls{system}'s equations of motion, not its time domain
response. The equation defining the motor's change in angular velocity over time
looks like
\begin{equation*}
  \dot{\omega} = -a\omega + bV
\end{equation*}

where $\dot{\omega}$ is the derivative of $\omega$ with respect to time, $V$ is
the input voltage, and $a$ and $b$ are constants specific to the motor. This
equation, unlike the one shown before, is actually linear because it only
consists of multiplications and additions relating the \gls{input} $V$ and
current \gls{state} $\omega$.

Also of note is that the relation between the input voltage and the angular
velocity of the output shaft is a linear regression. You'll see why if you model
a DC brushed motor as a voltage source and generator producing back-EMF (in the
equation above, $bV$ corresponds to the voltage source and $-a\omega$
corresponds to the back-EMF). As you increase the input voltage, the back-EMF
increases linearly with the motor's angular velocity. If there was a friction
term that varied with the angular velocity squared (air resistance is one
example), the relation from input to output would be a curve. Friction that
scales with just the angular velocity would result in a lower maximum angular
velocity, but because that term can be lumped into the back-EMF term, the
response is still linear.

\section{State-space notation}

\subsection{What is state-space?}

Recall from last chapter that 2D space has two axes: $x$ and $y$. We represent
locations within this space as a pair of numbers packaged in a vector, and each
coordinate is a measure of how far to move along the corresponding axis.
State-space is a Cartesian coordinate system with an axis for each \gls{state}
variable, and we represent locations within it the same way we do for 2D space:
with a list of numbers in a vector. Each element in the vector corresponds to a
\gls{state} of the \gls{system}.

In addition to the \gls{state}, \glspl{input} and \glspl{output} are represented
as vectors. Since the mapping from the current \glspl{state} and \glspl{input}
to the change in \gls{state} is a system of equations, it's natural to write it
in matrix form.

\subsection{Benefits over classical control}

State-space notation provides a more convenient and compact way to model and
analyze \glspl{system} with multiple \glspl{input} and \glspl{output}. For a
\gls{system} with $p$ \glspl{input} and $q$ \glspl{output}, we would have to
write $q \times p$ Laplace transforms to represent it. Not only is the resulting
algebra unwieldy, but it only works for linear \glspl{system}. Including nonzero
initial conditions complicates the algebra even more. State-space representation
uses the time domain instead of the Laplace domain, so it can model nonlinear
\glspl{system}\footnote{This book focuses on analysis and control of linear
\glspl{system}. See chapter \ref{ch:nonlinear_control} for more on nonlinear
control.} and trivially supports nonzero initial conditions.

Students are still taught classical control first because it provides a
framework within which to understand the results we get from the fancy
mathematical machinery of modern control.

\subsection{Definition}

Below are the continuous and discrete versions of state-space notation.

\begin{definition}[State-space notation]%
  \index{State-space controllers!open-loop}

  \begin{align}
    \dot{\mtx{x}} &= \mtx{A}\mtx{x} + \mtx{B}\mtx{u} \label{eq:ss_ctrl_x} \\
    \mtx{y} &= \mtx{C}\mtx{x} + \mtx{D}\mtx{u} \label{eq:ss_ctrl_y} \\
    \nonumber \\
    \mtx{x}_{k+1} &= \mtx{A}\mtx{x}_k + \mtx{B}\mtx{u}_k \label{eq:ssz_ctrl_x} \\
    \mtx{y}_{k+1} &= \mtx{C}\mtx{x}_k + \mtx{D}\mtx{u}_k \label{eq:ssz_ctrl_y}
  \end{align}

  \begin{figurekey}
    \begin{tabulary}{\linewidth}{LLLL}
      $\mtx{A}$ & system matrix      & $\mtx{x}$ & state vector \\
      $\mtx{B}$ & input matrix       & $\mtx{u}$ & input vector \\
      $\mtx{C}$ & output matrix      & $\mtx{y}$ & output vector \\
      $\mtx{D}$ & feedthrough matrix &  &  \\
    \end{tabulary}
  \end{figurekey}
\end{definition}

\begin{booktable}
  \begin{tabular}{|ll|ll|}
    \hline
    \rowcolor{headingbg}
    \textbf{Matrix} & \textbf{Rows $\times$ Columns} &
    \textbf{Matrix} & \textbf{Rows $\times$ Columns} \\
    \hline
    $\mtx{A}$ & states $\times$ states & $\mtx{x}$ & states $\times$ 1 \\
    $\mtx{B}$ & states $\times$ inputs & $\mtx{u}$ & inputs $\times$ 1 \\
    $\mtx{C}$ & outputs $\times$ states & $\mtx{y}$ & outputs $\times$ 1 \\
    $\mtx{D}$ & outputs $\times$ inputs &  &  \\
    \hline
  \end{tabular}
  \caption{State-space matrix dimensions}
  \label{tab:ss_matrix_dims}
\end{booktable}

In the continuous case, the change in \gls{state} and the \gls{output} are
linear combinations of the \gls{state} vector and the \gls{input} vector. The
$\mtx{A}$ and $\mtx{B}$ matrices are used to map the \gls{state} vector
$\mtx{x}$ and the \gls{input} vector $\mtx{u}$ to a change in the \gls{state}
vector $\dot{\mtx{x}}$. The $\mtx{C}$ and $\mtx{D}$ matrices are used to map the
\gls{state} vector $\mtx{x}$ and the \gls{input} vector $\mtx{u}$ to an
\gls{output} vector $\mtx{y}$.

\section{Controllability}
\index{Controller design!controllability}

\Gls{state} controllability implies that it is possible -- by admissible inputs
-- to steer the \glspl{state} from any initial value to any final value within
some finite time window.

\begin{theorem}[Controllability]
  A continuous \gls{time-invariant} linear state-space \gls{model} is
  controllable if and only if

  \begin{equation}
    \rank\left(
    \begin{bmatrix}
      \mtx{B} & \mtx{A}\mtx{B} & \mtx{A}^2\mtx{B} & \cdots &
      \mtx{A}^{n-1}\mtx{B}
    \end{bmatrix}
    \right) = n
    \label{eq:ctrl_rank}
  \end{equation}

  where rank is the number of linearly independent rows in a matrix and $n$ is
  the number of \gls{state} variables.
\end{theorem}

The matrix in equation \eqref{eq:ctrl_rank} being rank-deficient means the
\glspl{input} cannot apply transforms along all axes in the state-space; the
transformation the matrix represents is collapsed into a lower dimension.

The condition number of the controllability matrix $\mathbb{C}$ is defined as
$\frac{\sigma_{max}(\mathbb{C})}{\sigma_{min}(\mathbb{C})}$ where $\sigma_{max}$
is the maximum singular value\footnote{\label{footn:singular_val}Singular values
are a generalization of eigenvalues for nonsquare matrices.} and $\sigma_{min}$
is the minimum singular value. As this number approaches infinity, one or more
of the \glspl{state} becomes uncontrollable. This number can also be used to
tell us which actuators are better than others for the given \gls{system}; a
lower condition number means that the actuators have more control authority.

\section{Observability}
\index{controller design!observability}

Observability is a measure for how well internal \glspl{state} of a \gls{system}
can be inferred by knowledge of its external \glspl{output}. The observability
and controllability of a \gls{system} are mathematical duals (i.e., as
controllability proves that an \gls{input} is available that brings any initial
\gls{state} to any desired final \gls{state}, observability proves that knowing
enough \gls{output} values provides enough information to predict the initial
\gls{state} of the \gls{system}).
\begin{theorem}[Observability]
  A continuous \gls{time-invariant} linear state-space \gls{model} is observable
  if and only if
  \begin{equation}
    \rank\left(
    \begin{bmatrix}
      C \\
      CA \\
      \vdots \\
      CA^{n-1}
    \end{bmatrix}\right) = n \label{eq:obsv_rank}
  \end{equation}

  where rank is the number of linearly independent rows in a matrix and $n$ is
  the number of \gls{state} variables.
\end{theorem}

The matrix in equation \eqref{eq:obsv_rank} being rank-deficient means the
\glspl{output} do not contain contributions from every \gls{state}. That is, not
all \glspl{state} are mapped to a linear combination in the \gls{output}.
Therefore, the \glspl{output} alone are insufficient to estimate all the
\glspl{state}.

The condition number of the observability matrix $\mathcal{O}$ is defined as
$\frac{\sigma_{max}(\mathcal{O})}{\sigma_{min}(\mathcal{O})}$ where
$\sigma_{max}$ is the maximum singular value\footref{footn:singular_val} and
$\sigma_{min}$ is the minimum singular value. As this number approaches
infinity, one or more of the \glspl{state} becomes unobservable. This number can
also be used to tell us which sensors are better than others for the given
\gls{system}; a lower condition number means the \glspl{output} produced by the
sensors are better indicators of the \gls{system} \gls{state}.

\section{Closed-loop controller}

With the \gls{control law} $\mtx{u} = \mtx{K}(\mtx{r} - \mtx{x})$, we can derive
the closed-loop state-space equations. We'll discuss where this
\gls{control law} comes from in subsection \ref{sec:lqr}.

First is the \gls{state} update equation. Substitute the \gls{control law} into
equation (\ref{eq:ss_ctrl_x}).

\begin{align}
  \dot{\mtx{x}} &= \mtx{A}\mtx{x} + \mtx{B}\mtx{K}(\mtx{r} - \mtx{x}) \nonumber
    \\
  \dot{\mtx{x}} &= \mtx{A}\mtx{x} + \mtx{B}\mtx{K}\mtx{r} -
    \mtx{B}\mtx{K}\mtx{x} \nonumber \\
  \dot{\mtx{x}} &= (\mtx{A} - \mtx{B}\mtx{K})\mtx{x} + \mtx{B}\mtx{K}\mtx{r}
\end{align}

Now for the \gls{output} equation. Substitute the \gls{control law} into
equation (\ref{eq:ss_ctrl_y}).

\begin{align}
  \mtx{y} &= \mtx{C}\mtx{x} + \mtx{D}(\mtx{K}(\mtx{r} - \mtx{x})) \nonumber \\
  \mtx{y} &= \mtx{C}\mtx{x} + \mtx{D}\mtx{K}\mtx{r} - \mtx{D}\mtx{K}\mtx{x}
    \nonumber \\
  \mtx{y} &= (\mtx{C} - \mtx{D}\mtx{K})\mtx{x} + \mtx{D}\mtx{K}\mtx{r}
\end{align}

Now, we'll do the same for the discrete \gls{system}. We'd like to know whether
the \gls{system} defined by equation (\ref{eq:ssz_ctrl_x}) operating with the
\gls{control law} $\mtx{u}_k = \mtx{K}(\mtx{r}_k - \mtx{x}_k)$ converges to the
\gls{reference} $\mtx{r}_k$.

\begin{align*}
  \mtx{x}_{k+1} &= \mtx{A}\mtx{x}_k + \mtx{B}\mtx{u}_k \\
  \mtx{x}_{k+1} &= \mtx{A}\mtx{x}_k + \mtx{B}(\mtx{K}(\mtx{r}_k - \mtx{x}_k)) \\
  \mtx{x}_{k+1} &= \mtx{A}\mtx{x}_k + \mtx{B}\mtx{K}\mtx{r}_k -
    \mtx{B}\mtx{K}\mtx{x}_k \\
  \mtx{x}_{k+1} &= \mtx{A}\mtx{x}_k - \mtx{B}\mtx{K}\mtx{x}_k +
    \mtx{B}\mtx{K}\mtx{r}_k \\
  \mtx{x}_{k+1} &= (\mtx{A} - \mtx{B}\mtx{K})\mtx{x}_k + \mtx{B}\mtx{K}\mtx{r}_k
\end{align*}

\begin{theorem}[Closed-loop state-space controller]
  \index{State-space controllers!closed-loop}

  \begin{align}
    \dot{\mtx{x}} &= (\mtx{A} - \mtx{B}\mtx{K})\mtx{x} + \mtx{B}\mtx{K}\mtx{r}
      \label{eq:s_ref_ctrl_x} \\
    \mtx{y} &= (\mtx{C} - \mtx{D}\mtx{K})\mtx{x} + \mtx{D}\mtx{K}\mtx{r}
      \label{eq:s_ref_ctrl_y}
  \end{align}

  \begin{align}
    \mtx{x}_{k+1} &= (\mtx{A} - \mtx{B}\mtx{K})\mtx{x}_k +
      \mtx{B}\mtx{K}\mtx{r}_k \label{eq:z_ref_ctrl_x} \\
    \mtx{y}_k &= (\mtx{C} - \mtx{D}\mtx{K})\mtx{x}_k + \mtx{D}\mtx{K}\mtx{r}_k
      \label{eq:z_ref_ctrl_y}
  \end{align}

  \begin{figurekey}
    \begin{tabular}{llll}
      $\mtx{A}$ & system matrix      & $\mtx{K}$ & controller gain matrix \\
      $\mtx{B}$ & input matrix       & $\mtx{x}$ & state vector \\
      $\mtx{C}$ & output matrix      & $\mtx{r}$ & \gls{reference} vector \\
      $\mtx{D}$ & feedthrough matrix & $\mtx{y}$ & output vector \\
    \end{tabular}
  \end{figurekey}
\end{theorem}

\begin{booktable}
  \begin{tabular}{|ll|ll|}
    \hline
    \rowcolor{headingbg}
    \textbf{Matrix} & \textbf{Rows $\times$ Columns} &
    \textbf{Matrix} & \textbf{Rows $\times$ Columns} \\
    \hline
    $\mtx{A}$ & states $\times$ states & $\mtx{x}$ & states $\times$ 1 \\
    $\mtx{B}$ & states $\times$ inputs & $\mtx{u}$ & inputs $\times$ 1 \\
    $\mtx{C}$ & outputs $\times$ states & $\mtx{y}$ & outputs $\times$ 1 \\
    $\mtx{D}$ & outputs $\times$ inputs & $\mtx{r}$ & states $\times$ 1 \\
    $\mtx{K}$ & inputs $\times$ states &  &  \\
    \hline
  \end{tabular}
  \caption{Controller matrix dimensions}
  \label{tab:ctrl_matrix_dims}
\end{booktable}

\index{Stability!eigenvalues}
Instead of commanding the \gls{system} to a \gls{state} using the vector
$\mtx{u}$ directly, we can now specify a vector of desired \glspl{state} through
$\mtx{r}$ and the \gls{controller} will choose values of $\mtx{u}$ for us over
time to make the \gls{system} converge to the \gls{reference}. For equation
(\ref{eq:s_ref_ctrl_x}) to reach steady-state, the eigenvalues of
$\mtx{A} - \mtx{B}\mtx{K}$ must be in the left-half plane. For equation
(\ref{eq:z_ref_ctrl_x}) to have a bounded output, the eigenvalues of
$\mtx{A} - \mtx{B}\mtx{K}$ must be within the unit circle.

The eigenvalues of $\mtx{A} - \mtx{B}\mtx{K}$ are the poles of the closed-loop
\gls{system}. Therefore, the rate of convergence and stability of the
closed-loop \gls{system} can be changed by moving the poles via the eigenvalues
of $\mtx{A} - \mtx{B}\mtx{K}$. $\mtx{A}$ and $\mtx{B}$ are inherent to the
\gls{system}, but $\mtx{K}$ can be chosen arbitrarily by the controller
designer.

\section{Pole placement}
\index{Controller design!pole placement}

This is the practice of placing the poles of a closed-loop \gls{system} directly
to produce a desired response. Python Control offers several pole placement
algorithms for generating controller or observer gains from a set of poles. In
general, pole placement should only be used if you know what you're doing. It's
much easier to let LQR place the poles for you, which we'll discuss next.

\section{Linear-quadratic regulator} \label{sec:lqr}
\index{controller design!linear-quadratic regulator}
\index{optimal control!linear-quadratic regulator}

\subsection{The intuition}

We can demonstrate the basic idea behind the linear-quadratic regulator with the
following flywheel model.
\begin{equation*}
  \dot{x} = ax + bu
\end{equation*}

where $a$ is a negative constant representing the back-EMF of the motor, $x$ is
the angular velocity, $b$ is a positive constant that maps the input voltage to
some change in angular velocity (angular acceleration), $u$ is the voltage
applied to the motor, and $\dot{x}$ is the angular acceleration. Discretized,
this equation would look like
\begin{equation*}
  x_{k+1} = a_d x + b_d u_k
\end{equation*}

If the angular velocity starts from zero and we apply a positive voltage, we'd
see the motor spin up to some constant speed following an exponential decay,
then stay at that speed. If we throw in the control law $u_k = k_p(r_k - x_k)$,
we can make the system converge to a desired state $r_k$ through proportional
feedback. In what manner can we pick the constant $k_p$ that balances getting to
the target angular velocity quickly with getting there efficiently (minimal
oscillations or excessive voltage)?

We can solve this problem with something called the linear-quadratic regulator.
We'll define the following cost function that includes the states and inputs:
\begin{equation*}
  J = \sum_{k=0}^\infty (Q(r_k - x_k)^2 + Ru_k^2)
\end{equation*}

We want to minimize this while obeying the constraint that the system follow our
flywheel dynamics $x_{k+1} = a_d x_k + b_d u_k$.

The cost is the sum of the squares of the error and the input for all time. If
the controller gain $k_p$ we pick in the control law $u_k = k_p(r_k - x_k)$ is
stable, the error $r_k - x_k$ and the input $u_k$ will both go to zero and give
us a finite cost. $Q$ and $R$ let us decide how much the error and input
contribute to the cost (we will require that $Q \geq 0$ and $R > 0$ for reasons
that will be clear shortly\footnote{Lets consider the boundary conditions on the
weights $Q$ and $R$. If we set $Q$ to zero, error doesn't contribute to the
cost, so the optimal solution is to not move. This minimizes the sum of the
inputs over time. If we let $R$ be zero, the input doesn't contribute to the
cost, so infinite inputs are allowed as they minimize the sum of the errors over
time. This isn't useful, so we require that the input be penalized with a
nonzero $R$.}). Penalizing error more will make the controller more aggressive,
while penalizing the input more will make the controller less aggressive. We
want to pick a $k_p$ that minimizes the cost.

There's a common trick for finding the value of a variable that minimizes a
function of that variable. We'll take the derivative (the slope) of the cost
function with respect to the input $u_k$, set the derivative to zero, then solve
for $u_k$. When the slope is zero, the function is at a minimum or maximum. Now,
the cost function we picked is quadratic. All the terms are strictly positive on
account of the squared variables and nonnegative weights, so our cost is
strictly positive and the quadratic function is concave up. The $u_k$ we found
is therefore a minimum.

The actual process of solving for $u_k$ is mathematically intensive and outside
the scope of this explanation (appendix \ref{ch:deriv_lqr} references a
derivation for those curious). The rest of this section will describe the more
general form of the linear-quadratic regulator and how to use it.

\subsection{The mathematical definition}

Instead of placing the poles of a closed-loop \gls{system} manually, the
linear-quadratic regulator (LQR) places the poles for us based on acceptable
relative \gls{error} and \gls{control effort} costs. This method of controller
design uses a quadratic function for the cost-to-go defined as the sum of the
\gls{error} and \gls{control effort} over time for the linear \gls{system}
$\mat{x}_{k+1} = \mat{A}\mat{x}_k + \mat{B}\mat{u}_k$.
\begin{equation*}
  J = \sum_{k=0}^\infty \left(\mat{x}_k\T\mat{Q}\mat{x}_k +
    \mat{u}_k\T\mat{R}\mat{u}_k\right)
\end{equation*}

where $J$ represents a trade-off between \gls{state} excursion and
\gls{control effort} with the weighting factors $\mat{Q}$ and $\mat{R}$. LQR is
a \gls{control law} $\mat{u}$ that minimizes the cost function. $\mat{Q}$ and
$\mat{R}$ slide the cost along a Pareto boundary between state tracking and
\gls{control effort} (see figure \ref{fig:pareto_boundary}). Pareto optimality
for this problem means that an improvement in state \gls{tracking} cannot be
obtained without using more \gls{control effort} to do so. Also, a reduction in
\gls{control effort} cannot be obtained without sacrificing state \gls{tracking}
performance. Pole placement, on the other hand, will have a cost anywhere on,
above, or to the right of the Pareto boundary (no cost can be inside the
boundary).
\begin{svg}{build/\chapterpath/pareto_boundary}
  \caption{Pareto boundary for LQR}
  \label{fig:pareto_boundary}
\end{svg}

The minimum of LQR's cost function is found by setting the derivative of the
cost function to zero and solving for the \gls{control law} $\mat{u}_k$.
However, matrix calculus is used instead of normal calculus to take the
derivative.

The feedback \gls{control law} that minimizes $J$ is shown in theorem
\ref{thm:linear-quadratic_regulator}.
\begin{theorem}[Linear-quadratic regulator]
  \label{thm:linear-quadratic_regulator}
  \begin{align}
    \min_{\mat{u}_k} &\sum\limits_{k=0}^\infty
      \left(\mat{x}_k\T\mat{Q}\mat{x}_k + \mat{u}_k\T\mat{R}\mat{u}_k\right)
      \nonumber \\
    \text{subject to } &\mat{x}_{k+1} = \mat{A}\mat{x}_k + \mat{B}\mat{u}_k
  \end{align}

  If the \gls{system} is controllable, the optimal control policy $\mat{u}_k^*$
  that drives all the \glspl{state} to zero is $-\mat{K}\mat{x}_k$. To converge
  to nonzero \glspl{state}, a \gls{reference} vector $\mat{r}_k$ can be added to
  the \gls{state} $\mat{x}_k$.
  \begin{equation}
    \mat{u}_k = \mat{K}(\mat{r}_k - \mat{x}_k)
  \end{equation}
\end{theorem}
\index{controller design!linear-quadratic regulator!definition}
\index{optimal control!linear-quadratic regulator!definition}

This means that optimal control can be achieved with simply a set of
proportional gains on all the \glspl{state}. To use the \gls{control law}, we
need knowledge of the full \gls{state} of the \gls{system}. That means we either
have to measure all our \glspl{state} directly or estimate those we do not
measure.

See appendix \ref{ch:deriv_lqr} for how $\mat{K}$ is calculated. If the result
is finite, the controller is guaranteed to be stable and
\glslink{robustness}{robust} with a \gls{phase margin} of 60 degrees
\cite{bib:lqr_phase_margin}.
\begin{remark}
  LQR design's $\mat{Q}$ and $\mat{R}$ matrices don't need \gls{discretization},
  but the $\mat{K}$ calculated for continuous time and discrete time
  \glspl{system} will be different. The discrete time gains approach the
  continuous time gains as the sample period tends to zero.
\end{remark}

\subsection{Bryson's rule}
\index{controller design!linear-quadratic regulator!Bryson's rule}
\index{optimal control!linear-quadratic regulator!Bryson's rule}

The next obvious question is what values to choose for $\mat{Q}$ and $\mat{R}$.
While this can be more of an art than a science, Bryson's rule provides a good
starting point. With Bryson's rule, the diagonals of the $\mat{Q}$ and $\mat{R}$
matrices are chosen based on the maximum acceptable value for each \gls{state}
and actuator. The nondiagonal elements are zero. The balance between $\mat{Q}$
and $\mat{R}$ can be slid along the Pareto boundary using a weighting factor
$\rho$.
\begin{equation*}
  J = \sum_0^\infty \left(\rho \left[
    \left(\frac{x_1}{x_{1,max}}\right)^2 + \ldots +
    \left(\frac{x_m}{x_{m,max}}\right)^2\right] + \left[
    \left(\frac{u_1}{u_{1,max}}\right)^2 + \ldots +
    \left(\frac{u_n}{u_{n,max}}\right)^2\right]\right)
\end{equation*}
\begin{equation*}
  \mat{Q} = \begin{bmatrix}
    \frac{\rho}{x_{1,max}^2} & 0 & \ldots & 0 \\
    0 & \frac{\rho}{x_{2,max}^2} & & \vdots \\
    \vdots & & \ddots & 0 \\
    0 & \ldots & 0 & \frac{\rho}{x_{m,max}^2}
  \end{bmatrix}
  \quad
  \mat{R} = \begin{bmatrix}
    \frac{1}{u_{1,max}^2} & 0 & \ldots & 0 \\
    0 & \frac{1}{u_{2,max}^2} & & \vdots \\
    \vdots & & \ddots & 0 \\
    0 & \ldots & 0 & \frac{1}{u_{n,max}^2}
  \end{bmatrix}
\end{equation*}

The index subscript denotes a row of the state or input vector. Small values of
$\rho$ penalize \gls{control effort} while large values of $\rho$ penalize
\gls{state} excursions. Large values would be chosen in applications like
fighter jets where performance is necessary. Spacecrafts would use small values
to conserve their limited fuel supply.

\subsection{Pole placement vs LQR}

This example uses the following continuous second-order \gls{model} for a CIM
motor (a DC brushed motor).
\begin{align*}
  \mat{A} = \begin{bmatrix}
    -\frac{b}{J} & \frac{K_t}{J} \\
    -\frac{K_e}{L} & -\frac{R}{L}
  \end{bmatrix}
  \quad
  \mat{B} = \begin{bmatrix}
    0 \\
    \frac{1}{L}
  \end{bmatrix}
  \quad
  \mat{C} = \begin{bmatrix}
    1 & 0
  \end{bmatrix}
  \quad
  \mat{D} = \begin{bmatrix}
    0
  \end{bmatrix}
\end{align*}

Figure \ref{fig:case_study_pp_lqr} shows the response using various discrete
pole placements and using LQR with the following cost matrices.
\begin{align*}
  \mat{Q} = \begin{bmatrix}
    \frac{1}{20^2} & 0 \\
    0 & 0
  \end{bmatrix}
  \quad
  \mat{R} = \begin{bmatrix}
    \frac{1}{12^2}
  \end{bmatrix}
\end{align*}

With Bryson's rule, this means an angular velocity tolerance of $20$ rad/s, an
infinite current tolerance (in other words, we don't care what the current
does), and a voltage tolerance of $12$ V.
\begin{svg}{build/\chapterpath/case_study_pp_lqr}
  \caption{Second-order CIM motor response with pole placement and LQR}
  \label{fig:case_study_pp_lqr}
\end{svg}

Notice with pole placement that as the current pole moves toward the origin, the
\gls{control effort} becomes more aggressive.

\section{Model augmentation}

This section will teach various tricks for manipulating state-space
\glspl{model} with the goal of demystifying the matrix algebra at play. We will
use the augmentation techniques discussed here in the section on integral
control.

Matrix augmentation is the process of appending rows or columns to a matrix. In
state-space, there are several common types of augmentation used: \gls{plant}
augmentation, controller augmentation, and \gls{observer} augmentation.

\subsection{Plant augmentation}
\index{Model augmentation!of plant}

Plant augmentation is the process of adding a state to a model's state vector
and adding a corresponding row to the $\mtx{A}$ and $\mtx{B}$ matrices.

\subsection{Controller augmentation}
\index{Model augmentation!of controller}

Controller augmentation is the process of adding a column to a controller's
$\mtx{K}$ matrix. This is often done in combination with \gls{plant}
augmentation to add controller dynamics relating to a newly added \gls{state}.

\subsection{Observer augmentation}
\index{Model augmentation!of observer}

Observer augmentation is closely related to \gls{plant} augmentation. In
addition to adding entries to the \gls{observer} matrix $\mtx{L}$, the
\gls{observer} is using this augmented \gls{plant} for estimation purposes. This
is better explained with an example.

By augmenting the \gls{plant} with a bias term with no dynamics (represented by
zeroes in its rows in $\mtx{A}$ and $\mtx{B}$), the \gls{observer} will attempt
to estimate a value for this bias term that makes the \gls{model} best reflect
the measurements taken of the real \gls{system}. Note that we're not collecting
any data on this bias term directly; it's what's known as a hidden \gls{state}.
Rather than our \glspl{input} and other \glspl{state} affecting it directly, the
\gls{observer} determines a value for it based on what is most likely given the
\gls{model} and current measurements. We just tell the \gls{plant} what kind of
dynamics the term has and the \gls{observer} will estimate it for us.

\subsection{Output augmentation}
\index{Model augmentation!of output}

Output augmentation is the process of adding rows to the $\mtx{C}$ matrix. This
is done to help the controls designer visualize the behavior of internal states
or other aspects of the \gls{system} in MATLAB or Python Control. $\mtx{C}$
matrix augmentation doesn't affect \gls{state} feedback, so the designer has a
lot of freedom here. Noting that the \gls{output} is defined as
$\mtx{y} = \mtx{C}\mtx{x} + \mtx{D}\mtx{u}$, The following row augmentations of
$\mtx{C}$ may prove useful. Of course, $\mtx{D}$ needs to be augmented with
zeroes as well in these cases to maintain the correct matrix dimensionality.

Since $\mtx{u} = -\mtx{K}\mtx{x}$, augmenting $\mtx{C}$ with $-\mtx{K}$ makes
the \gls{observer} estimate the \gls{control input} $\mtx{u}$ applied.

\begin{align*}
  \mtx{y} &= \mtx{C}\mtx{x} + \mtx{D}\mtx{u} \\
  \begin{bmatrix}
    \mtx{y} \\
    \mtx{u}
  \end{bmatrix} &=
  \begin{bmatrix}
    \mtx{C} \\
    -\mtx{K}
  \end{bmatrix}
  \mtx{x} +
  \begin{bmatrix}
    \mtx{D} \\
    \mtx{0}
  \end{bmatrix}
  \mtx{u}
\end{align*}

This works because $\mtx{K}$ has the same number of columns as \glspl{state}.

Various \glspl{state} can also be produced in the \gls{output} with $\mtx{I}$
matrix augmentation.

\subsection{Examples}

Snippet \ref{lst:augment_concat} shows how one packs together the following
augmented matrix in Python using concatenation.

\begin{equation*}
  \begin{bmatrix}
    \mtx{A} & \mtx{B} \\
    \mtx{C} & \mtx{D}
  \end{bmatrix}
\end{equation*}

\begin{code}{Python}{code/snippets/augment_concat.py}
  \caption{Matrix augmentation example: concatenation}
  \label{lst:augment_concat}
\end{code}

Snippet \ref{lst:augment_slices} shows how one packs together the same augmented
matrix in Python using array slices.

\begin{code}{Python}{code/snippets/augment_slices.py}
  \caption{Matrix augmentation example: array slices}
  \label{lst:augment_slices}
\end{code}

Section \ref{sec:integral_control} demonstrates \gls{model} augmentation for
different types of integral control.

\section{Feedforward}

So far, we've used feedback control for \gls{reference} \gls{tracking} (making a
\gls{system}'s output follow a desired \gls{reference} signal). While this is
effective, it's a reactionary measure; the \gls{system} won't start applying
\gls{control effort} until the \gls{system} is already behind. If we could tell
the \gls{controller} about the desired movement and required input beforehand,
the \gls{system} could react quicker and the feedback \gls{controller} could do
less work. A \gls{controller} that feeds information forward into the
\gls{plant} like this is called a \gls{feedforward controller}.

A \gls{feedforward controller} injects information about the \gls{system}'s
dynamics (like a \gls{model} does) or the desired movement. The feedforward
handles parts of the control actions we already know must be applied to make a
\gls{system} track a \gls{reference}, then feedback compensates for what we do
not or cannot know about the \gls{system}'s behavior at runtime.

There are two types of feedforwards: model-based feedforward and feedforward for
unmodeled dynamics. The first solves a mathematical model of the system for the
inputs required to meet desired velocities and accelerations. The second
compensates for unmodeled forces or behaviors directly so the feedback
controller doesn't have to. Both types can facilitate simpler feedback
controllers; we'll cover examples of each.

\subsection{Plant inversion}
\label{subsec:plant_inversion}

\Gls{plant} inversion is a method of model-based feedforward for \gls{state}
feedback. It solves the \gls{plant} for the input that will make the \gls{plant}
track a desired state. This is called inversion because in a block diagram, the
inverted \gls{plant} feedforward and \gls{plant} cancel out to produce a unity
system from input to output.

While it can be an effective tool, the following should be kept in mind.
\begin{enumerate}
  \item Don't invert an unstable \gls{plant}. If the expected \gls{plant}
    doesn't match the real \gls{plant} exactly, the \gls{plant} inversion will
    still result in an unstable \gls{system}. Stabilize the \gls{plant} first
    with feedback, then inject an inversion.
  \item Don't invert a nonminimum phase system. The advice for pole-zero
    cancellation in subsection \ref{subsec:pole-zero_cancellation} applies here.
\end{enumerate}

\subsubsection{Necessary theorems}

The following theorem will be needed to derive the linear plant inversion
equation.
\begin{theorem}
  \label{thm:partial_xax}

  $\frac{\partial \mtx{x}^T\mtx{A}\mtx{x}}{\partial\mtx{x}} =
    2\mtx{A}\mtx{x}$ where $\mtx{A}$ is symmetric.
\end{theorem}

\subsubsection{Setup}

Let's start with the equation for the \gls{reference} dynamics
\begin{equation*}
  \mtx{r}_{k+1} = \mtx{A}\mtx{r}_k + \mtx{B}\mtx{u}_k
\end{equation*}

where $\mtx{u}_k$ is the feedforward input. Note that this feedforward equation
does not and should not take into account any feedback terms. We want to find
the optimal $\mtx{u}_k$ such that we minimize the \gls{tracking} error between
$\mtx{r}_{k+1}$ and $\mtx{r}_k$.
\begin{equation*}
  \mtx{r}_{k+1} - \mtx{A}\mtx{r}_k = \mtx{B}\mtx{u}_k
\end{equation*}

To solve for $\mtx{u}_k$, we need to take the inverse of the nonsquare matrix
$\mtx{B}$. This isn't possible, but we can find the pseudoinverse given some
constraints on the \gls{state} \gls{tracking} error and \gls{control effort}. To
find the optimal solution for these sorts of trade-offs, one can define a cost
function and attempt to minimize it. To do this, we'll first solve the
expression for $\mtx{0}$.
\begin{equation*}
  \mtx{0} = \mtx{B}\mtx{u}_k - (\mtx{r}_{k+1} - \mtx{A}\mtx{r}_k)
\end{equation*}

This expression will be the \gls{state} \gls{tracking} cost we use in the
following cost function as an $H_2$ norm.
\begin{equation*}
  \mtx{J} = (\mtx{B}\mtx{u}_k - (\mtx{r}_{k+1} - \mtx{A}\mtx{r}_k))^T
    (\mtx{B}\mtx{u}_k - (\mtx{r}_{k+1} - \mtx{A}\mtx{r}_k))
\end{equation*}

\subsubsection{Minimization}

Given theorem \ref{thm:partial_xax}, find the minimum of $\mtx{J}$ by taking the
partial derivative with respect to $\mtx{u}_k$ and setting the result to
$\mtx{0}$.
\begin{align*}
  \frac{\partial\mtx{J}}{\partial\mtx{u}_k} &= 2\mtx{B}^T
    (\mtx{B}\mtx{u}_k - (\mtx{r}_{k+1} - \mtx{A}\mtx{r}_k)) \\
  \mtx{0} &= 2\mtx{B}^T
    (\mtx{B}\mtx{u}_k - (\mtx{r}_{k+1} - \mtx{A}\mtx{r}_k)) \\
  \mtx{0} &= 2\mtx{B}^T\mtx{B}\mtx{u}_k -
    2\mtx{B}^T(\mtx{r}_{k+1} - \mtx{A}\mtx{r}_k) \\
  2\mtx{B}^T\mtx{B}\mtx{u}_k &=
    2\mtx{B}^T(\mtx{r}_{k+1} - \mtx{A}\mtx{r}_k) \\
  \mtx{B}^T\mtx{B}\mtx{u}_k &=
    \mtx{B}^T(\mtx{r}_{k+1} - \mtx{A}\mtx{r}_k) \\
  \mtx{u}_k &=
    (\mtx{B}^T\mtx{B})^{-1} \mtx{B}^T(\mtx{r}_{k+1} - \mtx{A}\mtx{r}_k)
\end{align*}

$(\mtx{B}^T\mtx{B})^{-1} \mtx{B}^T$ is the Moore-Penrose pseudoinverse of
$\mtx{B}$ denoted by $\mtx{B}^\dagger$.
\begin{theorem}[Linear plant inversion]
  \label{thm:linear_plant_inversion}

  Given the discrete model
  $\mtx{x}_{k+1} = \mtx{A}\mtx{x}_k + \mtx{B}\mtx{u}_k$, the plant inversion
  feedforward is
  \begin{equation}
    \mtx{u}_k = \mtx{B}^\dagger (\mtx{r}_{k+1} - \mtx{A}\mtx{r}_k)
  \end{equation}

  where $\mtx{B}^\dagger$ is the Moore-Penrose pseudoinverse of $\mtx{B}$,
  $\mtx{r}_{k+1}$ is the reference at the next timestep, and $\mtx{r}_k$ is the
  reference at the current timestep.
\end{theorem}
\index{feedforward!linear plant inversion}
\index{optimal control!linear plant inversion}

\subsubsection{Discussion}

Linear \gls{plant} inversion in theorem \ref{thm:linear_plant_inversion}
compensates for \gls{reference} dynamics that don't follow how the \gls{model}
inherently behaves. If they do follow the \gls{model}, the feedforward has
nothing to do as the \gls{model} already behaves in the desired manner. When
this occurs, $\mtx{r}_{k+1} - \mtx{A}\mtx{r}_k$ will return a zero vector.

For example, a constant \gls{reference} requires a feedforward that opposes
\gls{system} dynamics that would change the \gls{state} over time. If the
\gls{system} has no dynamics, then $\mtx{A} = \mtx{I}$ and thus
\begin{align*}
  \mtx{u}_k &= \mtx{B}_\dagger (\mtx{r}_{k+1} - \mtx{I}\mtx{r}_k) \\
  \mtx{u}_k &= \mtx{B}_\dagger (\mtx{r}_{k+1} - \mtx{r}_k)
\end{align*}

For a constant \gls{reference}, $\mtx{r}_{k+1} = \mtx{r}_k$.
\begin{align*}
  \mtx{u}_k &= \mtx{B}_\dagger (\mtx{r}_k - \mtx{r}_k) \\
  \mtx{u}_k &= \mtx{B}_\dagger (\mtx{0}) \\
  \mtx{u}_k &= \mtx{0}
\end{align*}

so no feedforward is required to hold a \gls{system} with no dynamics at a
constant \gls{reference}, as expected.

Figure \ref{fig:case_study_ff} shows \gls{plant} inversion applied to a
second-order CIM motor model. \Gls{plant} inversion accounts for the motor
back-EMF and eliminates steady-state error.
\begin{svg}{build/\chapterpath/case_study_ff}
  \caption{Second-order CIM motor response with plant inversion}
  \label{fig:case_study_ff}
\end{svg}

\subsection{Unmodeled dynamics}

In addition to \gls{plant} inversion, one can include feedforwards for unmodeled
dynamics. Consider an elevator model which doesn't include gravity. A constant
voltage offset can be used compensate for this. The feedforward takes the form
of a voltage constant because voltage is proportional to force applied, and the
force is acting in only one direction at all times.
\begin{equation}
  u_k = V_{app}
\end{equation}

where $V_{app}$ is a constant. Another feedforward holds a single-jointed arm
steady in the presence of gravity. It has the following form.
\begin{equation}
  u_k = V_{app} \cos\theta
\end{equation}

where $V_{app}$ is the voltage required to keep the single-jointed arm level
with the ground, and $\theta$ is the angle of the arm relative to the ground.
Therefore, the force applied is greatest when the arm is parallel with the
ground and zero when the arm is perpendicular to the ground (at that point, the
joint supports all the weight).

Note that the elevator model could be augmented easily enough to include gravity
and still be linear, but this wouldn't work for the single-jointed arm since a
trigonometric function is required to model the gravitational force in the arm's
rotating reference frame\footnote{While the applied torque of the motor is
constant throughout the arm's range of motion, the torque caused by gravity in
the opposite direction varies according to the arm's angle.}.

\section{Integral control}
\label{sec:integral_control}

A common way of implementing integral control is to add an additional
\gls{state} that is the integral of the \gls{error} of the variable intended to
have zero \gls{steady-state error}.

There are two drawbacks to this method. First, there is integral windup on a
unit \gls{step input}. That is, the integrator accumulates even if the
\gls{system} is \gls{tracking} the \gls{model} correctly. The second is
demonstrated by an example from Jared Russell of FRC team 254. Say there is a
position/velocity trajectory for some \gls{plant} to follow. Without integral
control, one can calculate a desired $\mtx{K}\mtx{x}$ to use as the
\gls{control input}. As a result of using both desired position and velocity,
\gls{reference} \gls{tracking} is good. With integral control added, the
\gls{reference} is always the desired position, but there is no way to tell the
controller the desired velocity.

Consider carefully whether integral control is necessary. One can get relatively
close without integral control, and integral adds all the issues listed above.
Below, it is assumed that the controls designer has determined that integral
control will be worth the inconvenience.

There are three methods FRC team 971 has used over the years:

\begin{enumerate}
  \item Augment the \gls{plant} as described earlier. For an arm, one would add
    an ``integral of position" state.
  \item Add an integrator to the output of the controller, then estimate the
    \gls{control effort} being applied. 971 has called this Delta U control. The
    upside is that it doesn't have the windup issue described above; the
    integrator only acts if the \gls{system} isn't behaving like the
    \gls{model}, which was the original intent. The downside is working with it
    is very confusing.
  \item Estimate the ``error" in the \gls{control input} (the difference between
    what was applied versus what was observed to happen) via the \gls{observer}
    and compensate for it.
\end{enumerate}

We'll present the first and third methods since the third is strictly better
than the second.

\subsection{Plant augmentation}
\index{Integral control!plant augmentation}

We want to augment the \gls{system} with an integral term that integrates the
\gls{error} $\mtx{e} = \mtx{r} - \mtx{y} = \mtx{r} - \mtx{C}\mtx{x}$.

\begin{align*}
  \mtx{x}_I &= \int \mtx{e} \,dt \\
  \dot{\mtx{x}}_I &= \mtx{e} = \mtx{r} - \mtx{C}\mtx{x}
\end{align*}

The \gls{plant} is augmented as

\begin{align*}
  \dot{\begin{bmatrix}
    \mtx{x} \\
    \mtx{x}_I
  \end{bmatrix}} &=
  \begin{bmatrix}
    \mtx{A} & \mtx{0} \\
    -\mtx{C} & \mtx{0}
  \end{bmatrix}
  \begin{bmatrix}
    \mtx{x} \\
    \mtx{x}_I
  \end{bmatrix} +
  \begin{bmatrix}
    \mtx{B} \\
    \mtx{0}
  \end{bmatrix}
  \mtx{u} +
  \begin{bmatrix}
    \mtx{0} \\
    \mtx{I}
  \end{bmatrix}
  \mtx{r} \\
  \dot{\begin{bmatrix}
    \mtx{x} \\
    \mtx{x}_I
  \end{bmatrix}} &=
  \begin{bmatrix}
    \mtx{A} & \mtx{0} \\
    -\mtx{C} & \mtx{0}
  \end{bmatrix}
  \begin{bmatrix}
    \mtx{x} \\
    \mtx{x}_I
  \end{bmatrix} +
  \begin{bmatrix}
    \mtx{B} & \mtx{0} \\
    \mtx{0} & \mtx{I}
  \end{bmatrix}
  \begin{bmatrix}
    \mtx{u} \\
    \mtx{r}
  \end{bmatrix}
\end{align*}

The controller is augmented as

\begin{align*}
  \mtx{u} &= \mtx{K} (\mtx{r} - \mtx{x}) - \mtx{K}_I\mtx{x}_I \\
  \mtx{u} &=
  \begin{bmatrix}
    \mtx{K} & \mtx{K}_I
  \end{bmatrix}
  \left(\begin{bmatrix}
    \mtx{r} \\
    \mtx{0}
  \end{bmatrix} -
  \begin{bmatrix}
    \mtx{x} \\
    \mtx{x}_I
  \end{bmatrix}\right)
\end{align*}

\subsection{U error estimation}
\label{subsec:u_error_estimation}
\index{Integral control!U error estimation}

Given the desired \gls{input} produced by a \gls{controller}, unmodeled
\glspl{disturbance} may cause the observed behavior of a \gls{system} to deviate
from its \gls{model}. U error estimation estimates the difference between the
desired \gls{input} and a hypothetical \gls{input} that makes the \gls{model}
match the observed behavior. This value can be added to the \gls{control input}
to make the \gls{controller} compensate for unmodeled \glspl{disturbance} and
make the \gls{model} better predict the \gls{system}'s future behavior.

Let $u_{error}$ be the difference between the \gls{input} actually applied to a
\gls{system} and the desired \gls{input}. The $u_{error}$ term is then added to
the \gls{system} as follows.

\begin{equation*}
  \dot{\mtx{x}} = \mtx{A}\mtx{x} + \mtx{B}\left(\mtx{u} + u_{error}\right)
\end{equation*}

$\mtx{u} + u_{error}$ is the hypothetical \gls{input} actually applied to the
\gls{system}.

\begin{equation*}
  \dot{\mtx{x}} = \mtx{A}\mtx{x} + \mtx{B}\mtx{u} + \mtx{B}u_{error}
\end{equation*}

For a multiple-output \gls{system}, this would be

\begin{equation*}
  \dot{\mtx{x}} = \mtx{A}\mtx{x} + \mtx{B}\mtx{u} + \mtx{B}_{error}u_{error}
\end{equation*}

where $\mtx{B}_{error}$ is the column vector that maps $u_{error}$ to changes in
the rest of the \gls{state} the same way $\mtx{B}$ does for $\mtx{u}$.
$\mtx{B}_{error}$ is only a column of $\mtx{B}$ if $u_{error}$ corresponds to an
existing \gls{input} within $\mtx{u}$.

The \gls{plant} is augmented as

\begin{align*}
  \dot{\begin{bmatrix}
    \mtx{x} \\
    u_{error}
  \end{bmatrix}} &=
  \begin{bmatrix}
    \mtx{A} & \mtx{B}_{error} \\
    \mtx{0} & \mtx{0}
  \end{bmatrix}
  \begin{bmatrix}
    \mtx{x} \\
    u_{error}
  \end{bmatrix} +
  \begin{bmatrix}
    \mtx{B} \\
    \mtx{0}
  \end{bmatrix}
  \mtx{u} \\
  \mtx{y} &= \begin{bmatrix}
    \mtx{C} & 0
  \end{bmatrix} \begin{bmatrix}
    \mtx{x} \\
    u_{error}
  \end{bmatrix} + \mtx{D}\mtx{u}
\end{align*}

With this \gls{model}, the \gls{observer} will estimate both the \gls{state} and
the $u_{error}$ term. The controller is augmented similarly. $\mtx{r}$ is
augmented with a zero for the goal $u_{error}$ term.

\begin{align*}
  \mtx{u} &= \mtx{K} \left(\mtx{r} - \mtx{x}\right) - \mtx{k}_{error}u_{error}
    \\
  \mtx{u} &=
  \begin{bmatrix}
    \mtx{K} & \mtx{k}_{error}
  \end{bmatrix}
  \left(\begin{bmatrix}
    \mtx{r} \\
    0
  \end{bmatrix} -
  \begin{bmatrix}
    \mtx{x} \\
    u_{error}
  \end{bmatrix}\right)
\end{align*}

where $\mtx{k}_{error}$ is a column vector with a $1$ in a given row if
$u_{error}$ should be applied to that \gls{input} or a $0$ otherwise.

This process can be repeated for an arbitrary \gls{error} which can be corrected
via some linear combination of the \glspl{input}.

