\section{Linear combinations, span, and basis vectors}

\begin{remark}
  See the corresponding \textit{Essence of linear algebra} video for a more
  visual presentation (10 minutes) \cite{bib:linalg_linear_combinations}.
\end{remark}

Vector coordinates were probably already familiar to you, but there's another
interesting way to think about these coordinates which is central to linear
algebra. When given a pair of numbers that's meant to describe a vector, like
$(3, 2)$, we want you to think about each coordinate as a scalar, meaning, think
about how each one stretches or compresses vectors.

\subsection{Basis vectors}
\index{Linear algebra!basis vectors}

In the xy-coordinate system, there are two special vectors: the one pointing to
the right with length $1$, commonly called ``i-hat", or the unit vector in the
x-direction ($\hat{i}$), and the one pointing straight up, with length $1$,
commonly called ``j-hat", or the unit vector in the y-direction ($\hat{j}$).

Now think of the x-coordinate of our vector as a scalar that scales $\hat{i}$,
stretching it by a factor of $3$, and the y-coordinate as a scalar that scales
$\hat{j}$, flipping it and stretching it by a factor of $2$. In this sense, the
vectors that these coordinates describe is the sum of two scaled vectors
$(3)\hat{i} + (-2)\hat{j}$. This idea of adding together two scaled vectors is a
surprisingly important concept. Those two vectors, $\hat{i}$ and $\hat{j}$, have
a special name, by the way. Together they're called the \textit{basis} of a
coordinate system ($\hat{i}$ and $\hat{j}$ are the ``basis vectors" of the
xy-coordinate system). When you think about coordinates as scalars, the basis
vectors are what those scalars actually scale.

By framing our coordinate system in terms of these two special basis vectors, it
raises an interesting and subtle point: we could have chosen different basis
vectors and had a completely reasonable, new coordinate system system. For
example, take some vector pointing up and to the right, along with some other
vector pointing down and to the right, in some way. Take a moment to think about
all the different vectors that you can get by choosing two scalars, using each
one to scale one of the vectors, then adding together what you get. Which
two-dimensional vectors can you reach by altering the choices of scalars? The
answer is that you can reach every possible two-dimensional vector. A new pair
of basis vectors like this still gives us a valid way to go back and forth
between pairs of numbers and two-dimensional vectors, but the association is
definitely different from the one that you get using the more standard basis of
$\hat{i}$ and $\hat{j}$.

\subsection{Linear combination}
\index{Linear algebra!linear combination}

Any time we describe vectors numerically, it depends on an implicit choice of
what basis vectors we're using. So any time that you're scaling two vectors and
adding them like this, it's called a \textit{linear combination} of those two
vectors. Below is a linear combination of vectors $\vec{v}$ and $\vec{w}$ with
scalars $a$ and $b$.

\begin{equation*}
  a \vec{v} + b \vec{w}
\end{equation*}

Where does this word ``linear" come from? Why does this have anything to do with
lines? This isn't the etymology, but if you fix one of those scalars and let the
other one change its value freely, the tip of the resulting vector draws a
straight line.

\subsection{Span}

Now, if you let both scalars range freely and consider every possible resultant
vector, there are three things that can happen. For most pairs of vectors,
you'll be able to reach every possible point in the plane; every two-dimensional
vector is within your grasp. However, in the unlucky case where your two original
vectors happen to line up, the tip of the resulting vector is limited to just a
single line passing through the origin. The vectors could also both be zero, in
which case the resultant vector is just at the origin.

The set of all possible vectors that you can reach with a linear combination of
a given pair of vectors is called the \textit{span} of those two vectors. So,
restating what we just discussed in this lingo, the span of most pairs of 2D
vectors is all vectors in 2D space, but when they line up, their span is all
vectors whose tip sits on a certain line.

Remember how we said that linear algebra revolves around vector addition and
scalar multiplication? The span of two vectors is a way of asking, ``What are
all the possible vectors one can reach using only these two fundamental
operations, vector addition and scalar multiplication?"

Thinking about a whole collection of vectors sitting on a line gets crowded, and
even more so to think about all two-dimensional vectors at once filling up the
plane. When dealing with collections of vectors like this, it's common to
represent each one with just a point in space where the tip of the vector was.
This way, if you want to think about every possible vector whose tip sits on a
certain line, just think about the line itself.

Likewise, to think about all possible two-dimensional vectors at once,
conceptualize each one as the point where its tip sits. In effect, you're
thinking about the infinite, flat sheet of two-dimensional space itself, leaving
the arrows out of it.

In general, if you're thinking about a vector on its own, think of it as an
arrow, and if you're dealing with a collection of vectors, it's convenient to
think of them all as points. Therefore, for our span example, the span of most
pairs of vectors ends up being the entire infinite sheet of two-dimensional
space, but if they line up, their span is just a line.

The idea of span gets more interesting if we start thinking about vectors in
three-dimensional space. For example, given two vectors in 3D space that are not
pointing in the same direction, what does it mean to take their span? Their span
is the collection of all possible linear combinations of those two vectors,
meaning all possible vectors you get by scaling each vector in some way, then
adding them together.

You can imagine turning two different knobs to change the two scalars defining
the linear combination, adding the scaled vectors and following the tip of the
resulting vector. That tip will trace out a flat sheet cutting through the
origin of three-dimensional space. This flat sheet is the span of the two
vectors. More precisely, the span of the two vectors is the set of all possible
vectors whose tips sit on that flat sheet.

So what happens if we add a third vector and consider the span of all three? A
linear combination of three vectors is defined similarly as it is for two;
you'll choose three different scalars, scale each of those vectors, then add
them all together. The linear combination of $\vec{v}$, $\vec{w}$, and $\vec{u}$
looks like

\begin{equation*}
  a\vec{v} + b\vec{w} + c\vec{u}
\end{equation*}

where $a$, $b$, and $c$ are allowed to vary. Again, the span of these vectors is
the set of all possible linear combinations.

Two different things could happen here. If your third vector happens to be
sitting on the span of the first two, then the span doesn't change; you're
trapped on that same flat sheet. In other words, adding a scaled version of that
third vector to the linear combination doesn't give you access to any new
vectors. However, if you just randomly choose a third vector, it's almost
certainly not sitting on the span of those first two. Then, since it's pointing
in a separate direction, it unlocks access to every possible three-dimensional
vector. As you scale that new third vector, it moves around that span sheet of
the first two, sweeping it through all of space.

Another way to think about it is that you're making full use of the three,
freely-changing scalars that you have at your disposal to access the full three
dimensions of space.

\subsection{Linear dependence and independence}

In the case where the third vector was already sitting on the span of the first
two, or the case where two vectors happen to line up, we want some terminology
to describe the fact that at least one of these vectors is redundant--not adding
anything to our span. When there are multiple vectors and one could be removed
without reducing the span, the relevant terminology is to say that they are
\textit{linearly dependent}.

In other words, one of the vectors can be expressed as a linear combination of
the others since it's already in the span of the others.

\begin{equation*}
  \vec{u} = a\vec{v} + b\vec{w} \text{ for some values of $a$ and $b$}
\end{equation*}

On the other hand, if each vector really does add another dimension to the span,
they're said to be \textit{linearly independent}.

\begin{equation*}
  \vec{w} \neq a\vec{v} \text{ for all values of $a$}
\end{equation*}

Now with all that terminology, and hopefully some good mental images to go with
it, the technical definition of a basis of a space is as follows.

\begin{definition}[Basis of a vector space]
  The \textit{basis} of a vector space is a set of \textit{linearly independent}
  vectors that \textit{span} the full space.
\end{definition}
