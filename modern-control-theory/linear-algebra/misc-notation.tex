\section{Miscellaneous notation}

This book works with two-dimensional matrices in the sense that they only have
rows and columns. The dimensionality of these matrices is specified by row
first, then column. For example, a matrix with two rows and three columns would
be a two-by-three matrix. A square matrix has the same number of rows as
columns. Matrices commonly use capital letters while vectors use lowercase
letters.

The matrix $\mat{I}$ is known as the identity matrix, which is a square matrix
with ones along its diagonal and zeroes elsewhere. For example
\begin{equation*}
  \begin{bmatrix}
    1 & 0 & 0 \\
    0 & 1 & 0 \\
    0 & 0 & 1
  \end{bmatrix}
\end{equation*}

The matrix denoted by $\mat{0}_{m \times n}$ is a matrix filled with zeroes with
$m$ rows and $n$ columns.

\index{matrices!transpose}
The $^T$ in $\mat{A}^T$ denotes transpose, which flips the matrix across its
diagonal such that the rows become columns and vice versa.

\index{matrices!pseudoinverse}
The $^\dagger$ in $\mat{B}^\dagger$ denotes the Moore-Penrose pseudoinverse
given by $\mat{B}^\dagger = (\mat{B}^T\mat{B})^{-1}\mat{B}^T$. The pseudoinverse
is used when the matrix is nonsquare and thus not invertible to produce a close
approximation of an inverse in the least squares sense.

\index{matrices!trace}
$\tr(\mat{A})$ denotes the trace of the square matrix $\mat{A}$, which is
defined as the sum of the elements on the main diagonal (top-left to
bottom-right).
