\section{Matrix calculus}
\label{sec:matrix_calculus}

Matrix calculus uses partial derivatives. See subsection
\ref{subsec:partial_derivatives} for how they work.

We'll need a vector-valued function to demonstrate some common operations in
matrix calculus. Let $\mat{f}(\mat{x}, \mat{u})$ be a vector-valued function
defined as
\begin{equation*}
  \mat{f}(\mat{x}, \mat{u}) =
  \begin{bmatrix}
    f_1(\mat{x}, \mat{u}) \\
    \vdots \\
    f_m(\mat{x}, \mat{u})
  \end{bmatrix}
  \text{where }
  \mat{x} =
  \begin{bmatrix}
    x_1 \\
    \vdots \\
    x_m
  \end{bmatrix}
  \text{and }
  \mat{u} =
  \begin{bmatrix}
    u_1 \\
    \vdots \\
    u_n
  \end{bmatrix}
\end{equation*}

\subsection{Jacobian}
\index{matrices!Jacobian}

The Jacobian is the first-order partial derivative of a vector-valued function
with respect to one of its vector arguments. The columns of the Jacobian of
$\mat{f}$ are filled with partial derivatives of $\mat{f}$'s rows with respect
to each of the argument's elements. For example, the Jacobian of $\mat{f}$ with
respect to $\mat{x}$ is
\begin{equation*}
  \frac{\partial \mat{f}(\mat{x}, \mat{u})}{\partial \mat{x}} =
  \begin{bmatrix}
    \frac{\partial \mat{f}(\mat{x}, \mat{u})}{\partial x_1} & \hdots &
      \frac{\partial \mat{f}(\mat{x}, \mat{u})}{\partial x_m}
  \end{bmatrix} =
  \begin{bmatrix}
    \frac{\partial f_1}{\partial x_1} & \hdots &
      \frac{\partial f_1}{\partial x_m} \\
    \vdots & \ddots & \vdots \\
    \frac{\partial f_m}{\partial x_1} & \hdots &
      \frac{\partial f_m}{\partial x_m}
  \end{bmatrix}
\end{equation*}

$\frac{\partial f_1}{\partial x_1}$ is the partial derivative of the first row
of $\mat{f}$ with respect to the first row of $\mat{x}$, and so on for all rows
of $\mat{f}$ and $\mat{x}$. This has $m^2$ permutations and thus produces a
square matrix.

The Jacobian of $\mat{f}$ with respect to $\mat{u}$ is
\begin{equation*}
  \frac{\partial \mat{f}(\mat{x}, \mat{u})}{\partial \mat{u}} =
  \begin{bmatrix}
    \frac{\partial \mat{f}(\mat{x}, \mat{u})}{\partial u_1} & \hdots &
      \frac{\partial \mat{f}(\mat{x}, \mat{u})}{\partial u_n}
  \end{bmatrix} =
  \begin{bmatrix}
    \frac{\partial f_1}{\partial u_1} & \hdots &
      \frac{\partial f_1}{\partial u_n} \\
    \vdots & \ddots & \vdots \\
    \frac{\partial f_m}{\partial u_1} & \hdots &
      \frac{\partial f_m}{\partial u_n}
  \end{bmatrix}
\end{equation*}

$\frac{\partial f_1}{\partial u_1}$ is the partial derivative of the first row
of $\mat{f}$ with respect to the first row of $\mat{u}$, and so on for all rows
of $\mat{f}$ and $\mat{u}$. This has $m \times n$ permutations and can produce a
nonsquare matrix if $m \neq n$.

\subsection{Hessian}
\index{matrices!Hessian}

The Hessian is the second-order partial derivative of a vector-valued function
with respect to one of its vector arguments. For example, the Hessian of
$\mat{f}$ with respect to $\mat{x}$ is
\begin{equation*}
  \frac{\partial^2 \mat{f}(\mat{x}, \mat{u})}{\partial \mat{x}^2} =
  \begin{bmatrix}
    \frac{\partial^2 \mat{f}(\mat{x}, \mat{u})}{\partial x_1^2} & \hdots &
      \frac{\partial^2 \mat{f}(\mat{x}, \mat{u})}{\partial x_m^2}
  \end{bmatrix} =
  \begin{bmatrix}
    \frac{\partial^2 f_1}{\partial x_1^2} & \hdots &
      \frac{\partial^2 f_1}{\partial x_m^2} \\
    \vdots & \ddots & \vdots \\
    \frac{\partial^2 f_m}{\partial x_1^2} & \hdots &
      \frac{\partial^2 f_m}{\partial x_m^2}
  \end{bmatrix}
\end{equation*}

and the Hessian of $\mat{f}$ with respect to $\mat{u}$ is
\begin{equation*}
  \frac{\partial^2 \mat{f}(\mat{x}, \mat{u})}{\partial \mat{u}^2} =
  \begin{bmatrix}
    \frac{\partial^2 \mat{f}(\mat{x}, \mat{u})}{\partial u_1^2} & \hdots &
      \frac{\partial^2 \mat{f}(\mat{x}, \mat{u})}{\partial u_n^2}
  \end{bmatrix} =
  \begin{bmatrix}
    \frac{\partial^2 f_1}{\partial u_1^2} & \hdots &
      \frac{\partial^2 f_1}{\partial u_n^2} \\
    \vdots & \ddots & \vdots \\
    \frac{\partial^2 f_m}{\partial u_1^2} & \hdots &
      \frac{\partial^2 f_m}{\partial u_n^2}
  \end{bmatrix}
\end{equation*}
