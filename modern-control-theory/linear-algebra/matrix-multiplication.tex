\section{Matrix multiplication as composition}
\index{Matrices!multiplication}

\begin{remark}
  See the corresponding \textit{Essence of linear algebra} video for a more
  visual presentation (10 minutes)
  \cite{bib:linalg_matrix_multiplication_as_composition}.
\end{remark}

Often, you find yourself wanting to describe the effect of applying one
transformation and then another. For example, you may want to describe what
happens when you first rotate the plane $90\degree$ counterclockwise then apply
a shear. The overall effect here, from start to finish, is another linear
transformation distinct from the rotation and the shear. This new linear
transformation is commonly called the ``composition" of the two separate
transformations we applied, and like any linear transformation, it can be
described with a matrix all its own by following $\hat{i}$ and $\hat{j}$. In
this example, the ultimate landing spot for $\hat{i}$ after both transformations
is $(1, 1)$, so that's the first column of the matrix. Likewise, $\hat{j}$
ultimately ends up at the location $(-1, 0)$, so we make that the second column
of the matrix.

\begin{equation*}
  \begin{bmatrix}
    1 & -1 \\
    1 & -0
  \end{bmatrix}
\end{equation*}

This new matrix captures the overall effect of applying a rotation then a shear
but as one single action rather than two successive ones.

Here's one way to think about that new matrix: if you were to feed some vector
through the rotation then the shear, the long way to compute where it ends up is
to, first, multiply it on the left by the rotation matrix; then, take whatever
you get and multiply that on the left by the shear matrix.

\begin{equation*}
  \begin{bmatrix}
    1 & 1 \\
    0 & 1
  \end{bmatrix}\left(
  \begin{bmatrix}
    0 & -1 \\
    1 & 0
  \end{bmatrix}
  \begin{bmatrix}
    x \\
    y
  \end{bmatrix}\right)
\end{equation*}

This is, numerically speaking, what it means to apply a rotation then a shear to
a given vector, but the result should be the same as just applying this new
composition matrix we found to that same vector. This applies to any vector
because this new matrix is supposed to capture the same overall effect as the
rotation-then-shear action.

\begin{equation*}
  \begin{bmatrix}
    1 & 1 \\
    0 & 1
  \end{bmatrix}\left(
  \begin{bmatrix}
    0 & -1 \\
    1 & 0
  \end{bmatrix}
  \begin{bmatrix}
    x \\
    y
  \end{bmatrix}\right) =
  \begin{bmatrix}
    1 & -1 \\
    1 & 0
  \end{bmatrix} \begin{bmatrix}
    x \\
    y
  \end{bmatrix}
\end{equation*}

Based on how things are written down here, it's reasonable to call this new
matrix the ``product" of the original two matrices.

\begin{equation*}
  \begin{bmatrix}
    1 & 1 \\
    0 & 1
  \end{bmatrix}
  \begin{bmatrix}
    0 & -1 \\
    1 & 0
  \end{bmatrix} =
  \begin{bmatrix}
    1 & -1 \\
    1 & 0
  \end{bmatrix}
\end{equation*}

We can think about how to compute that product more generally in just a moment,
but it's way too easy to get lost in the forest of numbers. Always remember
that multiplying two matrices like this has the geometric meaning of applying
one transformation then another.

One oddity here is that we are reading the transformations from right to left;
you first apply the transformation represented by the matrix on the right, then
you apply the transformation represented by the matrix on the left. This stems
from function notation, since we write functions on the left of variables, so
every time you compose two functions, you always have to read it right to left.

Let's look at another example. Take the matrix with columns $(1, 1)$ and
$(-2, 0)$.

\begin{equation*}
  M_1 = \begin{bmatrix}
    1 & -2 \\
    1 & 0
  \end{bmatrix}
\end{equation*}

Next, take the matrix with columns $(0, 1)$ and $(2, 0)$.

\begin{equation*}
  M_2 = \begin{bmatrix}
    0 & 2 \\
    1 & 0
  \end{bmatrix}
\end{equation*}

The total effect of applying $M_1$ then $M_2$ gives us a new transformation, so
let's find its matrix. First, we need to determine where $\hat{i}$ goes. After
applying $M_1$, the new coordinates of $\hat{i}$, by definition, are given by
that first column of $M_1$, namely, $(1, 1)$. To see what happens after applying
$M_2$, multiply the matrix for $M_2$ by that vector $(1, 1)$. Working it out the
way described in the last section, you'll get the vector $(2, 1)$.

\begin{equation*}
  \begin{bmatrix}
    0 & 2 \\
    1 & 0
  \end{bmatrix}
  \begin{bmatrix}
    1 \\
    1
  \end{bmatrix} = 1
  \begin{bmatrix}
    0 \\
    1
  \end{bmatrix} + 1
  \begin{bmatrix}
    2 \\
    0
  \end{bmatrix} =
  \begin{bmatrix}
    2 \\
    1
  \end{bmatrix}
\end{equation*}

This will be the first column of the composition matrix. Likewise, to follow
$\hat{j}$, the second column of $M_1$ tells us that it first lands on $(-2, 0)$.
Then, when we apply $M_2$ to that vector, you can work out the matrix-vector
product to get $(0, -2)$.

\begin{equation*}
  \begin{bmatrix}
    0 & 2 \\
    1 & 0
  \end{bmatrix}
  \begin{bmatrix}
    -2 \\
    0
  \end{bmatrix} = -2
  \begin{bmatrix}
    0 \\
    1
  \end{bmatrix} + 0
  \begin{bmatrix}
    2 \\
    0
  \end{bmatrix} =
  \begin{bmatrix}
    0 \\
    -2
  \end{bmatrix}
\end{equation*}

This will be the second column of the composition matrix.

\begin{equation*}
  \begin{bmatrix}
    0 & 2 \\
    1 & 0
  \end{bmatrix}
  \begin{bmatrix}
    1 & -2 \\
    1 & 0
  \end{bmatrix} =
  \begin{bmatrix}
    2 & 0 \\
    1 & -2
  \end{bmatrix}
\end{equation*}

\subsection{General matrix multiplication}

Let's go through that same process again, but this time, we'll use variable
entries in each matrix, just to show that the same line of reasoning works for
any matrices. This is more symbol-heavy, but it should be pretty satisfying for
anyone who has previously been taught matrix multiplication the more rote way.

\begin{equation*}
  \begin{bmatrix}
    a & b \\
    c & d
  \end{bmatrix}
  \begin{bmatrix}
    e & f \\
    g & h
  \end{bmatrix} =
  \begin{bmatrix}
    ? & ? \\
    ? & ?
  \end{bmatrix}
\end{equation*}

To follow where $\hat{i}$ goes, start by looking at the first column of the
matrix on the right, since this is where $\hat{i}$ initially lands. Multiplying
that column by the matrix on the left is how you can tell where the intermediate
version of $\hat{i}$ ends up after applying the second transformation.

\begin{equation*}
  \begin{bmatrix}
    a & b \\
    c & d
  \end{bmatrix}
  \begin{bmatrix}
    e \\
    g
  \end{bmatrix} = e
  \begin{bmatrix}
    a \\
    c
  \end{bmatrix} + g
  \begin{bmatrix}
    b \\
    d
  \end{bmatrix} =
  \begin{bmatrix}
    ae + bg \\
    ce + dg
  \end{bmatrix}
\end{equation*}

So the first column of the composition matrix will always equal the left matrix
times the first column of the right matrix. Likewise, $\hat{j}$ will always
initially land on the second column of the right matrix, so multiplying by this
second column will give its final location, and hence, that's the second column
of the composition matrix.

\begin{equation*}
  \begin{bmatrix}
    a & b \\
    c & d
  \end{bmatrix}
  \begin{bmatrix}
    f \\
    h
  \end{bmatrix} = f
  \begin{bmatrix}
    a \\
    c
  \end{bmatrix} + h
  \begin{bmatrix}
    b \\
    d
  \end{bmatrix} =
  \begin{bmatrix}
    af + bh \\
    cf + dh
  \end{bmatrix}
\end{equation*}

So the complete composition matrix is

\begin{equation*}
  \begin{bmatrix}
    a & b \\
    c & d
  \end{bmatrix}
  \begin{bmatrix}
    e & f \\
    g & h
  \end{bmatrix} =
  \begin{bmatrix}
    ae + bg & af + bh \\
    ce + dg & cf + dh
  \end{bmatrix}
\end{equation*}

Notice there's a lot of symbols here, and it's common to be taught this formula
as something to memorize along with a certain algorithmic process to help
remember it. Before memorizing that process, you should get in the habit of
thinking about what matrix multiplication really represents: applying one
transformation after another. This will give you a much better conceptual
framework that makes the properties of matrix multiplication much easier to
understand.

\subsection{Matrix multiplication associativity}

For example, here's a question: does it matter what order we put the two
matrices in when we multiply them? Let's think through a simple example. Take a
shear which fixes $\hat{i}$ and moves $\hat{j}$ over to the right, and a
$90\degree$ rotation. If you first do the shear then rotate, we can see that
$\hat{i}$ ends up at $(0, 1)$ and $\hat{j}$ ends up at $(-1, 1)$. Both are
generally pointing close together. If you first rotate then do the shear,
$\hat{i}$ ends up over at $(1, 1)$ and $\hat{j}$ is off on a different direction
at $(-1, 0)$ and they're pointing farther apart. The overall effect here is
clearly different, so evidently, order totally does matter. Notice by thinking
in terms of transformations, that's the kind of thing that you can do in your
head by visualizing. No matrix multiplication necessary.

Let's consider trying to prove that matrix multiplication is associative. This
means that if you have three matrices $A$, $B$, and $C$, and you multiply them
all together, it shouldn't matter if you first compute $A$ times $B$ then
multiply the result by $C$, or if you first multiply $B$ times $C$ then multiply
that result by $A$ on the left. In other words, it doesn't matter where you put
the parenthesis.

\begin{equation*}
  (AB)C \stackrel{?}{=} A(BC)
\end{equation*}

If you try to work through this numerically, it's horrible, and unenlightening
for that matter. However, when you think about matrix multiplication as applying
one transformation after another, this property is just trivial. Can you see
why? What it's saying is that if you first apply $C$ then $B$, then $A$, it's
the same as applying $C$, then $B$ then $A$. There's nothing to prove, you're
just applying the same three things one after the other all in the same order.
This might feel like cheating, but it's not. This is a valid proof that matrix
multiplication is associative, and even better than that, it's a good
explanation for why that property should be true.
