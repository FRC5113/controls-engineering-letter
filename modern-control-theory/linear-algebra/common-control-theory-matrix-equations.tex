\section{Common control theory matrix equations}

Here's some common matrix equations from control theory we'll use later on.
Solvers for them exist in \href{https://github.com/RobotLocomotion/drake}{Drake}
(C++) and \href{https://github.com/scipy/scipy}{SciPy} (Python).

\subsection{Continuous algebraic Riccati equation (CARE)}
\index{algebraic Riccati equation!continuous}

The continuous algebraic Riccati equation (CARE) appears in the solution to the
continuous time LQ problem.
\begin{equation}
  \mat{A}\mat{X} + \mat{X}\mat{A} - \mat{X}\mat{B}\mat{R}^{-1}\mat{B}\T\mat{X} +
    \mat{Q} = 0
\end{equation}

\subsection{Discrete algebraic Riccati equation (DARE)}
\index{algebraic Riccati equation!discrete}

The discrete algebraic Riccati equation (DARE) appears in the solution to the
discrete time LQ problem.
\begin{equation}
  \mat{X} = \mat{A}\T\mat{X}\mat{A} - (\mat{A}\T\mat{X}\mat{B})(\mat{R} +
    \mat{B}\T\mat{X}\mat{B})^{-1} \mat{B}\T\mat{X}\mat{A} + \mat{Q}
\end{equation}

Snippet \ref{lst:dare} computes the unique stabilizing solution to the discrete
algebraic Riccati equation. A robust implementation should also check that:
\begin{enumerate}
  \item $\mat{A}$ is states $\times$ states.
  \item $\mat{B}$ is states $\times$ inputs.
  \item $\mat{Q}$ is states $\times$ states and symmetric positive semidefinite.
  \item $\mat{R}$ is inputs $\times$ inputs and symmetric positive definite.
  \item The number of inputs isn't greater than the number of states.
  \item $(\mat{A}, \mat{B})$ is a stabilizable pair (see subsection
    \ref{subsec:stabilizability}).
  \item $(\mat{A}, \mat{C})$ is a detectable pair where
    $\mat{Q} = \mat{C}\T\mat{C}$ (see section \ref{subsec:detectability}).
\end{enumerate}
\begin{coderemote}{cpp}{snippets/DARE.cpp}
  \caption{Discrete algebraic Riccati equation solver in C++}
  \label{lst:dare}
\end{coderemote}

\subsection{Continuous Lyapunov equation}
\index{Lyapunov equation!continuous}

The continuous Lyapunov equation appears in controllability/observability
analysis of continuous time systems.
\begin{equation}
  \mat{A}\mat{X} + \mat{X}\mat{A}\T + \mat{Q} = 0
\end{equation}

\subsection{Discrete Lyapunov equation}
\index{Lyapunov equation!discrete}

The discrete Lyapunov equation appears in controllability/observability analysis
of discrete time systems.
\begin{equation}
  \mat{A}\mat{X}\mat{A}\T - \mat{X} + \mat{Q} = 0
\end{equation}
