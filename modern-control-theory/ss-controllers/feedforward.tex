\section{Feedforward}

So far, we've used feedback control for \gls{reference} \gls{tracking} (making a
\gls{system}'s output follow a desired \gls{reference} signal). While this is
effective, it's a reactionary measure; the \gls{system} won't start applying
\gls{control effort} until the \gls{system} is already behind. If we could tell
the \gls{controller} about the desired movement and required input beforehand,
the \gls{system} could react quicker and the feedback \gls{controller} could do
less work. A \gls{controller} that feeds information forward into the
\gls{plant} like this is called a \gls{feedforward controller}.

A \gls{feedforward controller} injects information about the \gls{system}'s
dynamics (like a \gls{model} does) or the desired movement. The feedforward
handles parts of the control actions we already know must be applied to make a
\gls{system} track a \gls{reference}, then feedback compensates for what we do
not or cannot know about the \gls{system}'s behavior at runtime.

There are two types of feedforwards: model-based feedforward and feedforward for
unmodeled dynamics. The first solves a mathematical model of the system for the
inputs required to meet desired velocities and accelerations. The second
compensates for unmodeled forces or behaviors directly so the feedback
controller doesn't have to. Both types can facilitate simpler feedback
controllers; we'll cover examples of each.

\subsection{Plant inversion}
\label{subsec:plant_inversion}

\Gls{plant} inversion is a method of model-based feedforward for \gls{state}
feedback. It solves the \gls{plant} for the input that will make the \gls{plant}
track a desired state. This is called inversion because in a block diagram, the
inverted \gls{plant} feedforward and \gls{plant} cancel out to produce a unity
system from input to output.

While it can be an effective tool, the following should be kept in mind.
\begin{enumerate}
  \item Don't invert an unstable \gls{plant}. If the expected \gls{plant}
    doesn't match the real \gls{plant} exactly, the \gls{plant} inversion will
    still result in an unstable \gls{system}. Stabilize the \gls{plant} first
    with feedback, then inject an inversion.
  \item Don't invert a nonminimum phase system. The advice for pole-zero
    cancellation in subsection \ref{subsec:pole-zero_cancellation} applies here.
\end{enumerate}

\subsubsection{Setup}

Let's start with the equation for the \gls{reference} dynamics
\begin{equation*}
  \mat{r}_{k+1} = \mat{A}\mat{r}_k + \mat{B}\mat{u}_k
\end{equation*}

where $\mat{u}_k$ is the feedforward input. Note that this feedforward equation
does not and should not take into account any feedback terms. We want to find
the optimal $\mat{u}_k$ such that we minimize the \gls{tracking} error between
$\mat{r}_{k+1}$ and $\mat{r}_k$.
\begin{equation*}
  \mat{r}_{k+1} - \mat{A}\mat{r}_k = \mat{B}\mat{u}_k
\end{equation*}

To solve for $\mat{u}_k$, we need to take the inverse of the nonsquare matrix
$\mat{B}$. This isn't possible, but we can find the pseudoinverse given some
constraints on the \gls{state} \gls{tracking} error and \gls{control effort}. To
find the optimal solution for these sorts of trade-offs, one can define a cost
function and attempt to minimize it. To do this, we'll first solve the
expression for $\mat{0}$.
\begin{equation*}
  \mat{0} = \mat{B}\mat{u}_k - (\mat{r}_{k+1} - \mat{A}\mat{r}_k)
\end{equation*}

This expression will be the \gls{state} \gls{tracking} cost we use in the
following cost function as an $H_2$ norm.
\begin{equation*}
  \mat{J} = (\mat{B}\mat{u}_k - (\mat{r}_{k+1} - \mat{A}\mat{r}_k))\T
    (\mat{B}\mat{u}_k - (\mat{r}_{k+1} - \mat{A}\mat{r}_k))
\end{equation*}

\subsubsection{Minimization}

Given theorem \ref{thm:partial_xax}, find the minimum of $\mat{J}$ by taking the
partial derivative with respect to $\mat{u}_k$ and setting the result to
$\mat{0}$.
\begin{align*}
  \frac{\partial\mat{J}}{\partial\mat{u}_k} &= 2\mat{B}\T
    (\mat{B}\mat{u}_k - (\mat{r}_{k+1} - \mat{A}\mat{r}_k)) \\
  \mat{0} &= 2\mat{B}\T
    (\mat{B}\mat{u}_k - (\mat{r}_{k+1} - \mat{A}\mat{r}_k)) \\
  \mat{0} &= 2\mat{B}\T\mat{B}\mat{u}_k -
    2\mat{B}\T(\mat{r}_{k+1} - \mat{A}\mat{r}_k) \\
  2\mat{B}\T\mat{B}\mat{u}_k &=
    2\mat{B}\T(\mat{r}_{k+1} - \mat{A}\mat{r}_k) \\
  \mat{B}\T\mat{B}\mat{u}_k &=
    \mat{B}\T(\mat{r}_{k+1} - \mat{A}\mat{r}_k) \\
  \mat{u}_k &=
    (\mat{B}\T\mat{B})^{-1} \mat{B}\T(\mat{r}_{k+1} - \mat{A}\mat{r}_k)
\end{align*}

$(\mat{B}\T\mat{B})^{-1} \mat{B}\T$ is the Moore-Penrose pseudoinverse of
$\mat{B}$ denoted by $\mat{B}^\dagger$.
\begin{theorem}[Linear plant inversion]
  \label{thm:linear_plant_inversion}

  Given the discrete model
  $\mat{x}_{k+1} = \mat{A}\mat{x}_k + \mat{B}\mat{u}_k$, the plant inversion
  feedforward is
  \begin{equation}
    \mat{u}_k = \mat{B}^\dagger (\mat{r}_{k+1} - \mat{A}\mat{r}_k)
  \end{equation}

  where $\mat{B}^\dagger$ is the Moore-Penrose pseudoinverse of $\mat{B}$,
  $\mat{r}_{k+1}$ is the reference at the next timestep, and $\mat{r}_k$ is the
  reference at the current timestep.
\end{theorem}
\index{feedforward!linear plant inversion}
\index{optimal control!linear plant inversion}

\subsubsection{Discussion}

Linear \gls{plant} inversion in theorem \ref{thm:linear_plant_inversion}
compensates for \gls{reference} dynamics that don't follow how the \gls{model}
inherently behaves. If they do follow the \gls{model}, the feedforward has
nothing to do as the \gls{model} already behaves in the desired manner. When
this occurs, $\mat{r}_{k+1} - \mat{A}\mat{r}_k$ will return a zero vector.

For example, a constant \gls{reference} requires a feedforward that opposes
\gls{system} dynamics that would change the \gls{state} over time. If the
\gls{system} has no dynamics, then $\mat{A} = \mat{I}$ and thus
\begin{align*}
  \mat{u}_k &= \mat{B}_\dagger (\mat{r}_{k+1} - \mat{I}\mat{r}_k) \\
  \mat{u}_k &= \mat{B}_\dagger (\mat{r}_{k+1} - \mat{r}_k)
  \intertext{For a constant \gls{reference}, $\mat{r}_{k+1} = \mat{r}_k$.}
  \mat{u}_k &= \mat{B}_\dagger (\mat{r}_k - \mat{r}_k) \\
  \mat{u}_k &= \mat{B}_\dagger (\mat{0}) \\
  \mat{u}_k &= \mat{0}
\end{align*}

so no feedforward is required to hold a \gls{system} with no dynamics at a
constant \gls{reference}, as expected.

Figure \ref{fig:case_study_ff} shows \gls{plant} inversion applied to a
second-order CIM motor model. \Gls{plant} inversion accounts for the motor
back-EMF and eliminates steady-state error.
\begin{svg}{build/\chapterpath/case_study_ff}
  \caption{Second-order CIM motor response with plant inversion}
  \label{fig:case_study_ff}
\end{svg}

\subsection{Unmodeled dynamics}

In addition to \gls{plant} inversion, one can include feedforwards for unmodeled
dynamics. Consider an elevator model which doesn't include gravity. A constant
voltage offset can be used compensate for this. The feedforward takes the form
of a voltage constant because voltage is proportional to force applied, and the
force is acting in only one direction at all times.
\begin{align}
  u_k &= V_{app}
  \intertext{where $V_{app}$ is a constant. Another feedforward holds a
    single-jointed arm steady in the presence of gravity. It has the following
    form.}
  u_k &= V_{app} \cos\theta
\end{align}

where $V_{app}$ is the voltage required to keep the single-jointed arm level
with the ground, and $\theta$ is the angle of the arm relative to the ground.
Therefore, the force applied is greatest when the arm is parallel with the
ground and zero when the arm is perpendicular to the ground (at that point, the
joint supports all the weight).

Note that the elevator model could be augmented easily enough to include gravity
and still be linear, but this wouldn't work for the single-jointed arm since a
trigonometric function is required to model the gravitational force in the arm's
rotating reference frame\footnote{While the applied torque of the motor is
constant throughout the arm's range of motion, the torque caused by gravity in
the opposite direction varies according to the arm's angle.}.
