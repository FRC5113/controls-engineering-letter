\section{Feedforwards}

Feedforwards are used to inject information about either the \gls{system}'s
dynamics (like a \gls{model} does) or the intended movement into a controller.
Feedforward is generally used to handle parts of the control actions we already
know must be applied to make a \gls{system} track a \gls{reference}, then let
the feedback controller correct for at runtime what we do not or cannot know
about the \gls{system}'s behavior. There are two types: model-based feedforward
and feedforward for unmodeled dynamics. First, we will present \gls{plant}
inversion as a method of model-based feedforwards for \gls{state} feedback.

\subsection{Plant inversion}
\label{subsec:plant_inversion}

\Gls{plant} inversion solves the \gls{plant} for the input that will make the
\gls{plant} track a desired output. This is called inversion because in a block
diagram, the inverted \gls{plant} feedforward and \gls{plant} cancel out to
produce a unity system from input to output.

While it can be an effective tool, the following should be kept in mind.

\begin{enumerate}
  \item Don't invert an unstable \gls{plant}. If the expected \gls{plant}
    doesn't match the real \gls{plant} exactly, the \gls{plant} inversion will
    still result in an unstable \gls{system}. Stabilize the \gls{plant} first
    with feedback, then inject an inversion.
  \item Don't invert a nonminimum phase system. The advice for pole-zero
    cancellation in subsection \ref{subsec:pole-zero_cancellation} applies here.
\end{enumerate}

Let's start with the equation for the \gls{reference} dynamics

\begin{equation*}
  \mtx{r}_{k+1} = \mtx{A}\mtx{r}_k + \mtx{B}\mtx{u}_{ff}
\end{equation*}

where $\mtx{u}_{ff}$ is the feedforward input. Note that this feedforward
equation does not and should not take into account any feedback terms. We want
to find the optimal $\mtx{u}_{ff}$ such that we minimize the \gls{tracking}
error between $\mtx{r}_{k+1}$ and $\mtx{r}_k$.

\begin{equation*}
  \mtx{r}_{k+1} - \mtx{A}\mtx{r}_k = \mtx{B}\mtx{u}_{ff}
\end{equation*}

To solve for $\mtx{u}_{ff}$, we need to take the inverse of the nonsquare matrix
$\mtx{B}$. This isn't possible, but we can find the pseudoinverse given some
constraints on the \gls{state} \gls{tracking} error and \gls{control effort}. To
find the optimal solution for these sorts of trade-offs, one can define a cost
function and attempt to minimize it. To do this, we'll first solve the
expression for $\mtx{0}$.

\begin{equation*}
  \mtx{0} = \mtx{B}\mtx{u}_{ff} - (\mtx{r}_{k+1} - \mtx{A}\mtx{r}_k)
\end{equation*}

This expression will be the \gls{state} \gls{tracking} cost we use in our cost
function.

Our cost function will use an $H_2$ norm with $\mtx{Q}$ as the \gls{state} cost
matrix with dimensionality $states \times states$ and $\mtx{R}$ as the
\gls{control input} cost matrix with dimensionality $inputs \times inputs$.

\begin{equation*}
  \mtx{J} = (\mtx{B}\mtx{u}_{ff} - (\mtx{r}_{k+1} - \mtx{A}\mtx{r}_k))^T \mtx{Q}
    (\mtx{B}\mtx{u}_{ff} - (\mtx{r}_{k+1} - \mtx{A}\mtx{r}_k)) +
    \mtx{u}_{ff}^T\mtx{R}\mtx{u}_{ff}
\end{equation*}

The following theorems will be needed to find the minimum of $\mtx{J}$.

\begin{theorem}
  \label{thm:partial_xax}

  $\frac{\partial \mtx{x}^T\mtx{A}\mtx{x}}{\partial\mtx{x}} =
    2\mtx{A}\mtx{x}$ where $\mtx{A}$ is symmetric.
\end{theorem}

\begin{theorem}
  \label{thm:partial_ax_b}

  $\frac{\partial (\mtx{A}\mtx{x} + \mtx{b})^T\mtx{C}
    (\mtx{D}\mtx{x} + \mtx{e})}{\partial\mtx{x}} =
    \mtx{A}^T\mtx{C}(\mtx{D}\mtx{x} + \mtx{e}) + \mtx{D}^T\mtx{C}^T
    (\mtx{A}\mtx{x} + \mtx{b})$
\end{theorem}

\begin{corollary}
  \label{cor:partial_ax_b}

  $\frac{\partial (\mtx{A}\mtx{x} + \mtx{b})^T\mtx{C}
    (\mtx{A}\mtx{x} + \mtx{b})}{\partial\mtx{x}} =
    2\mtx{A}^T\mtx{C}(\mtx{A}\mtx{x} + \mtx{b})$ where $\mtx{C}$ is symmetric.

  Proof:
  \begin{align*}
    \frac{\partial (\mtx{A}\mtx{x} + \mtx{b})^T\mtx{C}
      (\mtx{A}\mtx{x} + \mtx{b})}{\partial\mtx{x}} &=
      \mtx{A}^T\mtx{C}(\mtx{A}\mtx{x} + \mtx{b}) + \mtx{A}^T\mtx{C}^T
      (\mtx{A}\mtx{x} + \mtx{b}) \\
    \frac{\partial (\mtx{A}\mtx{x} + \mtx{b})^T\mtx{C}
      (\mtx{A}\mtx{x} + \mtx{b})}{\partial\mtx{x}} &=
      (\mtx{A}^T\mtx{C} + \mtx{A}^T\mtx{C}^T)(\mtx{A}\mtx{x} + \mtx{b})
  \end{align*}

  $\mtx{C}$ is symmetric, so

  \begin{align*}
    \frac{\partial (\mtx{A}\mtx{x} + \mtx{b})^T\mtx{C}
      (\mtx{A}\mtx{x} + \mtx{b})}{\partial\mtx{x}} &=
      (\mtx{A}^T\mtx{C} + \mtx{A}^T\mtx{C})(\mtx{A}\mtx{x} + \mtx{b}) \\
    \frac{\partial (\mtx{A}\mtx{x} + \mtx{b})^T\mtx{C}
      (\mtx{A}\mtx{x} + \mtx{b})}{\partial\mtx{x}} &=
      2\mtx{A}^T\mtx{C}(\mtx{A}\mtx{x} + \mtx{b})
  \end{align*}
\end{corollary}

Given theorem \ref{thm:partial_xax} and corollary \ref{cor:partial_ax_b}, find
the minimum of $\mtx{J}$ by taking the partial derivative with respect to
$\mtx{u}_{ff}$ and setting the result to $\mtx{0}$.

\begin{align*}
  \frac{\partial\mtx{J}}{\partial\mtx{u}_{ff}} &= 2\mtx{B}^T\mtx{Q}
    (\mtx{B}\mtx{u}_{ff} - (\mtx{r}_{k+1} - \mtx{A}\mtx{r}_k)) +
    2\mtx{R}\mtx{u}_{ff} \\
  \mtx{0} &= 2\mtx{B}^T\mtx{Q}
    (\mtx{B}\mtx{u}_{ff} - (\mtx{r}_{k+1} - \mtx{A}\mtx{r}_k)) +
    2\mtx{R}\mtx{u}_{ff} \\
  \mtx{0} &= \mtx{B}^T\mtx{Q}
    (\mtx{B}\mtx{u}_{ff} - (\mtx{r}_{k+1} - \mtx{A}\mtx{r}_k)) +
    \mtx{R}\mtx{u}_{ff} \\
  \mtx{0} &= \mtx{B}^T\mtx{Q}\mtx{B}\mtx{u}_{ff} -
    \mtx{B}^T\mtx{Q}(\mtx{r}_{k+1} - \mtx{A}\mtx{r}_k) + \mtx{R}\mtx{u}_{ff} \\
  \mtx{B}^T\mtx{Q}\mtx{B}\mtx{u}_{ff} + \mtx{R}\mtx{u}_{ff} &=
    \mtx{B}^T\mtx{Q}(\mtx{r}_{k+1} - \mtx{A}\mtx{r}_k) \\
  (\mtx{B}^T\mtx{Q}\mtx{B} + \mtx{R})\mtx{u}_{ff} &=
    \mtx{B}^T\mtx{Q}(\mtx{r}_{k+1} - \mtx{A}\mtx{r}_k) \\
  \mtx{u}_{ff} &= (\mtx{B}^T\mtx{Q}\mtx{B} + \mtx{R})^{-1}
    \mtx{B}^T\mtx{Q}(\mtx{r}_{k+1} - \mtx{A}\mtx{r}_k)
\end{align*}

\begin{theorem}[Linear plant inversion]
  \label{thm:linear_plant_inversion}

  \begin{align}
    &\mtx{u}_{ff} = \mtx{K}_{ff} (\mtx{r}_{k+1} - \mtx{A}\mtx{r}_k) \\
    &\text{where } \mtx{K}_{ff} =
      (\mtx{B}^T\mtx{Q}\mtx{B} + \mtx{R})^{-1}\mtx{B}^T\mtx{Q}
  \end{align}

  The exact solution without $\mtx{Q}$ and $\mtx{R}$ weights gives

  \begin{equation}
    \mtx{u}_{ff} = \mtx{B}^\dagger (\mtx{r}_{k+1} - \mtx{A}\mtx{r}_k)
  \end{equation}

  Proof:

  \begin{equation*}
    \mtx{J} = (\mtx{B}\mtx{u}_{ff} - (\mtx{r}_{k+1} - \mtx{A}\mtx{r}_k))^T
      (\mtx{B}\mtx{u}_{ff} - (\mtx{r}_{k+1} - \mtx{A}\mtx{r}_k))
  \end{equation*}

  Given theorem \ref{thm:partial_xax} and corollary \ref{cor:partial_ax_b}, find
  the minimum of $\mtx{J}$ by taking the partial derivative with respect to
  $\mtx{u}_{ff}$ and setting the result to $\mtx{0}$.

  \begin{align*}
    \frac{\partial\mtx{J}}{\partial\mtx{u}_{ff}} &= 2\mtx{B}^T
      (\mtx{B}\mtx{u}_{ff} - (\mtx{r}_{k+1} - \mtx{A}\mtx{r}_k)) \\
    \mtx{0} &= 2\mtx{B}^T
      (\mtx{B}\mtx{u}_{ff} - (\mtx{r}_{k+1} - \mtx{A}\mtx{r}_k)) \\
    \mtx{0} &= 2\mtx{B}^T\mtx{B}\mtx{u}_{ff} -
      2\mtx{B}^T(\mtx{r}_{k+1} - \mtx{A}\mtx{r}_k) \\
    2\mtx{B}^T\mtx{B}\mtx{u}_{ff} &=
      2\mtx{B}^T(\mtx{r}_{k+1} - \mtx{A}\mtx{r}_k) \\
    \mtx{B}^T\mtx{B}\mtx{u}_{ff} &=
      \mtx{B}^T(\mtx{r}_{k+1} - \mtx{A}\mtx{r}_k) \\
    \mtx{u}_{ff} &=
      (\mtx{B}^T\mtx{B})^{-1} \mtx{B}^T(\mtx{r}_{k+1} - \mtx{A}\mtx{r}_k)
  \end{align*}

  $(\mtx{B}^T\mtx{B})^{-1} \mtx{B}^T$ is the definition of the Moore-Penrose
  pseudoinverse denoted by $\mtx{B}^\dagger$.
\end{theorem}
\index{Feedforward!linear plant inversion}
\index{Optimal control!linear plant inversion}

Linear \gls{plant} inversion in theorem \ref{thm:linear_plant_inversion}
compensates for \gls{reference} dynamics that don't follow how the \gls{model}
inherently behaves. If they do follow the \gls{model}, the feedforward has
nothing to do as the \gls{model} already behaves in the desired manner. When
this occurs, $\mtx{r}_{k+1} - \mtx{A}\mtx{r}_k$ will return a zero vector.

For example, a constant \gls{reference} requires a feedforward that opposes
\gls{system} dynamics that would change the \gls{state} over time. If the
\gls{system} has no dynamics, then $\mtx{A} = \mtx{I}$ and thus

\begin{align*}
  \mtx{u}_{ff} &= \mtx{K}_{ff} (\mtx{r}_{k+1} - \mtx{I}\mtx{r}_k) \\
  \mtx{u}_{ff} &= \mtx{K}_{ff} (\mtx{r}_{k+1} - \mtx{r}_k)
\end{align*}

For a constant \gls{reference}, $\mtx{r}_{k+1} = \mtx{r}_k$.

\begin{align*}
  \mtx{u}_{ff} &= \mtx{K}_{ff} (\mtx{r}_k - \mtx{r}_k) \\
  \mtx{u}_{ff} &= \mtx{K}_{ff} (\mtx{0}) \\
  \mtx{u}_{ff} &= \mtx{0}
\end{align*}

so no feedforward is required to hold a \gls{system} with no dynamics at a
constant \gls{reference}, as expected.

Figure \ref{fig:case_study_ff} shows \gls{plant} inversion applied to a
second-order CIM motor model.

\begin{svg}{build/code/case_study_ff}
  \caption{Second-order CIM motor response with plant inversion}
  \label{fig:case_study_ff}
\end{svg}

\Gls{plant} inversion isn't as effective with both $\mtx{Q}$ and $\mtx{R}$ cost
because the $\mtx{R}$ matrix penalized \gls{control effort}. The \gls{reference}
\gls{tracking} with no cost matrices is much better.

\subsection{Unmodeled dynamics}

In addition to \gls{plant} inversion, one can include feedforwards for unmodeled
dynamics. Consider an elevator model which doesn't include gravity. A constant
voltage offset can be used compensate for this. The feedforward takes the form
of a voltage constant because voltage is proportional to force applied, and the
force is acting in only one direction at all times.

\begin{equation}
  u_{ff} = V_{app}
\end{equation}

where $V_{app}$ is a constant. Another feedforward holds a single-jointed arm
steady in the presence of gravity. It has the following form.

\begin{equation}
  u_{ff} = V_{app} \cos\theta
\end{equation}

where $V_{app}$ is the voltage required to keep the single-jointed arm level
with the ground, and $\theta$ is the angle of the arm relative to the ground.
Therefore, the force applied is greatest when the arm is parallel with the
ground and zero when the arm is perpendicular to the ground (at that point, the
joint supports all the weight).

Note that the elevator model could be augmented easily enough to include gravity
and still be linear, but this wouldn't work for the single-jointed arm since a
trigonometric function is required to model the gravitational force in the arm's
rotating reference frame\footnote{While the applied torque of the motor is
constant throughout the arm's range of motion, the torque caused by gravity in
the opposite direction varies according to the arm's angle.}.
