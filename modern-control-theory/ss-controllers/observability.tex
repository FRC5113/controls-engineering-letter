\section{Observability}
\index{controller design!observability}

Observability is a measure for how well internal \glspl{state} of a \gls{system}
can be inferred by knowledge of its external \glspl{output}. The observability
and controllability of a \gls{system} are mathematical duals (i.e., as
controllability proves that an \gls{input} is available that brings any initial
\gls{state} to any desired final \gls{state}, observability proves that knowing
enough \gls{output} values provides enough information to predict the initial
\gls{state} of the \gls{system}).
\begin{theorem}[Observability]
  A continuous \gls{time-invariant} linear state-space \gls{model} is observable
  if and only if
  \begin{equation}
    \rank\left(
    \begin{bmatrix}
      C \\
      CA \\
      \vdots \\
      CA^{n-1}
    \end{bmatrix}\right) = n \label{eq:obsv_rank}
  \end{equation}

  where rank is the number of linearly independent rows in a matrix and $n$ is
  the number of \gls{state} variables.
\end{theorem}

The matrix in equation \eqref{eq:obsv_rank} being rank-deficient means the
\glspl{output} do not contain contributions from every \gls{state}. That is, not
all \glspl{state} are mapped to a linear combination in the \gls{output}.
Therefore, the \glspl{output} alone are insufficient to estimate all the
\glspl{state}.

The condition number of the observability matrix $\mathcal{O}$ is defined as
$\frac{\sigma_{max}(\mathcal{O})}{\sigma_{min}(\mathcal{O})}$ where
$\sigma_{max}$ is the maximum singular value\footref{footn:singular_val} and
$\sigma_{min}$ is the minimum singular value. As this number approaches
infinity, one or more of the \glspl{state} becomes unobservable. This number can
also be used to tell us which sensors are better than others for the given
\gls{system}; a lower condition number means the \glspl{output} produced by the
sensors are better indicators of the \gls{system} \gls{state}.
