\section{Controllability}
\index{controller design!controllability}

A \gls{system} is controllable if it can be steered from any \gls{state} to any
\gls{state} by a finite sequence of admissible \glspl{input}.

\subsection{Controllability matrix}

The controllability matrix can be used to determine if a system is controllable.
\begin{theorem}[Controllability]
  A continuous \gls{time-invariant} linear state-space \gls{model} is
  controllable if and only if
  \begin{equation}
    \rank\left(
    \begin{bmatrix}
      \mat{B} & \mat{A}\mat{B} & \cdots & \mat{A}^{n-1}\mat{B}
    \end{bmatrix}
    \right) = n
    \label{eq:ctrl_rank}
  \end{equation}

  where rank is the number of linearly independent rows in a matrix and $n$ is
  the number of \glspl{state}.
\end{theorem}

The controllability matrix in equation \eqref{eq:ctrl_rank} being rank-deficient
means the \glspl{input} cannot apply transforms along all axes in the
state-space; the transformation the matrix represents is collapsed into a lower
dimension.

The condition number of the controllability matrix $\mathcal{C}$ is defined as
$\frac{\sigma_{max}(\mathcal{C})}{\sigma_{min}(\mathcal{C})}$ where
$\sigma_{max}$ is the maximum singular
value\footnote{\label{footn:singular_val}Singular values are a generalization of
eigenvalues for nonsquare matrices.} and $\sigma_{min}$ is the minimum singular
value. As this number approaches infinity, one or more of the \glspl{state}
becomes uncontrollable. This number can also be used to tell us which actuators
are better than others for the given \gls{system}; a lower condition number
means that the actuators have more control authority.

\subsection{Controllability Gramian}
\index{controller design!controllability Gramian}

While the rank of the observability matrix can tell us whether the system is
controllable, it won't tell us which specific states are controllable or how
controllable. The controllability Gramian can be used to determine these things.

If $\mat{A}$ is stable, the controllability Gramian $\mat{W}_c$ is the unique
solution to the following continuous Lyapunov equation.
\begin{equation}
  \mat{A}\mat{W}_c + \mat{W}_c\mat{A}\T + \mat{B}\mat{B}\T = 0
\end{equation}

Alternatively,
\begin{equation}
  \mat{W}_c =
    \int_0^\infty e^{\mat{A}\tau} \mat{B}\mat{B}\T e^{\mat{A}\T\tau} \,d\tau
\end{equation}

If the solution is positive definite, the system is controllable. The
eigenvalues of $\mat{W}_c$ represent how controllable their respective states
are (larger means more controllable).

\subsection{Controllability of specific states}

If you want to know if a specific state is controllable, first find its
corresponding eigenvalue $\lambda$ in $\mat{A}$. Then, that state is
controllable if
\begin{equation}
  \rank\left(
  \begin{bmatrix}
    \mat{A} - \lambda\mat{I} & \mat{B}
  \end{bmatrix}\right) = n
\end{equation}

where $n$ is the number of \glspl{state}.

\subsection{Stabilizability}
\index{controller design!stabilizability}

Stabilizability is a weaker form of controllability. A system is considered
stabilizable if one of the following conditions is true:
\begin{enumerate}
  \item All uncontrollable states can be stabilized
  \item All unstable states are controllable
\end{enumerate}
