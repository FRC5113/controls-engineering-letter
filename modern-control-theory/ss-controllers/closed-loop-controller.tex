\section{Closed-loop controller}

With the \gls{control law} $\mat{u} = \mat{K}(\mat{r} - \mat{x})$, we can derive
the closed-loop state-space equations. We'll discuss where this
\gls{control law} comes from in subsection \ref{sec:lqr}.

First is the \gls{state} update equation. Substitute the \gls{control law} into
equation \eqref{eq:ss_ctrl_x}.
\begin{align}
  \dot{\mat{x}} &= \mat{A}\mat{x} + \mat{B}\mat{K}(\mat{r} - \mat{x}) \nonumber
    \\
  \dot{\mat{x}} &= \mat{A}\mat{x} + \mat{B}\mat{K}\mat{r} -
    \mat{B}\mat{K}\mat{x} \nonumber \\
  \dot{\mat{x}} &= (\mat{A} - \mat{B}\mat{K})\mat{x} + \mat{B}\mat{K}\mat{r}
\end{align}

Now for the \gls{output} equation. Substitute the \gls{control law} into
equation \eqref{eq:ss_ctrl_y}.
\begin{align}
  \mat{y} &= \mat{C}\mat{x} + \mat{D}(\mat{K}(\mat{r} - \mat{x})) \nonumber \\
  \mat{y} &= \mat{C}\mat{x} + \mat{D}\mat{K}\mat{r} - \mat{D}\mat{K}\mat{x}
    \nonumber \\
  \mat{y} &= (\mat{C} - \mat{D}\mat{K})\mat{x} + \mat{D}\mat{K}\mat{r}
\end{align}

Now, we'll do the same for the discrete \gls{system}.
\begin{align*}
  \mat{x}_{k+1} &= \mat{A}\mat{x}_k + \mat{B}\mat{u}_k \\
  \mat{x}_{k+1} &= \mat{A}\mat{x}_k + \mat{B}(\mat{K}(\mat{r}_k - \mat{x}_k)) \\
  \mat{x}_{k+1} &= \mat{A}\mat{x}_k + \mat{B}\mat{K}\mat{r}_k -
    \mat{B}\mat{K}\mat{x}_k \\
  \mat{x}_{k+1} &= \mat{A}\mat{x}_k - \mat{B}\mat{K}\mat{x}_k +
    \mat{B}\mat{K}\mat{r}_k \\
  \mat{x}_{k+1} &= (\mat{A} - \mat{B}\mat{K})\mat{x}_k + \mat{B}\mat{K}\mat{r}_k
\end{align*}
\begin{theorem}[Closed-loop state-space controller]
  \index{state-space controllers!closed-loop}
  \begin{align}
    \intertext{Continuous:}
    \dot{\mat{x}} &= (\mat{A} - \mat{B}\mat{K})\mat{x} + \mat{B}\mat{K}\mat{r}
      \label{eq:s_ref_ctrl_x} \\
    \mat{y} &= (\mat{C} - \mat{D}\mat{K})\mat{x} + \mat{D}\mat{K}\mat{r}
    \intertext{Discrete:}
    \mat{x}_{k+1} &= (\mat{A} - \mat{B}\mat{K})\mat{x}_k +
      \mat{B}\mat{K}\mat{r}_k \label{eq:z_ref_ctrl_x} \\
    \mat{y}_k &= (\mat{C} - \mat{D}\mat{K})\mat{x}_k + \mat{D}\mat{K}\mat{r}_k
  \end{align}
  \begin{figurekey}
    \begin{tabular}{llll}
      $\mat{A}$ & system matrix      & $\mat{K}$ & controller gain matrix \\
      $\mat{B}$ & input matrix       & $\mat{r}$ & \gls{reference} vector \\
      $\mat{C}$ & output matrix      & $\mat{x}$ & state vector \\
      $\mat{D}$ & feedthrough matrix & $\mat{y}$ & output vector \\
    \end{tabular}
  \end{figurekey}
\end{theorem}
\begin{booktable}
  \begin{tabular}{|ll|ll|}
    \hline
    \rowcolor{headingbg}
    \textbf{Matrix} & \textbf{Rows $\times$ Columns} &
    \textbf{Matrix} & \textbf{Rows $\times$ Columns} \\
    \hline
    $\mat{A}$ & states $\times$ states & $\mat{r}$ & states $\times$ 1 \\
    $\mat{B}$ & states $\times$ inputs & $\mat{x}$ & states $\times$ 1 \\
    $\mat{C}$ & outputs $\times$ states & $\mat{u}$ & inputs $\times$ 1 \\
    $\mat{D}$ & outputs $\times$ inputs & $\mat{y}$ & outputs $\times$ 1 \\
    $\mat{K}$ & inputs $\times$ states &  &  \\
    \hline
  \end{tabular}
  \caption{Controller matrix dimensions}
\end{booktable}

\index{stability!eigenvalues}
Instead of commanding the \gls{system} to a \gls{state} using the vector
$\mat{u}$ directly, we can now specify a vector of desired \glspl{state} through
$\mat{r}$ and the \gls{controller} will choose values of $\mat{u}$ for us over
time to make the \gls{system} converge to the \gls{reference}. For equation
\eqref{eq:s_ref_ctrl_x} to reach steady-state, the eigenvalues of
$\mat{A} - \mat{B}\mat{K}$ must be in the left-half plane. For equation
\eqref{eq:z_ref_ctrl_x} to have a bounded output, the eigenvalues of
$\mat{A} - \mat{B}\mat{K}$ must be within the unit circle.

The eigenvalues of $\mat{A} - \mat{B}\mat{K}$ are the poles of the closed-loop
\gls{system}. Therefore, the rate of convergence and stability of the
closed-loop \gls{system} can be changed by moving the poles via the eigenvalues
of $\mat{A} - \mat{B}\mat{K}$. $\mat{A}$ and $\mat{B}$ are inherent to the
\gls{system}, but $\mat{K}$ can be chosen arbitrarily by the controller
designer.
