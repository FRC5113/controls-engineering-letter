\section{Eigenvalues and stability}

\index{stability!poles}
The eigenvalues of $\mat{A}$ are called \textit{poles}. The locations of the
closed-loop poles in the complex plane determine the stability of the
\gls{system}. Each pole represents a frequency mode of the \gls{system}, and
their location determines how much of each response is induced for a given input
frequency. Figure \ref{fig:impulse_response_eig} shows the \glspl{impulse
response} in the time domain for \glspl{system} with various pole locations.
They all have an initial condition of $1$.
\begin{bookfigure}
  \begin{tikzpicture}[auto, >=latex']
  % \draw [help lines] (-4,-2) grid (4,4);

  % Draw main axes
  \draw[->] (-4,0) -- (4,0) node[below] {\small Re($\sigma$)};
  \draw[->] (0,-2) -- (0,4) node[right] {\small Im($j\omega$)};

  % Stable: e^-1.75t * cos(1.75wt) (80/3*w for readability)
  \drawtimeplot{-2.125cm}{2.5cm}{0.125cm}{0.44375cm}{
    exp(-1.75 * \x) * cos(80/3 * 1.75 * deg(\x))}
  \drawpole{-1.75cm}{1.75cm}

  % Stable: e^-2.5t
  \drawtimeplot{-2.25cm}{0.75cm}{0.125cm}{0.125cm}{exp(-2 * \x)}
  \drawpole{-2cm}{0cm}

  % Stable: e^-t
  \drawtimeplot{-1.125cm}{-0.75cm}{0.125cm}{0.125cm}{exp(-\x)}
  \drawpole{-1cm}{0cm}

  % Marginally stable: cos(wt) (80/3*w for readability)
  \drawtimeplot{-0.75cm}{1.125cm}{0.125cm}{0.44375cm}{cos(80/3 * deg(\x))}
  \drawpole{0cm}{1cm}

  % Marginally stable cos(2wt) (80/3*w for readability)
  \drawtimeplot{0cm}{2.75cm}{0.125cm}{0.44375cm}{cos(80/3 * 2 * deg(\x))}
  \drawpole{0cm}{2cm}

  % Integrator
  \drawtimeplot{0.25cm}{-0.75cm}{0.125cm}{0.125cm}{1}
  \drawpole{0cm}{0cm}

  % Unstable: e^t
  \drawtimeplot{1.125cm}{0.75cm}{0.125cm}{0.125cm}{exp(\x)}
  \drawpole{1cm}{0cm}

  % Unstable: e^2t
  \drawtimeplot{2.25cm}{-0.75cm}{0.125cm}{0.125cm}{exp(2 * \x)}
  \drawpole{2cm}{0cm}

  % Unstable: e^0.75t * cos(1.75wt) (80/3*w for readability)
  \drawtimeplot{1.5cm}{2.25cm}{0.125cm}{0.44375cm}{
    exp(0.75 * \x) * cos(80/3 * 1.75 * deg(\x))}
  \drawpole{0.75cm}{1.75cm}

  % LHP and RHP labels
  \draw (-3.5,1.5) node {LHP};
  \draw (3.5,1.5) node {RHP};

  % Stable and unstable labels
  \draw (-2.5,3.5) node {\small Stable};
  \draw (2.5,3.5) node {\small Unstable};
\end{tikzpicture}

  \caption{Impulse response vs pole location}
  \label{fig:impulse_response_eig}
\end{bookfigure}

Poles in the left half-plane (LHP) are stable; the \gls{system}'s output may
oscillate but it converges to steady-state. Poles on the imaginary axis are
marginally stable; the \gls{system}'s output oscillates at a constant amplitude
forever. Poles in the right half-plane (RHP) are unstable; the \gls{system}'s
output grows without bound.
\begin{remark}
  Imaginary poles always come in complex conjugate pairs (e.g., $-2 + 3i$,
  $-2 - 3i$).
\end{remark}
