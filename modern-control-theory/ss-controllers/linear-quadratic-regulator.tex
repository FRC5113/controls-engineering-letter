\section{Linear-quadratic regulator} \label{sec:lqr}
\index{controller design!linear-quadratic regulator}
\index{optimal control!linear-quadratic regulator}

Instead of placing the poles of a closed-loop \gls{system} manually, the
linear-quadratic regulator (LQR) places the poles for us based on acceptable
relative \gls{error} and \gls{control effort} costs. This method of controller
design uses a quadratic function for the cost-to-go defined as the sum of the
\gls{error} and \gls{control effort} over time for the linear \gls{system}
$\dot{\mtx{x}} = \mtx{A}\mtx{x} + \mtx{B}\mtx{u}$.
\begin{equation*}
  J = \int\limits_0^\infty \left(\mtx{x}^T\mtx{Q}\mtx{x} +
    \mtx{u}^T\mtx{R}\mtx{u}\right) dt
\end{equation*}

where $J$ represents a trade-off between \gls{state} excursion and
\gls{control effort} with the weighting factors $\mtx{Q}$ and $\mtx{R}$. LQR
finds a \gls{control law} $\mtx{u}$ that minimizes the cost function. $\mtx{Q}$
and $\mtx{R}$ slide the cost along a Pareto boundary between state tracking and
\gls{control effort} (see figure \ref{fig:pareto_boundary}). Pareto optimality
for this problem means that an improvement in state \gls{tracking} cannot be
obtained without using more \gls{control effort} to do so. Also, a reduction in
\gls{control effort} cannot be obtained without sacrificing state \gls{tracking}
performance. Pole placement, on the other hand, will have a cost anywhere on,
above, or to the right of the Pareto boundary (no cost can be inside the
boundary).
\begin{svg}{build/\chapterpath/pareto_boundary}
  \caption{Pareto boundary for LQR}
  \label{fig:pareto_boundary}
\end{svg}

The minimum of LQR's cost function is found by setting the derivative of the
cost function to zero and solving for the \gls{control law} $\mtx{u}$. However,
matrix calculus is used instead of normal calculus to take the derivative.

The feedback \gls{control law} that minimizes $J$ is shown in theorem
\ref{thm:linear-quadratic_regulator}.
\begin{theorem}[Linear-quadratic regulator]
  \label{thm:linear-quadratic_regulator}

  \begin{align}
    \min_{\mtx{u}} &\int\limits_0^\infty \left(\mtx{x}^T\mtx{Q}\mtx{x} +
      \mtx{u}^T\mtx{R}\mtx{u}\right) dt \nonumber \\
    \text{subject to } &\dot{\mtx{x}} = \mtx{A}\mtx{x} + \mtx{B}\mtx{u}
  \end{align}

  If the \gls{system} is controllable, the optimal control policy $\mtx{u}^*$
  that drives all the \glspl{state} to zero is $-\mtx{K}\mtx{x}$. To converge to
  nonzero \glspl{state}, a \gls{reference} vector $\mtx{r}$ can be added to the
  \gls{state} $\mtx{x}$.

  \begin{equation}
    \mtx{u} = \mtx{K}(\mtx{r} - \mtx{x})
  \end{equation}
\end{theorem}
\index{controller design!linear-quadratic regulator!definition}
\index{optimal control!linear-quadratic regulator!definition}

This means that optimal control can be achieved with simply a set of
proportional gains on all the \glspl{state}. To use the \gls{control law}, we
need knowledge of the full \gls{state} of the \gls{system}. That means we either
have to measure all our \glspl{state} directly or estimate those we do not
measure.

See appendix \ref{ch:deriv_lqr} for how $\mtx{K}$ is calculated in Python. If
the result is finite, the controller is guaranteed to be stable and
\glslink{robustness}{robust} with a \gls{phase margin} of 60 degrees
\cite{bib:lqr_phase_margin}.
\begin{remark}
  LQR design's $\mtx{Q}$ and $\mtx{R}$ matrices don't need \gls{discretization},
  but the $\mtx{K}$ calculated for continuous time and discrete time
  \glspl{system} will be different. The discrete time gains approach the
  continuous time gains as the sample period tends to zero.
\end{remark}

\subsection{Bryson's rule}
\index{controller design!linear-quadratic regulator!Bryson's rule}
\index{optimal control!linear-quadratic regulator!Bryson's rule}

The next obvious question is what values to choose for $\mtx{Q}$ and $\mtx{R}$.
With Bryson's rule, the diagonals of the $\mtx{Q}$ and $\mtx{R}$ matrices are
chosen based on the maximum acceptable value for each \gls{state} and actuator.
The nondiagonal elements are zero. The balance between $\mtx{Q}$ and $\mtx{R}$
can be slid along the Pareto boundary using a weighting factor $\rho$.
\begin{equation*}
  J = \int\limits_0^\infty \left(\rho \left[
    \left(\frac{x_1}{x_{1,max}}\right)^2 + \ldots +
    \left(\frac{x_m}{x_{m,max}}\right)^2\right] + \left[
    \left(\frac{u_1}{u_{1,max}}\right)^2 + \ldots +
    \left(\frac{u_n}{u_{n,max}}\right)^2\right]\right) dt
\end{equation*}
\begin{equation*}
  \begin{array}{cc}
    \mtx{Q} = \begin{bmatrix}
      \frac{\rho}{x_{1,max}^2} & 0 & \ldots & 0 \\
      0 & \frac{\rho}{x_{2,max}^2} & & \vdots \\
      \vdots & & \ddots & 0 \\
      0 & \ldots & 0 & \frac{\rho}{x_{m,max}^2}
    \end{bmatrix} &
    \mtx{R} = \begin{bmatrix}
      \frac{1}{u_{1,max}^2} & 0 & \ldots & 0 \\
      0 & \frac{1}{u_{2,max}^2} & & \vdots \\
      \vdots & & \ddots & 0 \\
      0 & \ldots & 0 & \frac{1}{u_{n,max}^2}
    \end{bmatrix}
  \end{array}
\end{equation*}

Small values of $\rho$ penalize \gls{control effort} while large values of
$\rho$ penalize \gls{state} excursions. Large values would be chosen in
applications like fighter jets where performance is necessary. Spacecrafts would
use small values to conserve their limited fuel supply.

\subsection{Pole placement vs LQR}

This example uses the following second-order \gls{model} for a CIM motor (a DC
brushed motor).
\begin{align*}
  \begin{array}{cccc}
    \mtx{A} = \begin{bmatrix}
      -\frac{b}{J} & \frac{K_t}{J} \\
      -\frac{K_e}{L} & -\frac{R}{L}
    \end{bmatrix} &
    \mtx{B} = \begin{bmatrix}
      0 \\
      \frac{1}{L}
    \end{bmatrix} &
    \mtx{C} = \begin{bmatrix}
      1 & 0
    \end{bmatrix} &
    \mtx{D} = \begin{bmatrix}
      0
    \end{bmatrix}
  \end{array}
\end{align*}

Figure \ref{fig:case_study_pp_lqr} shows the response using discrete
poles\footnote{See section \ref{sec:s-plane_to_z-plane} for pole mappings of
discrete systems (inside unit circle is stable). The pole mappings mentioned so
far (LHP is stable) only apply to continuous systems.} placed at $(0.1, 0)$ and
$(0.9, 0)$ and LQR with the following cost matrices.
\begin{align*}
  \begin{array}{cc}
    \mtx{Q} = \begin{bmatrix}
      \frac{1}{20^2} & 0 \\
      0 & \frac{1}{40^2}
    \end{bmatrix} &
    \mtx{R} = \begin{bmatrix}
      \frac{1}{12^2}
    \end{bmatrix}
  \end{array}
\end{align*}
\begin{svg}{build/\chapterpath/case_study_pp_lqr}
  \caption{Second-order CIM motor response with pole placement and LQR}
  \label{fig:case_study_pp_lqr}
\end{svg}

LQR selected poles at $(0.593, 0)$ and $(0.955, 0)$. Notice with pole placement
that as the current pole moves left, the \gls{control effort} becomes more
aggressive.
