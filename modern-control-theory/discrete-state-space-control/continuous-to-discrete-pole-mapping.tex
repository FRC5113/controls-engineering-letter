\section{Continuous to discrete pole mapping}

When a continuous system is discretized, its poles in the LHP map to the inside
of a unit circle. Table \ref{tab:c2d_mapping} contains a few common points and
figure \ref{fig:c2d_mapping} shows the mapping visually.
\begin{booktable}
  \begin{tabular}{|cc|}
    \hline
    \rowcolor{headingbg}
    \textbf{Continuous} & \textbf{Discrete} \\
    \hline
    $(0, 0)$ & $(1, 0)$ \\
    imaginary axis & edge of unit circle \\
    $(-\infty, 0)$ & $(0, 0)$ \\
    \hline
  \end{tabular}
  \caption{Mapping from continuous to discrete}
  \label{tab:c2d_mapping}
\end{booktable}
\begin{bookfigure}
  \begin{minisvg}{2}{build/\chapterpath/s_plane}
  \end{minisvg}
  \hfill
  \begin{minisvg}{2}{build/\chapterpath/z_plane}
  \end{minisvg}
  \caption{Mapping of axes from continuous (left) to discrete (right)}
  \label{fig:c2d_mapping}
\end{bookfigure}

\subsection{Discrete system stability}

Eigenvalues of a \gls{system} that are within the unit circle are stable, but
why is that? Let's consider a scalar equation $x_{k + 1} = ax_k$. $a < 1$ makes
$x_{k + 1}$ converge to zero. The same applies to a complex number like
$z = x + yi$ for $x_{k + 1} = zx_k$. If the magnitude of the complex number $z$
is less than one, $x_{k+1}$ will converge to zero. Values with a magnitude of
$1$ oscillate forever because $x_{k+1}$ never decays.

\subsection{Discrete system behavior}

As $\omega$ increases in $s = j\omega$, a pole in the discrete plane moves
around the perimeter of the unit circle. Once it hits $\frac{\omega_s}{2}$ (half
the sampling frequency) at $(-1, 0)$, the pole wraps around. This is due to
poles faster than the sample frequency folding down to below the sample
frequency (that is, higher frequency signals \textit{alias} to lower frequency
ones).

You may notice that poles can be placed at $(0, 0)$ in the discrete plane. This
is known as a deadbeat controller. An $\rm N^{th}$-order deadbeat controller
decays to the \gls{reference} in N timesteps. While this sounds great, there are
other considerations like \gls{control effort}, \gls{robustness}, and
\gls{noise immunity}.

If poles from $(1, 0)$ to $(0, 0)$ on the x-axis approach infinity, then what do
poles from $(-1, 0)$ to $(0, 0)$ represent? Them being faster than infinity
doesn't make sense. Poles in this location exhibit oscillatory behavior similar
to complex conjugate pairs. See figures \ref{fig:continuous_oscillations_1p} and
\ref{fig:discrete_oscillations_2p}. The jaggedness of these signals is due to
the frequency of the \gls{system} dynamics being above the Nyquist frequency
(twice the sample frequency). The \glslink{discretization}{discretized} signal
doesn't have enough samples to reconstruct the continuous \gls{system}'s
dynamics.
\begin{bookfigure}
  \begin{minisvg}{2}{build/\chapterpath/z_oscillations_1p}
    \caption{Single poles in various locations in discrete plane}
    \label{fig:continuous_oscillations_1p}
  \end{minisvg}
  \hfill
  \begin{minisvg}{2}{build/\chapterpath/z_oscillations_2p}
    \caption{Complex conjugate poles in various locations in discrete plane}
    \label{fig:discrete_oscillations_2p}
  \end{minisvg}
\end{bookfigure}

\subsection{Nyquist frequency}
\index{digital signal processing!Nyquist frequency}
\index{digital signal processing!aliasing}

To completely reconstruct a signal, the Nyquist-Shannon sampling theorem states
that it must be sampled at a frequency at least twice the maximum frequency it
contains. The highest frequency a given sample rate can capture is called the
Nyquist frequency, which is half the sample frequency. This is why recorded
audio is sampled at $44.1$ kHz. The maximum frequency a typical human can hear
is about $20$ kHz, so the Nyquist frequency is $20$ kHz and the minimum sampling
frequency is $40$ kHz. ($44.1$ kHz in particular was chosen for unrelated
historical reasons.)

Frequencies above the Nyquist frequency are folded down across it. The higher
frequency and the folded down lower frequency are said to alias each
other\footnote{The aliases of a frequency $f$ can be expressed as
$f_{alias}(N) \stackrel{def}{=} |f - Nf_s|$. For example, if a $200$ Hz sine
wave is sampled at $150$ Hz, the \gls{observer} will see a $50$ Hz signal
instead of a $200$ Hz one.}. Figure \ref{fig:c2d_aliasing} demonstrates
aliasing.
\begin{svg}{build/\chapterpath/aliasing}
  \caption{The original signal is a $1.5$ Hz sine wave, which means its Nyquist
    frequency is $1.5$ Hz. The signal is being sampled at $2$ Hz, so the aliased
    signal is a $0.5$ Hz sine wave.}
    \label{fig:c2d_aliasing}
\end{svg}

The effect of these high-frequency aliases can be reduced with a low-pass filter
(called an anti-aliasing filter in this application).
