\section{Effects of discretization on controller performance}

\subsection{Sample delay}

Implementing a discrete control system is easier than implementing a continuous
one, but \gls{discretization} has drawbacks. A microcontroller updates the
system input in discrete intervals of duration $T$; it's held constant between
updates. This introduces an average sample delay of $\frac{T}{2}$. Large delays
can make a stable controller in the continuous domain become unstable in the
discrete domain since it can't react as quickly to output changes. Here are a
few ways to combat this.
\begin{itemize}
  \item Run the controller with a high sample rate.
  \item Designing the controller in the analog domain with enough
    \gls{phase margin} to compensate for any phase loss that occurs as part of
    \gls{discretization}.
  \item Convert the \gls{plant} to the digital domain and design the controller
    completely in the digital domain.
\end{itemize}

\subsection{Sample rate}

Running a feedback controller at a faster update rate doesn't always mean better
control. In fact, you may be using more computational resources than you need.
However, here are some reasons for running at a faster update rate.

Firstly, if you have a discrete \gls{model} of the \gls{system}, that
\gls{model} can more accurately approximate the underlying continuous
\gls{system}. Not all controllers use a \gls{model} though.

Secondly, the controller can better handle fast \gls{system} dynamics. If the
\gls{system} can move from its initial state to the desired one in under $250$
ms, you obviously want to run the controller with a period less than $250$ ms.
When you reduce the sample period, you're making the discrete controller more
accurately reflect what the equivalent continuous controller would do
(controllers built from analog circuit components like op-amps are continuous).

Running at a lower sample rate only causes problems if you don't take into
account the response time of your \gls{system}. Some \glspl{system} like heaters
have \glspl{output} that change on the order of minutes. Running a control loop
at $1$ kHz doesn't make sense for this because the \gls{plant} \gls{input} the
controller computes won't change much, if at all, in $1$ ms.
