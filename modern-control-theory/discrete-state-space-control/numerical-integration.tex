\section{Numerical integration methods}

Most systems don't have linear dynamics and their differential equations can't
be solved analytically. Instead, we'll have to approximate their solutions with
numerical integration.

\subsection{Butcher tableaus}

Butcher tableaus are a more succinct representation for explicit Runge-Kutta
numerical integration methods. Here's the general structure.
\begin{equation*}
  \renewcommand\arraystretch{1.2}
  \begin{array}{c|cccc}
    0 \\
    c_2    & a_{2,1} \\
    \vdots & \vdots & \ddots \\
    c_s    & a_{s,1} & \hdots & a_{s,s-1} \\
    \hline
           & b_1    & \hdots & \hdots    & b_s
  \end{array}
\end{equation*}

where $s$ is the number of stages in the method, the matrix $[a_{ij}]$ is the
Runge-Kutta matrix, $b_1, \ldots, b_s$ are the weights, and $c_1, \ldots, c_s$
are the nodes. The top-left quadrant contains the sums of the rows in the
top-right quadrant. Each column in the right half corresponds to a $\mat{k}$
coefficient from $\mat{k}_1$ to $\mat{k}_s$.

The family of solutions to $\dot{\mat{x}} = f(t, \mat{x})$ is given by
\begin{align*}
  \mat{k}_1 &= f(t_k, \mat{x}_k) \\
  \mat{k}_2 &= f(t_k + c_2 h, \mat{x}_k + h (a_{2,1} \mat{k}_1)) \\
  &\ \ \vdots \\
  \mat{k}_s &= f(t_k + c_s h, \mat{x}_k +
    h (a_{s,1} \mat{k}_1 + \ldots + a_{s,s-1} \mat{k}_{s-1})) \\
  \mat{x}_{k+1} &= \mat{x}_k + h \sum_{i=1}^s b_i \mat{k}_i
\end{align*}

where $h$ is the timestep duration.

\subsection{Runge-Kutta fourth-order method}
\index{numerical integration!Runge-Kutta fourth-order}

The first and most common method we'll cover is Runge-Kutta fourth-order (RK4).
It's simple and accurate for most systems we'll see in FRC. To demonstrate how
to translate a Butcher tableau into equations, we'll be integrating
$\dot{\mat{x}} = f(t, \mat{x})$ from $0$ to $h$.
\begin{center}
  \begin{minipage}{0.35\linewidth}
    \centering
    \begin{alignat*}{7}
      \mat{k}_1 &= f(t +
        && {\color{blue}0} h,
        && \mat{x}_k)
        &&
        &&
        &&
        && \\
      \mat{k}_2 &= f(t +
        && {\color{blue}\frac{1}{2}} h,
        && \mat{x}_k + h (
        && {\color{deeporange}\frac{1}{2}} \mat{k}_1))
        &&
        &&
        && \\
      \mat{k}_3 &= f(t +
        && {\color{blue}\frac{1}{2}} h,
        && \mat{x}_k + h (
        && {\color{deeporange}0} \mat{k}_1 +
        && {\color{deeporange}\frac{1}{2}} \mat{k}_2))
        &&
        && \\
      \mat{k}_4 &= f(t +
        && {\color{blue}1} h,
        && \mat{x}_k + h (
        && {\color{deeporange}0} \mat{k}_1 +
        && {\color{deeporange}0} \mat{k}_2 +
        && {\color{deeporange}1} \mat{k}_3))
        && \\
      \mat{x}_{k+1} &=
        &&
        && \mat{x}_k + h (
        && {\color{deepgreen}\frac{1}{6}} \mat{k}_1 +
        && {\color{deepgreen}\frac{1}{3}} \mat{k}_2 +
        && {\color{deepgreen}\frac{1}{3}} \mat{k}_3 +
        && {\color{deepgreen}\frac{1}{6}} \mat{k}_4)
    \end{alignat*}
  \end{minipage}
  \quad
  \begin{minipage}{0.35\linewidth}
    \centering
    \begin{equation*}
      \renewcommand\arraystretch{1.2}
      \begin{array}{c|cccc}
        {\color{blue}0} \\
        {\color{blue}\frac{1}{2}} & {\color{deeporange}\frac{1}{2}} \\
        {\color{blue}\frac{1}{2}} & {\color{deeporange}0}           & {\color{deeporange}\frac{1}{2}} \\
        {\color{blue}1}           & {\color{deeporange}0}           & {\color{deeporange}0}           & {\color{deeporange}1} \\
        \hline
                                  & {\color{deepgreen}\frac{1}{6}}  & {\color{deepgreen}\frac{1}{3}}  & {\color{deepgreen}\frac{1}{3}} & {\color{deepgreen}\frac{1}{6}}
      \end{array}
    \end{equation*}
  \end{minipage}
\end{center}

Removing zeroed out terms gives
\begin{align*}
  \mat{k}_1 &= f(t, \mat{x}_k) \\
  \mat{k}_2 &= f(t + \frac{1}{2} h, \mat{x}_k + h \frac{1}{2} \mat{k}_1) \\
  \mat{k}_3 &= f(t + \frac{1}{2} h, \mat{x}_k + h \frac{1}{2} \mat{k}_2) \\
  \mat{k}_4 &= f(t + h, \mat{x}_k + h \mat{k}_3) \\
  \mat{x}_{k+1} &= \mat{x}_k + h \left(
    \frac{1}{6} \mat{k}_1 +
    \frac{1}{3} \mat{k}_2 +
    \frac{1}{3} \mat{k}_3 +
    \frac{1}{6} \mat{k}_4\right)
  \intertext{$\frac{1}{6}$ is usually factored out of the last equation to
    reduce the number of floating point operations.}
  \mat{x}_{k+1} &= \mat{x}_k + h \frac{1}{6} (
    \mat{k}_1 + 2\mat{k}_2 + 2\mat{k}_3 + \mat{k}_4)
\end{align*}

In FRC, our differential equations are of the form
$\dot{\mat{x}} = f(\mat{x}, \mat{u})$ where $\mat{u}$ is held constant between
timesteps. Since it's time-invariant, we can ignore the time argument of the
integration method. This gives theorem \ref{thm:rk4}.
\begin{theorem}[Fourth-order Runge-Kutta integration]
  \label{thm:rk4}

  Given the differential equation $\dot{\mat{x}} = f(\mat{x}_k, \mat{u}_k)$,
  this method solves for $\mat{x}_{k+1}$ at $dt$ seconds in the future.
  $\mat{u}$ is assumed to be held constant between timesteps.
  \begin{center}
    \begin{minipage}{0.35\linewidth}
      \centering
      \begin{align*}
        \mat{k}_1 &= f(\mat{x}_k, \mat{u}_k) \\
        \mat{k}_2 &= f(\mat{x}_k + dt \frac{1}{2}\mat{k}_1, \mat{u}_k) \\
        \mat{k}_3 &= f(\mat{x}_k + dt \frac{1}{2}\mat{k}_2, \mat{u}_k) \\
        \mat{k}_4 &= f(\mat{x}_k + dt \mat{k}_3, \mat{u}_k) \\
        \mat{x}_{k+1} &= \mat{x}_k + dt \frac{1}{6} (\mat{k}_1 + 2\mat{k}_2 +
          2\mat{k}_3 + \mat{k}_4)
      \end{align*}
    \end{minipage}
    \quad
    \begin{minipage}{0.35\linewidth}
      \centering
      \begin{equation*}
        \renewcommand\arraystretch{1.2}
        \begin{array}{c|cccc}
          0 \\
          \frac{1}{2} & \frac{1}{2} \\
          \frac{1}{2} & 0 & \frac{1}{2} \\
          1 & 0 & 0 & 1 \\
          \hline
          & \frac{1}{6} & \frac{1}{3} & \frac{1}{3} & \frac{1}{6}
        \end{array}
      \end{equation*}
    \end{minipage}
  \end{center}
\end{theorem}

Here's a reference implementation.
\begin{coderemote}{cpp}{snippets/RK4.cpp}
  \caption{RK4 implementation in C++}
\end{coderemote}

Other methods of Runge-Kutta integration exist with various properties
\cite{bib:wiki_rk4}, but the one presented here is popular for its high accuracy
relative to the amount of floating point operations (FLOPs) it requires.

\subsection{Runge-Kutta-Fehlberg method}
\index{numerical integration!Runge-Kutta-Fehlberg}

Runge-Kutta-Fehlberg (RKF45) is a fourth-order method like Runge-Kutta, but it
uses an adaptive stepsize to enforce an upper bound on the integration error.
\begin{theorem}[Runge-Kutta-Fehlberg integration]
  Given the differential equation $\dot{\mat{x}} = f(\mat{x}_k, \mat{u}_k)$,
  this method solves for $\mat{x}_{k+1}$ at $dt$ seconds in the future.
  $\mat{u}$ is assumed to be held constant between timesteps. It has the
  following Butcher tableau.
  \begin{equation*}
    \renewcommand\arraystretch{1.2}
    \begin{array}{c|ccccccc}
      0 \\
      \frac{1}{4} & \frac{1}{4} \\
      \frac{3}{8} & \frac{3}{32} & \frac{9}{32} \\
      \frac{12}{13} & \frac{1932}{2197} & -\frac{7200}{2197} & \frac{7296}{2197}
        \\
      1 & \frac{439}{216} & -8 & \frac{3680}{513} & -\frac{845}{4104} \\
      \frac{1}{2} & -\frac{8}{27} & 2 & -\frac{3544}{2565} & \frac{1859}{4104} &
        -\frac{11}{40} \\
      \hline
      & \frac{16}{135} & 0 & \frac{6656}{12825} & \frac{28561}{56430} &
        -\frac{9}{50} & \frac{2}{55} \\
      & \frac{25}{216} & 0 & \frac{1408}{2565} & \frac{2197}{4104} &
        -\frac{1}{5} & 0
    \end{array}
  \end{equation*}
  The first row of coefficients below the table divider gives the fifth-order
  accurate solution. The second row gives an alternative solution which,
  when subtracted from the first solution, gives the error estimate.
\end{theorem}

Here's a reference implementation.
\begin{coderemote}{cpp}{snippets/RKF45.cpp}
  \caption{RKF45 implementation in C++}
\end{coderemote}

\subsection{Dormand-Prince method}
\index{numerical integration!Dormand-Prince}

Dormand-Prince (RKDP) is similar to RKF45, but it has higher precision per unit
work.
\begin{theorem}[Dormand-Prince integration]
  Given the differential equation $\dot{\mat{x}} = f(\mat{x}_k, \mat{u}_k)$,
  this method solves for $\mat{x}_{k+1}$ at $dt$ seconds in the future.
  $\mat{u}$ is assumed to be held constant between timesteps. It has the
  following Butcher tableau.
  \begin{equation*}
    \renewcommand\arraystretch{1.2}
    \begin{array}{c|ccccccc}
      0 \\
      \frac{1}{5} & \frac{1}{5} \\
      \frac{3}{10} & \frac{3}{40} & \frac{9}{40} \\
      \frac{4}{5} & \frac{44}{45} & -\frac{56}{15} & \frac{32}{9} \\
      \frac{8}{9} & \frac{19372}{6561} & -\frac{25360}{2187} &
        \frac{64448}{6561} & -\frac{212}{729} \\
      1 & \frac{9017}{3168} & -\frac{355}{33} & \frac{46732}{5247} &
        \frac{49}{176} & -\frac{5103}{18656} \\
      1 & \frac{35}{384} & 0 & \frac{500}{1113} & \frac{125}{192} &
        -\frac{2187}{6784} & \frac{11}{84} \\
      \hline
      & \frac{35}{384} & 0 & \frac{500}{1113} & \frac{125}{192} &
        -\frac{2187}{6784} & \frac{11}{84} & 0 \\
      & \frac{5179}{57600} & 0 & \frac{7571}{16695} & \frac{393}{640} &
        -\frac{92097}{339200} & \frac{187}{2100} & \frac{1}{40}
    \end{array}
  \end{equation*}
  The first row of coefficients below the table divider gives the fifth-order
  accurate solution. The second row gives an alternative solution which,
  when subtracted from the first solution, gives the error estimate.
\end{theorem}

Here's a reference implementation.
\begin{coderemote}{cpp}{snippets/RKDP.cpp}
  \caption{RKDP implementation in C++}
\end{coderemote}
