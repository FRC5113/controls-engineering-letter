\section{Discretization methods}
\index{discretization}

\Gls{discretization} is done using a zero-order hold. That is, the \gls{system}
\gls{state} is only updated at discrete intervals and it's held constant between
samples (see figure \ref{fig:zoh}). The exact method of applying this uses the
matrix exponential, but this can be computationally expensive. Instead,
approximations such as the following are used.
\begin{enumerate}
  \item Forward Euler method. This is defined as
    $y_{n+1} = y_n + f(t_n, y_n) \Delta t$.
    \index{discretization!forward Euler method}
  \item Backward Euler method. This is defined as
    $y_{n+1} = y_n + f(t_{n+1}, y_{n+1}) \Delta t$.
    \index{discretization!backward Euler method}
  \item Bilinear transform. The first-order bilinear approximation is
    $s = \frac{2}{T} \frac{1 - z^{-1}}{1 + z^{-1}}$.
    \index{discretization!bilinear transform}
\end{enumerate}

where the function $f(t_n, y_n)$ is the slope of $y$ at $n$ and $T$ is the
sample period for the discrete \gls{system}. Each of these methods is
essentially finding the area underneath a curve. The forward and backward Euler
methods use rectangles to approximate that area while the bilinear transform
uses trapezoids (see figures \ref{fig:discretization_methods_vel} and
\ref{fig:discretization_methods_pos}). Since these are approximations, there is
distortion between the real discrete \gls{system}'s poles and the approximate
poles. This is in addition to the phase loss introduced by discretizing at a
given sample rate in the first place. For fast-changing \glspl{system}, this
distortion can quickly lead to instability.
\begin{bookfigure}
  \begin{minisvg}{2}{build/\chapterpath/zoh}
      \caption{Zero-order hold of a system response}
      \label{fig:zoh}
  \end{minisvg}
  \hfill
  \begin{minisvg}{2}{build/\chapterpath/discretization_methods_vel}
    \caption{Discretization methods applied to velocity data}
    \label{fig:discretization_methods_vel}
  \end{minisvg}
  \hfill
  \begin{minisvg}{2}{build/\chapterpath/discretization_methods_pos}
    \caption{Position plot of discretization methods applied to velocity data}
    \label{fig:discretization_methods_pos}
  \end{minisvg}
\end{bookfigure}
