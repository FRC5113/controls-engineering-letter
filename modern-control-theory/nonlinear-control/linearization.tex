\section{Linearization}
\index{nonlinear control!linearization}

One way to control nonlinear \glspl{system} is to
\glslink{linearization}{linearize} the \gls{model} around a reference point.
Then, all the powerful tools that exist for linear controls can be applied. This
is done by taking the Jacobians of $f$ and $h$.
\begin{align*}
  \begin{array}{cccc}
    \mat{A} = \frac{\partial f(\mat{x}, \mat{u}, \mat{w})}{\partial \mat{x}} &
    \mat{B} = \frac{\partial f(\mat{x}, \mat{u}, \mat{w})}{\partial \mat{u}} &
    \mat{C} = \frac{\partial h(\mat{x}, \mat{u}, \mat{v})}{\partial \mat{x}} &
    \mat{D} = \frac{\partial h(\mat{x}, \mat{u}, \mat{v})}{\partial \mat{u}}
  \end{array}
\end{align*}

Let $f(\mat{x}, \mat{u})$ be defined as
\begin{equation*}
  \dot{\mat{x}} = f(\mat{x}, \mat{u}) =
  \begin{bmatrix}
    f_1(\mat{x}, \mat{u}) \\
    f_2(\mat{x}, \mat{u}) \\
    \vdots \\
    f_m(\mat{x}, \mat{u})
  \end{bmatrix}
\end{equation*}

The subscript denotes a row of $f$, where each row represents the dynamics of a
state.

The Jacobian is the partial derivative of a vector-valued function with respect
to one of the vector arguments. The Jacobian of $f$ has as many rows as $f$, and
the columns are filled with partial deriviatives of $f$'s rows with respect to
each of the argument's elements. For example, the Jacobian of $f$ with respect
to $\mat{x}$ is
\begin{equation*}
  \mat{A} = \frac{\partial f(\mat{x}, \mat{u})}{\partial \mat{x}} =
  \begin{bmatrix}
    \frac{\partial f_1}{\partial \mat{x}_1} &
      \frac{\partial f_1}{\partial \mat{x}_2} & \hdots &
      \frac{\partial f_1}{\partial \mat{x}_m} \\
    \frac{\partial f_2}{\partial \mat{x}_1} &
      \frac{\partial f_2}{\partial \mat{x}_2} & \hdots &
      \frac{\partial f_2}{\partial \mat{x}_m} \\
    \vdots & \vdots & \ddots & \vdots \\
    \frac{\partial f_m}{\partial \mat{x}_1} &
      \frac{\partial f_m}{\partial \mat{x}_2} & \hdots &
      \frac{\partial f_m}{\partial \mat{x}_m}
  \end{bmatrix}
\end{equation*}

$\frac{\partial f_1}{\partial \mat{x}_1}$ is the partial derivative of the first
row of $f$ with respect to the first state, and so on for all rows of $f$ and
states. This has $n^2$ permutations and thus produces a square matrix.

The Jacobian of $f$ with respect to $\mat{u}$ is
\begin{equation*}
  \mat{B} = \frac{\partial f(\mat{x}, \mat{u})}{\partial \mat{u}} =
  \begin{bmatrix}
    \frac{\partial f_1}{\partial \mat{u}_1} &
      \frac{\partial f_1}{\partial \mat{u}_2} & \hdots &
      \frac{\partial f_1}{\partial \mat{u}_n} \\
    \frac{\partial f_2}{\partial \mat{u}_1} &
      \frac{\partial f_2}{\partial \mat{u}_2} & \hdots &
      \frac{\partial f_2}{\partial \mat{u}_n} \\
    \vdots & \vdots & \ddots & \vdots \\
    \frac{\partial f_m}{\partial \mat{u}_1} &
      \frac{\partial f_m}{\partial \mat{u}_2} & \hdots &
      \frac{\partial f_m}{\partial \mat{u}_n}
  \end{bmatrix}
\end{equation*}

$\frac{\partial f_1}{\partial \mat{u}_1}$ is the partial derivative of the first
row of $f$ with respect to the first input, and so on for all rows of $f$ and
inputs. This has $m \times n$ permutations and can produce a nonsquare matrix if
$m \neq n$.

Linearization of a nonlinear equation is a Taylor series expansion to only the
first-order terms (that is, terms whose variables have exponents on the order of
$x^1$). This is where the small angle approximations for $\sin\theta$ and
$\cos\theta$ ($\theta$ and $1$ respectively) come from.

Higher order partial derivatives can be added to better approximate the
nonlinear dynamics. We typically only \glslink{linearization}{linearize} around
equilibrium points\footnote{Equilibrium points are points where
$\dot{\mat{x}} = \mat{0}$. At these points, the system is in steady-state.}
because we are interested in how the \gls{system} behaves when perturbed from
equilibrium. An FAQ on this goes into more detail
\cite{bib:linearize_equilibrium_point}. To be clear though,
\glslink{linearization}{linearizing} the \gls{system} around the current
\gls{state} as the \gls{system} evolves does give a closer approximation over
time.

Note that linearization with static matrices (that is, with a time-invariant
linear \gls{system}) only works if the original \gls{system} in question is
feedback linearizable.
