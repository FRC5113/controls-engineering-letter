\section{Linearization}
\index{nonlinear control!linearization}

One way to control nonlinear \glspl{system} is to
\glslink{linearization}{linearize} the \gls{model} around a reference point.
Then, all the powerful tools that exist for linear controls can be applied. This
is done by taking the Jacobians of $f$ and $h$ with respect to the state and
input vectors. See section \ref{sec:matrix_calculus} for more on Jacobians.
\begin{align*}
  &\mat{A} = \frac{\partial f(\mat{x}, \mat{u}, \mat{w})}{\partial \mat{x}}
  &\mat{B} = \frac{\partial f(\mat{x}, \mat{u}, \mat{w})}{\partial \mat{u}} \\
  &\mat{C} = \frac{\partial h(\mat{x}, \mat{u}, \mat{v})}{\partial \mat{x}}
  &\mat{D} = \frac{\partial h(\mat{x}, \mat{u}, \mat{v})}{\partial \mat{u}}
\end{align*}

Linearization of a nonlinear equation is a Taylor series expansion to only the
first-order terms (that is, terms whose variables have exponents on the order of
$x^1$). This is where the small angle approximations for $\sin\theta$ and
$\cos\theta$ ($\theta$ and $1$ respectively) come from.

Higher order partial derivatives can be added to better approximate the
nonlinear dynamics. We typically only \glslink{linearization}{linearize} around
equilibrium points\footnote{Equilibrium points are points where
$\dot{\mat{x}} = \mat{0}$. At these points, the system is in steady-state.}
because we are interested in how the \gls{system} behaves when perturbed from
equilibrium. An FAQ on this goes into more detail
\cite{bib:linearize_equilibrium_point}. To be clear though,
\glslink{linearization}{linearizing} the \gls{system} around the current
\gls{state} as the \gls{system} evolves does give a closer approximation over
time.

Note that linearization with static matrices (that is, with a time-invariant
linear \gls{system}) only works if the original \gls{system} in question is
feedback linearizable.
