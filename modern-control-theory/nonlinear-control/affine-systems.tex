\section{Affine systems}

Let $\mat{x} = \mat{x}_0 + \delta\mat{x}$ and
$\mat{u} = \mat{u}_0 + \delta\mat{u}$ where $\delta\mat{x}$ and $\delta\mat{u}$
are perturbations from $(\mat{x}_0, \mat{u}_0)$. A first-order linearization of
$\dot{\mat{x}} = f(\mat{x}, \mat{u})$ around $(\mat{x}_0, \mat{u}_0)$ gives
\begin{align*}
  \dot{\mat{x}} &\approx f(\mat{x}_0, \mat{u}_0) +
    \left.\frac{\partial f(\mat{x}, \mat{u})}{\partial \mat{x}}
    \right|_{\mat{x}_0, \mat{u}_0}\delta\mat{x} +
    \left.\frac{\partial f(\mat{x}, \mat{u})}{\partial \mat{u}}
    \right|_{\mat{x}_0, \mat{u}_0}\delta\mat{u} \\
  \dot{\mat{x}} &= f(\mat{x}_0, \mat{u}_0) +
    \left.\frac{\partial f(\mat{x}, \mat{u})}{\partial \mat{x}}
    \right|_{\mat{x}_0, \mat{u}_0}\delta\mat{x} +
    \left.\frac{\partial f(\mat{x}, \mat{u})}{\partial \mat{u}}
    \right|_{\mat{x}_0, \mat{u}_0}\delta\mat{u} \\
\end{align*}

An affine system is a linear system with a constant offset in the dynamics. If
$(\mat{x}_0, \mat{u}_0)$ is an equilibrium point,
$f(\mat{x}_0, \mat{u}_0) = \mat{0}$, the resulting \gls{model} is linear, and
LQR works as usual. If $(\mat{x}_0, \mat{u}_0)$ is, say, the current operating
point rather than an equilibrium point, the easiest way to correctly apply LQR
is
\begin{enumerate}
  \item Find a control input $\mat{u}_0$ that makes $(\mat{x}_0, \mat{u}_0)$ an
    equilibrium point.
  \item Obtain an LQR for the linearized system.
  \item Add $\mat{u}_0$ to the LQR's control input.
\end{enumerate}

A control-affine system is of the form
$\dot{\mat{x}} = f(\mat{x}) + g(\mat{x})\mat{u}$. Since it has separable control
inputs, $\mat{u}_0$ can be derived via plant inversion as follows.
\begin{align}
  \dot{\mat{x}} &= f(\mat{x}_0) + g(\mat{x}_0)\mat{u}_0 \nonumber \\
  \mat{0} &= f(\mat{x}_0) + g(\mat{x}_0)\mat{u}_0 \nonumber \\
  g(\mat{x}_0)\mat{u}_0 &= -f(\mat{x}_0) \nonumber \\
  \mat{u}_0 &= -g^{-1}(\mat{x}_0) f(\mat{x}_0)
\end{align}

For the control-affine \gls{system}
$\dot{\mat{x}} = f(\mat{x}) + \mat{B}\mat{u}$, $\mat{u}_0$ would be
\begin{align}
  \mat{u}_0 &= -\mat{B}^+ f(\mat{x}_0)
\end{align}

\subsection{Feedback linearization for reference tracking}

Feedback linearization lets us erase the nonlinear dynamics of a system so we
can apply our own (usually linear) dynamics for \gls{reference} tracking. To do
this, we will perform a similar procedure as in subsection
\ref{subsec:plant_inversion} and solve for $\mat{u}$ given the \gls{reference}
dynamics in $\dot{\mat{r}}$.
\begin{align}
  \dot{\mat{r}} &= f(\mat{x}) + \mat{B}\mat{u} \nonumber \\
  \mat{B}\mat{u} &= \dot{\mat{r}} - f(\mat{x}) \nonumber \\
  \mat{u} &= \mat{B}^+ (\dot{\mat{r}} - f(\mat{x}))
    \label{eq:control_affine_plant_invert}
\end{align}
\begin{remark}
  To use equation \eqref{eq:control_affine_plant_invert} in a discrete
  controller, one can approximate $\dot{\mat{r}}$ with
  $\frac{\mat{r}_{k + 1} - \mat{r}_k}{T}$ where $T$ is the time period between
  the two \glspl{reference}.
\end{remark}
