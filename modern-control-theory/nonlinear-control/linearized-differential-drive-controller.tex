\section{Linearized differential drive controller}

First, we'll start with theorem \ref{thm:ramsete_decoupled_ref_tracker}.

\begin{equation*}
  \begin{array}{ccc}
    \mtx{x} =
    \begin{bmatrix}
      v_l \\
      v_r
    \end{bmatrix} &
    \mtx{y} =
    \begin{bmatrix}
      v_l \\
      v_r
    \end{bmatrix} &
    \mtx{u} =
    \begin{bmatrix}
      V_l \\
      V_r
    \end{bmatrix}
  \end{array}
\end{equation*}

\begin{equation}
  \begin{array}{ll}
    \mtx{A} =
    \begin{bmatrix}
      \left(\frac{1}{m} + \frac{r_b^2}{J}\right) C_1 &
        \left(\frac{1}{m} - \frac{r_b^2}{J}\right) C_3 \\
      \left(\frac{1}{m} - \frac{r_b^2}{J}\right) C_1 &
        \left(\frac{1}{m} + \frac{r_b^2}{J}\right) C_3
    \end{bmatrix} &
    \mtx{B} =
    \begin{bmatrix}
      \left(\frac{1}{m} + \frac{r_b^2}{J}\right) C_2 &
        \left(\frac{1}{m} - \frac{r_b^2}{J}\right) C_4 \\
      \left(\frac{1}{m} - \frac{r_b^2}{J}\right) C_2 &
        \left(\frac{1}{m} + \frac{r_b^2}{J}\right) C_4
    \end{bmatrix} \\
    \mtx{C} =
    \begin{bmatrix}
      1 & 0 \\
      0 & 1 \\
    \end{bmatrix} &
    \mtx{D} = \mtx{0}_{2 \times 2}
  \end{array}
\end{equation}

where $C_1 = -\frac{G_l^2 K_t}{K_v R r^2}$, $C_2 = \frac{G_l K_t}{Rr}$,
$C_3 = -\frac{G_r^2 K_t}{K_v R r^2}$, and $C_4 = \frac{G_r K_t}{Rr}$.

The change in global pose is defined by these three equations.

\begin{align*}
  \dot{x} &= \frac{v_r + v_l}{2}\cos\theta = \frac{v_r}{2}\cos\theta +
    \frac{v_l}{2}\cos\theta \\
  \dot{y} &= \frac{v_r + v_l}{2}\sin\theta = \frac{v_r}{2}\sin\theta +
    \frac{v_l}{2}\sin\theta \\
  \dot{\theta} &= \frac{v_r - v_l}{2r_b} = \frac{v_r}{2r_b} - \frac{v_l}{2r_b}
\end{align*}

Next, we'll augment the reference tracker's state with the global pose $x$, $y$,
and $\theta$. Here's the model as a vector function where
$\mtx{x} = \begin{bmatrix} x & y & \theta & v_l & v_r \end{bmatrix}^T$ and
$\mtx{u} = \begin{bmatrix} V_l & V_r \end{bmatrix}^T$.

\begin{equation}
  f(\mtx{x}, \mtx{u}) =
  \begin{bmatrix}
    \frac{v_r}{2}\cos\theta + \frac{v_l}{2}\cos\theta \\
    \frac{v_r}{2}\sin\theta + \frac{v_l}{2}\sin\theta \\
    \frac{v_r}{2r_b} - \frac{v_l}{2r_b} \\
    \left(\frac{1}{m} + \frac{r_b^2}{J}\right) C_1 v_l +
      \left(\frac{1}{m} - \frac{r_b^2}{J}\right) C_3 v_r +
      \left(\frac{1}{m} + \frac{r_b^2}{J}\right) C_2 V_l +
      \left(\frac{1}{m} - \frac{r_b^2}{J}\right) C_4 V_r \\
    \left(\frac{1}{m} - \frac{r_b^2}{J}\right) C_1 v_l +
      \left(\frac{1}{m} + \frac{r_b^2}{J}\right) C_3 v_r +
      \left(\frac{1}{m} - \frac{r_b^2}{J}\right) C_2 V_l +
      \left(\frac{1}{m} + \frac{r_b^2}{J}\right) C_4 V_r
  \end{bmatrix}
  \label{eq:linearized_diff_drive_f}
\end{equation}

To create an LQR, we need to linearize this.

\begin{align*}
  \frac{\partial f(\mtx{x}, \mtx{u})}{\partial\mtx{x}} &=
  \begin{bmatrix}
    0 & 0 & -\frac{v_r + v_l}{2}\sin\theta & \frac{1}{2}\cos\theta &
      \frac{1}{2}\cos\theta \\
    0 & 0 & \frac{v_r + v_l}{2}\cos\theta & \frac{1}{2}\sin\theta &
      \frac{1}{2}\sin\theta \\
    0 & 0 & 0 & -\frac{1}{2r_b} & \frac{1}{2r_b} \\
    0 & 0 & 0 & \left(\frac{1}{m} + \frac{r_b^2}{J}\right) C_1 &
      \left(\frac{1}{m} - \frac{r_b^2}{J}\right) C_3 \\
    0 & 0 & 0 & \left(\frac{1}{m} - \frac{r_b^2}{J}\right) C_1 &
      \left(\frac{1}{m} + \frac{r_b^2}{J}\right) C_3
  \end{bmatrix} \\
  \frac{\partial f(\mtx{x}, \mtx{u})}{\partial\mtx{u}} &=
  \begin{bmatrix}
    0 & 0 \\
    0 & 0 \\
    0 & 0 \\
    \left(\frac{1}{m} + \frac{r_b^2}{J}\right) C_2 &
    \left(\frac{1}{m} - \frac{r_b^2}{J}\right) C_4 \\
    \left(\frac{1}{m} - \frac{r_b^2}{J}\right) C_2 &
    \left(\frac{1}{m} + \frac{r_b^2}{J}\right) C_4
  \end{bmatrix}
\end{align*}

Therefore,

\begin{theorem}[Linearized differential drive state-space model]
  \label{thm:linearized_diff_drive_model}
  \begin{equation*}
    \begin{array}{ccc}
      \mtx{x} =
      \begin{bmatrix}
        x \\
        y \\
        \theta \\
        v_l \\
        v_r
      \end{bmatrix} &
      \mtx{y} =
      \begin{bmatrix}
        \theta \\
        v_l \\
        v_r
      \end{bmatrix} &
      \mtx{u} =
      \begin{bmatrix}
        V_l \\
        V_r
      \end{bmatrix}
    \end{array}
  \end{equation*}

  \begin{equation}
    \begin{array}{ll}
      \mtx{A} =
      \begin{bmatrix}
        0 & 0 & -\frac{v_r + v_l}{2}\sin\theta & \frac{1}{2}\cos\theta &
          \frac{1}{2}\cos\theta \\
        0 & 0 & \frac{v_r + v_l}{2}\cos\theta & \frac{1}{2}\sin\theta &
          \frac{1}{2}\sin\theta \\
        0 & 0 & 0 & -\frac{1}{2r_b} & \frac{1}{2r_b} \\
        0 & 0 & 0 & \left(\frac{1}{m} + \frac{r_b^2}{J}\right) C_1 &
          \left(\frac{1}{m} - \frac{r_b^2}{J}\right) C_3 \\
        0 & 0 & 0 & \left(\frac{1}{m} - \frac{r_b^2}{J}\right) C_1 &
          \left(\frac{1}{m} + \frac{r_b^2}{J}\right) C_3
      \end{bmatrix} \\
      \mtx{B} =
      \begin{bmatrix}
        0 & 0 \\
        0 & 0 \\
        0 & 0 \\
        \left(\frac{1}{m} + \frac{r_b^2}{J}\right) C_2 &
        \left(\frac{1}{m} - \frac{r_b^2}{J}\right) C_4 \\
        \left(\frac{1}{m} - \frac{r_b^2}{J}\right) C_2 &
        \left(\frac{1}{m} + \frac{r_b^2}{J}\right) C_4
      \end{bmatrix} \\
      \mtx{C} =
      \begin{bmatrix}
        0 & 0 & 1 & 0 & 0 \\
        0 & 0 & 0 & 1 & 0 \\
        0 & 0 & 0 & 0 & 1
      \end{bmatrix} &
      \mtx{D} = \mtx{0}_{3 \times 2}
    \end{array}
  \end{equation}

  where $C_1 = -\frac{G_l^2 K_t}{K_v R r^2}$, $C_2 = \frac{G_l K_t}{Rr}$,
  $C_3 = -\frac{G_r^2 K_t}{K_v R r^2}$, and $C_4 = \frac{G_r K_t}{Rr}$.

  The LQR gain should be recomputed around the current operating point regularly
  due to the high nonlinearity of this system. A less computationally expensive
  controller will be developed in later sections.
\end{theorem}

We can also use this in an extended Kalman filter as is since the measurement
model ($\mtx{y} = \mtx{C}\mtx{x} + \mtx{D}\mtx{u}$) is linear.

With this \gls{controller}, $\theta$ becomes uncontrollable at $v = 0$ due to
the $x$ and $y$ dynamics being equivalent to a unicycle; it can't change its
heading unless it's rolling (just like a bicycle). However, a differential
drive \textit{can} rotate in place. To address this controller's limitation at
$v = 0$, one can temporarily switch to an LQR of just $\theta$, $v_l$, and
$v_r$; linearize the controller around a slightly nonzero state; or plan a new
trajectory after the previous one completes that provides nonzero wheel
velocities to rotate the robot.

\subsection{Improving model accuracy}

Figures \ref{fig:linearized_diff_drive_nonrotated_firstorder_xy} and
\ref{fig:linearized_diff_drive_nonrotated_firstorder_response} demonstrate the
tracking behavior of the linearized differential drive controller.

\begin{bookfigure}
  \begin{minisvg}{2}{build/code/linearized_diff_drive_nonrotated_firstorder_xy}
    \caption{Linearized differential drive controller x-y plot (first order)}
    \label{fig:linearized_diff_drive_nonrotated_firstorder_xy}
  \end{minisvg}
  \hfill
  \begin{minisvg}{2}{build/code/linearized_diff_drive_nonrotated_firstorder_response}
    \caption{Linearized differential drive controller response (first order)}
    \label{fig:linearized_diff_drive_nonrotated_firstorder_response}
  \end{minisvg}
\end{bookfigure}

The linearized differential drive model doesn't track well because the
first-order linearization of $\mtx{A}$ doesn't capture the full heading
dynamics, making the \gls{model} update inaccurate. This linearization
inaccuracy is evident in the Hessian matrix (second partial derivative with
respect to the state vector) being nonzero.

\begin{equation*}
  \frac{\partial^2 f(\mtx{x}, \mtx{u})}{\partial\mtx{x}^2} =
  \begin{bmatrix}
    0 & 0 & -\frac{v_r + v_l}{2}\cos\theta & 0 & 0 \\
    0 & 0 & -\frac{v_r + v_l}{2}\sin\theta & 0 & 0 \\
    0 & 0 & 0 & 0 & 0 \\
    0 & 0 & 0 & 0 & 0 \\
    0 & 0 & 0 & 0 & 0
  \end{bmatrix}
\end{equation*}

The second-order Taylor series expansion of the \gls{model} around $\mtx{x}_0$
would be

\begin{equation*}
  f(\mtx{x}, \mtx{u}_0) \approx f(\mtx{x}_0, \mtx{u}_0) +
    \frac{\partial f(\mtx{x}, \mtx{u})}{\partial\mtx{x}}(\mtx{x} - \mtx{x}_0) +
    \frac{1}{2}\frac{\partial^2 f(\mtx{x}, \mtx{u})}{\partial\mtx{x}^2}
    (\mtx{x} - \mtx{x}_0)^2
\end{equation*}

To include higher-order dynamics in the linearized differential drive model
integration, we recommend using the fourth-order Runge-Kutta (RK4) integration
method on equation (\ref{eq:linearized_diff_drive_f}). See snippet
\ref{lst:runge-kutta} for an implementation of RK4.

\begin{code}{Python}{build/frccontrol/frccontrol/runge_kutta.py}
  \caption{Fourth-order Runge-Kutta integration in Python}
  \label{lst:runge-kutta}
\end{code}

Figures \ref{fig:linearized_diff_drive_nonrotated_xy} and
\ref{fig:linearized_diff_drive_nonrotated_response} show a simulation using RK4
instead of the first-order \gls{model}.

\begin{bookfigure}
  \begin{minisvg}{2}{build/code/linearized_diff_drive_nonrotated_xy}
    \caption{Linearized differential drive controller (global reference frame
      formulation) x-y plot}
    \label{fig:linearized_diff_drive_nonrotated_xy}
  \end{minisvg}
  \hfill
  \begin{minisvg}{2}{build/code/linearized_diff_drive_nonrotated_response}
    \caption{Linearized differential drive controller (global reference frame
      formulation) response}
    \label{fig:linearized_diff_drive_nonrotated_response}
  \end{minisvg}
\end{bookfigure}

\subsection{Cross track error controller}

\begin{bookfigure}
  \begin{minisvg}{2}{build/code/ramsete_traj_xy}
    \caption{Ramsete nonlinear controller x-y plot}
    \label{fig:ramsete_traj_xy2}
  \end{minisvg}
  \hfill
  \begin{minisvg}{2}{build/code/ramsete_traj_response}
    \caption{Ramsete nonlinear controller response}
    \label{fig:ramsete_traj_response2}
  \end{minisvg}
\end{bookfigure}

The tracking performance of the linearized differential drive controller
(figures \ref{fig:linearized_diff_drive_nonrotated_xy} and
\ref{fig:linearized_diff_drive_nonrotated_response}) and Ramsete (figures
\ref{fig:ramsete_traj_xy2} and \ref{fig:ramsete_traj_response2}) for a given
trajectory are similar, but the former's performance-effort trade-off can be
tuned more intuitively via the Q and R gains. However, if the $x$ and $y$ error
cost are too high, the $x$ and $y$ components of the controller will fight each
other, and it will take longer to converge to the path. This can be fixed by
applying a counterclockwise rotation matrix to the global tracking error to
transform it into the robot's coordinate frame.

\begin{equation*}
  \crdfrm{R}{\begin{bmatrix}
    e_x \\
    e_y \\
    e_\theta
  \end{bmatrix}} =
  \begin{bmatrix}
    \cos\theta & \sin\theta & 0 \\
    -\sin\theta & \cos\theta & 0 \\
    0 & 0 & 1
  \end{bmatrix}
  \crdfrm{G}{\begin{bmatrix}
    e_x \\
    e_y \\
    e_\theta
  \end{bmatrix}}
\end{equation*}

where the the superscript $R$ represents the robot's coordinate frame and the
superscript $G$ represents the global coordinate frame.

With this transformation, the $x$ and $y$ error cost in LQR penalize the error
ahead of the robot and cross-track error respectively instead of global pose
error. Since the cross-track error is always measured from the robot's
coordinate frame, the \gls{model} used to compute the LQR should be linearized
around $\theta = 0$ at all times.

\begin{align*}
  \mtx{A} &=
  \begin{bmatrix}
    0 & 0 & -\frac{v_r + v_l}{2}\sin 0 & \frac{1}{2}\cos 0 &
      \frac{1}{2}\cos 0 \\
    0 & 0 & \frac{v_r + v_l}{2}\cos 0 & \frac{1}{2}\sin 0 &
      \frac{1}{2}\sin 0 \\
    0 & 0 & 0 & -\frac{1}{2r_b} & \frac{1}{2r_b} \\
    0 & 0 & 0 & \left(\frac{1}{m} + \frac{r_b^2}{J}\right) C_1 &
      \left(\frac{1}{m} - \frac{r_b^2}{J}\right) C_3 \\
    0 & 0 & 0 & \left(\frac{1}{m} - \frac{r_b^2}{J}\right) C_1 &
      \left(\frac{1}{m} + \frac{r_b^2}{J}\right) C_3
  \end{bmatrix} \\
  \mtx{A} &=
  \begin{bmatrix}
    0 & 0 & 0 & \frac{1}{2} & \frac{1}{2} \\
    0 & 0 & \frac{v_r + v_l}{2} & 0 & 0 \\
    0 & 0 & 0 & -\frac{1}{2r_b} & \frac{1}{2r_b} \\
    0 & 0 & 0 & \left(\frac{1}{m} + \frac{r_b^2}{J}\right) C_1 &
      \left(\frac{1}{m} - \frac{r_b^2}{J}\right) C_3 \\
    0 & 0 & 0 & \left(\frac{1}{m} - \frac{r_b^2}{J}\right) C_1 &
      \left(\frac{1}{m} + \frac{r_b^2}{J}\right) C_3
  \end{bmatrix}
\end{align*}

This \gls{model} results in figures \ref{fig:linearized_diff_drive_exact_xy} and
\ref{fig:linearized_diff_drive_exact_response}, which show slightly better
tracking performance than the previous formulation.

\begin{bookfigure}
  \begin{minisvg}{2}{build/code/linearized_diff_drive_exact_xy}
    \caption{Linearized differential drive controller x-y plot}
    \label{fig:linearized_diff_drive_exact_xy}
  \end{minisvg}
  \hfill
  \begin{minisvg}{2}{build/code/linearized_diff_drive_exact_response}
    \caption{Linearized differential drive controller response}
    \label{fig:linearized_diff_drive_exact_response}
  \end{minisvg}
\end{bookfigure}

\subsection{Explicit time-varying control law}

Another downside of this controller over Ramsete is that the user must generate
controller gains for every state they visit in the state-space (an implicit
control law) whereas Ramsete has two closed-form functions for its explicit
control law.

A possible solution to this is fitting a vector function of the states to the
linearized differential drive controller gains. Fortunately, the formulation of
the controller dealing with cross-track error only requires linearization around
different values of $v$ rather than $v$ and $\theta$; only a two-dimensional
function is needed rather than a three-dimensional one. See figures
\ref{fig:linearized_diff_drive_lqr_fit_0} through
\ref{fig:linearized_diff_drive_lqr_fit_4} for plots of each controller gain for
a range of velocities.

\begin{bookfigure}
  \begin{minisvg}{2}{build/code/linearized_diff_drive_lqr_fit_0}
    \caption{Linearized differential drive controller LQR gain regression fit
      ($x$)}
    \label{fig:linearized_diff_drive_lqr_fit_0}
  \end{minisvg}
  \hfill
  \begin{minisvg}{2}{build/code/linearized_diff_drive_lqr_fit_1}
    \caption{Linearized differential drive controller LQR gain fit regression
      fit ($y$)}
    \label{fig:linearized_diff_drive_lqr_fit_1}
  \end{minisvg}
  \hfill
  \begin{minisvg}{2}{build/code/linearized_diff_drive_lqr_fit_2}
    \caption{Linearized differential drive controller LQR gain regression fit
      ($\theta$)}
    \label{fig:linearized_diff_drive_lqr_fit_2}
  \end{minisvg}
  \hfill
  \begin{minisvg}{2}{build/code/linearized_diff_drive_lqr_fit_3}
    \caption{Linearized differential drive controller LQR gain regression fit
      ($v_l$)}
    \label{fig:linearized_diff_drive_lqr_fit_3}
  \end{minisvg}
  \hfill
  \begin{minisvg}{2}{build/code/linearized_diff_drive_lqr_fit_4}
    \caption{Linearized differential drive controller LQR gain regression fit
      ($v_r$)}
    \label{fig:linearized_diff_drive_lqr_fit_4}
  \end{minisvg}
\end{bookfigure}

With the exception of the $x$ gain plot, all functions are a variation of a
square root. $v = 0$ and $v = 1$ were used to scale each function to produce a
close approximation.

\begin{theorem}[Linearized differential drive controller]
  The following $\mtx{A}$ and $\mtx{B}$ matrices are used to compute the LQR.

  \begin{equation}
    \begin{array}{ll}
      \mtx{A} =
      \begin{bmatrix}
        0 & 0 & 0 & \frac{1}{2} & \frac{1}{2} \\
        0 & 0 & v & 0 & 0 \\
        0 & 0 & 0 & -\frac{1}{2r_b} & \frac{1}{2r_b} \\
        0 & 0 & 0 & \left(\frac{1}{m} + \frac{r_b^2}{J}\right) C_1 &
          \left(\frac{1}{m} - \frac{r_b^2}{J}\right) C_3 \\
        0 & 0 & 0 & \left(\frac{1}{m} - \frac{r_b^2}{J}\right) C_1 &
          \left(\frac{1}{m} + \frac{r_b^2}{J}\right) C_3
      \end{bmatrix} &
      \mtx{x} =
      \begin{bmatrix}
        x \\
        y \\
        \theta \\
        v_l \\
        v_r
      \end{bmatrix} \\
      \mtx{B} =
      \begin{bmatrix}
        0 & 0 \\
        0 & 0 \\
        0 & 0 \\
        \left(\frac{1}{m} + \frac{r_b^2}{J}\right) C_2 &
        \left(\frac{1}{m} - \frac{r_b^2}{J}\right) C_4 \\
        \left(\frac{1}{m} - \frac{r_b^2}{J}\right) C_2 &
        \left(\frac{1}{m} + \frac{r_b^2}{J}\right) C_4
      \end{bmatrix} &
      \mtx{u} =
      \begin{bmatrix}
        V_l \\
        V_r
      \end{bmatrix}
    \end{array}
  \end{equation}

  where $v = \frac{v_l + v_r}{2}$, $C_1 = -\frac{G_l^2 K_t}{K_v R r^2}$,
  $C_2 = \frac{G_l K_t}{Rr}$, $C_3 = -\frac{G_r^2 K_t}{K_v R r^2}$, and
  $C_4 = \frac{G_r K_t}{Rr}$.

  The Pareto optimal controller for this model is
  $\mtx{u} = \mtx{K}(v) (\mtx{r} - \mtx{x})$ where

  \begin{align}
    \mtx{K}(v) &= \begin{bmatrix}
      k_x & k_{1, 2}(v) \sign(v) & k_{\theta, 1} \sqrt{|v|} & k_{1, 4}(v) &
        k_{2, 4}(v) \\
      k_x & -k_{1, 2}(v) \sign(v) & -k_{\theta, 1} \sqrt{|v|} & k_{2, 4}(v) &
        k_{1, 4}(v)
    \end{bmatrix} \\
    k_{1, 2}(v) &= k_{y, 0} + (k_{y, 1} - k_{y, 0}) \sqrt{|v|} \\
    k_{1, 4}(v) &= k_{v^+, 0} + (k_{v^+, 1} - k_{v^+, 0}) \sqrt{|v|} \\
    k_{2, 4}(v) &= k_{v^-, 0} - (k_{v^+, 1} - k_{v^+, 0}) \sqrt{|v|} \\
    \sign(v) &= \begin{cases}
      -1 & v < 0 \\
      0 & v = 0 \\
      1 & v > 0
    \end{cases}
  \end{align}

  Using $\mtx{K}$ computed via LQR at $v = 0$
  \begin{figurekey}
    \begin{tabular}{llll}
      $k_x = \mtx{K}_{1, 1}$ &
        $k_{y, 0} = \mtx{K}_{1, 2}$ &
        $k_{v^+, 0} = \mtx{K}_{1, 4}$ &
        $k_{v^-, 0} = \mtx{K}_{2, 4}$
    \end{tabular}
  \end{figurekey}

  Using $\mtx{K}$ computed via LQR at $v = 1$
  \begin{figurekey}
    \begin{tabular}{lll}
        $k_{y, 1} = \mtx{K}_{1, 2}$ &
        $k_{\theta, 1} = \mtx{K}_{1, 3}$ &
        $k_{v^+, 1} = \mtx{K}_{1, 4}$
    \end{tabular}
  \end{figurekey}
\end{theorem}

Figures \ref{fig:linearized_diff_drive_approx_xy} through
\ref{fig:linearized_diff_drive_exact_response2} show the responses of the exact
and approximate solutions are the same.

\begin{bookfigure}
  \begin{minisvg}{2}{build/code/linearized_diff_drive_approx_xy}
    \caption{Linearized differential drive controller x-y plot (approximate)}
    \label{fig:linearized_diff_drive_approx_xy}
  \end{minisvg}
  \hfill
  \begin{minisvg}{2}{build/code/linearized_diff_drive_exact_xy}
    \caption{Linearized differential drive controller x-y plot (exact)}
    \label{fig:linearized_diff_drive_exact_xy2}
  \end{minisvg}
  \hfill
  \begin{minisvg}{2}{build/code/linearized_diff_drive_approx_response}
    \caption{Linearized differential drive controller response (approximate)}
    \label{fig:linearized_diff_drive_approx_response}
  \end{minisvg}
  \hfill
  \begin{minisvg}{2}{build/code/linearized_diff_drive_exact_response}
    \caption{Linearized differential drive controller response (exact)}
    \label{fig:linearized_diff_drive_exact_response2}
  \end{minisvg}
\end{bookfigure}
