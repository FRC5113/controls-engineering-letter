\section{Ramsete unicycle controller}

Ramsete is a nonlinear time-varying feedback controller for unicycle
\glspl{model} that drives the \gls{model} to a desired \gls{pose} along a
two-dimensional trajectory. Why would we need a nonlinear control law in
addition to the linear ones we have used so far? If we use the original approach
with an LQR \gls{controller} for left and right position and velocity
\glspl{state}, the \gls{controller} only deals with the local \gls{pose}. If the
robot deviates from the path, there is no way for the \gls{controller} to
correct and the robot may not reach the desired global \gls{pose}. This is due
to multiple endpoints existing for the robot which have the same encoder path
arc lengths.

Instead of using wheel path arc lengths (which are in the robot's local
coordinate frame), nonlinear controllers like pure pursuit and Ramsete use
global pose. The \gls{controller} uses this extra information to guide a linear
\glslink{tracking}{reference tracker} like an LQR \gls{controller} back in by
adjusting the \glspl{reference} of the LQR \gls{controller}.

The paper \textit{Control of Wheeled Mobile Robots: An Experimental Overview}
describes a nonlinear controller for a wheeled vehicle with unicycle-like
kinematics; a global \gls{pose} consisting of $x$, $y$, and $\theta$; and a
desired \gls{pose} consisting of $x_d$, $y_d$, and $\theta_d$
\cite{bib:ctrl_wheeled_mobile_robots}. We'll call it Ramsete because that's the
acronym for the title of the book it came from in Italian (``Robotica Articolata
e Mobile per i SErvizi e le TEcnologie").

\subsection{Velocity and turning rate command derivation}

The \gls{state} tracking \gls{error} $\mat{e}$ in the vehicle's coordinate frame
is defined as
\begin{equation*}
  \begin{bmatrix}
    e_x \\
    e_y \\
    e_\theta
  \end{bmatrix} =
  \begin{bmatrix}
    \cos\theta & \sin\theta & 0 \\
    -\sin\theta & \cos\theta & 0 \\
    0 & 0 & 1
  \end{bmatrix}
  \begin{bmatrix}
    x_d - x \\
    y_d - y \\
    \theta_d - \theta
  \end{bmatrix}
\end{equation*}

where $e_x$ is the tracking \gls{error} in $x$, $e_y$ is the tracking
\gls{error} in $y$, and $e_\theta$ is the tracking \gls{error} in $\theta$.
The $3 \times 3$ matrix is a rotation matrix that transforms the \gls{error} in
the \gls{pose} (represented by $x_d - x$, $y_d - y$, and $\theta_d - \theta$)
from the global coordinate frame into the vehicle's coordinate frame.

We will use the following control laws $u_1$ and $u_2$ for velocity and turning
rate respectively.
\begin{equation}
  \begin{aligned}
    u_1 &= -k_1 e_x \\
    u_2 &= -k_3 e_\theta - k_2 v_d \sinc(e_\theta) e_y
  \end{aligned}
  \label{eq:ramsete_u}
\end{equation}

where $k_1$, $k_2$, and $k_3$ are time-varying gains and $\sinc(e_\theta)$ is
defined as $\frac{\sin{e_\theta}}{e_\theta}$. This choice of control law may
seem arbitrary, and that's because it is. The only requirement on our choice is
that there exist an associated Lyapunov candidate function that proves the
control law is globally asymptotically stable. We'll provide a sketch of a proof
in theorem \ref{thm:ramsete_lyapunov_stability}.

Our velocity and turning rate commands for the vehicle will use the following
nonlinear transformation of the inputs.
\begin{align*}
  v &= v_d\cos e_\theta - u_1 \\
  \omega &= \omega_d - u_2
  \intertext{Substituting the control laws $u_1$ and $u_2$ into these equations
    gives}
  v &= v_d\cos{e_\theta} + k_1 e_x \\
  \omega &= k_3 e_\theta + \omega_d + k_2 v_d \sinc(e_\theta) e_y
\end{align*}
\begin{theorem}
  \label{thm:ramsete_lyapunov_stability}

  Assuming that $v_d$ and $\omega_d$ are bounded with bounded derivatives, and
  that $v_d(t) \rightarrow 0$ or $\omega_d(t) \rightarrow 0$ when
  $t \rightarrow \infty$, the control laws in equation \eqref{eq:ramsete_u}
  globally asymptotically stabilize the origin $\mat{e} = 0$.

  Proof:

  To prove convergence, the paper previously mentioned uses the following
  Lyapunov function.
  \begin{equation*}
    V = \frac{k_2}{2}(e_x^2 + e_y^2) + \frac{e_\theta^2}{2}
  \end{equation*}

  where $k_2$ is a tuning constant, $e_x$ is the tracking error in $x$, $e_y$ is
  the tracking error in $y$, and $e_\theta$ is the tracking error in $\theta$.

  The time derivative along the solutions of the closed-loop \gls{system} is
  nonincreasing since
  \begin{equation*}
    \dot{V} = -k_1 k_2 e_x^2 - k_3 e_\theta^2 \leq 0
  \end{equation*}

  Thus, $\lVert e(t) \rVert$ is bounded, $\dot{V}(t)$ is uniformly continuous,
  and $V(t)$ tends to some limit value. Using the Barbalat lemma, $\dot{V}(t)$
  tends to zero \cite{bib:ctrl_wheeled_mobile_robots}.
\end{theorem}

\subsection{Nonlinear controller equations}

Let $k_2 = b$ and
$k = k_1(v_d, \omega_d) = k_3(v_d, \omega_d) = 2\zeta\sqrt{\omega_d^2 + bv_d^2}$.
\begin{theorem}[Ramsete unicycle controller]
  \begin{align}
    \begin{bmatrix}
      e_x \\
      e_y \\
      e_\theta
    \end{bmatrix} &=
    \begin{bmatrix}
      \cos\theta & \sin\theta & 0 \\
      -\sin\theta & \cos\theta & 0 \\
      0 & 0 & 1
    \end{bmatrix}
    \begin{bmatrix}
      x_d - x \\
      y_d - y \\
      \theta_d - \theta
    \end{bmatrix} \\
    v &= v_d \cos{e_\theta} + k e_x \label{eq:ramsete_eq1} \\
    \omega &= \omega_d + k e_\theta + b v_d \sinc(e_\theta) e_y
      \label{eq:ramsete_eq2} \\
    k &= 2\zeta\sqrt{\omega_d^2 + bv_d^2} \label{eq:ramsete_eq3} \\
    \sinc(e_\theta) &= \frac{\sin{e_\theta}}{e_\theta}
  \end{align}
  \begin{figurekey}
    \begin{tabular}{llll}
      $v$ & velocity command & $v_d$ & desired velocity \\
      $\omega$ & turning rate command & $\omega_d$ & desired turning rate \\
      $x$ & actual $x$ position in global coordinate frame & $x_d$ &
        desired $x$ position \\
      $y$ & actual $y$ position in global coordinate frame & $y_d$ &
        desired $y$ position \\
      $\theta$ & actual angle in global coordinate frame & $\theta_d$ &
        desired angle
    \end{tabular}
  \end{figurekey}

  $b$ and $\zeta$ are tuning parameters where
  $b > 0~\frac{\text{rad}^2}{\text{m}^2}$ and
  $\zeta \in (0, 1)~\text{rad}^{-1}$. Larger values of $b$ make convergence more
  aggressive (like a proportional term), and larger values of $\zeta$ provide
  more damping.
\end{theorem}

$v$ and $\omega$ should be the \glspl{reference} for a drivetrain
\gls{reference} tracker. A good choice would be using equations
\eqref{eq:diff_vl} and \eqref{eq:diff_vr} to convert $v$ and $\omega$ to left
and right velocities, then applying LQR to the model in theorem
\ref{thm:diff_drive_velocity_state-space_model}.

$x$, $y$, and $\theta$ are obtained via a \gls{pose} estimator (see chapter
\ref{ch:pose_estimation} for how to implement one). The desired velocity,
turning rate, and \gls{pose} can be varied over time according to a desired
trajectory.

Figures \ref{fig:ramsete_traj_xy} and \ref{fig:ramsete_traj_response} show the
tracking performance of Ramsete for a given trajectory.
\begin{bookfigure}
  \begin{minisvg}{2}{build/\chapterpath/ramsete_traj_xy}
    \caption{Ramsete nonlinear controller x-y plot}
    \label{fig:ramsete_traj_xy}
  \end{minisvg}
  \hfill
  \begin{minisvg}{2}{build/\chapterpath/ramsete_traj_response}
    \caption{Ramsete nonlinear controller response}
    \label{fig:ramsete_traj_response}
  \end{minisvg}
\end{bookfigure}

\subsection{Dimensional analysis}

$[v]$ denotes the dimension of $v$. $A$ means angle units, $L$ means length
units, and $T$ means time units.

\subsubsection{Units of $\sinc(e_\theta)$}

First, we'll find the units of $\sinc(e_\theta)$ for later use.
\begin{align*}
  \sinc(e_\theta) &= \frac{\sin(e_\theta)}{e_\theta} \\
  [\sinc(e_\theta)] &= \frac{1}{A} \\
  [\sinc(e_\theta)] &= A^{-1}
\end{align*}

\subsubsection{Ramsete velocity command equation}

Start from equation \eqref{eq:ramsete_eq1}.
\begin{align*}
  v &= v_d \cos e_\theta + k e_x \\
  [v] &= [v_d] [\cos e_\theta] + [k] [e_x] \\
  L T^{-1} &= L T^{-1} \cdot 1 + [k] L \\
  L T^{-1} &= L T^{-1} + [k] L \\
  L T^{-1} &= [k] L \\
  T^{-1} &= [k] \\
  [k] &= T^{-1}
\end{align*}

Therefore, the units of $k$ are $T^{-1}$.

\subsubsection{Ramsete angular velocity command equation}

Start from equation \eqref{eq:ramsete_eq2}.
\begin{align}
  \omega &= \omega_d + ke_\theta + bv_d \sinc(e_\theta) e_y \nonumber \\
  [\omega] &= [\omega_d] + [k][e_\theta] + [b][v_d] [\sinc(e_\theta)] [e_y]
    \nonumber \\
  [\omega] &= [\omega_d] + [k][e_\theta] + [b][v_d] [\sinc(e_\theta)] [e_y]
    \nonumber \\
  A T^{-1} &= A T^{-1} + [k] A + [b] L T^{-1} \cdot A^{-1} \cdot L \nonumber \\
  A T^{-1} &= A T^{-1} + [k] A + [b] A^{-1} L^{2} T \nonumber \\
  A T^{-1} &= [k] A + [b] A^{-1} L^{2} T
    \label{eq:ramsete_eq2_intermediate_units}
\end{align}

First, we'll find the units of $k$.

The left-hand side and first term of equation
\eqref{eq:ramsete_eq2_intermediate_units} must have equivalent units.
\begin{align*}
  A T^{-1} &= [k] A \\
  T^{-1} &= [k] \\
  [k] &= T^{-1}
\end{align*}

This is consistent with the result from the Ramsete velocity command equation.

Next, we'll find the units of $b$.

The left-hand side and second term of equation
\eqref{eq:ramsete_eq2_intermediate_units} must have equivalent units.
\begin{align*}
  A T^{-1} &= [b] A^{-1} L^{2} T^{-1} \\
  A^{2} L^{-2} &= [b] \\
  [b] &= A^{2} L^{-2}
\end{align*}

\subsubsection{Ramsete $k$ equation}

Start from equation \eqref{eq:ramsete_eq3}.
\begin{align}
  k &= 2\zeta \sqrt{\omega_d^{2} + bv_d^{2}} \nonumber \\
  [k] &= [\zeta] \sqrt{[\omega_d]^{2} + [b][v_d]^{2}} \nonumber \\
  T^{-1} &= [\zeta] \sqrt{(A T^{-1})^{2} + [b] (L T^{-1})^{2}} \nonumber \\
  T^{-1} &= [\zeta] \sqrt{A^{2} T^{-2} + [b] L^{2} T^{-2}}
    \label{eq:ramsete_eq3_intermediate_units}
\end{align}

First, we'll find the units of $b$.

The additive terms under the square root must have equivalent units.
\begin{align*}
  (A T^{-1})^{2} &= [b] (L T^{-1})^{2} \\
  A^{2} T^{-2} &= [b] L^{2} T^{-2} \\
  A^{2} L^{-2} &= [b] \\
  [b] &= A^{2} L^{-2}
\end{align*}

This is consistent with the result from the angular velocity command equation,
so we can use it when determining the units of $\zeta$.

Next, we'll find the units of $\zeta$.

Substitute $[b]$ into equation \eqref{eq:ramsete_eq3_intermediate_units}.
\begin{align*}
  T^{-1} &= [\zeta] \sqrt{A^{2} T^{-2} + [b] L^{2} T^{-2}} \\
  T^{-1} &= [\zeta] \sqrt{A^{2} T^{-2} + A^{2} L^{-2} \cdot L^{2} T^{-2}} \\
  T^{-1} &= [\zeta] \sqrt{A^{2} T^{-2} + A^{2} T^{-2}} \\
  T^{-1} &= [\zeta] \sqrt{A^{2} T^{-2}} \\
  T^{-1} &= [\zeta] A T^{-1} \\
  A^{-1} &= [\zeta] \\
  [\zeta] &= A^{-1}
\end{align*}

\subsubsection{Units of tunable parameters $b$ and $\zeta$}
\begin{align}
  [b] &= A^{2} L^{-2} \\
  [\zeta] &= A^{-1}
\end{align}
