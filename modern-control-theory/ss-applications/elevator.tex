\section{Elevator}
\label{sec:ss_model_elevator}

This elevator consists of a DC brushed motor attached to a pulley that drives a
mass up or down.
\begin{bookfigure}
  \begin{tikzpicture}[auto, >=latex', circuit ee IEC,
                    set resistor graphic=var resistor IEC graphic]
  % \draw [help lines] (-1,-3) grid (7,4);

  % Electrical equivalent circuit
  \draw (0,2) to [voltage source={direction info'={->}, info'=$V$}] (0,0);
  \draw (0,2) to [current direction={info=$I$}] (0,3);
  \draw (0,3) -- (0.5,3);
  \draw (0.5,3) to [resistor={info={$R$}}] (2,3);

  \draw (2,3) -- (2.5,3);
  \draw (2.5,3) to [voltage source={direction info'={->}, info'=$V_{emf}$}]
    (2.5,0);
  \draw (0,0) -- (2.5,0);

  % Motor
  \begin{scope}[xshift=2.4cm,yshift=1.05cm]
    \draw[fill=black] (0,0) rectangle (0.2,0.9);
    \draw[fill=white] (0.1,0.45) ellipse (0.3 and 0.3);
  \end{scope}

  % Transmission gear one
  \begin{scope}[xshift=3.75cm,yshift=1.17cm]
    \draw[fill=black!50] (0.2,0.33) ellipse (0.08 and 0.33);
    \draw[fill=black!50, color=black!50] (0,0) rectangle (0.2,0.66);
    \draw[fill=white] (0,0.33) ellipse (0.08 and 0.33);
    \draw (0,0.66) -- (0.2,0.66);
    \draw (0,0) -- (0.2,0) node[pos=0.5,below] {$G$};
  \end{scope}

  % Output shaft of motor
  \begin{scope}[xshift=2.8cm,yshift=1.45cm]
    \draw[fill=black!50] (0,0) rectangle (0.95,0.1);
  \end{scope}

  % Angular velocity arrow of drive -> transmission
  \draw[line width=0.7pt,<-] (3.2,1) arc (-30:30:1) node[above] {$\omega_m$};

  % Transmission gear two
  \begin{scope}[xshift=3.75cm,yshift=1.83cm]
    \draw[fill=black!50] (0.2,0.68) ellipse (0.13 and 0.67);
    \draw[fill=black!50, color=black!50] (0,0) rectangle (0.2,1.35);
    \draw[fill=white] (0,0.68) ellipse (0.13 and 0.67);
    \draw (0,1.35) -- (0.2,1.35);
    \draw (0,0) -- (0.2,0);
  \end{scope}

  % Pulley rear chain
  \begin{scope}[xshift=5.03cm,yshift=0.32cm]
    \draw[fill=black!70, color=black!70] (0.01,2.17) rectangle (0.09,0);
    \draw (0,2.17) -- (0,0);
    \draw (0.1,2.17) -- (0.1,0);
  \end{scope}

  % Upper pulley
  \begin{scope}[xshift=5.05cm,yshift=2.09cm]
    \draw[fill=black!50] (0.2,0.4) ellipse (0.13 and 0.4);
    \draw[fill=black!70] (0.15,0.4) ellipse (0.13 and 0.4);
    \draw[fill=black!50, color=black!50] (0,0) rectangle (0.1,0.8);
    \draw[fill=black!70, color=black!70] (0.1,0) rectangle (0.15,0.8);
    \draw[fill=black!50] (0.05,0.4) ellipse (0.13 and 0.4);
    \draw[fill=black!50, color=black!50] (0,0) rectangle (0.05,0.8);
    \draw[fill=white] (0,0.4) ellipse (0.13 and 0.4);
    \draw (0,0) -- (0.2,0);
    \draw (0,0.8) -- (0.2,0.8);
  \end{scope}

  % Lower pulley
  \begin{scope}[xshift=5.05cm,yshift=-0.05cm]
    \draw[fill=black!50] (0.2,0.4) ellipse (0.13 and 0.4);
    \draw[fill=black!70] (0.15,0.4) ellipse (0.13 and 0.4);
    \draw[fill=black!50, color=black!50] (0,0) rectangle (0.1,0.8);
    \draw[fill=black!70, color=black!70] (0.1,0) rectangle (0.15,0.8);
    \draw[fill=black!50] (0.05,0.4) ellipse (0.13 and 0.4);
    \draw[fill=black!50, color=black!50] (0,0) rectangle (0.05,0.8);
    \draw[fill=white] (0,0.4) ellipse (0.13 and 0.4);
    \draw (0,0) -- (0.2,0);
    \draw (0,0.8) -- (0.2,0.8);
  \end{scope}

  % Transmission shaft from gear two to pulley
  \begin{scope}[xshift=4.09cm,yshift=2.42cm]
    \draw[fill=black!50] (0,0) rectangle (0.96,0.1);
  \end{scope}

  % Angular velocity arrow between transmission and pulley
  \draw[line width=0.7pt,->] (4.54,1.97) arc (-30:30:1) node[above]
    {$\omega_p$};

  % Pulley front chain
  \begin{scope}[xshift=5.23cm,yshift=0.32cm]
    \draw[fill=black!70, color=black!70] (0.01,2.17) rectangle (0.09,0);
    \draw (0,2.17) -- (0,0);
    \draw (0.1,2.17) -- (0.1,0);
  \end{scope}

  % Pulley radius arrow
  \begin{scope}[xshift=5.54cm,yshift=2.49]
    \draw[line width=0.7pt,<->] (0,-0.15) -- node[right] {$r$} (0,0.35);
  \end{scope}

  % Mass
  \begin{scope}[xshift=4.89cm,yshift=0.82cm]
    \fill[fill=white] (0,0.8) -- (0,0.2) -- (0.2,0) -- (0.2,0.2)
      -- (0.98,0.2) -- (0.78,0.8) -- cycle;
    \draw (0,0.8) -- (0.78,0.8);
    \draw (0,0.8) -- (0,0.2);
    \draw (0,0.2) -- (0.2,0);
    \draw (0,0.8) -- (0.2,0.6);
    \draw (0.78,0.8) -- (0.98,0.6);
    \draw[fill=white] (0.2,0.6) rectangle (0.98,0);
  \end{scope}

  % Mass velocity arrow
  \begin{scope}[xshift=6.04cm,yshift=0.95cm]
    \draw[line width=0.7pt,<-] (0,0.4) -- node {$v$} (0,0);
  \end{scope}

  % Descriptions inside graphic
  \draw (5.48,1.12) node {$m$};

  % Descriptions of subsystems under graphic
  \begin{scope}[xshift=-0.5cm,yshift=-0.28cm]
    \draw[decorate,decoration={brace,amplitude=10pt}]
      (3.5,0) -- (0,0) node[midway,yshift=-20pt] {circuit};
    \draw[decorate,decoration={brace,amplitude=10pt}]
      (7.05,0) -- (3.75,0) node[midway,yshift=-20pt] {mechanics};
  \end{scope}
\end{tikzpicture}

  \caption{Elevator system diagram}
\end{bookfigure}

\subsection{Continuous state-space model}
\index{FRC models!elevator equations}

The position and velocity derivatives of the elevator can be written as
\begin{align}
  \dot{x} &= v \\
  \dot{v} &= a \label{eq:elevator_cont_ss_vel}
\end{align}

where by equation \eqref{eq:elevator_accel},
\begin{equation*}
  a = \frac{GK_t}{Rrm} V - \frac{G^2 K_t}{Rr^2 m K_v} v
\end{equation*}

Substitute this into equation \eqref{eq:elevator_cont_ss_vel}.
\begin{align*}
  \dot{v} &= \frac{GK_t}{Rrm} V - \frac{G^2 K_t}{Rr^2 m K_v} v \\
  \dot{v} &= -\frac{G^2 K_t}{Rr^2 m K_v} v + \frac{GK_t}{Rrm} V
  \intertext{Factor out $v$ and $V$ into column vectors.}
  \dot{\begin{bmatrix}
    v
  \end{bmatrix}} &=
  \begin{bmatrix}
    -\frac{G^2 K_t}{Rr^2 m K_v}
  \end{bmatrix}
  \begin{bmatrix}
    v
  \end{bmatrix} +
  \begin{bmatrix}
    \frac{GK_t}{Rrm}
  \end{bmatrix}
  \begin{bmatrix}
    V
  \end{bmatrix}
  \intertext{Augment the matrix equation with the position state $x$, which has
    the model equation $\dot{x} = v$. The matrix elements corresponding to $v$
    will be $1$, and the others will be $0$ since they don't appear, so
    $\dot{x} = 0x + 1v + 0V$. The existing rows will have zeroes inserted where
    $x$ is multiplied in.}
  \dot{\begin{bmatrix}
    x \\
    v
  \end{bmatrix}} &=
  \begin{bmatrix}
    0 & 1 \\
    0 & -\frac{G^2 K_t}{Rr^2 m K_v}
  \end{bmatrix}
  \begin{bmatrix}
    x \\
    v
  \end{bmatrix} +
  \begin{bmatrix}
    0 \\
    \frac{GK_t}{Rrm}
  \end{bmatrix}
  \begin{bmatrix}
    V
  \end{bmatrix}
\end{align*}
\begin{theorem}[Elevator state-space model]
  \begin{align*}
    \dot{\mat{x}} &= \mat{A} \mat{x} + \mat{B} \mat{u} \\
    \mat{y} &= \mat{C} \mat{x} + \mat{D} \mat{u}
  \end{align*}
  \begin{equation*}
    \mat{x} =
    \begin{bmatrix}
      x \\
      v
    \end{bmatrix}
    \quad
    \mat{y} = x
    \quad
    \mat{u} = V
  \end{equation*}
  \begin{equation}
    \mat{A} =
    \begin{bmatrix}
      0 & 1 \\
      0 & -\frac{G^2 K_t}{Rr^2 mK_v}
    \end{bmatrix}
    \quad
    \mat{B} =
    \begin{bmatrix}
      0 \\
      \frac{GK_t}{Rrm}
    \end{bmatrix}
    \quad
    \mat{C} =
    \begin{bmatrix}
      1 & 0
    \end{bmatrix}
    \quad
    \mat{D} = 0
  \end{equation}
\end{theorem}

\subsection{Model augmentation}

As per subsection \ref{subsec:input_error_estimation}, we will now augment the
\gls{model} so a $u_{error}$ state is added to the \gls{control input}.

The \gls{plant} and \gls{observer} augmentations should be performed before the
\gls{model} is \glslink{discretization}{discretized}. After the \gls{controller}
gain is computed with the unaugmented discrete \gls{model}, the controller may
be augmented. Therefore, the \gls{plant} and \gls{observer} augmentations assume
a continuous \gls{model} and the \gls{controller} augmentation assumes a
discrete \gls{controller}.
\begin{equation*}
  \mat{x}_{aug} =
  \begin{bmatrix}
    x \\
    v \\
    u_{error}
  \end{bmatrix}
  \quad
  \mat{y} = x
  \quad
  \mat{u} = V
\end{equation*}
\begin{equation}
  \mat{A}_{aug} =
  \begin{bmatrix}
    \mat{A} & \mat{B} \\
    \mat{0}_{1 \times 2} & 0
  \end{bmatrix}
  \quad
  \mat{B}_{aug} =
  \begin{bmatrix}
    \mat{B} \\
    0
  \end{bmatrix}
  \quad
  \mat{C}_{aug} = \begin{bmatrix}
    \mat{C} & 0
  \end{bmatrix}
  \quad
  \mat{D}_{aug} = \mat{D}
\end{equation}
\begin{equation}
  \mat{K}_{aug} = \begin{bmatrix}
    \mat{K} & 1
  \end{bmatrix}
  \quad
  \mat{r}_{aug} = \begin{bmatrix}
    \mat{r} \\
    0
  \end{bmatrix}
\end{equation}

This will compensate for unmodeled dynamics such as gravity. However, using a
constant voltage feedforward to counteract gravity is preferred over $u_{error}$
estimation in this case because it results in a simpler controller with similar
performance.

\subsection{Gravity feedforward}

Input voltage is proportional to force and gravity is a constant force, so a
constant voltage feedforward can compensate for gravity. We'll model gravity as
a disturbance described by $-mg$. To compensate for it, we want to find a
voltage that is equal and opposite to it. The bottom row of the continuous
elevator model contains the acceleration terms.
\begin{align*}
  Bu_{ff} &= -(\text{unmodeled dynamics})
  \intertext{where $B$ is the motor acceleration term from $\mat{B}$ and
    $u_{ff}$ is the voltage feedforward.}
  Bu_{ff} &= -(-mg) \\
  Bu_{ff} &= mg \\
  \frac{G K_t}{Rrm} u_{ff} &= mg \\
  u_{ff} &= \frac{Rrm^2 g}{G K_t}
\end{align*}

\subsection{Simulation}

Python Control will be used to \glslink{discretization}{discretize} the
\gls{model} and simulate it. One of the frccontrol
examples\footnote{\url{https://github.com/calcmogul/frccontrol/blob/main/examples/elevator.py}}
creates and tests a controller for it. Figure \ref{fig:elevator_response} shows
the closed-loop \gls{system} response.
\begin{svg}{build/frccontrol/examples/elevator_response}
  \caption{Elevator response}
  \label{fig:elevator_response}
\end{svg}

\subsection{Implementation}

C++ and Java implementations of this elevator controller are available
online\footnote{\url{https://github.com/wpilibsuite/allwpilib/blob/main/wpilibcExamples/src/main/cpp/examples/StateSpaceElevator/cpp/Robot.cpp}}
\footnote{\url{https://github.com/wpilibsuite/allwpilib/blob/main/wpilibjExamples/src/main/java/edu/wpi/first/wpilibj/examples/statespaceelevator/Robot.java}}.
