\section{Flywheel}
\label{sec:ss_model_flywheel}

This flywheel consists of a DC brushed motor attached to a spinning mass of
non-negligible moment of inertia.
\begin{bookfigure}
  \begin{tikzpicture}[auto, >=latex', circuit ee IEC,
                    set resistor graphic=var resistor IEC graphic]
  % \draw [help lines] (-1,-3) grid (7,4);

  % Electrical equivalent circuit
  \draw (0,2) to [voltage source={direction info'={->}, info'=$V$}] (0,0);
  \draw (0,2) to [current direction={info=$I$}] (0,3);
  \draw (0,3) -- (0.5,3);
  \draw (0.5,3) to [resistor={info={$R$}}] (2,3);

  \draw (2,3) -- (2.5,3);
  \draw (2.5,3) to [voltage source={direction info'={->}, info'=$V_{emf}$}]
    (2.5,0);
  \draw (0,0) -- (2.5,0);

  % Motor
  \begin{scope}[xshift=2.4cm,yshift=1.05cm]
    \draw[fill=black] (0,0) rectangle (0.2,0.9);
    \draw[fill=white] (0.1,0.45) ellipse (0.3 and 0.3);
  \end{scope}

  % Transmission gear one
  \begin{scope}[xshift=3.75cm,yshift=1.17cm]
    \draw[fill=black!50] (0.2,0.33) ellipse (0.08 and 0.33);
    \draw[fill=black!50, color=black!50] (0,0) rectangle (0.2,0.66);
    \draw[fill=white] (0,0.33) ellipse (0.08 and 0.33);
    \draw (0,0.66) -- (0.2,0.66);
    \draw (0,0) -- (0.2,0) node[pos=0.5,below] {$G$};
  \end{scope}

  % Output shaft of motor
  \begin{scope}[xshift=2.8cm,yshift=1.45cm]
    \draw[fill=black!50] (0,0) rectangle (0.95,0.1);
  \end{scope}

  % Angular velocity arrow of drive -> transmission
  \draw[line width=0.7pt,<-] (3.2,1) arc (-30:30:1) node[above] {$\omega_m$};

  % Transmission gear two
  \begin{scope}[xshift=3.75cm,yshift=1.83cm]
    \draw[fill=black!50] (0.2,0.68) ellipse (0.13 and 0.67);
    \draw[fill=black!50, color=black!50] (0,0) rectangle (0.2,1.35);
    \draw[fill=white] (0,0.68) ellipse (0.13 and 0.67);
    \draw (0,1.35) -- (0.2,1.35);
    \draw (0,0) -- (0.2,0);
  \end{scope}

  % Flywheel
  \begin{scope}[xshift=5.05cm,yshift=2.09cm]
    \draw[fill=white] (0.6,0.4) ellipse (0.13 and 0.4);
    \draw[fill=white,color=white] (0,0.8) rectangle (0.6,0);
    \draw[fill=white] (0,0.4) ellipse (0.13 and 0.4);
    \draw (0,0) -- (0.6,0);
    \draw (0,0.8) -- (0.6,0.8);
  \end{scope}

  % Transmission shaft from gear two to flywheel
  \begin{scope}[xshift=4.09cm,yshift=2.42cm]
    \draw[fill=black!50] (0,0) rectangle (0.96,0.1);
  \end{scope}

  % Angular velocity arrow between transmission and flywheel
  \draw[line width=0.7pt,->] (4.54,1.97) arc (-30:30:1) node[above]
    {$\omega_f$};

  % Descriptions inside graphic
  \draw (5.45,2.49) node {$J$};

  % Descriptions of subsystems under graphic
  \begin{scope}[xshift=-0.5cm,yshift=-0.28cm]
    \draw[decorate,decoration={brace,amplitude=10pt}]
      (3.5,0) -- (0,0) node[midway,yshift=-20pt] {circuit};
    \draw[decorate,decoration={brace,amplitude=10pt}]
      (6.55,0) -- (3.75,0) node[midway,yshift=-20pt] {mechanics};
  \end{scope}
\end{tikzpicture}

  \caption{Flywheel system diagram}
\end{bookfigure}

\subsection{Continuous state-space model}
\index{FRC models!flywheel equations}

By equation \eqref{eq:dot_omega_flywheel}
\begin{align*}
  \dot{\omega} &= -\frac{G^2 K_t}{K_v RJ} \omega + \frac{G K_t}{RJ} V
  \intertext{Factor out $\omega$ and $V$ into column vectors.}
  \dot{\begin{bmatrix}
    \omega
  \end{bmatrix}} &=
  \begin{bmatrix}
    -\frac{G^2 K_t}{K_v RJ}
  \end{bmatrix}
  \begin{bmatrix}
    \omega
  \end{bmatrix} +
  \begin{bmatrix}
    \frac{GK_t}{RJ}
  \end{bmatrix}
  \begin{bmatrix}
    V
  \end{bmatrix}
\end{align*}
\begin{theorem}[Flywheel state-space model]
  \begin{align*}
    \dot{\mat{x}} &= \mat{A} \mat{x} + \mat{B} \mat{u} \\
    \mat{y} &= \mat{C} \mat{x} + \mat{D} \mat{u}
  \end{align*}
  \begin{equation*}
    \mat{x} = \omega
    \quad
    \mat{y} = \omega
    \quad
    \mat{u} = V
  \end{equation*}
  \begin{equation}
    \mat{A} = -\frac{G^2 K_t}{K_v RJ}
    \quad
    \mat{B} = \frac{G K_t}{RJ}
    \quad
    \mat{C} = 1
    \quad
    \mat{D} = 0
  \end{equation}
\end{theorem}

\subsection{Model augmentation}

As per subsection \ref{subsec:input_error_estimation}, we will now augment the
\gls{model} so a $u_{error}$ state is added to the \gls{control input}.

The \gls{plant} and \gls{observer} augmentations should be performed before the
\gls{model} is \glslink{discretization}{discretized}. After the \gls{controller}
gain is computed with the unaugmented discrete \gls{model}, the controller may
be augmented. Therefore, the \gls{plant} and \gls{observer} augmentations assume
a continuous \gls{model} and the \gls{controller} augmentation assumes a
discrete \gls{controller}.
\begin{equation*}
  \mat{x} =
  \begin{bmatrix}
    \omega \\
    u_{error}
  \end{bmatrix}
  \quad
  \mat{y} = \omega
  \quad
  \mat{u} = V
\end{equation*}
\begin{equation}
  \mat{A}_{aug} =
  \begin{bmatrix}
    \mat{A} & \mat{B} \\
    0 & 0
  \end{bmatrix}
  \quad
  \mat{B}_{aug} =
  \begin{bmatrix}
    \mat{B} \\
    0
  \end{bmatrix}
  \quad
  \mat{C}_{aug} = \begin{bmatrix}
    \mat{C} & 0
  \end{bmatrix}
  \quad
  \mat{D}_{aug} = \mat{D}
\end{equation}
\begin{equation}
  \mat{K}_{aug} = \begin{bmatrix}
    \mat{K} & 1
  \end{bmatrix}
  \quad
  \mat{r}_{aug} = \begin{bmatrix}
    \mat{r} \\
    0
  \end{bmatrix}
\end{equation}

This will compensate for unmodeled dynamics such as projectiles slowing down the
flywheel.

\subsection{Simulation}

Python Control will be used to \glslink{discretization}{discretize} the
\gls{model} and simulate it. One of the frccontrol
examples\footnote{\url{https://github.com/calcmogul/frccontrol/blob/main/examples/flywheel.py}}
creates and tests a controller for it. Figure \ref{fig:flywheel_response} shows
the closed-loop \gls{system} response.
\begin{svg}{build/frccontrol/examples/flywheel_response}
  \caption{Flywheel response}
  \label{fig:flywheel_response}
\end{svg}

Notice how the \gls{control effort} when the \gls{reference} is reached is
nonzero. This is a plant inversion feedforward compensating for the \gls{system}
dynamics attempting to slow the flywheel down when no voltage is applied.

\subsection{Implementation}

C++ and Java implementations of this flywheel controller are available
online\footnote{\url{https://github.com/wpilibsuite/allwpilib/blob/main/wpilibcExamples/src/main/cpp/examples/StateSpaceFlywheel/cpp/Robot.cpp}}
\footnote{\url{https://github.com/wpilibsuite/allwpilib/blob/main/wpilibjExamples/src/main/java/edu/wpi/first/wpilibj/examples/statespaceflywheel/Robot.java}}.

\subsection{Flywheel model without encoder}

In the FIRST Robotics Competition, we can get the current drawn for specific
channels on the power distribution panel. We can theoretically use this to
estimate the angular velocity of a DC motor without an encoder. We'll start with
the flywheel model derived earlier as equation \eqref{eq:dot_omega_flywheel}.
\begin{align*}
  \dot{\omega} &= \frac{G K_t}{RJ} V - \frac{G^2 K_t}{K_v RJ} \omega \\
  \dot{\omega} &= -\frac{G^2 K_t}{K_v RJ} \omega + \frac{G K_t}{RJ} V
  \intertext{Next, we'll derive the current $I$ as an output.}
  V &= IR + \frac{\omega}{K_v} \\
  IR &= V - \frac{\omega}{K_v} \\
  I &= -\frac{1}{K_v R} \omega + \frac{1}{R} V
\end{align*}

Therefore,
\begin{theorem}[Flywheel state-space model without encoder]
  \begin{align*}
    \dot{\mat{x}} &= \mat{A} \mat{x} + \mat{B} \mat{u} \\
    \mat{y} &= \mat{C} \mat{x} + \mat{D} \mat{u}
  \end{align*}
  \begin{equation*}
    \mat{x} = \omega
    \quad
    \mat{y} = I
    \quad
    \mat{u} = V
  \end{equation*}
  \begin{equation}
    \mat{A} = -\frac{G^2 K_t}{K_v RJ}
    \quad
    \mat{B} = \frac{G K_t}{RJ}
    \quad
    \mat{C} = -\frac{1}{K_v R}
    \quad
    \mat{D} = \frac{1}{R}
  \end{equation}
\end{theorem}

Notice that in this \gls{model}, the \gls{output} doesn't provide any direct
measurements of the \gls{state}. To estimate the full \gls{state} (also known as
full observability), we only need the \glspl{output} to collectively include
linear combinations of every \gls{state}\footnote{While the flywheel model's
outputs are a linear combination of both the states and inputs, \glspl{input}
don't provide new information about the \glspl{state}. Therefore, they don't
affect whether the system is observable.}. We'll revisit this in chapter
\ref{ch:stochastic_control_theory} with an example that uses range measurements
to estimate an object's orientation.

The effectiveness of this \gls{model}'s \gls{observer} is heavily dependent on
the quality of the current sensor used. If the sensor's noise isn't zero-mean,
the \gls{observer} won't converge to the true \gls{state}.

\subsection{Voltage compensation}

To improve controller \gls{tracking}, one may want to use the voltage
renormalized to the power rail voltage to compensate for voltage drop when
current spikes occur. This can be done as follows.
\begin{equation}
  V = V_{cmd} \frac{V_{nominal}}{V_{rail}}
\end{equation}

where $V$ is the \gls{controller}'s new input voltage, $V_{cmd}$ is the old
input voltage, $V_{nominal}$ is the rail voltage when effects like voltage drop
due to current draw are ignored, and $V_{rail}$ is the real rail voltage.

To drive the \gls{model} with a more accurate voltage that includes voltage
drop, the reciprocal can be used.
\begin{equation}
  V = V_{cmd} \frac{V_{rail}}{V_{nominal}}
\end{equation}

where $V$ is the \gls{model}'s new input voltage. Note that if both the
\gls{controller} compensation and \gls{model} compensation equations are
applied, the original voltage is obtained. The \gls{model} input only drops from
ideal if the compensated \gls{controller} voltage saturates.
