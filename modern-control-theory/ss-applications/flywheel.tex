\section{Flywheel}
\label{sec:ss_model_flywheel}

\subsection{Continuous state-space model}
\index{FRC models!flywheel equations}

By equation \eqref{eq:dot_omega_flywheel}

\begin{equation*}
  \dot{\omega} = -\frac{G^2 K_t}{K_v RJ} \omega + \frac{G K_t}{RJ} V
\end{equation*}

Factor out $\omega$ and $V$ into column vectors.

\begin{align*}
  \dot{\begin{bmatrix}
    \omega
  \end{bmatrix}} &=
  \begin{bmatrix}
    -\frac{G^2 K_t}{K_v RJ}
  \end{bmatrix}
  \begin{bmatrix}
    \omega
  \end{bmatrix} +
  \begin{bmatrix}
    \frac{GK_t}{RJ}
  \end{bmatrix}
  \begin{bmatrix}
    V
  \end{bmatrix}
\end{align*}

\begin{theorem}[Flywheel state-space model]
  \begin{align*}
    \dot{\mtx{x}} &= \mtx{A} \mtx{x} + \mtx{B} \mtx{u} \\
    \mtx{y} &= \mtx{C} \mtx{x} + \mtx{D} \mtx{u}
  \end{align*}
  \begin{equation*}
    \begin{array}{ccc}
      \mtx{x} = \omega &
      \mtx{y} = \omega &
      \mtx{u} = V
    \end{array}
  \end{equation*}
  \begin{equation}
    \begin{array}{cccc}
      \mtx{A} = -\frac{G^2 K_t}{K_v RJ} &
      \mtx{B} = \frac{G K_t}{RJ} &
      \mtx{C} = 1 &
      \mtx{D} = 0
    \end{array}
  \end{equation}
\end{theorem}

\subsection{Model augmentation}

As per subsection \ref{subsec:input_error_estimation}, we will now augment the
\gls{model} so a $u_{error}$ state is added to the \gls{control input}.

The \gls{plant} and \gls{observer} augmentations should be performed before the
\gls{model} is \glslink{discretization}{discretized}. After the \gls{controller}
gain is computed with the unaugmented discrete \gls{model}, the controller may
be augmented. Therefore, the \gls{plant} and \gls{observer} augmentations assume
a continuous \gls{model} and the \gls{controller} augmentation assumes a
discrete \gls{controller}.

\begin{equation*}
  \begin{array}{ccc}
    \mtx{x} =
    \begin{bmatrix}
      \omega \\
      u_{error}
    \end{bmatrix} &
    \mtx{y} = \omega &
    \mtx{u} = V
  \end{array}
\end{equation*}

\begin{equation}
  \begin{array}{cccc}
    \mtx{A}_{aug} =
    \begin{bmatrix}
      \mtx{A} & \mtx{B} \\
      0 & 0
    \end{bmatrix} &
    \mtx{B}_{aug} =
    \begin{bmatrix}
      \mtx{B} \\
      0
    \end{bmatrix} &
    \mtx{C}_{aug} = \begin{bmatrix}
      \mtx{C} & 0
    \end{bmatrix} &
    \mtx{D}_{aug} = \mtx{D}
  \end{array}
\end{equation}

\begin{equation}
  \begin{array}{cc}
    \mtx{K}_{aug} = \begin{bmatrix}
      \mtx{K} & 1
    \end{bmatrix} &
    \mtx{r}_{aug} = \begin{bmatrix}
      \mtx{r} \\
      0
    \end{bmatrix}
  \end{array}
\end{equation}

This will compensate for unmodeled dynamics such as projectiles slowing down the
flywheel.

\subsection{Simulation}

Python Control will be used to \glslink{discretization}{discretize} the
\gls{model} and simulate it. One of the frccontrol
examples\footnote{\url{https://github.com/calcmogul/frccontrol/blob/master/examples/flywheel.py}}
creates and tests a controller for it. Figure \ref{fig:flywheel_response} shows
the closed-loop \gls{system} response.

\begin{svg}{build/frccontrol/examples/flywheel_response}
  \caption{Flywheel response}
  \label{fig:flywheel_response}
\end{svg}

Notice how the \gls{control effort} when the \gls{reference} is reached is
nonzero. This is a plant inversion feedforward compensating for the \gls{system}
dynamics attempting to slow the flywheel down when no voltage is applied.

\subsection{Implementation}

The script linked above also generates two files: FlywheelCoeffs.h and
FlywheelCoeffs.cpp. These can be used with the WPILib StateSpacePlant,
StateSpaceController, and StateSpaceObserver classes in C++ and Java. A C++
implementation of this flywheel controller is available online\footnote{
\url{https://github.com/calcmogul/allwpilib/tree/state-space/wpilibcExamples/src/main/cpp/examples/StateSpaceFlywheel}}.

\subsection{Flywheel model without encoder}

In the FIRST Robotics Competition, we can get the current drawn for specific
channels on the power distribution panel. We can theoretically use this to
estimate the angular velocity of a DC motor without an encoder. We'll start with
the flywheel model derived earlier as equation \eqref{eq:dot_omega_flywheel}.

\begin{align*}
  \dot{\omega} &= \frac{G K_t}{RJ} V - \frac{G^2 K_t}{K_v RJ} \omega \\
  \dot{\omega} &= -\frac{G^2 K_t}{K_v RJ} \omega + \frac{G K_t}{RJ} V
\end{align*}

Next, we'll derive the current $I$ as an output.

\begin{align*}
  V &= IR + \frac{\omega}{K_v} \\
  IR &= V - \frac{\omega}{K_v} \\
  I &= -\frac{1}{K_v R} \omega + \frac{1}{R} V
\end{align*}

Therefore,

\begin{theorem}[Flywheel state-space model without encoder]
  \begin{align*}
    \dot{\mtx{x}} &= \mtx{A} \mtx{x} + \mtx{B} \mtx{u} \\
    \mtx{y} &= \mtx{C} \mtx{x} + \mtx{D} \mtx{u}
  \end{align*}
  \begin{equation*}
    \begin{array}{ccc}
      \mtx{x} = \omega &
      \mtx{y} = I &
      \mtx{u} = V
    \end{array}
  \end{equation*}
  \begin{equation}
    \begin{array}{cccc}
      \mtx{A} = -\frac{G^2 K_t}{K_v RJ} &
      \mtx{B} = \frac{G K_t}{RJ} &
      \mtx{C} = -\frac{1}{K_v R} &
      \mtx{D} = \frac{1}{R}
    \end{array}
  \end{equation}
\end{theorem}

Notice that in this \gls{model}, the \gls{output} doesn't provide any direct
measurements of the \gls{state}. To estimate the full \gls{state} (alsa known as
full observability), we only need the \glspl{output} to collectively include
linear combinations of every \gls{state}\footnote{While the flywheel model's
outputs are a linear combination of both the states and inputs, \glspl{input}
don't provide new information about the \glspl{state}. Therefore, they don't
affect whether the system is observable.}. We'll revisit this in chapter
\ref{ch:stochastic_control_theory} with an example that uses range measurements
to estimate an object's orientation.

The effectiveness of this \gls{model}'s \gls{observer} is heavily dependent on
the quality of the current sensor used. If the sensor's noise isn't zero-mean,
the \gls{observer} won't converge to the true \gls{state}.

\subsection{Voltage compensation}

To improve controller \gls{tracking}, one may want to use the voltage
renormalized to the power rail voltage to compensate for voltage drop when
current spikes occur. This can be done as follows.

\begin{equation}
  V = V_{cmd} \frac{V_{nominal}}{V_{rail}}
\end{equation}

where $V$ is the \gls{controller}'s new input voltage, $V_{cmd}$ is the old
input voltage, $V_{nominal}$ is the rail voltage when effects like voltage drop
due to current draw are ignored, and $V_{rail}$ is the real rail voltage.

To drive the \gls{model} with a more accurate voltage that includes voltage
drop, the reciprocal can be used.

\begin{equation}
  V = V_{cmd} \frac{V_{rail}}{V_{nominal}}
\end{equation}

where $V$ is the \gls{model}'s new input voltage. Note that if both the
\gls{controller} compensation and \gls{model} compensation equations are
applied, the original voltage is obtained. The \gls{model} input only drops from
ideal if the compensated \gls{controller} voltage saturates.
