\section{Pendulum}

\subsection{State-space model}

Below is the \gls{model} for a pendulum
\begin{equation*}
  \ddot{\theta} = -\frac{g}{l}\sin\theta
\end{equation*}

where $\theta$ is the angle of the pendulum and $l$ is the length of the
pendulum.

Since state-space representation requires that only single derivatives be used,
they should be broken up as separate \glspl{state}. We'll reassign
$\dot{\theta}$ to be $\omega$ so the derivatives are easier to keep straight for
state-space representation.
\begin{equation*}
  \dot{\omega} = -\frac{g}{l}\sin\theta
\end{equation*}

Now separate the \glspl{state}.
\begin{align*}
  \dot{\theta} &= \omega \\
  \dot{\omega} &= -\frac{g}{l} \sin\theta
\end{align*}

This makes our state vector $\begin{bmatrix}\theta & \omega\end{bmatrix}\T$ and
our nonlinear model
\begin{equation*}
  f(\mat{x}, \mat{u}) =
  \begin{bmatrix}
    \omega \\
    -\frac{g}{l}\sin\theta
  \end{bmatrix}
\end{equation*}

\subsubsection{Linearization around $\theta = 0$}

To apply our tools for linear control theory, the \gls{model} must be a linear
combination of the \glspl{state} and \glspl{input} (addition and multiplication
by constants). Since this \gls{model} is nonlinear on account of the sine
function, we should \glslink{linearization}{linearize}
\index{nonlinear control!linearization} it.

Linearization finds a tangent line to the nonlinear dynamics at a desired point
in the state-space. The Taylor series is a way to approximate arbitrary
functions with polynomials, so we can use it for linearization.

The taylor series expansion for $\sin\theta$ around $\theta = 0$ is
$\theta - \frac{1}{6} \theta^3 + \frac{1}{120} \theta^5 - \ldots$. We'll take
just the first-order term $\theta$ to obtain a linear function.
\begin{align*}
  \dot{\theta} &= \omega \\
  \dot{\omega} &= -\frac{g}{l} \theta
\end{align*}

Now write the \gls{model} in state-space representation. We'll write out the
system of equations with the zeroed variables included to assist with this.
\begin{align*}
  \dot{\theta} &= \;\;\;\,0 \theta + 1 \omega \\
  \dot{\omega} &= -\frac{g}{l} \theta + 0 \omega
\end{align*}

Factor out $\theta$ and $\omega$ into a column vector.
\begin{align}
  \dot{
  \begin{bmatrix}
    \theta \\
    \omega
  \end{bmatrix}} =
  \begin{bmatrix}
    0 & 1 \\
    -\frac{g}{l} & 0
  \end{bmatrix}
  \begin{bmatrix}
    \theta \\
    \omega
  \end{bmatrix}
\end{align}

\subsubsection{Linearization with the Jacobian}

Here's the original nonlinear model in state-space representation.
\begin{equation*}
  f(\mat{x}, \mat{u}) =
  \begin{bmatrix}
    \omega \\
    -\frac{g}{l}\sin\theta
  \end{bmatrix}
\end{equation*}

If we want to linearize around an arbitrary point, we can take the Jacobian with
respect to $\mat{x}$.
\begin{equation*}
  \frac{\partial f(\mat{x}, \mat{u})}{\partial\mat{x}} =
  \begin{bmatrix}
    0 & 1 \\
    -\frac{g}{l}\cos\theta & 0
  \end{bmatrix}
\end{equation*}

For full \gls{state} feedback, knowledge of all \glspl{state} is required. If
not all \glspl{state} are measured directly, an estimator can be used to
supplement them.

We may only be measuring $\theta$ in the pendulum example, not $\dot{\theta}$,
so we'll need to estimate the latter. The $\mat{C}$ matrix the \gls{observer}
would use in this case is
\begin{equation*}
  \mat{C} = \begin{bmatrix}
    1 & 0 \\
  \end{bmatrix}
\end{equation*}

Therefore, the measurement vector is
\begin{align*}
  \mat{y} &= \mat{C}\mat{x} \\
  \mat{y} &= \begin{bmatrix}
    1 & 0
  \end{bmatrix}
  \begin{bmatrix}
    \theta \\
    \omega
  \end{bmatrix} \\
  \mat{y} &= 1\theta + 0\omega \\
  \mat{y} &= \theta
\end{align*}
\begin{theorem}[Linear time-varying pendulum state-space model with no input]
  \begin{align*}
    \dot{\mat{x}} &= \mat{A} \mat{x} \\
    \mat{y} &= \mat{C} \mat{x}
  \end{align*}
  \begin{equation*}
    \mat{x} =
    \begin{bmatrix}
      \theta \\
      \omega
    \end{bmatrix}
    \quad
    \mat{y} = \theta
  \end{equation*}
  \begin{equation}
    \mat{A} =
    \begin{bmatrix}
      0 & 1 \\
      -\frac{g}{l}\cos\theta & 0
    \end{bmatrix}
    \quad
    \mat{C} =
    \begin{bmatrix}
      1 & 0
    \end{bmatrix}
  \end{equation}
\end{theorem}
