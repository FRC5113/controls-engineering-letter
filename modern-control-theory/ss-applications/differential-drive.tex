\section{Differential drive}
\label{sec:ss_model_differential_drive}

This drivetrain consists of two DC brushed motors per side which are chained
together on their respective sides and drive wheels which are assumed to be
massless.
\begin{bookfigure}
  \begin{tikzpicture}[auto, >=latex', circuit ee IEC,
                    set resistor graphic=var resistor IEC graphic]
  % \draw [help lines] (-1,-3) grid (7,4);

  % Right wheel
  \begin{scope}[xshift=5.78cm,yshift=1.83cm]
    \draw[fill=black!50] (0.2,0.68) ellipse (0.13 and 0.67);
    \draw[fill=black!50, color=black!50] (0,0) rectangle (0.2,1.35);
    \draw[fill=white] (0,0.68) ellipse (0.13 and 0.67);
    \draw (0,1.35) -- (0.2,1.35);
    \draw (0,0) -- (0.2,0);
  \end{scope}

  % Right transmission shaft
  \begin{scope}[xshift=5.32cm,yshift=2.42cm]
    \draw[fill=black!50] (0,0) rectangle (0.46,0.1);
  \end{scope}

  % Chassis
  \begin{scope}[xshift=4.44cm,yshift=2.09cm]
    \fill[fill=white] (0,0.8) -- (0,0.2) -- (0.2,0) -- (0.2,0.2)
      -- (0.98,0.2) -- (0.78,0.8) -- cycle;
    \draw (0,0.8) -- (0.78,0.8);
    \draw (0,0.8) -- (0,0.2);
    \draw (0,0.2) -- (0.2,0);
    \draw (0,0.8) -- (0.2,0.6);
    \draw (0.78,0.8) -- (0.98,0.6);
    \draw[fill=white] (0.2,0.6) rectangle (0.98,0);
  \end{scope}

  % Left transmission shaft
  \begin{scope}[xshift=4.09cm,yshift=2.42cm]
    \draw[fill=black!50] (0,0) rectangle (0.46,0.1);
  \end{scope}

  % Left wheel
  \begin{scope}[xshift=3.75cm,yshift=1.83cm]
    \draw[fill=black!50] (0.2,0.68) ellipse (0.13 and 0.67);
    \draw[fill=black!50, color=black!50] (0,0) rectangle (0.2,1.35);
    \draw[fill=white] (0,0.68) ellipse (0.13 and 0.67);
    \draw (0,1.35) -- (0.2,1.35);
    \draw (0,0) -- (0.2,0);
  \end{scope}

  % Angular velocity arrow for left wheel
  \draw[line width=0.7pt,->] (4.24,1.97) arc (-30:30:1) node[above]
    {$\omega_l$};

  % Angular velocity arrow for right wheel
  \draw[line width=0.7pt,->] (5.44,1.97) arc (-30:30:1) node[above]
    {$\omega_r$};

  % Wheel radius arrow
  \begin{scope}[xshift=3.5cm,yshift=1.83cm]
    \draw[line width=0.7pt,<->] (0,0) -- node[left] {$r$} (0,0.67);
  \end{scope}

  % Robot radius arrow
  \begin{scope}[xshift=4.65cm,yshift=1.83cm]
    \draw[line width=0.7pt,<->] (0,0) -- node[below] {$r_b$} (0.39,0);
  \end{scope}

  % Descriptions inside graphic
  \draw (4.99,2.42) node {$J$};
\end{tikzpicture}

  \caption{Differential drive system diagram}
\end{bookfigure}

\subsection{Velocity subspace state-space model}
\index{FRC models!differential drive equations}

By equations \eqref{eq:diff_drive_model_right} and
\eqref{eq:diff_drive_model_left}
\begin{align*}
  \dot{v}_l &= \left(\frac{1}{m} + \frac{r_b^2}{J}\right)
    \left(C_1 v_l + C_2 V_l\right) +
    \left(\frac{1}{m} - \frac{r_b^2}{J}\right) \left(C_3 v_r + C_4 V_r\right) \\
  \dot{v}_r &= \left(\frac{1}{m} - \frac{r_b^2}{J}\right)
    \left(C_1 v_l + C_2 V_l\right) +
    \left(\frac{1}{m} + \frac{r_b^2}{J}\right) \left(C_3 v_r + C_4 V_r\right)
  \intertext{Regroup the terms into states $v_l$ and $v_r$ and inputs $V_l$ and
    $V_r$.}
  \dot{v}_l &= \left(\frac{1}{m} + \frac{r_b^2}{J}\right) C_1 v_l +
    \left(\frac{1}{m} + \frac{r_b^2}{J}\right) C_2 V_l +
    \left(\frac{1}{m} - \frac{r_b^2}{J}\right) C_3 v_r +
    \left(\frac{1}{m} - \frac{r_b^2}{J}\right) C_4 V_r \\
  \dot{v}_r &= \left(\frac{1}{m} - \frac{r_b^2}{J}\right) C_1 v_l +
    \left(\frac{1}{m} - \frac{r_b^2}{J}\right) C_2 V_l +
    \left(\frac{1}{m} + \frac{r_b^2}{J}\right) C_3 v_r +
    \left(\frac{1}{m} + \frac{r_b^2}{J}\right) C_4 V_r
  \shortintertext{}
  \dot{v}_l &= \left(\frac{1}{m} + \frac{r_b^2}{J}\right) C_1 v_l +
    \left(\frac{1}{m} - \frac{r_b^2}{J}\right) C_3 v_r +
    \left(\frac{1}{m} + \frac{r_b^2}{J}\right) C_2 V_l +
    \left(\frac{1}{m} - \frac{r_b^2}{J}\right) C_4 V_r \\
  \dot{v}_r &= \left(\frac{1}{m} - \frac{r_b^2}{J}\right) C_1 v_l +
    \left(\frac{1}{m} + \frac{r_b^2}{J}\right) C_3 v_r +
    \left(\frac{1}{m} - \frac{r_b^2}{J}\right) C_2 V_l +
    \left(\frac{1}{m} + \frac{r_b^2}{J}\right) C_4 V_r
  \intertext{Factor out $v_l$ and $v_r$ into a column vector and $V_l$ and $V_r$
    into a column vector.}
  \dot{\begin{bmatrix}
    v_l \\
    v_r
  \end{bmatrix}} &=
  \begin{bmatrix}
    \left(\frac{1}{m} + \frac{r_b^2}{J}\right) C_1 &
    \left(\frac{1}{m} - \frac{r_b^2}{J}\right) C_3 \\
    \left(\frac{1}{m} - \frac{r_b^2}{J}\right) C_1 &
    \left(\frac{1}{m} + \frac{r_b^2}{J}\right) C_3
  \end{bmatrix}
  \begin{bmatrix}
    v_l \\
    v_r
  \end{bmatrix} +
  \begin{bmatrix}
    \left(\frac{1}{m} + \frac{r_b^2}{J}\right) C_2 &
    \left(\frac{1}{m} - \frac{r_b^2}{J}\right) C_4 \\
    \left(\frac{1}{m} - \frac{r_b^2}{J}\right) C_2 &
    \left(\frac{1}{m} + \frac{r_b^2}{J}\right) C_4
  \end{bmatrix}
  \begin{bmatrix}
    V_l \\
    V_r
  \end{bmatrix}
\end{align*}
\begin{theorem}[Differential drive velocity state-space model]
  \label{thm:diff_drive_velocity_ss_model}
  \begin{align*}
    \dot{\mat{x}} &= \mat{A} \mat{x} + \mat{B} \mat{u} \\
    \mat{y} &= \mat{C} \mat{x} + \mat{D} \mat{u}
  \end{align*}
  \begin{equation*}
    \mat{x} =
    \begin{bmatrix}
      v_l \\
      v_r
    \end{bmatrix}
    \quad
    \mat{y} =
    \begin{bmatrix}
      v_l \\
      v_r
    \end{bmatrix}
    \quad
    \mat{u} =
    \begin{bmatrix}
      V_l \\
      V_r
    \end{bmatrix}
  \end{equation*}
  \begin{equation}
    \label{eq:diff_drive_ss_model}
    \begin{array}{ll}
      \mat{A} =
      \begin{bmatrix}
        \left(\frac{1}{m} + \frac{r_b^2}{J}\right) C_1 & \left(\frac{1}{m} - \frac{r_b^2}{J}\right) C_3 \\
        \left(\frac{1}{m} - \frac{r_b^2}{J}\right) C_1 & \left(\frac{1}{m} + \frac{r_b^2}{J}\right) C_3
      \end{bmatrix} &
      \mat{B} =
      \begin{bmatrix}
        \left(\frac{1}{m} + \frac{r_b^2}{J}\right) C_2 & \left(\frac{1}{m} - \frac{r_b^2}{J}\right) C_4 \\
        \left(\frac{1}{m} - \frac{r_b^2}{J}\right) C_2 & \left(\frac{1}{m} + \frac{r_b^2}{J}\right) C_4
      \end{bmatrix} \\
      \mat{C} =
      \begin{bmatrix}
        1 & 0 \\
        0 & 1 \\
      \end{bmatrix} &
      \mat{D} = \mat{0}_{2 \times 2}
    \end{array}
  \end{equation}

  where $C_1 = -\frac{G_l^2 K_t}{K_v R r^2}$, $C_2 = \frac{G_l K_t}{Rr}$,
  $C_3 = -\frac{G_r^2 K_t}{K_v R r^2}$, and $C_4 = \frac{G_r K_t}{Rr}$.
\end{theorem}

\subsubsection{Simulation}

Python Control will be used to \glslink{discretization}{discretize} the
\gls{model} and simulate it. One of the frccontrol
examples\footnote{\url{https://github.com/calcmogul/frccontrol/blob/main/examples/differential_drive.py}}
creates and tests a controller for it. Figure \ref{fig:diff_drive_response}
shows the closed-loop \gls{system} response.
\begin{svg}{build/frccontrol/examples/differential_drive_response}
  \caption{Drivetrain response}
  \label{fig:diff_drive_response}
\end{svg}

Given the high inertia in drivetrains, it's better to drive the \gls{reference}
with a motion profile instead of a \gls{step input} for reproducibility.

\subsection{Linear time-varying model}
\index{controller design!linear time-varying control}
\index{nonlinear control!linear time-varying control}
\index{optimal control!linear time-varying control}

The model in theorem \ref{thm:diff_drive_velocity_ss_model} is linear, but only
includes the velocity dynamics, not the dynamics of the drivetrain's global
pose. The change in global pose is defined by these three equations.
\begin{align*}
  \dot{x} &= \frac{v_l + v_r}{2}\cos\theta = \frac{v_r}{2}\cos\theta +
    \frac{v_l}{2}\cos\theta \\
  \dot{y} &= \frac{v_l + v_r}{2}\sin\theta = \frac{v_r}{2}\sin\theta +
    \frac{v_l}{2}\sin\theta \\
  \dot{\theta} &= \frac{v_r - v_l}{2r_b} = \frac{v_r}{2r_b} - \frac{v_l}{2r_b}
\end{align*}

Next, we'll augment the linear subspace's state with the global pose $x$, $y$,
and $\theta$. Here's the model as a vector function where
$\mat{x} = \begin{bmatrix} x & y & \theta & v_l & v_r \end{bmatrix}\T$ and
$\mat{u} = \begin{bmatrix} V_l & V_r \end{bmatrix}\T$.
\begin{equation}
  f(\mat{x}, \mat{u}) =
  \begin{bmatrix}
    \frac{v_r}{2}\cos\theta + \frac{v_l}{2}\cos\theta \\
    \frac{v_r}{2}\sin\theta + \frac{v_l}{2}\sin\theta \\
    \frac{v_r}{2r_b} - \frac{v_l}{2r_b} \\
    \left(\frac{1}{m} + \frac{r_b^2}{J}\right) C_1 v_l +
      \left(\frac{1}{m} - \frac{r_b^2}{J}\right) C_3 v_r +
      \left(\frac{1}{m} + \frac{r_b^2}{J}\right) C_2 V_l +
      \left(\frac{1}{m} - \frac{r_b^2}{J}\right) C_4 V_r \\
    \left(\frac{1}{m} - \frac{r_b^2}{J}\right) C_1 v_l +
      \left(\frac{1}{m} + \frac{r_b^2}{J}\right) C_3 v_r +
      \left(\frac{1}{m} - \frac{r_b^2}{J}\right) C_2 V_l +
      \left(\frac{1}{m} + \frac{r_b^2}{J}\right) C_4 V_r
  \end{bmatrix}
  \label{eq:ltv_diff_drive_f}
\end{equation}

As mentioned in chapter \ref{ch:nonlinear_control}, one can approximate a
nonlinear system via linearizations around points of interest in the state-space
and design controllers for those linearized subspaces. If we sample
linearization points progressively closer together, we converge on a control
policy for the original nonlinear system. Since the linear \gls{plant} being
controlled varies with time, its controller is called a linear time-varying
(LTV) controller.

If we use LQRs for the linearized subspaces, the nonlinear control policy will
also be locally optimal. We'll be taking this approach with a differential
drive. To create an LQR, we need to linearize equation
\eqref{eq:ltv_diff_drive_f}.
\begin{align*}
  \frac{\partial f(\mat{x}, \mat{u})}{\partial\mat{x}} &=
  \begin{bmatrix}
    0 & 0 & -\frac{v_l + v_r}{2}\sin\theta & \frac{1}{2}\cos\theta &
      \frac{1}{2}\cos\theta \\
    0 & 0 & \frac{v_l + v_r}{2}\cos\theta & \frac{1}{2}\sin\theta &
      \frac{1}{2}\sin\theta \\
    0 & 0 & 0 & -\frac{1}{2r_b} & \frac{1}{2r_b} \\
    0 & 0 & 0 & \left(\frac{1}{m} + \frac{r_b^2}{J}\right) C_1 &
      \left(\frac{1}{m} - \frac{r_b^2}{J}\right) C_3 \\
    0 & 0 & 0 & \left(\frac{1}{m} - \frac{r_b^2}{J}\right) C_1 &
      \left(\frac{1}{m} + \frac{r_b^2}{J}\right) C_3
  \end{bmatrix} \\
  \frac{\partial f(\mat{x}, \mat{u})}{\partial\mat{u}} &=
  \begin{bmatrix}
    0 & 0 \\
    0 & 0 \\
    0 & 0 \\
    \left(\frac{1}{m} + \frac{r_b^2}{J}\right) C_2 &
    \left(\frac{1}{m} - \frac{r_b^2}{J}\right) C_4 \\
    \left(\frac{1}{m} - \frac{r_b^2}{J}\right) C_2 &
    \left(\frac{1}{m} + \frac{r_b^2}{J}\right) C_4
  \end{bmatrix}
\end{align*}

Therefore,
\begin{theorem}[Linear time-varying differential drive state-space model]
  \begin{align*}
    \dot{\mat{x}} &= \mat{A}\mat{x} + \mat{B}\mat{u} \\
    \mat{y} &= \mat{C}\mat{x} + \mat{D}\mat{u}
  \end{align*}
  \begin{equation*}
    \mat{x} =
    \begin{bmatrix}
      x & y & \theta & v_l & v_r
    \end{bmatrix}\T
    \quad
    \mat{y} =
    \begin{bmatrix}
      \theta & v_l & v_r
    \end{bmatrix}\T
    \quad
    \mat{u} =
    \begin{bmatrix}
      V_l & V_r
    \end{bmatrix}\T
  \end{equation*}
  \begin{equation}
    \begin{array}{ll}
      \mat{A} =
      \begin{bmatrix}
        0 & 0 & -vs & \frac{1}{2}c & \frac{1}{2}c \\
        0 & 0 & vc & \frac{1}{2}s & \frac{1}{2}s \\
        0 & 0 & 0 & -\frac{1}{2r_b} & \frac{1}{2r_b} \\
        0 & 0 & 0 & \left(\frac{1}{m} + \frac{r_b^2}{J}\right) C_1 &
          \left(\frac{1}{m} - \frac{r_b^2}{J}\right) C_3 \\
        0 & 0 & 0 & \left(\frac{1}{m} - \frac{r_b^2}{J}\right) C_1 &
          \left(\frac{1}{m} + \frac{r_b^2}{J}\right) C_3
      \end{bmatrix} &
      \mat{B} =
      \begin{bmatrix}
        0 & 0 \\
        0 & 0 \\
        0 & 0 \\
        \left(\frac{1}{m} + \frac{r_b^2}{J}\right) C_2 &
        \left(\frac{1}{m} - \frac{r_b^2}{J}\right) C_4 \\
        \left(\frac{1}{m} - \frac{r_b^2}{J}\right) C_2 &
        \left(\frac{1}{m} + \frac{r_b^2}{J}\right) C_4
      \end{bmatrix} \\
      \mat{C} =
      \begin{bmatrix}
        0 & 0 & 1 & 0 & 0 \\
        0 & 0 & 0 & 1 & 0 \\
        0 & 0 & 0 & 0 & 1
      \end{bmatrix} &
      \mat{D} = \mat{0}_{3 \times 2}
    \end{array}
  \end{equation}

  where $v = \frac{v_l + v_r}{2}$, $c = \cos\theta$, $s = \sin\theta$,
  $C_1 = -\frac{G_l^2 K_t}{K_v R r^2}$, $C_2 = \frac{G_l K_t}{Rr}$,
  $C_3 = -\frac{G_r^2 K_t}{K_v R r^2}$, and $C_4 = \frac{G_r K_t}{Rr}$. The
  constants $C_1$ through $C_4$ are from the derivation in section
  \ref{sec:differential_drive}.
\end{theorem}

We can also use this in an extended Kalman filter as is since the measurement
model ($\mat{y} = \mat{C}\mat{x} + \mat{D}\mat{u}$) is linear.

\subsection{Improving model accuracy}

Figures \ref{fig:ltv_diff_drive_nonrotated_firstorder_xy} and
\ref{fig:ltv_diff_drive_nonrotated_firstorder_response} demonstrate the
tracking behavior of the linearized differential drive controller.
\begin{bookfigure}
  \begin{minisvg}{2}{build/\chapterpath/ltv_diff_drive_nonrotated_firstorder_xy}
    \caption{Linear time-varying differential drive controller x-y plot
      (first-order)}
    \label{fig:ltv_diff_drive_nonrotated_firstorder_xy}
  \end{minisvg}
  \hfill
  \begin{minisvg}{2}{build/\chapterpath/ltv_diff_drive_nonrotated_firstorder_response}
    \caption{Linear time-varying differential drive controller response
      (first-order)}
    \label{fig:ltv_diff_drive_nonrotated_firstorder_response}
  \end{minisvg}
\end{bookfigure}

The linearized differential drive model doesn't track well because the
first-order linearization of $\mat{A}$ doesn't capture the full heading
dynamics, making the \gls{model} update inaccurate. This linearization
inaccuracy is evident in the Hessian matrix (second partial derivative with
respect to the state vector) being nonzero.
\begin{equation*}
  \frac{\partial^2 f(\mat{x}, \mat{u})}{\partial\mat{x}^2} =
  \begin{bmatrix}
    0 & 0 & -\frac{v_l + v_r}{2}\cos\theta & 0 & 0 \\
    0 & 0 & -\frac{v_l + v_r}{2}\sin\theta & 0 & 0 \\
    0 & 0 & 0 & 0 & 0 \\
    0 & 0 & 0 & 0 & 0 \\
    0 & 0 & 0 & 0 & 0
  \end{bmatrix}
\end{equation*}

The second-order Taylor series expansion of the \gls{model} around $\mat{x}_0$
would be
\begin{equation*}
  f(\mat{x}, \mat{u}_0) \approx f(\mat{x}_0, \mat{u}_0) +
    \frac{\partial f(\mat{x}, \mat{u})}{\partial\mat{x}}(\mat{x} - \mat{x}_0) +
    \frac{1}{2}\frac{\partial^2 f(\mat{x}, \mat{u})}{\partial\mat{x}^2}
    (\mat{x} - \mat{x}_0)^2
\end{equation*}

To include higher-order dynamics in the linearized differential drive model
integration, we'll apply the fourth-order Runge-Kutta integration method (RK4)
from theorem \ref{thm:rk4} to equation \eqref{eq:ltv_diff_drive_f}.

Figures \ref{fig:ltv_diff_drive_nonrotated_xy} and
\ref{fig:ltv_diff_drive_nonrotated_response} show a simulation using RK4
instead of the first-order \gls{model}.
\begin{bookfigure}
  \begin{minisvg}{2}{build/\chapterpath/ltv_diff_drive_nonrotated_xy}
    \caption{Linear time-varying differential drive controller (global reference
        frame formulation) x-y plot}
    \label{fig:ltv_diff_drive_nonrotated_xy}
  \end{minisvg}
  \hfill
  \begin{minisvg}{2}{build/\chapterpath/ltv_diff_drive_nonrotated_response}
    \caption{Linear time-varying differential drive controller (global reference
        frame formulation) response}
    \label{fig:ltv_diff_drive_nonrotated_response}
  \end{minisvg}
\end{bookfigure}

\subsection{Cross track error controller}

Figures \ref{fig:ltv_diff_drive_nonrotated_xy} and
\ref{fig:ltv_diff_drive_nonrotated_response} show the tracking performance of
the linearized differential drive controller for a given trajectory. The
performance-effort trade-off can be tuned rather intuitively via the Q and R
gains. However, if the $x$ and $y$ error cost are too high, the $x$ and $y$
components of the controller will fight each other, and it will take longer to
converge to the path. This can be fixed by applying a clockwise rotation matrix
to the global tracking error to transform it into the robot's coordinate frame.
\begin{equation*}
  \crdfrm{R}{\begin{bmatrix}
    e_x \\
    e_y \\
    e_\theta
  \end{bmatrix}} =
  \begin{bmatrix}
    \cos\theta & \sin\theta & 0 \\
    -\sin\theta & \cos\theta & 0 \\
    0 & 0 & 1
  \end{bmatrix}
  \crdfrm{G}{\begin{bmatrix}
    e_x \\
    e_y \\
    e_\theta
  \end{bmatrix}}
\end{equation*}

where the the superscript $R$ represents the robot's coordinate frame and the
superscript $G$ represents the global coordinate frame.

With this transformation, the $x$ and $y$ error cost in LQR penalize the error
ahead of the robot and cross-track error respectively instead of global pose
error. Since the cross-track error is always measured from the robot's
coordinate frame, the \gls{model} used to compute the LQR should be linearized
around $\theta = 0$ at all times.
\begin{align*}
  \mat{A} &=
  \begin{bmatrix}
    0 & 0 & -\frac{v_l + v_r}{2}\sin 0 & \frac{1}{2}\cos 0 &
      \frac{1}{2}\cos 0 \\
    0 & 0 & \frac{v_l + v_r}{2}\cos 0 & \frac{1}{2}\sin 0 &
      \frac{1}{2}\sin 0 \\
    0 & 0 & 0 & -\frac{1}{2r_b} & \frac{1}{2r_b} \\
    0 & 0 & 0 & \left(\frac{1}{m} + \frac{r_b^2}{J}\right) C_1 &
      \left(\frac{1}{m} - \frac{r_b^2}{J}\right) C_3 \\
    0 & 0 & 0 & \left(\frac{1}{m} - \frac{r_b^2}{J}\right) C_1 &
      \left(\frac{1}{m} + \frac{r_b^2}{J}\right) C_3
  \end{bmatrix} \\
  \mat{A} &=
  \begin{bmatrix}
    0 & 0 & 0 & \frac{1}{2} & \frac{1}{2} \\
    0 & 0 & \frac{v_l + v_r}{2} & 0 & 0 \\
    0 & 0 & 0 & -\frac{1}{2r_b} & \frac{1}{2r_b} \\
    0 & 0 & 0 & \left(\frac{1}{m} + \frac{r_b^2}{J}\right) C_1 &
      \left(\frac{1}{m} - \frac{r_b^2}{J}\right) C_3 \\
    0 & 0 & 0 & \left(\frac{1}{m} - \frac{r_b^2}{J}\right) C_1 &
      \left(\frac{1}{m} + \frac{r_b^2}{J}\right) C_3
  \end{bmatrix}
\end{align*}
\begin{theorem}[Linear time-varying differential drive controller]
  \label{thm:linear_time-varying_diff_drive_controller}

  Let the differential drive dynamics be of the form
  $\dot{\mat{x}} = f(\mat{x}) + \mat{B}\mat{u}$ where
  $\mat{x} = \begin{bmatrix}x & y & \theta & v_l & v_r\end{bmatrix}\T$,
  $\mat{u} = \begin{bmatrix}V_l & V_r\end{bmatrix}\T$, and
  $\mat{A} = \left.\frac{\partial f(\mat{x})}{\partial\mat{x}}\right|_{\theta = 0}$.
  \begin{equation}
    \mat{A} =
    \begin{bmatrix}
      0 & 0 & 0 & \frac{1}{2} & \frac{1}{2} \\
      0 & 0 & v & 0 & 0 \\
      0 & 0 & 0 & -\frac{1}{2r_b} & \frac{1}{2r_b} \\
      0 & 0 & 0 & \left(\frac{1}{m} + \frac{r_b^2}{J}\right) C_1 &
        \left(\frac{1}{m} - \frac{r_b^2}{J}\right) C_3 \\
      0 & 0 & 0 & \left(\frac{1}{m} - \frac{r_b^2}{J}\right) C_1 &
        \left(\frac{1}{m} + \frac{r_b^2}{J}\right) C_3
    \end{bmatrix}
    \quad
    \mat{B} =
    \begin{bmatrix}
      0 & 0 \\
      0 & 0 \\
      0 & 0 \\
      \left(\frac{1}{m} + \frac{r_b^2}{J}\right) C_2 &
      \left(\frac{1}{m} - \frac{r_b^2}{J}\right) C_4 \\
      \left(\frac{1}{m} - \frac{r_b^2}{J}\right) C_2 &
      \left(\frac{1}{m} + \frac{r_b^2}{J}\right) C_4
    \end{bmatrix}
  \end{equation}

  where $v = \frac{v_l + v_r}{2}$, $C_1 = -\frac{G_l^2 K_t}{K_v R r^2}$,
  $C_2 = \frac{G_l K_t}{Rr}$, $C_3 = -\frac{G_r^2 K_t}{K_v R r^2}$, and
  $C_4 = \frac{G_r K_t}{Rr}$. The constants $C_1$ through $C_4$ are from the
  derivation in section \ref{sec:differential_drive}.

  The linear time-varying differential drive controller is
  \begin{equation}
    \mat{u} = \mat{K}
    \left[
      \begin{array}{c|c}
        \begin{array}{cc}
          \cos\theta & \sin\theta \\
          -\sin\theta & \cos\theta
        \end{array} & \mat{0}_{2 \times 3} \\
        \hline
        \mat{0}_{3 \times 2} & \mat{I}_{3 \times 3}
      \end{array}
    \right]
    (\mat{r} - \mat{x})
  \end{equation}

  At each timestep, the LQR controller gain $\mat{K}$ is computed for the
  $(\mat{A}, \mat{B})$ pair evaluated at the current state.
\end{theorem}

With the \gls{model} in theorem
\ref{thm:linear_time-varying_diff_drive_controller}, $y$ is uncontrollable at
$v = 0$ because nonholonomic drivetrains are unable to move sideways. Some DARE
solvers throw errors in this case, but one can avoid it by linearizing the model
around a slightly nonzero velocity instead.

The controller in theorem \ref{thm:linear_time-varying_diff_drive_controller}
results in figures \ref{fig:ltv_diff_drive_traj_xy} and
\ref{fig:ltv_diff_drive_traj_response}, which show slightly better tracking
performance than the previous formulation.
\begin{bookfigure}
  \begin{minisvg}{2}{build/\chapterpath/ltv_diff_drive_traj_xy}
    \caption{Linear time-varying differential drive controller x-y plot}
    \label{fig:ltv_diff_drive_traj_xy}
  \end{minisvg}
  \hfill
  \begin{minisvg}{2}{build/\chapterpath/ltv_diff_drive_traj_response}
    \caption{Linear time-varying differential drive controller response}
    \label{fig:ltv_diff_drive_traj_response}
  \end{minisvg}
\end{bookfigure}

\subsection{Nonlinear observer design}

\subsubsection{Encoder position augmentation}

Estimation of the global pose can be significantly improved if encoder position
measurements are used instead of velocity measurements. By augmenting the plant
with the line integral of each wheel's velocity over time, we can provide a
mapping from model states to position measurements. We can augment the linear
subspace of the model as follows.

Augment the matrix equation with position states $x_l$ and $x_r$, which have the
model equations $\dot{x}_l = v_l$ and $\dot{x}_r = v_r$. The matrix elements
corresponding to $v_l$ in the first equation and $v_r$ in the second equation
will be $1$, and the others will be $0$ since they don't appear, so
$\dot{x}_l = 1v_l + 0v_r + 0x_l + 0x_r + 0V_l + 0V_r$ and
$\dot{x}_r = 0v_l + 1v_r + 0x_l + 0x_r + 0V_l + 0V_r$. The existing rows will
have zeroes inserted where $x_l$ and $x_r$ are multiplied in.
\begin{equation*}
  \dot{\begin{bmatrix}
    x_l \\
    x_r
  \end{bmatrix}} =
  \begin{bmatrix}
    1 & 0 \\
    0 & 1
  \end{bmatrix}
  \begin{bmatrix}
    v_l \\
    v_r
  \end{bmatrix} +
  \begin{bmatrix}
    0 & 0 \\
    0 & 0
  \end{bmatrix}
  \begin{bmatrix}
    V_l \\
    V_r
  \end{bmatrix}
\end{equation*}

This produces the following linear subspace over
$\mat{x} = \begin{bmatrix}v_l & v_r & x_l & x_r\end{bmatrix}\T$.
\begin{equation}
  \mat{A} =
  \begin{bmatrix}
    \left(\frac{1}{m} + \frac{r_b^2}{J}\right) C_1 &
      \left(\frac{1}{m} - \frac{r_b^2}{J}\right) C_3 & 0 & 0 \\
    \left(\frac{1}{m} - \frac{r_b^2}{J}\right) C_1 &
      \left(\frac{1}{m} + \frac{r_b^2}{J}\right) C_3 & 0 & 0 \\
    1 & 0 & 0 & 0 \\
    0 & 1 & 0 & 0
  \end{bmatrix}
  \quad
  \mat{B} =
  \begin{bmatrix}
    \left(\frac{1}{m} + \frac{r_b^2}{J}\right) C_2 &
      \left(\frac{1}{m} - \frac{r_b^2}{J}\right) C_4 \\
    \left(\frac{1}{m} - \frac{r_b^2}{J}\right) C_2 &
      \left(\frac{1}{m} + \frac{r_b^2}{J}\right) C_4 \\
    0 & 0 \\
    0 & 0
  \end{bmatrix}
  \label{eq:diff_drive_linear_subspace}
\end{equation}

The measurement model for the complete nonlinear model is now
$\mat{y} = \begin{bmatrix}\theta & x_l & x_r\end{bmatrix}\T$ instead of
$\mat{y} = \begin{bmatrix}\theta & v_l & v_r\end{bmatrix}\T$.

\subsubsection{U error estimation}

As per subsection \ref{subsec:input_error_estimation}, we will now augment the
\gls{model} so $u_{error}$ states are added to the \glspl{control input}.

The \gls{plant} and \gls{observer} augmentations should be performed before the
\gls{model} is \glslink{discretization}{discretized}. After the \gls{controller}
gain is computed with the unaugmented discrete \gls{model}, the controller may
be augmented. Therefore, the \gls{plant} and \gls{observer} augmentations assume
a continuous \gls{model} and the \gls{controller} augmentation assumes a
discrete \gls{controller}.

The three $u_{error}$ states we'll be adding are $u_{error,l}$, $u_{error,r}$,
and $u_{error,heading}$ for left voltage error, right voltage error, and heading
error respectively. The left and right wheel positions are filtered encoder
positions and are not adjusted for heading error. The turning angle computed
from the left and right wheel positions is adjusted by the gyroscope heading.
The heading $u_{error}$ state is the heading error between what the wheel
positions imply and the gyroscope measurement.

The full state is thus
$\mat{x} = \begin{bmatrix}x & y & \theta & v_l & v_r & x_l & x_r & u_{error,l} &
  u_{error,r} & u_{error,heading}\end{bmatrix}\T$.

The complete nonlinear model is as follows. Let $v = \frac{v_l + v_r}{2}$. The
three $u_{error}$ states augment the linear subspace, so the nonlinear pose
dynamics are the same.
\begin{align}
  \dot{\begin{bmatrix}
    x \\
    y \\
    \theta
  \end{bmatrix}} &=
    \begin{bmatrix}
      v\cos\theta \\
      v\sin\theta \\
      \frac{v_r}{2r_b} - \frac{v_l}{2r_b}
    \end{bmatrix}
\end{align}

The left and right voltage error states should be mapped to the corresponding
velocity states, so the system matrix should be augmented with $\mat{B}$.

The heading $u_{error}$ is measuring counterclockwise encoder understeer
relative to the gyroscope heading, so it should add to the left position and
subtract from the right position for clockwise correction of encoder positions.
That corresponds to the following input mapping vector.
\begin{equation*}
  \mat{B}_{\theta} = \begin{bmatrix}
    0 \\
    0 \\
    1 \\
    -1
  \end{bmatrix}
\end{equation*}

Now we'll augment the linear system matrix horizontally to accomodate the
$u_{error}$ states.
\begin{equation}
  \dot{\begin{bmatrix}
    v_l \\
    v_r \\
    x_l \\
    x_r
  \end{bmatrix}} =
    \begin{bmatrix}
      \mat{A} & \mat{B} & \mat{B}_{\theta}
    \end{bmatrix}
    \begin{bmatrix}
      v_l \\
      v_r \\
      x_l \\
      x_r \\
      u_{error,l} \\
      u_{error,r} \\
      u_{error,heading}
    \end{bmatrix} + \mat{B}\mat{u}
\end{equation}

$\mat{A}$ and $\mat{B}$ are the linear subspace from equation
\eqref{eq:diff_drive_linear_subspace}.

The $u_{error}$ states have no dynamics. The observer selects them to minimize
the difference between the expected and actual measurements.
\begin{equation}
  \dot{\begin{bmatrix}
    u_{error,l} \\
    u_{error,r} \\
    u_{error,heading}
  \end{bmatrix}} = \mat{0}_{3 \times 1}
\end{equation}

The controller is augmented as follows.
\begin{equation}
  \mat{K}_{error} =
  \begin{bmatrix}
    1 & 0 & 0 \\
    0 & 1 & 0
  \end{bmatrix}
  \quad
  \mat{K}_{aug} = \begin{bmatrix}
    \mat{K} & \mat{K}_{error}
  \end{bmatrix}
  \quad
  \mat{r}_{aug} = \begin{bmatrix}
    \mat{r} \\
    0 \\
    0 \\
    0
  \end{bmatrix}
\end{equation}

This controller augmentation compensates for unmodeled dynamics like:
\begin{enumerate}
  \item Understeer caused by wheel friction inherent in skid-steer robots
  \item Battery voltage drop under load, which reduces the available control
    authority
\end{enumerate}
\begin{remark}
  The process noise for the voltage error states should be how much the voltage
  can be expected to drop. The heading error state should be the encoder
  \gls{model} uncertainty.
\end{remark}
