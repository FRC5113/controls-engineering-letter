\section{Differential drive}
\label{sec:ss_model_differential_drive}

\subsection{Velocity subspace state-space model}
\index{FRC models!differential drive equations}

By equations \eqref{eq:diff_drive_model_right} and
\eqref{eq:diff_drive_model_left}
\begin{align*}
  \dot{v}_l &= \left(\frac{1}{m} + \frac{r_b^2}{J}\right)
    \left(C_1 v_l + C_2 V_l\right) +
    \left(\frac{1}{m} - \frac{r_b^2}{J}\right) \left(C_3 v_r + C_4 V_r\right) \\
  \dot{v}_r &= \left(\frac{1}{m} - \frac{r_b^2}{J}\right)
    \left(C_1 v_l + C_2 V_l\right) +
    \left(\frac{1}{m} + \frac{r_b^2}{J}\right) \left(C_3 v_r + C_4 V_r\right)
\end{align*}

Regroup the terms into states $v_l$ and $v_r$ and inputs $V_l$ and $V_r$.
\begin{align*}
  \dot{v}_l &= \left(\frac{1}{m} + \frac{r_b^2}{J}\right) C_1 v_l +
    \left(\frac{1}{m} + \frac{r_b^2}{J}\right) C_2 V_l +
    \left(\frac{1}{m} - \frac{r_b^2}{J}\right) C_3 v_r +
    \left(\frac{1}{m} - \frac{r_b^2}{J}\right) C_4 V_r \\
  \dot{v}_r &= \left(\frac{1}{m} - \frac{r_b^2}{J}\right) C_1 v_l +
    \left(\frac{1}{m} - \frac{r_b^2}{J}\right) C_2 V_l +
    \left(\frac{1}{m} + \frac{r_b^2}{J}\right) C_3 v_r +
    \left(\frac{1}{m} + \frac{r_b^2}{J}\right) C_4 V_r
\end{align*}
\begin{align*}
  \dot{v}_l &= \left(\frac{1}{m} + \frac{r_b^2}{J}\right) C_1 v_l +
    \left(\frac{1}{m} - \frac{r_b^2}{J}\right) C_3 v_r +
    \left(\frac{1}{m} + \frac{r_b^2}{J}\right) C_2 V_l +
    \left(\frac{1}{m} - \frac{r_b^2}{J}\right) C_4 V_r \\
  \dot{v}_r &= \left(\frac{1}{m} - \frac{r_b^2}{J}\right) C_1 v_l +
    \left(\frac{1}{m} + \frac{r_b^2}{J}\right) C_3 v_r +
    \left(\frac{1}{m} - \frac{r_b^2}{J}\right) C_2 V_l +
    \left(\frac{1}{m} + \frac{r_b^2}{J}\right) C_4 V_r
\end{align*}

Factor out $v_l$ and $v_r$ into a column vector and $V_l$ and $V_r$ into a
column vector.
\begin{align*}
  \dot{\begin{bmatrix}
    v_l \\
    v_r
  \end{bmatrix}} &=
  \begin{bmatrix}
    \left(\frac{1}{m} + \frac{r_b^2}{J}\right) C_1 &
    \left(\frac{1}{m} - \frac{r_b^2}{J}\right) C_3 \\
    \left(\frac{1}{m} - \frac{r_b^2}{J}\right) C_1 &
    \left(\frac{1}{m} + \frac{r_b^2}{J}\right) C_3
  \end{bmatrix}
  \begin{bmatrix}
    v_l \\
    v_r
  \end{bmatrix} +
  \begin{bmatrix}
    \left(\frac{1}{m} + \frac{r_b^2}{J}\right) C_2 &
    \left(\frac{1}{m} - \frac{r_b^2}{J}\right) C_4 \\
    \left(\frac{1}{m} - \frac{r_b^2}{J}\right) C_2 &
    \left(\frac{1}{m} + \frac{r_b^2}{J}\right) C_4
  \end{bmatrix}
  \begin{bmatrix}
    V_l \\
    V_r
  \end{bmatrix}
\end{align*}
\begin{theorem}[Differential drive velocity state-space model]
  \label{thm:diff_drive_velocity_ss_model}

  \begin{align*}
    \dot{\mtx{x}} &= \mtx{A} \mtx{x} + \mtx{B} \mtx{u} \\
    \mtx{y} &= \mtx{C} \mtx{x} + \mtx{D} \mtx{u}
  \end{align*}
  \begin{equation*}
    \begin{array}{ccc}
      \mtx{x} =
      \begin{bmatrix}
        v_l \\
        v_r
      \end{bmatrix} &
      \mtx{y} =
      \begin{bmatrix}
        v_l \\
        v_r
      \end{bmatrix} &
      \mtx{u} =
      \begin{bmatrix}
        V_l \\
        V_r
      \end{bmatrix}
    \end{array}
  \end{equation*}
  \begin{equation}
    \label{eq:diff_drive_ss_model}
    \begin{array}{ll}
      \mtx{A} =
      \begin{bmatrix}
        \left(\frac{1}{m} + \frac{r_b^2}{J}\right) C_1 & \left(\frac{1}{m} - \frac{r_b^2}{J}\right) C_3 \\
        \left(\frac{1}{m} - \frac{r_b^2}{J}\right) C_1 & \left(\frac{1}{m} + \frac{r_b^2}{J}\right) C_3
      \end{bmatrix} &
      \mtx{B} =
      \begin{bmatrix}
        \left(\frac{1}{m} + \frac{r_b^2}{J}\right) C_2 & \left(\frac{1}{m} - \frac{r_b^2}{J}\right) C_4 \\
        \left(\frac{1}{m} - \frac{r_b^2}{J}\right) C_2 & \left(\frac{1}{m} + \frac{r_b^2}{J}\right) C_4
      \end{bmatrix} \\
      \mtx{C} =
      \begin{bmatrix}
        1 & 0 \\
        0 & 1 \\
      \end{bmatrix} &
      \mtx{D} = \mtx{0}_{2 \times 2}
    \end{array}
  \end{equation}

  where $C_1 = -\frac{G_l^2 K_t}{K_v R r^2}$, $C_2 = \frac{G_l K_t}{Rr}$,
  $C_3 = -\frac{G_r^2 K_t}{K_v R r^2}$, and $C_4 = \frac{G_r K_t}{Rr}$.
\end{theorem}

\subsubsection{Simulation}

Python Control will be used to \glslink{discretization}{discretize} the
\gls{model} and simulate it. One of the frccontrol
examples\footnote{\url{https://github.com/calcmogul/frccontrol/blob/master/examples/differential_drive.py}}
creates and tests a controller for it. Figure \ref{fig:diff_drive_response}
shows the closed-loop \gls{system} response.
\begin{svg}{build/frccontrol/examples/differential_drive_response}
  \caption{Drivetrain response}
  \label{fig:diff_drive_response}
\end{svg}

Given the high inertia in drivetrains, it's better to drive the \gls{reference}
with a motion profile instead of a \gls{step input} for reproducibility.

\subsection{Linear time-varying model}
\index{controller design!linear time-varying control}
\index{nonlinear control!linear time-varying control}
\index{optimal control!linear time-varying control}

The model in theorem \ref{thm:diff_drive_velocity_ss_model} is linear, but only
includes the velocity dynamics, not the dynamics of the drivetrain's global
pose. The change in global pose is defined by these three equations.
\begin{align*}
  \dot{x} &= \frac{v_l + v_r}{2}\cos\theta = \frac{v_r}{2}\cos\theta +
    \frac{v_l}{2}\cos\theta \\
  \dot{y} &= \frac{v_l + v_r}{2}\sin\theta = \frac{v_r}{2}\sin\theta +
    \frac{v_l}{2}\sin\theta \\
  \dot{\theta} &= \frac{v_r - v_l}{2r_b} = \frac{v_r}{2r_b} - \frac{v_l}{2r_b}
\end{align*}

Next, we'll augment the linear subspace's state with the global pose $x$, $y$,
and $\theta$. Here's the model as a vector function where
$\mtx{x} = \begin{bmatrix} x & y & \theta & v_l & v_r \end{bmatrix}^T$ and
$\mtx{u} = \begin{bmatrix} V_l & V_r \end{bmatrix}^T$.
\begin{equation}
  f(\mtx{x}, \mtx{u}) =
  \begin{bmatrix}
    \frac{v_r}{2}\cos\theta + \frac{v_l}{2}\cos\theta \\
    \frac{v_r}{2}\sin\theta + \frac{v_l}{2}\sin\theta \\
    \frac{v_r}{2r_b} - \frac{v_l}{2r_b} \\
    \left(\frac{1}{m} + \frac{r_b^2}{J}\right) C_1 v_l +
      \left(\frac{1}{m} - \frac{r_b^2}{J}\right) C_3 v_r +
      \left(\frac{1}{m} + \frac{r_b^2}{J}\right) C_2 V_l +
      \left(\frac{1}{m} - \frac{r_b^2}{J}\right) C_4 V_r \\
    \left(\frac{1}{m} - \frac{r_b^2}{J}\right) C_1 v_l +
      \left(\frac{1}{m} + \frac{r_b^2}{J}\right) C_3 v_r +
      \left(\frac{1}{m} - \frac{r_b^2}{J}\right) C_2 V_l +
      \left(\frac{1}{m} + \frac{r_b^2}{J}\right) C_4 V_r
  \end{bmatrix}
  \label{eq:ltv_diff_drive_f}
\end{equation}

As mentioned in chapter \ref{ch:nonlinear_control}, one can approximate a
nonlinear system via linearizations around points of interest in the state-space
and design controllers for those linearized subspaces. If we sample
linearization points progressively closer together, we converge on a control
policy for the original nonlinear system. Since the linear \gls{plant} being
controlled varies with time, its controller is called a linear time-varying
(LTV) controller.

If we use LQRs for the linearized subspaces, the nonlinear control policy will
also be locally optimal. We'll be taking this approach with a differential
drive. To create an LQR, we need to linearize equation
\eqref{eq:ltv_diff_drive_f}.
\begin{align*}
  \frac{\partial f(\mtx{x}, \mtx{u})}{\partial\mtx{x}} &=
  \begin{bmatrix}
    0 & 0 & -\frac{v_l + v_r}{2}\sin\theta & \frac{1}{2}\cos\theta &
      \frac{1}{2}\cos\theta \\
    0 & 0 & \frac{v_l + v_r}{2}\cos\theta & \frac{1}{2}\sin\theta &
      \frac{1}{2}\sin\theta \\
    0 & 0 & 0 & -\frac{1}{2r_b} & \frac{1}{2r_b} \\
    0 & 0 & 0 & \left(\frac{1}{m} + \frac{r_b^2}{J}\right) C_1 &
      \left(\frac{1}{m} - \frac{r_b^2}{J}\right) C_3 \\
    0 & 0 & 0 & \left(\frac{1}{m} - \frac{r_b^2}{J}\right) C_1 &
      \left(\frac{1}{m} + \frac{r_b^2}{J}\right) C_3
  \end{bmatrix} \\
  \frac{\partial f(\mtx{x}, \mtx{u})}{\partial\mtx{u}} &=
  \begin{bmatrix}
    0 & 0 \\
    0 & 0 \\
    0 & 0 \\
    \left(\frac{1}{m} + \frac{r_b^2}{J}\right) C_2 &
    \left(\frac{1}{m} - \frac{r_b^2}{J}\right) C_4 \\
    \left(\frac{1}{m} - \frac{r_b^2}{J}\right) C_2 &
    \left(\frac{1}{m} + \frac{r_b^2}{J}\right) C_4
  \end{bmatrix}
\end{align*}

Therefore,
\begin{theorem}[Linear time-varying differential drive state-space model]
  \begin{align*}
    \dot{\mtx{x}} &= \mtx{A}\mtx{x} + \mtx{B}\mtx{u} \\
    \mtx{y} &= \mtx{C}\mtx{x} + \mtx{D}\mtx{u}
  \end{align*}
  \begin{equation*}
    \begin{array}{ccc}
      \mtx{x} =
      \begin{bmatrix}
        x & y & \theta & v_l & v_r
      \end{bmatrix}^T &
      \mtx{y} =
      \begin{bmatrix}
        \theta & v_l & v_r
      \end{bmatrix}^T &
      \mtx{u} =
      \begin{bmatrix}
        V_l & V_r
      \end{bmatrix}^T
    \end{array}
  \end{equation*}

  \begin{equation}
    \begin{array}{ll}
      \mtx{A} =
      \begin{bmatrix}
        0 & 0 & -vs & \frac{1}{2}c & \frac{1}{2}c \\
        0 & 0 & vc & \frac{1}{2}s & \frac{1}{2}s \\
        0 & 0 & 0 & -\frac{1}{2r_b} & \frac{1}{2r_b} \\
        0 & 0 & 0 & \left(\frac{1}{m} + \frac{r_b^2}{J}\right) C_1 &
          \left(\frac{1}{m} - \frac{r_b^2}{J}\right) C_3 \\
        0 & 0 & 0 & \left(\frac{1}{m} - \frac{r_b^2}{J}\right) C_1 &
          \left(\frac{1}{m} + \frac{r_b^2}{J}\right) C_3
      \end{bmatrix} &
      \mtx{B} =
      \begin{bmatrix}
        0 & 0 \\
        0 & 0 \\
        0 & 0 \\
        \left(\frac{1}{m} + \frac{r_b^2}{J}\right) C_2 &
        \left(\frac{1}{m} - \frac{r_b^2}{J}\right) C_4 \\
        \left(\frac{1}{m} - \frac{r_b^2}{J}\right) C_2 &
        \left(\frac{1}{m} + \frac{r_b^2}{J}\right) C_4
      \end{bmatrix} \\
      \mtx{C} =
      \begin{bmatrix}
        0 & 0 & 1 & 0 & 0 \\
        0 & 0 & 0 & 1 & 0 \\
        0 & 0 & 0 & 0 & 1
      \end{bmatrix} &
      \mtx{D} = \mtx{0}_{3 \times 2}
    \end{array}
  \end{equation}

  where $v = \frac{v_l + v_r}{2}$, $c = \cos\theta$, $s = \sin\theta$,
  $C_1 = -\frac{G_l^2 K_t}{K_v R r^2}$, $C_2 = \frac{G_l K_t}{Rr}$,
  $C_3 = -\frac{G_r^2 K_t}{K_v R r^2}$, and $C_4 = \frac{G_r K_t}{Rr}$. The
  constants $C_1$ through $C_4$ are from the derivation in section
  \ref{sec:differential_drive}.

  The LQR gain should be recomputed around the current operating point regularly
  due to the high nonlinearity of this system. A less computationally expensive
  controller will be developed in later sections.
\end{theorem}

We can also use this in an extended Kalman filter as is since the measurement
model ($\mtx{y} = \mtx{C}\mtx{x} + \mtx{D}\mtx{u}$) is linear.

With this \gls{controller}, $\theta$ becomes uncontrollable at $v = 0$ due to
the $x$ and $y$ dynamics being equivalent to a unicycle; it can't change its
heading unless it's rolling (just like a bicycle). However, a differential
drive \textit{can} rotate in place. To address this controller's limitation at
$v = 0$, one can temporarily switch to an LQR of just $\theta$, $v_l$, and
$v_r$; linearize the controller around a slightly nonzero state; or plan a new
trajectory after the previous one completes that provides nonzero wheel
velocities to rotate the robot.

\subsection{Improving model accuracy}

Figures \ref{fig:ltv_diff_drive_nonrotated_firstorder_xy} and
\ref{fig:ltv_diff_drive_nonrotated_firstorder_response} demonstrate the
tracking behavior of the linearized differential drive controller.
\begin{bookfigure}
  \begin{minisvg}{2}{build/\chapterpath/ltv_diff_drive_nonrotated_firstorder_xy}
    \caption{Linear time-varying differential drive controller x-y plot
      (first-order)}
    \label{fig:ltv_diff_drive_nonrotated_firstorder_xy}
  \end{minisvg}
  \hfill
  \begin{minisvg}{2}{build/\chapterpath/ltv_diff_drive_nonrotated_firstorder_response}
    \caption{Linear time-varying differential drive controller response
      (first-order)}
    \label{fig:ltv_diff_drive_nonrotated_firstorder_response}
  \end{minisvg}
\end{bookfigure}

The linearized differential drive model doesn't track well because the
first-order linearization of $\mtx{A}$ doesn't capture the full heading
dynamics, making the \gls{model} update inaccurate. This linearization
inaccuracy is evident in the Hessian matrix (second partial derivative with
respect to the state vector) being nonzero.
\begin{equation*}
  \frac{\partial^2 f(\mtx{x}, \mtx{u})}{\partial\mtx{x}^2} =
  \begin{bmatrix}
    0 & 0 & -\frac{v_l + v_r}{2}\cos\theta & 0 & 0 \\
    0 & 0 & -\frac{v_l + v_r}{2}\sin\theta & 0 & 0 \\
    0 & 0 & 0 & 0 & 0 \\
    0 & 0 & 0 & 0 & 0 \\
    0 & 0 & 0 & 0 & 0
  \end{bmatrix}
\end{equation*}

The second-order Taylor series expansion of the \gls{model} around $\mtx{x}_0$
would be
\begin{equation*}
  f(\mtx{x}, \mtx{u}_0) \approx f(\mtx{x}_0, \mtx{u}_0) +
    \frac{\partial f(\mtx{x}, \mtx{u})}{\partial\mtx{x}}(\mtx{x} - \mtx{x}_0) +
    \frac{1}{2}\frac{\partial^2 f(\mtx{x}, \mtx{u})}{\partial\mtx{x}^2}
    (\mtx{x} - \mtx{x}_0)^2
\end{equation*}

To include higher-order dynamics in the linearized differential drive model
integration, we recommend using the fourth-order Runge-Kutta (RK4) integration
method on equation \eqref{eq:ltv_diff_drive_f}. See snippet
\ref{lst:runge-kutta} for an implementation of RK4.
\begin{code}{Python}{build/frccontrol/frccontrol/runge_kutta.py}
  \caption{Fourth-order Runge-Kutta integration in Python}
  \label{lst:runge-kutta}
\end{code}

Figures \ref{fig:ltv_diff_drive_nonrotated_xy} and
\ref{fig:ltv_diff_drive_nonrotated_response} show a simulation using RK4
instead of the first-order \gls{model}.
\begin{bookfigure}
  \begin{minisvg}{2}{build/\chapterpath/ltv_diff_drive_nonrotated_xy}
    \caption{Linear time-varying differential drive controller (global reference
        frame formulation) x-y plot}
    \label{fig:ltv_diff_drive_nonrotated_xy}
  \end{minisvg}
  \hfill
  \begin{minisvg}{2}{build/\chapterpath/ltv_diff_drive_nonrotated_response}
    \caption{Linear time-varying differential drive controller (global reference
        frame formulation) response}
    \label{fig:ltv_diff_drive_nonrotated_response}
  \end{minisvg}
\end{bookfigure}

\subsection{Cross track error controller}

Figures \ref{fig:ltv_diff_drive_nonrotated_xy} and
\ref{fig:ltv_diff_drive_nonrotated_response} show the tracking performance of
the linearized differential drive controller for a given trajectory. The
performance-effort trade-off can be tuned rather intuitively via the Q and R
gains. However, if the $x$ and $y$ error cost are too high, the $x$ and $y$
components of the controller will fight each other, and it will take longer to
converge to the path. This can be fixed by applying a counterclockwise rotation
matrix to the global tracking error to transform it into the robot's coordinate
frame.
\begin{equation*}
  \crdfrm{R}{\begin{bmatrix}
    e_x \\
    e_y \\
    e_\theta
  \end{bmatrix}} =
  \begin{bmatrix}
    \cos\theta & \sin\theta & 0 \\
    -\sin\theta & \cos\theta & 0 \\
    0 & 0 & 1
  \end{bmatrix}
  \crdfrm{G}{\begin{bmatrix}
    e_x \\
    e_y \\
    e_\theta
  \end{bmatrix}}
\end{equation*}

where the the superscript $R$ represents the robot's coordinate frame and the
superscript $G$ represents the global coordinate frame.

With this transformation, the $x$ and $y$ error cost in LQR penalize the error
ahead of the robot and cross-track error respectively instead of global pose
error. Since the cross-track error is always measured from the robot's
coordinate frame, the \gls{model} used to compute the LQR should be linearized
around $\theta = 0$ at all times.
\begin{align*}
  \mtx{A} &=
  \begin{bmatrix}
    0 & 0 & -\frac{v_l + v_r}{2}\sin 0 & \frac{1}{2}\cos 0 &
      \frac{1}{2}\cos 0 \\
    0 & 0 & \frac{v_l + v_r}{2}\cos 0 & \frac{1}{2}\sin 0 &
      \frac{1}{2}\sin 0 \\
    0 & 0 & 0 & -\frac{1}{2r_b} & \frac{1}{2r_b} \\
    0 & 0 & 0 & \left(\frac{1}{m} + \frac{r_b^2}{J}\right) C_1 &
      \left(\frac{1}{m} - \frac{r_b^2}{J}\right) C_3 \\
    0 & 0 & 0 & \left(\frac{1}{m} - \frac{r_b^2}{J}\right) C_1 &
      \left(\frac{1}{m} + \frac{r_b^2}{J}\right) C_3
  \end{bmatrix} \\
  \mtx{A} &=
  \begin{bmatrix}
    0 & 0 & 0 & \frac{1}{2} & \frac{1}{2} \\
    0 & 0 & \frac{v_l + v_r}{2} & 0 & 0 \\
    0 & 0 & 0 & -\frac{1}{2r_b} & \frac{1}{2r_b} \\
    0 & 0 & 0 & \left(\frac{1}{m} + \frac{r_b^2}{J}\right) C_1 &
      \left(\frac{1}{m} - \frac{r_b^2}{J}\right) C_3 \\
    0 & 0 & 0 & \left(\frac{1}{m} - \frac{r_b^2}{J}\right) C_1 &
      \left(\frac{1}{m} + \frac{r_b^2}{J}\right) C_3
  \end{bmatrix}
\end{align*}

This \gls{model} results in figures \ref{fig:ltv_diff_drive_exact_xy} and
\ref{fig:ltv_diff_drive_exact_response}, which show slightly better
tracking performance than the previous formulation.
\begin{bookfigure}
  \begin{minisvg}{2}{build/\chapterpath/ltv_diff_drive_exact_xy}
    \caption{Linear time-varying differential drive controller x-y plot}
    \label{fig:ltv_diff_drive_exact_xy}
  \end{minisvg}
  \hfill
  \begin{minisvg}{2}{build/\chapterpath/ltv_diff_drive_exact_response}
    \caption{Linear time-varying differential drive controller response}
    \label{fig:ltv_diff_drive_exact_response}
  \end{minisvg}
\end{bookfigure}

\subsection{Explicit time-varying control law}

Another downside of this controller is that the user must generate controller
gains for every state they visit in the state-space (an implicit control law).

A possible solution to this is fitting a vector function of the states to the
linearized differential drive controller gains. Fortunately, the formulation of
the controller dealing with cross-track error only requires linearization around
different values of $v$ rather than $v$ and $\theta$; only a two-dimensional
function is needed rather than a three-dimensional one. See figures
\ref{fig:ltv_diff_drive_lqr_fit_0} through
\ref{fig:ltv_diff_drive_lqr_fit_4} for plots of each controller gain for
a range of velocities.
\begin{bookfigure}
  \begin{minisvg}{2}{build/\chapterpath/ltv_diff_drive_lqr_fit_0}
    \caption{Linear time-varying differential drive controller LQR gain
      regression fit ($x$)}
    \label{fig:ltv_diff_drive_lqr_fit_0}
  \end{minisvg}
  \hfill
  \begin{minisvg}{2}{build/\chapterpath/ltv_diff_drive_lqr_fit_1}
    \caption{Linear time-varying differential drive controller LQR gain fit
      regression fit ($y$)}
  \end{minisvg}
  \hfill
  \begin{minisvg}{2}{build/\chapterpath/ltv_diff_drive_lqr_fit_2}
    \caption{Linear time-varying differential drive controller LQR gain
      regression fit ($\theta$)}
  \end{minisvg}
  \hfill
  \begin{minisvg}{2}{build/\chapterpath/ltv_diff_drive_lqr_fit_3}
    \caption{Linear time-varying differential drive controller LQR gain
      regression fit ($v_l$)}
  \end{minisvg}
  \hfill
  \begin{minisvg}{2}{build/\chapterpath/ltv_diff_drive_lqr_fit_4}
    \caption{Linear time-varying differential drive controller LQR gain
      regression fit ($v_r$)}
    \label{fig:ltv_diff_drive_lqr_fit_4}
  \end{minisvg}
\end{bookfigure}

With the exception of the $x$ gain plot, all functions are a variation of a
square root. $v = 0$ and $v = 1$ were used to scale each function to produce a
close approximation. The sign function\footnote{The sign function is defined as
follows:
\begin{equation*}
  \sgn(x) = \begin{cases}
    -1 & x < 0 \\
    0 & x = 0 \\
    1 & x > 0
  \end{cases}
\end{equation*}} is used for symmetry around the origin in the regression for
$y$.
\begin{theorem}[Linear time-varying differential drive controller]
  \begin{equation*}
    \begin{array}{ccc}
      \mtx{x} =
      \begin{bmatrix}
        x & y & \theta & v_l & v_r
      \end{bmatrix}^T &
      \mtx{y} =
      \begin{bmatrix}
        \theta & v_l & v_r
      \end{bmatrix}^T &
      \mtx{u} =
      \begin{bmatrix}
        V_l & V_r
      \end{bmatrix}^T
    \end{array}
  \end{equation*}

  The following $\mtx{A}$ and $\mtx{B}$ matrices of a continuous system are used
  to compute the LQR.

  \begin{equation}
    \begin{array}{ll}
      \mtx{A} =
      \begin{bmatrix}
        0 & 0 & 0 & \frac{1}{2} & \frac{1}{2} \\
        0 & 0 & v & 0 & 0 \\
        0 & 0 & 0 & -\frac{1}{2r_b} & \frac{1}{2r_b} \\
        0 & 0 & 0 & \left(\frac{1}{m} + \frac{r_b^2}{J}\right) C_1 &
          \left(\frac{1}{m} - \frac{r_b^2}{J}\right) C_3 \\
        0 & 0 & 0 & \left(\frac{1}{m} - \frac{r_b^2}{J}\right) C_1 &
          \left(\frac{1}{m} + \frac{r_b^2}{J}\right) C_3
      \end{bmatrix} &
      \mtx{B} =
      \begin{bmatrix}
        0 & 0 \\
        0 & 0 \\
        0 & 0 \\
        \left(\frac{1}{m} + \frac{r_b^2}{J}\right) C_2 &
        \left(\frac{1}{m} - \frac{r_b^2}{J}\right) C_4 \\
        \left(\frac{1}{m} - \frac{r_b^2}{J}\right) C_2 &
        \left(\frac{1}{m} + \frac{r_b^2}{J}\right) C_4
      \end{bmatrix}
    \end{array}
  \end{equation}

  where $v = \frac{v_l + v_r}{2}$, $C_1 = -\frac{G_l^2 K_t}{K_v R r^2}$,
  $C_2 = \frac{G_l K_t}{Rr}$, $C_3 = -\frac{G_r^2 K_t}{K_v R r^2}$, and
  $C_4 = \frac{G_r K_t}{Rr}$. The constants $C_1$ through $C_4$ are from the
  derivation in section \ref{sec:differential_drive}.

  The locally optimal controller for this model is
  $\mtx{u} = \mtx{K}(v) (\mtx{r} - \mtx{x})$ where

  \begin{align}
    \mtx{K}(v) &= \begin{bmatrix}
      k_x & k_y(v) \sgn(v) & k_\theta(v) & k_{1, 4}(v) & k_{2, 4}(v) \\
      k_x & -k_y(v) \sgn(v) & -k_\theta(v) & k_{2, 4}(v) & k_{1, 4}(v)
    \end{bmatrix} \\
    k_y(v) &= k_{y, 0} + (k_{y, 1} - k_{y, 0}) \sqrt{|v|} \\
    k_\theta(v) &= k_{\theta, 0} + (k_{\theta, 1} - k_{\theta, 0}) \sqrt{|v|} \\
    k_{1, 4}(v) &= k_{v^+, 0} + (k_{v^+, 1} - k_{v^+, 0}) \sqrt{|v|} \\
    k_{2, 4}(v) &= k_{v^-, 0} - (k_{v^+, 1} - k_{v^+, 0}) \sqrt{|v|}
  \end{align}

  Using $\mtx{K}$ computed via LQR at $v = 0$
  \begin{figurekey}
    \begin{tabular}{lllll}
      $k_x = \mtx{K}_{1, 1}$ &
      $k_{y, 0} = \mtx{K}_{1, 2}$ &
      $k_{\theta, 0} = \mtx{K}_{1, 3}$ &
      $k_{v^+, 0} = \mtx{K}_{1, 4}$ &
      $k_{v^-, 0} = \mtx{K}_{2, 4}$
    \end{tabular}
  \end{figurekey}

  Using $\mtx{K}$ computed via LQR at $v = 1$
  \begin{figurekey}
    \begin{tabular}{lll}
        $k_{y, 1} = \mtx{K}_{1, 2}$ &
        $k_{\theta, 1} = \mtx{K}_{1, 3}$ &
        $k_{v^+, 1} = \mtx{K}_{1, 4}$
    \end{tabular}
  \end{figurekey}
\end{theorem}

Figures \ref{fig:ltv_diff_drive_approx_xy} through
\ref{fig:ltv_diff_drive_exact_response2} show the responses of the exact
and approximate solutions are the same.
\begin{bookfigure}
  \begin{minisvg}{2}{build/\chapterpath/ltv_diff_drive_approx_xy}
    \caption{Linear time-varying differential drive controller x-y plot
      (approximate)}
    \label{fig:ltv_diff_drive_approx_xy}
  \end{minisvg}
  \hfill
  \begin{minisvg}{2}{build/\chapterpath/ltv_diff_drive_exact_xy}
    \caption{Linear time-varying differential drive controller x-y plot (exact)}
  \end{minisvg}
  \hfill
  \begin{minisvg}{2}{build/\chapterpath/ltv_diff_drive_approx_response}
    \caption{Linear time-varying differential drive controller response
      (approximate)}
  \end{minisvg}
  \hfill
  \begin{minisvg}{2}{build/\chapterpath/ltv_diff_drive_exact_response}
    \caption{Linear time-varying differential drive controller response (exact)}
    \label{fig:ltv_diff_drive_exact_response2}
  \end{minisvg}
\end{bookfigure}

\subsection{Nonlinear observer design}

\subsubsection{Encoder position augmentation}

Estimation of the global pose can be significantly improved if encoder position
measurements are used instead of velocity measurements. By augmenting the plant
with the line integral of each wheel's velocity over time, we can provide a
mapping from model states to position measurements. We can augment the linear
subspace of the model as follows.

Augment the matrix equation with position states $x_l$ and $x_r$, which have the
model equations $\dot{x}_l = v_l$ and $\dot{x}_r = v_r$. The matrix elements
corresponding to $v_l$ in the first equation and $v_r$ in the second equation
will be $1$, and the others will be $0$ since they don't appear, so
$\dot{x}_l = 1v_l + 0v_r + 0x_l + 0x_r + 0V_l + 0V_r$ and
$\dot{x}_r = 0v_l + 1v_r + 0x_l + 0x_r + 0V_l + 0V_r$. The existing rows will
have zeroes inserted where $x_l$ and $x_r$ are multiplied in.
\begin{align*}
  \dot{\begin{bmatrix}
    x_l \\
    x_r
  \end{bmatrix}} &=
  \begin{bmatrix}
    1 & 0 \\
    0 & 1
  \end{bmatrix}
  \begin{bmatrix}
    v_l \\
    v_r
  \end{bmatrix} +
  \begin{bmatrix}
    0 & 0 \\
    0 & 0
  \end{bmatrix}
  \begin{bmatrix}
    V_l \\
    V_r
  \end{bmatrix}
\end{align*}

This produces the following linear subspace over
$\mtx{x} = \begin{bmatrix}v_l & v_r & x_l & x_r\end{bmatrix}^T$.
\begin{equation}
  \begin{array}{ll}
    \mtx{A} =
    \begin{bmatrix}
      \left(\frac{1}{m} + \frac{r_b^2}{J}\right) C_1 &
        \left(\frac{1}{m} - \frac{r_b^2}{J}\right) C_3 & 0 & 0 \\
      \left(\frac{1}{m} - \frac{r_b^2}{J}\right) C_1 &
        \left(\frac{1}{m} + \frac{r_b^2}{J}\right) C_3 & 0 & 0 \\
      1 & 0 & 0 & 0 \\
      0 & 1 & 0 & 0
    \end{bmatrix} &
    \mtx{B} =
    \begin{bmatrix}
      \left(\frac{1}{m} + \frac{r_b^2}{J}\right) C_2 &
        \left(\frac{1}{m} - \frac{r_b^2}{J}\right) C_4 \\
      \left(\frac{1}{m} - \frac{r_b^2}{J}\right) C_2 &
        \left(\frac{1}{m} + \frac{r_b^2}{J}\right) C_4 \\
      0 & 0 \\
      0 & 0
    \end{bmatrix}
    \label{eq:diff_drive_linear_subspace}
  \end{array}
\end{equation}

The measurement model for the complete nonlinear model is now
$\mtx{y} = \begin{bmatrix}\theta & x_l & x_r\end{bmatrix}^T$ instead of
$\mtx{y} = \begin{bmatrix}\theta & v_l & v_r\end{bmatrix}^T$.

\subsubsection{U error estimation}

As per subsection \ref{subsec:input_error_estimation}, we will now augment the
\gls{model} so $u_{error}$ states are added to the \glspl{control input}.

The \gls{plant} and \gls{observer} augmentations should be performed before the
\gls{model} is \glslink{discretization}{discretized}. After the \gls{controller}
gain is computed with the unaugmented discrete \gls{model}, the controller may
be augmented. Therefore, the \gls{plant} and \gls{observer} augmentations assume
a continuous \gls{model} and the \gls{controller} augmentation assumes a
discrete \gls{controller}.

The three $u_{error}$ states we'll be adding are $u_{error,l}$, $u_{error,r}$,
and $u_{error,heading}$ for left voltage error, right voltage error, and heading
error respectively. The left and right wheel positions are filtered encoder
positions and are not adjusted for heading error. The turning angle computed
from the left and right wheel positions is adjusted by the gyroscope heading.
The heading $u_{error}$ state is the heading error between what the wheel
positions imply and the gyroscope measurement.

The full state is thus
$\mtx{x} = \begin{bmatrix}x & y & \theta & v_l & v_r & x_l & x_r & u_{error,l} &
  u_{error,r} & u_{error,heading}\end{bmatrix}^T$.

The complete nonlinear model is as follows. Let $v = \frac{v_l + v_r}{2}$. The
three $u_{error}$ states augment the linear subspace, so the nonlinear pose
dynamics are the same.
\begin{align}
  \dot{\begin{bmatrix}
    x \\
    y \\
    \theta
  \end{bmatrix}} &=
    \begin{bmatrix}
      v\cos\theta \\
      v\sin\theta \\
      \frac{v_r}{2r_b} - \frac{v_l}{2r_b}
    \end{bmatrix}
\end{align}

The left and right voltage error states should be mapped to the corresponding
velocity states, so the system matrix should be augmented with $\mtx{B}$.

The heading $u_{error}$ is measuring counterclockwise encoder understeer
relative to the gyroscope heading, so it should add to the left position and
subtract from the right position for clockwise correction of encoder positions.
That corresponds to the following input mapping vector.
\begin{equation*}
  \mtx{B}_{\theta} = \begin{bmatrix}
    0 \\
    0 \\
    1 \\
    -1
  \end{bmatrix}
\end{equation*}

Now we'll augment the linear system matrix horizontally to accomodate the
$u_{error}$ states.
\begin{equation}
  \dot{\begin{bmatrix}
    v_l \\
    v_r \\
    x_l \\
    x_r
  \end{bmatrix}} =
    \begin{bmatrix}
      \mtx{A} & \mtx{B} & \mtx{B}_{\theta}
    \end{bmatrix}
    \begin{bmatrix}
      v_l \\
      v_r \\
      x_l \\
      x_r \\
      u_{error,l} \\
      u_{error,r} \\
      u_{error,heading}
    \end{bmatrix} + \mtx{B}\mtx{u}
\end{equation}

$\mtx{A}$ and $\mtx{B}$ are the linear subspace from equation
\eqref{eq:diff_drive_linear_subspace}.

The $u_{error}$ states have no dynamics. The observer selects them to minimize
the difference between the expected and actual measurements.
\begin{equation}
  \dot{\begin{bmatrix}
    u_{error,l} \\
    u_{error,r} \\
    u_{error,heading}
  \end{bmatrix}} = \mtx{0}_{3 \times 1}
\end{equation}

The controller is augmented as follows.
\begin{equation}
  \begin{array}{ccc}
    \mtx{K}_{error} =
    \begin{bmatrix}
      1 & 0 & 0 \\
      0 & 1 & 0
    \end{bmatrix} &
    \mtx{K}_{aug} = \begin{bmatrix}
      \mtx{K} & \mtx{K}_{error}
    \end{bmatrix} &
    \mtx{r}_{aug} = \begin{bmatrix}
      \mtx{r} \\
      0 \\
      0 \\
      0
    \end{bmatrix}
  \end{array}
\end{equation}

This controller augmentation compensates for unmodeled dynamics like:
\begin{enumerate}
  \item Understeer caused by wheel friction inherent in skid-steer robots
  \item Battery voltage drop under load, which reduces the available control
    authority
\end{enumerate}
\begin{remark}
  The process noise for the voltage error states should be how much the voltage
  can be expected to drop. The heading error state should be the encoder
  \gls{model} uncertainty.
\end{remark}
