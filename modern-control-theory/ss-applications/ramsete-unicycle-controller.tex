\section{Ramsete unicycle controller}
\label{sec:ramsete_unicycle_controller}

Ramsete is a nonlinear time-varying feedback controller for unicycle
\glspl{model} that drives the \gls{model} to a desired \gls{pose} along a
two-dimensional trajectory. Why would we need a nonlinear control law in
addition to the linear ones we have used so far? If we use the original approach
with an LQR \gls{controller} for left and right position and velocity
\glspl{state}, the \gls{controller} only deals with the local \gls{pose}. If the
robot deviates from the path, there is no way for the \gls{controller} to
correct and the robot may not reach the desired global \gls{pose}. This is due
to multiple endpoints existing for the robot which have the same encoder path
arc lengths.

Instead of using wheel path arc lengths (which are in the robot's local
coordinate frame), nonlinear controllers like pure pursuit and Ramsete use
global pose. The \gls{controller} uses this extra information to guide a linear
\glslink{tracking}{reference tracker} like an LQR \gls{controller} back in by
adjusting the \glspl{reference} of the LQR \gls{controller}.

The paper \textit{Control of Wheeled Mobile Robots: An Experimental Overview}
describes a nonlinear controller for a wheeled vehicle with unicycle-like
kinematics; a global \gls{pose} consisting of $x$, $y$, and $\theta$; and a
desired \gls{pose} consisting of $x_d$, $y_d$, and $\theta_d$
\cite{bib:ctrl_wheeled_mobile_robots}. We'll call it Ramsete because that's the
acronym for the title of the book it came from in Italian (``Robotica Articolata
e Mobile per i SErvizi e le TEcnologie").

\subsection{Velocity and turning rate command derivation}

The \gls{state} tracking \gls{error} $\mat{e}$ in the vehicle's coordinate frame
is defined as
\begin{equation*}
  \begin{bmatrix}
    e_x \\
    e_y \\
    e_\theta
  \end{bmatrix} =
  \begin{bmatrix}
    \cos\theta & \sin\theta & 0 \\
    -\sin\theta & \cos\theta & 0 \\
    0 & 0 & 1
  \end{bmatrix}
  \begin{bmatrix}
    x_d - x \\
    y_d - y \\
    \theta_d - \theta
  \end{bmatrix}
\end{equation*}

where $e_x$ is the tracking \gls{error} in $x$, $e_y$ is the tracking
\gls{error} in $y$, and $e_\theta$ is the tracking \gls{error} in $\theta$.
The $3 \times 3$ matrix is a rotation matrix that transforms the \gls{error} in
the \gls{pose} (represented by $x_d - x$, $y_d - y$, and $\theta_d - \theta$)
from the global coordinate frame into the vehicle's coordinate frame.

We will use the following control laws $u_1$ and $u_2$ for velocity and turning
rate respectively.
\begin{equation}
  \begin{aligned}
    u_1 &= -k_1 e_x \\
    u_2 &= -k_3 e_\theta - k_2 v_d \sinc(e_\theta) e_y
  \end{aligned}
  \label{eq:ramsete_u}
\end{equation}

where $k_1$, $k_2$, and $k_3$ are time-varying gains and $\sinc(e_\theta)$ is
defined as $\frac{\sin{e_\theta}}{e_\theta}$. This choice of control law may
seem arbitrary, and that's because it is. The only requirement on our choice is
that there exist an associated Lyapunov candidate function that proves the
control law is globally asymptotically stable. We'll provide a sketch of a proof
in theorem \ref{thm:ramsete_lyapunov_stability}.

Our velocity and turning rate commands for the vehicle will use the following
nonlinear transformation of the inputs.
\begin{align*}
  v &= v_d\cos e_\theta - u_1 \\
  \omega &= \omega_d - u_2
\end{align*}

Substituting the control laws $u_1$ and $u_2$ into these equations gives
\begin{align*}
  v &= v_d\cos{e_\theta} + k_1 e_x \\
  \omega &= k_3 e_\theta + \omega_d + k_2 v_d \sinc(e_\theta) e_y
\end{align*}
\begin{theorem}
  \label{thm:ramsete_lyapunov_stability}

  Assuming that $v_d$ and $\omega_d$ are bounded with bounded derivatives, and
  that $v_d(t) \rightarrow 0$ or $\omega_d(t) \rightarrow 0$ when
  $t \rightarrow \infty$, the control laws in equation \eqref{eq:ramsete_u}
  globally asymptotically stabilize the origin $\mat{e} = 0$.

  Proof:

  To prove convergence, the paper previously mentioned uses the following
  Lyapunov function.
  \begin{equation*}
    V = \frac{k_2}{2}(e_x^2 + e_y^2) + \frac{e_\theta^2}{2}
  \end{equation*}

  where $k_2$ is a tuning constant, $e_x$ is the tracking error in $x$, $e_y$ is
  the tracking error in $y$, and $e_\theta$ is the tracking error in $\theta$.

  The time derivative along the solutions of the closed-loop \gls{system} is
  nonincreasing since
  \begin{equation*}
    \dot{V} = -k_1 k_2 e_x^2 - k_3 e_\theta^2 \leq 0
  \end{equation*}

  Thus, $\lVert e(t) \rVert$ is bounded, $\dot{V}(t)$ is uniformly continuous,
  and $V(t)$ tends to some limit value. Using the Barbalat lemma, $\dot{V}(t)$
  tends to zero \cite{bib:ctrl_wheeled_mobile_robots}.
\end{theorem}

\subsection{Nonlinear controller equations}

Let $k_2 = b$ and
$k = k_1(v_d, \omega_d) = k_3(v_d, \omega_d) = 2\zeta\sqrt{\omega_d^2 + bv_d^2}$.
\begin{theorem}[Ramsete unicycle controller]
  \begin{align}
    \begin{bmatrix}
      e_x \\
      e_y \\
      e_\theta
    \end{bmatrix} &=
    \begin{bmatrix}
      \cos\theta & \sin\theta & 0 \\
      -\sin\theta & \cos\theta & 0 \\
      0 & 0 & 1
    \end{bmatrix}
    \begin{bmatrix}
      x_d - x \\
      y_d - y \\
      \theta_d - \theta
    \end{bmatrix} \\
    v &= v_d \cos{e_\theta} + k e_x \\
    \omega &= \omega_d + k e_\theta + b v_d \sinc(e_\theta) e_y \\
    k &= 2\zeta\sqrt{\omega_d^2 + bv_d^2} \\
    \sinc(e_\theta) &= \frac{\sin{e_\theta}}{e_\theta}
  \end{align}
  \begin{figurekey}
    \begin{tabular}{llll}
      $v$ & velocity command & $v_d$ & desired velocity \\
      $\omega$ & turning rate command & $\omega_d$ & desired turning rate \\
      $x$ & actual $x$ position in global coordinate frame & $x_d$ &
        desired $x$ position \\
      $y$ & actual $y$ position in global coordinate frame & $y_d$ &
        desired $y$ position \\
      $\theta$ & actual angle in global coordinate frame & $\theta_d$ &
        desired angle
    \end{tabular}
  \end{figurekey}

  $b$ and $\zeta$ are tuning parameters where $b > 0$ and $\zeta \in (0, 1)$.
  Larger values of $b$ make convergence more aggressive (like a proportional
  term), and larger values of $\zeta$ provide more damping.
\end{theorem}

$v$ and $\omega$ should be the \glspl{reference} for a \gls{reference} tracker
for the drivetrain. This can be a linear controller (see subsection
\ref{subsec:linear_reference_tracker}). $x$, $y$, and $\theta$ are obtained via
a \gls{pose} estimator (see chapter \ref{ch:pose_estimation} for how to
implement one). The desired velocity, turning rate, and \gls{pose} can be varied
over time according to a desired trajectory.

Figures \ref{fig:ramsete_traj_xy} and \ref{fig:ramsete_traj_response} show the
tracking performance of Ramsete for a given trajectory.
\begin{bookfigure}
  \begin{minisvg}{2}{build/\chapterpath/ramsete_traj_xy}
    \caption{Ramsete nonlinear controller x-y plot}
    \label{fig:ramsete_traj_xy}
  \end{minisvg}
  \hfill
  \begin{minisvg}{2}{build/\chapterpath/ramsete_traj_response}
    \caption{Ramsete nonlinear controller response}
    \label{fig:ramsete_traj_response}
  \end{minisvg}
\end{bookfigure}

\subsection{Linear reference tracker}
\label{subsec:linear_reference_tracker}

We need a velocity \gls{reference} tracker for the nonlinear \gls{controller}'s
commands. Starting from equation \eqref{eq:diff_drive_ss_model}, we'll derive
two \glspl{model} for this purpose and compare their responses with a straight
profile and as part of a nonlinear trajectory follower.

\subsubsection{Left/right velocity reference tracker}
\begin{equation*}
  \begin{array}{ccc}
    \mat{x} =
    \begin{bmatrix}
      x_l \\
      v_l \\
      x_r \\
      v_r
    \end{bmatrix} &
    \mat{y} =
    \begin{bmatrix}
      x_l \\
      x_r
    \end{bmatrix} &
    \mat{u} =
    \begin{bmatrix}
      V_l \\
      V_r
    \end{bmatrix}
  \end{array}
\end{equation*}
\begin{equation*}
  \begin{array}{ll}
    \mat{A} =
    \begin{bmatrix}
      0 & 1 & 0 & 0 \\
      0 & \left(\frac{1}{m} + \frac{r_b^2}{J}\right) C_1 & 0 & \left(\frac{1}{m} - \frac{r_b^2}{J}\right) C_3 \\
      0 & 0 & 0 & 1 \\
      0 & \left(\frac{1}{m} - \frac{r_b^2}{J}\right) C_1 & 0 & \left(\frac{1}{m} + \frac{r_b^2}{J}\right) C_3
    \end{bmatrix} &
    \mat{B} =
    \begin{bmatrix}
      0 & 0 \\
      \left(\frac{1}{m} + \frac{r_b^2}{J}\right) C_2 & \left(\frac{1}{m} - \frac{r_b^2}{J}\right) C_4 \\
      0 & 0 \\
      \left(\frac{1}{m} - \frac{r_b^2}{J}\right) C_2 & \left(\frac{1}{m} + \frac{r_b^2}{J}\right) C_4
    \end{bmatrix} \\
    \mat{C} =
    \begin{bmatrix}
      1 & 0 & 0 & 0 \\
      0 & 0 & 1 & 0 \\
    \end{bmatrix} &
    \mat{D} = \mat{0}_{2 \times 2}
  \end{array}
\end{equation*}

where $C_1 = -\frac{G_l^2 K_t}{K_v R r^2}$, $C_2 = \frac{G_l K_t}{Rr}$,
$C_3 = -\frac{G_r^2 K_t}{K_v R r^2}$, and $C_4 = \frac{G_r K_t}{Rr}$.

To obtain a left/right velocity \gls{reference} tracker, we can just remove the
position \glspl{state} from the \gls{model}.
\begin{theorem}[Left/right velocity reference tracker]
  \label{thm:ramsete_decoupled_ref_tracker}
  \begin{equation*}
    \begin{array}{ccc}
      \mat{x} =
      \begin{bmatrix}
        v_l \\
        v_r
      \end{bmatrix} &
      \mat{y} =
      \begin{bmatrix}
        v_l \\
        v_r
      \end{bmatrix} &
      \mat{u} =
      \begin{bmatrix}
        V_l \\
        V_r
      \end{bmatrix}
    \end{array}
  \end{equation*}
  \begin{equation}
    \begin{array}{ll}
      \mat{A} =
      \begin{bmatrix}
        \left(\frac{1}{m} + \frac{r_b^2}{J}\right) C_1 &
          \left(\frac{1}{m} - \frac{r_b^2}{J}\right) C_3 \\
        \left(\frac{1}{m} - \frac{r_b^2}{J}\right) C_1 &
          \left(\frac{1}{m} + \frac{r_b^2}{J}\right) C_3
      \end{bmatrix} &
      \mat{B} =
      \begin{bmatrix}
        \left(\frac{1}{m} + \frac{r_b^2}{J}\right) C_2 &
          \left(\frac{1}{m} - \frac{r_b^2}{J}\right) C_4 \\
        \left(\frac{1}{m} - \frac{r_b^2}{J}\right) C_2 &
          \left(\frac{1}{m} + \frac{r_b^2}{J}\right) C_4
      \end{bmatrix} \\
      \mat{C} =
      \begin{bmatrix}
        1 & 0 \\
        0 & 1 \\
      \end{bmatrix} &
      \mat{D} = \mat{0}_{2 \times 2}
    \end{array}
  \end{equation}

  where $C_1 = -\frac{G_l^2 K_t}{K_v R r^2}$, $C_2 = \frac{G_l K_t}{Rr}$,
  $C_3 = -\frac{G_r^2 K_t}{K_v R r^2}$, and $C_4 = \frac{G_r K_t}{Rr}$.
\end{theorem}

See
\url{https://github.com/calcmogul/controls-engineering-in-frc/blob/main/modern-control-theory/ss-applications/ramsete_decoupled.py}
for an implementation.

\subsubsection{Linear/angular velocity reference tracker}

Since the Ramsete controller produces velocity and turning rate commands, it
would be more convenient if the \glspl{state} are velocity and turning rate
instead of left and right wheel velocity. We'll create a second model that has
velocity and angular velocity states and see how well it performs.
\begin{align*}
  \dot{v}_l &= \left(\frac{1}{m} + \frac{r_b^2}{J}\right)
    \left(C_1 v_l + C_2 V_l\right) +
    \left(\frac{1}{m} - \frac{r_b^2}{J}\right) \left(C_3 v_r + C_4 V_r\right) \\
  \dot{v}_r &= \left(\frac{1}{m} - \frac{r_b^2}{J}\right)
    \left(C_1 v_l + C_2 V_l\right) +
    \left(\frac{1}{m} + \frac{r_b^2}{J}\right) \left(C_3 v_r + C_4 V_r\right)
\end{align*}

Substitute in equations \eqref{eq:diff_vl} and \eqref{eq:diff_vr}.
\begin{align}
  \dot{v}_c + \dot{\omega} r_b &= \left(\frac{1}{m} + \frac{r_b^2}{J}\right)
    \left(C_1(v_c - \omega r_b) + C_2 V_l\right) +
    \left(\frac{1}{m} - \frac{r_b^2}{J}\right) \left(C_3(v_c + \omega r_b) +
      C_4 V_r\right) \label{eq:diff_vc_l} \\
  \dot{v}_c - \dot{\omega} r_b &= \left(\frac{1}{m} - \frac{r_b^2}{J}\right)
    \left(C_1(v_c - \omega r_b) + C_2 V_l\right) +
    \left(\frac{1}{m} + \frac{r_b^2}{J}\right) \left(C_3(v_c + \omega r_b) +
      C_4 V_r\right) \label{eq:diff_vc_r}
\end{align}

Now, we need to solve for $\dot{v}_c$ and $\dot{\omega}$. First, we'll add
equation \eqref{eq:diff_vc_l} to equation \eqref{eq:diff_vc_r}.
\begin{align*}
  2\dot{v}_c &= \frac{2}{m} \left(C_1(v_c - \omega r_b) + C_2 V_l\right) +
    \frac{2}{m} \left(C_3(v_c + \omega r_b) + C_4 V_r\right) \\
  \dot{v}_c &= \frac{1}{m} \left(C_1(v_c - \omega r_b) + C_2 V_l\right) +
    \frac{1}{m} \left(C_3(v_c + \omega r_b) + C_4 V_r\right) \\
  \dot{v}_c &= \frac{1}{m} (C_1 + C_3) v_c +
    \frac{1}{m} (-C_1 + C_3) \omega r_b + \frac{1}{m} C_2 V_l +
    \frac{1}{m} C_4 V_r \\
  \dot{v}_c &= \frac{1}{m} (C_1 + C_3) v_c + \frac{r_b}{m} (-C_1 + C_3) \omega +
    \frac{1}{m} C_2 V_l + \frac{1}{m} C_4 V_r
\end{align*}

Next, we'll subtract equation \eqref{eq:diff_vc_r} from equation
\eqref{eq:diff_vc_l}.
\begin{align*}
  2\dot{\omega} r_b &= \frac{2r_b^2}{J}
    \left(C_1(v_c - \omega r_b) + C_2 V_l\right) -
    \frac{2r_b^2}{J} \left(C_3(v_c + \omega r_b) + C_4 V_r\right) \\
  \dot{\omega} &= \frac{r_b}{J} \left(C_1(v_c - \omega r_b) + C_2 V_l\right) -
    \frac{r_b}{J} \left(C_3(v_c + \omega r_b) + C_4 V_r\right) \\
  \dot{\omega} &= \frac{r_b}{J} (C_1 - C_3) v_c +
    \frac{r_b}{J} (-C_1 - C_3) \omega r_b + \frac{r_b}{J} C_2 V_l -
    \frac{r_b}{J} C_4 V_r \\
  \dot{\omega} &= \frac{r_b}{J} (C_1 - C_3) v_c +
    \frac{r_b^2}{J} (-C_1 - C_3) \omega + \frac{r_b}{J} C_2 V_l -
    \frac{r_b}{J} C_4 V_r
\end{align*}

Now, just convert the two equations to state-space notation.
\begin{theorem}[Linear/angular velocity reference tracker]
  \label{thm:ramsete_coupled_ref_tracker}
  \begin{equation*}
    \begin{array}{ccc}
      \mat{x} =
      \begin{bmatrix}
        v \\
        \omega
      \end{bmatrix} &
      \mat{y} =
      \begin{bmatrix}
        v_l \\
        v_r
      \end{bmatrix} &
      \mat{u} =
      \begin{bmatrix}
        V_l \\
        V_r
      \end{bmatrix}
    \end{array}
  \end{equation*}
  \begin{equation}
    \begin{array}{ll}
      \mat{A} =
      \begin{bmatrix}
        \frac{1}{m}(C_1 + C_3) & \frac{r_b}{m}(-C_1 + C_3) \\
        \frac{r_b}{J}(C_1 - C_3) & \frac{r_b^2}{J}(-C_1 - C_3)
      \end{bmatrix} &
      \mat{B} =
      \begin{bmatrix}
        \frac{1}{m}C_2 & \frac{1}{m}C_4 \\
        \frac{r_b}{J}C_2 & -\frac{r_b}{J}C_4
      \end{bmatrix} \\
      \mat{C} =
      \begin{bmatrix}
        1 & -r_b \\
        1 & r_b
      \end{bmatrix} &
      \mat{D} = \mat{0}_{2 \times 2}
    \end{array}
  \end{equation}

  where $C_1 = -\frac{G_l^2 K_t}{K_v R r^2}$, $C_2 = \frac{G_l K_t}{Rr}$,
  $C_3 = -\frac{G_r^2 K_t}{K_v R r^2}$, and $C_4 = \frac{G_r K_t}{Rr}$.
\end{theorem}

See
\url{https://github.com/calcmogul/controls-engineering-in-frc/blob/main/modern-control-theory/ss-applications/ramsete_coupled.py}
for an implementation.

\subsubsection{Reference tracker comparison}

Figures \ref{fig:ramsete_decoupled_vel_lqr_profile} and
\ref{fig:ramsete_coupled_vel_lqr_profile} shows how well each \gls{reference}
tracker performs.
\begin{bookfigure}
  \begin{minisvg}{2}{build/\chapterpath/ramsete_decoupled_vel_lqr_profile}
    \caption{Left/right velocity reference tracker response to a motion profile}
    \label{fig:ramsete_decoupled_vel_lqr_profile}
  \end{minisvg}
  \hfill
  \begin{minisvg}{2}{build/\chapterpath/ramsete_coupled_vel_lqr_profile}
    \caption{Linear/angular velocity reference tracker response to a motion
      profile}
    \label{fig:ramsete_coupled_vel_lqr_profile}
  \end{minisvg}
\end{bookfigure}

For a simple s-curve motion profile, they behave similarly.

Figure \ref{fig:ramsete_decoupled_response} demonstrates the Ramsete controller
with the left/right velocity \gls{reference} tracker for a typical FRC
drivetrain with the \gls{reference} tracking behavior shown in figure
\ref{fig:ramsete_decoupled_vel_lqr_response} and figure
\ref{fig:ramsete_coupled_response} demonstrates the Ramsete controller with the
velocity / angular velocity \gls{reference} tracker for a typical FRC drivetrain
with the \gls{reference} tracking behavior shown in figure
\ref{fig:ramsete_coupled_vel_lqr_response}.
\begin{bookfigure}
  \begin{minisvg}{2}{build/\chapterpath/ramsete_decoupled_response}
    \caption{Ramsete controller response with left/right velocity reference
      tracker ($b = 2$, $\zeta = 0.7$)}
    \label{fig:ramsete_decoupled_response}
  \end{minisvg}
  \hfill
  \begin{minisvg}{2}{build/\chapterpath/ramsete_decoupled_vel_lqr_response}
    \caption{Ramsete controller's left/right velocity reference tracker
      response}
    \label{fig:ramsete_decoupled_vel_lqr_response}
  \end{minisvg} \\
  \begin{minisvg}{2}{build/\chapterpath/ramsete_coupled_response}
    \caption{Ramsete controller response with velocity / angular velocity
      reference tracker ($b = 2$, $\zeta = 0.7$)}
    \label{fig:ramsete_coupled_response}
  \end{minisvg}
  \hfill
  \begin{minisvg}{2}{build/\chapterpath/ramsete_coupled_vel_lqr_response}
    \caption{Ramsete controller's velocity / angular velocity reference tracker
      response}
    \label{fig:ramsete_coupled_vel_lqr_response}
  \end{minisvg}
\end{bookfigure}

The Ramsete \gls{controller} behaves relatively the same in each case, but the
left/right velocity \gls{reference} tracker tracks the \glspl{reference}
produced by the Ramsete \gls{controller} better and with smaller
\glspl{control effort} overall. This is likely due to the second
\gls{controller} having coupled dynamics. That is, the control inputs don't act
independently on the system \glspl{state}. In the coupled \gls{controller}, $v$
and $\omega$ require opposing control actions to reach their respective
\glspl{reference}, which results in them fighting each other. This hypothesis is
supported by the condition number of the coupled \gls{controller}'s
controllability matrix being larger ($2.692$ for coupled and $1.314$ for
decoupled).

Therefore, theorem \ref{thm:ramsete_decoupled_ref_tracker} is a superior
\gls{reference} tracker to theorem \ref{thm:ramsete_coupled_ref_tracker} due to
its decoupled dynamics, even though conversions are required between it and the
Ramsete \gls{controller}'s commands.
