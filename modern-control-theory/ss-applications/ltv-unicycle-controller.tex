\section{Linear time-varying unicycle controller}

One can also create a linear time-varying controller with a cascaded control
architecture like Ramsete. This section will derive a locally optimal
replacement for Ramsete.

The change in global pose for a unicycle is defined by the following three
equations.
\begin{align*}
  \dot{x} &= v\cos\theta \\
  \dot{y} &= v\sin\theta \\
  \dot{\theta} &= \omega
\end{align*}

Here's the model as a vector function where
$\mtx{x} = \begin{bmatrix} x & y & \theta \end{bmatrix}^T$ and
$\mtx{u} = \begin{bmatrix} v & \omega \end{bmatrix}^T$.
\begin{equation}
  f(\mtx{x}, \mtx{u}) =
  \begin{bmatrix}
    v\cos\theta \\
    v\sin\theta \\
    \omega
  \end{bmatrix}
\end{equation}

To create an LQR, we need to linearize this.
\begin{equation*}
  \begin{array}{ll}
    \frac{\partial f(\mtx{x}, \mtx{u})}{\partial\mtx{x}} =
    \begin{bmatrix}
      0 & 0 & -v\sin\theta \\
      0 & 0 & v\cos\theta \\
      0 & 0 & 0
    \end{bmatrix} &
    \frac{\partial f(\mtx{x}, \mtx{u})}{\partial\mtx{u}} =
    \begin{bmatrix}
      \cos\theta & 0 \\
      \sin\theta & 0 \\
      0 & 1
    \end{bmatrix}
  \end{array}
\end{equation*}

We're going to make a cross-track error controller, so we'll apply a clockwise
rotation matrix to the global tracking error to transform it into the robot's
coordinate frame. Since the cross-track error is always measured from the
robot's coordinate frame, the \gls{model} used to compute the LQR should be
linearized around $\theta = 0$ at all times.
\begin{equation*}
  \begin{array}{ll}
    \mtx{A} =
    \begin{bmatrix}
      0 & 0 & -v\sin 0 \\
      0 & 0 & v\cos 0 \\
      0 & 0 & 0
    \end{bmatrix} &
    \mtx{B} =
    \begin{bmatrix}
      \cos 0 & 0 \\
      \sin 0 & 0 \\
      0 & 1
    \end{bmatrix} \\
    \mtx{A} =
    \begin{bmatrix}
      0 & 0 & 0 \\
      0 & 0 & v \\
      0 & 0 & 0
    \end{bmatrix} &
    \mtx{B} =
    \begin{bmatrix}
      1 & 0 \\
      0 & 0 \\
      0 & 1
    \end{bmatrix}
  \end{array}
\end{equation*}

Therefore,
\begin{theorem}[Linear time-varying unicycle controller]
  \label{thm:linear_time-varying_unicycle_controller}

  \begin{align*}
    \dot{\mtx{x}} &= \mtx{A}\mtx{x} + \mtx{B}\mtx{u} \\
    \mtx{y} &= \mtx{C}\mtx{x} + \mtx{D}\mtx{u}
  \end{align*}
  \begin{equation*}
    \begin{array}{ccc}
      \mtx{x} =
      \begin{bmatrix}
        x & y & \theta
      \end{bmatrix}^T &
      \mtx{y} =
      \begin{bmatrix}
        \theta
      \end{bmatrix} &
      \mtx{u} =
      \begin{bmatrix}
        v & \omega
      \end{bmatrix}^T
    \end{array}
  \end{equation*}
  \begin{equation}
    \begin{array}{llll}
      \mtx{A} =
      \begin{bmatrix}
        0 & 0 & 0 \\
        0 & 0 & v \\
        0 & 0 & 0
      \end{bmatrix} &
      \mtx{B} =
      \begin{bmatrix}
        1 & 0 \\
        0 & 0 \\
        0 & 1
      \end{bmatrix} &
      \mtx{C} =
      \begin{bmatrix}
        0 & 0 & 1
      \end{bmatrix} &
      \mtx{D} = \mtx{0}_{1 \times 2}
    \end{array}
  \end{equation}
  \begin{equation}
    \mtx{u} = \mtx{K}
    \begin{bmatrix}
      \cos\theta & \sin\theta & 0 \\
      -\sin\theta & \cos\theta & 0 \\
      0 & 0 & 1
    \end{bmatrix}
    (\mtx{r} - \mtx{x})
  \end{equation}

  The controller gain $\mtx{K}$ from LQR should be recomputed from the model
  linearized around the current state during every timestep.
\end{theorem}

The controller in theorem \ref{thm:linear_time-varying_unicycle_controller}
results in figures \ref{fig:ltv_unicycle_traj_xy} and
\ref{fig:ltv_unicycle_traj_response}.
\begin{bookfigure}
  \begin{minisvg}{2}{build/\chapterpath/ltv_unicycle_traj_xy}
    \caption{Linear time-varying unicycle controller x-y plot}
    \label{fig:ltv_unicycle_traj_xy}
  \end{minisvg}
  \hfill
  \begin{minisvg}{2}{build/\chapterpath/ltv_unicycle_traj_response}
    \caption{Linear time-varying unicycle controller response}
    \label{fig:ltv_unicycle_traj_response}
  \end{minisvg}
\end{bookfigure}
