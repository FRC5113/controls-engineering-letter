\chapterimage{discrete-state-space-control.jpg}{Chaparral by Merril Apartments at UCSC}

\chapter{Discrete state-space control}
\label{ch:discrete_state-space_control}

The complex plane discussed so far deals with continuous \glspl{system}. In
decades past, \glspl{plant} and controllers were implemented using analog
electronics, which are continuous in nature. Nowadays, microprocessors can be
used to achieve cheaper, less complex controller designs. \Gls{discretization}
converts the continuous \gls{model} we've worked with so far from a differential
equation like
\begin{align}
  \dot{x} &= x - 3 \label{eq:differential_equ_example}
  \intertext{to a difference equation like}
  \frac{x_{k+1} - x_k}{\Delta T} &= x_k - 3 \nonumber \\
  x_{k+1} - x_k &= (x_k - 3) \Delta T \nonumber \\
  x_{k+1} &= x_k + (x_k - 3) \Delta T \label{eq:difference_equ_example}
\end{align}

where $x_k$ refers to the value of $x$ at the $k^{th}$ timestep. The difference
equation is run with some update period denoted by $T$, by $\Delta T$, or
sometimes sloppily by $dt$\footnote{The discretization of equation
\eqref{eq:differential_equ_example} to equation
\eqref{eq:difference_equ_example} uses the forward Euler discretization
method.}.

While higher order terms of a differential equation are derivatives of the
\gls{state} variable (e.g., $\ddot{x}$ in relation to equation
\eqref{eq:differential_equ_example}), higher order terms of a difference
equation are delayed copies of the \gls{state} variable (e.g., $x_{k-1}$ with
respect to $x_k$ in equation \eqref{eq:difference_equ_example}).

\renewcommand*{\chapterpath}{\partpath/discrete-state-space-control}
\section{Continuous to discrete pole mapping}

When a continuous system is discretized, its poles in the LHP map to the inside
of a unit circle. Table \ref{tab:c2d_mapping} contains a few common points and
figure \ref{fig:c2d_mapping} shows the mapping visually.
\begin{booktable}
  \begin{tabular}{|cc|}
    \hline
    \rowcolor{headingbg}
    \textbf{Continuous} & \textbf{Discrete} \\
    \hline
    $(0, 0)$ & $(1, 0)$ \\
    imaginary axis & edge of unit circle \\
    $(-\infty, 0)$ & $(0, 0)$ \\
    \hline
  \end{tabular}
  \caption{Mapping from continuous to discrete}
  \label{tab:c2d_mapping}
\end{booktable}
\begin{bookfigure}
  \begin{minisvg}{2}{build/figs/s_plane}
  \end{minisvg}
  \hfill
  \begin{minisvg}{2}{build/figs/z_plane}
  \end{minisvg}
  \caption{Mapping of complex plane from continuous (left) to discrete (right)}
  \label{fig:c2d_mapping}
\end{bookfigure}

\subsection{Discrete system stability}

Eigenvalues of a \gls{system} that are within the unit circle are stable, but
why is that? Let's consider a scalar equation $x_{k + 1} = ax_k$. $a < 1$ makes
$x_{k + 1}$ converge to zero. The same applies to a complex number like
$z = x + yi$ for $x_{k + 1} = zx_k$. If the magnitude of the complex number $z$
is less than one, $x_{k+1}$ will converge to zero. Values with a magnitude of
$1$ oscillate forever because $x_{k+1}$ never decays.

\subsection{Discrete system behavior}

As $\omega$ increases in $s = j\omega$, a pole in the discrete plane moves
around the perimeter of the unit circle. Once it hits $\frac{\omega_s}{2}$ (half
the sampling frequency) at $(-1, 0)$, the pole wraps around. This is due to
poles faster than the sample frequency folding down to below the sample
frequency (that is, higher frequency signals \textit{alias} to lower frequency
ones).

You may notice that poles can be placed at $(0, 0)$ in the discrete plane. This
is known as a deadbeat controller. An $\rm N^{th}$-order deadbeat controller
decays to the \gls{reference} in N timesteps. While this sounds great, there are
other considerations like \gls{control effort}, \gls{robustness}, and
\gls{noise immunity}.

If poles from $(1, 0)$ to $(0, 0)$ on the x-axis approach infinity, then what do
poles from $(-1, 0)$ to $(0, 0)$ represent? Them being faster than infinity
doesn't make sense. Poles in this location exhibit oscillatory behavior similar
to complex conjugate pairs. See figures \ref{fig:continuous_oscillations_1p} and
\ref{fig:discrete_oscillations_2p}. The jaggedness of these signals is due to
the frequency of the \gls{system} dynamics being above the Nyquist frequency
(twice the sample frequency). The \glslink{discretization}{discretized} signal
doesn't have enough samples to reconstruct the continuous \gls{system}'s
dynamics.
\begin{bookfigure}
  \begin{minisvg}{2}{build/\chapterpath/z_oscillations_1p}
    \caption{Single poles in various locations in discrete plane}
    \label{fig:continuous_oscillations_1p}
  \end{minisvg}
  \hfill
  \begin{minisvg}{2}{build/\chapterpath/z_oscillations_2p}
    \caption{Complex conjugate poles in various locations in discrete plane}
    \label{fig:discrete_oscillations_2p}
  \end{minisvg}
\end{bookfigure}

\subsection{Nyquist frequency}
\index{digital signal processing!Nyquist frequency}
\index{digital signal processing!aliasing}

To completely reconstruct a signal, the Nyquist-Shannon sampling theorem states
that it must be sampled at a frequency at least twice the maximum frequency it
contains. The highest frequency a given sample rate can capture is called the
Nyquist frequency, which is half the sample frequency. This is why recorded
audio is sampled at $44.1$ kHz. The maximum frequency a typical human can hear
is about $20$ kHz, so the Nyquist frequency is $20$ kHz and the minimum sampling
frequency is $40$ kHz. ($44.1$ kHz in particular was chosen for unrelated
historical reasons.)

Frequencies above the Nyquist frequency are folded down across it. The higher
frequency and the folded down lower frequency are said to alias each
other.\footnote{The aliases of a frequency $f$ can be expressed as
$f_{alias}(N) \stackrel{def}{=} |f - Nf_s|$. For example, if a $200$ Hz sine
wave is sampled at $150$ Hz, the \gls{observer} will see a $50$ Hz signal
instead of a $200$ Hz one.} Figure \ref{fig:c2d_aliasing} demonstrates
aliasing.
\begin{svg}{build/\chapterpath/aliasing}
  \caption{The original signal is a $1.5$ Hz sine wave, which means its Nyquist
    frequency is $1.5$ Hz. The signal is being sampled at $2$ Hz, so the aliased
    signal is a $0.5$ Hz sine wave.}
    \label{fig:c2d_aliasing}
\end{svg}

The effect of these high-frequency aliases can be reduced with a low-pass filter
(called an anti-aliasing filter in this application).

\section{Sample delay}

Implementing a discrete control system is easier than implementing a continuous
one, but \gls{discretization} has drawbacks. A microcontroller updates the
system input in discrete intervals of duration $T$; it's held constant between
updates. This introduces an average sample delay of $\frac{T}{2}$. Large delays
can make a stable controller in the continuous domain become unstable in the
discrete domain. Here are a few ways to combat this.
\begin{itemize}
  \item Run the controller with a high sample rate.
  \item Designing the controller in the analog domain with enough
    \gls{phase margin} to compensate for any phase loss that occurs as part of
    \gls{discretization}.
  \item Convert the \gls{plant} to the digital domain and design the controller
    completely in the digital domain.
\end{itemize}

\section{Effects of discretization on controller performance}

Running a feedback controller at a faster update rate doesn't always mean better
control. In fact, you may be using more computational resources than you need.
However, here are some reasons for running at a faster update rate.

Firstly, if you have a discrete \gls{model} of the \gls{system}, that
\gls{model} can more accurately approximate the underlying continuous
\gls{system}. Not all controllers use a \gls{model} though.

Secondly, the controller can better handle fast \gls{system} dynamics. If the
\gls{system} can move from its initial state to the desired one in under $250$
ms, you obviously want to run the controller with a period less than $250$ ms.
When you reduce the sample period, you're making the discrete controller more
accurately reflect what the equivalent continuous controller would do
(controllers built from analog circuit components like op-amps are continuous).

Running at a lower sample rate only causes problems if you don't take into
account the response time of your \gls{system}. Some \glspl{system} like heaters
have \glspl{output} that change on the order of minutes. Running a control loop
at $1$ kHz doesn't make sense for this because the \gls{plant} \gls{input} the
controller computes won't change much, if at all, in $1$ ms.

Figures \ref{fig:sampling_simulation_0.1}, \ref{fig:sampling_simulation_0.05},
and \ref{fig:sampling_simulation_0.01} show simulations of the same controller
for different sampling methods and sample rates, which have varying levels of
fidelity to the real \gls{system}.
\begin{bookfigure}
  \begin{minisvg}{2}{build/\chapterpath/sampling_simulation_010}
    \caption{Sampling methods for system simulation with $T = 0.1$ s}
    \label{fig:sampling_simulation_0.1}
  \end{minisvg}
  \hfill
  \begin{minisvg}{2}{build/\chapterpath/sampling_simulation_005}
    \caption{Sampling methods for system simulation with $T = 0.05$ s}
    \label{fig:sampling_simulation_0.05}
  \end{minisvg}
  \hfill
  \begin{minisvg}{2}{build/\chapterpath/sampling_simulation_004}
    \caption{Sampling methods for system simulation with $T = 0.01$ s}
    \label{fig:sampling_simulation_0.01}
  \end{minisvg}
\end{bookfigure}

Forward Euler is numerically unstable for low sample rates. The bilinear
transform is a significant improvement due to it being a second-order
approximation, but zero-order hold performs best due to the matrix exponential
including much higher orders (we'll cover the matrix exponential in the next
section).

\section{Taylor series}
\index{discretization!Taylor series}
\begin{remark}
  Watch the ``Taylor series" video from 3Blue1Brown's \textit{Essence of
  calculus} series (22 minutes) \cite{bib:calculus_taylor_series} for an
  explanation of how the Taylor series expansion works.
\end{remark}

The definition for the matrix exponential and the approximations below all use
the \textit{Taylor series expansion}. The Taylor series is a method of
approximating a function like $e^t$ via the summation of weighted polynomial
terms like $t^k$. $e^t$ has the following Taylor series around $t = 0$.
\begin{equation*}
  e^t = \sum_{n = 0}^\infty \frac{t^n}{n!}
\end{equation*}

where a finite upper bound on the number of terms produces an approximation of
$e^t$. As $n$ increases, the polynomial terms increase in power and the weights
by which they are multiplied decrease. For $e^t$ and some other functions, the
Taylor series expansion equals the original function for all values of $t$ as
the number of terms approaches infinity\footnote{Functions for which their
Taylor series expansion converges to and also equals it are called analytic
functions.}. Figure \ref{fig:taylor_series} shows the Taylor series expansion of
$e^t$ around $t = 0$ for a varying number of terms.
\begin{svg}{build/\chapterpath/taylor_series}
  \caption{Taylor series expansions of $e^t$ around $t = 0$ for $n$ terms}
  \label{fig:taylor_series}
\end{svg}

We'll expand the first few terms of the Taylor series expansion in equation
\eqref{eq:mat_exp} for $\mat{X} = \mat{A}T$ so we can compare it with other
methods.
\begin{equation*}
  \sum_{k=0}^3 \frac{1}{k!} (\mat{A}T)^k = \mat{I} + \mat{A}T +
    \frac{1}{2}\mat{A}^2T^2 + \frac{1}{6}\mat{A}^3T^3
\end{equation*}

Table \ref{tab:disc_approx_matrix} compares the Taylor series expansions of the
\gls{discretization} methods for the matrix case. These use a more complex
formula which we won't present here.
\begin{booktable}
  \begin{tabular}{|cll|}
    \hline
    \rowcolor{headingbg}
    \multicolumn{1}{|c}{\textbf{Method}} &
      \multicolumn{1}{c}{\textbf{Conversion}} &
      \multicolumn{1}{c|}{\textbf{Taylor series expansion}} \\
    \hline
    Zero-order hold &
      $\mat{A}_d = e^{\mat{A}_c T}$ &
      $\mat{A}_d = \mat{I} + \mat{A}_c T + \frac{1}{2}\mat{A}_c^2T^2 +
        \frac{1}{6}\mat{A}_c^3T^3 + \ldots$ \\
    Bilinear &
      $\mat{A}_d =
        \left(\mat{I} + \frac{1}{2}\mat{A}_c T\right)
        \left(\mat{I} - \frac{1}{2}\mat{A}_c T\right)^{-1}$ &
      $\mat{A}_d = \mat{I} + \mat{A}_c T + \frac{1}{2}\mat{A}_c^2T^2 +
        \frac{1}{4}\mat{A}_c^3T^3 + \ldots$ \\
    Forward Euler &
      $\mat{A}_d = \mat{I} + \mat{A}_c T$ &
      $\mat{A}_d = \mat{I} + \mat{A}_c T$ \\
    Reverse Euler &
      $\mat{A}_d = \left(\mat{I} - \mat{A}_c T\right)^{-1}$ &
      $\mat{A}_d =
        \mat{I} + \mat{A}_c T + \mat{A}_c^2T^2 + \mat{A}_c^3T^3 + \ldots$ \\
    \hline
  \end{tabular}
  \caption{Taylor series expansions of discretization methods (matrix case).
    The zero-order hold discretization method is exact.}
  \label{tab:disc_approx_matrix}
\end{booktable}

Each of them has different stability properties. The bilinear transform
preserves the (in)stability of the continuous time \gls{system}.

\section{Matrix exponential}
\index{discretization!matrix exponential}

The matrix exponential (and \gls{system} \gls{discretization} in general) is
typically solved with a computer. Python Control's \texttt{StateSpace.sample()}
with the ``zoh" method (the default) does this.

\begin{definition}[Matrix exponential]
  Let $\mtx{X}$ be an $n \times n$ matrix. The exponential of $\mtx{X}$ denoted
  by $e^{\mtx{X}}$ is the $n \times n$ matrix given by the following power
  series.

  \begin{equation}
    e^{\mtx{X}} = \sum_{k=0}^\infty \frac{1}{k!} \mtx{X}^k \label{eq:mat_exp}
  \end{equation}

  where $\mtx{X}^0$ is defined to be the identity matrix $\mtx{I}$ with the same
  dimensions as $\mtx{X}$.
\end{definition}

To understand why the matrix exponential is used in the \gls{discretization}
process, consider the set of differential equations
$\dot{\mtx{x}} = \mtx{A}\mtx{x}$ we use to describe \glspl{system}
(\glspl{system} also have a $\mtx{B}\mtx{u}$ term, but we'll ignore it for
clarity). The solution to this type of differential equation uses an
exponential. Since we are using matrices and vectors here, we use the matrix
exponential.

\begin{equation*}
  \mtx{x}(t) = e^{\mtx{A}t} \mtx{x}_0
\end{equation*}

where $\mtx{x}_0$ contains the initial conditions. If the initial \gls{state} is
the current system \gls{state}, then we can describe the \gls{system}'s
\gls{state} over time as

\begin{equation*}
  \mtx{x}_{k+1} = e^{\mtx{A}T} \mtx{x}_k
\end{equation*}

where $T$ is the time between samples $\mtx{x}_k$ and $\mtx{x}_{k+1}$.

\section{Zero-order hold for state-space}
\index{discretization!zero-order hold}

We're going to discretize the following continuous time state-space model
\begin{align*}
  \dot{\mat{x}} &= \mat{A}_c\mat{x} + \mat{B}_c\mat{u} + \mat{w} \\
  \mat{y} &= \mat{C}_c\mat{x} + \mat{D}_c\mat{u} + \mat{v}
\end{align*}

where $\mat{w}$ is the process noise, $\mat{v}$ is the measurement noise, and
both are zero-mean white noise sources with covariances of $\mat{Q}_c$ and
$\mat{R}_c$ respectively. $\mat{w}$ and $\mat{v}$ are defined as normally
distributed random variables.
\begin{align*}
  \mat{w} &\sim N(0, \mat{Q}_c) \\
  \mat{v} &\sim N(0, \mat{R}_c)
\end{align*}

\subsection{Taylor series}
\index{discretization!Taylor series}
\begin{remark}
  Watch the ``Taylor series" video from 3Blue1Brown's \textit{Essence of
  calculus} series (22 minutes) \cite{bib:calculus_taylor_series} for an
  explanation of how the Taylor series expansion works.
\end{remark}

The definition for the matrix exponential and the approximations below all use
the \textit{Taylor series expansion}. The Taylor series is a method of
approximating a function like $e^t$ via the summation of weighted polynomial
terms like $t^k$. $e^t$ has the following Taylor series around $t = 0$.
\begin{equation*}
  e^t = \sum_{n = 0}^\infty \frac{t^n}{n!}
\end{equation*}

where a finite upper bound on the number of terms produces an approximation of
$e^t$. As $n$ increases, the polynomial terms increase in power and the weights
by which they are multiplied decrease. For $e^t$ and some other functions, the
Taylor series expansion equals the original function for all values of $t$ as
the number of terms approaches infinity\footnote{Functions for which their
Taylor series expansion converges to and also equals it are called analytic
functions.}. Figure \ref{fig:taylor_series} shows the Taylor series expansion of
$e^t$ around $t = 0$ for a varying number of terms.
\begin{svg}{build/\chapterpath/taylor_series}
  \caption{Taylor series expansions of $e^t$ around $t = 0$ for $n$ terms}
  \label{fig:taylor_series}
\end{svg}

We'll expand the first few terms of the Taylor series expansion in equation
\eqref{eq:mat_exp} for $\mat{X} = \mat{A}T$ so we can compare it with other
methods.
\begin{equation*}
  \sum_{k=0}^3 \frac{1}{k!} (\mat{A}T)^k = \mat{I} + \mat{A}T +
    \frac{1}{2}\mat{A}^2T^2 + \frac{1}{6}\mat{A}^3T^3
\end{equation*}

Table \ref{tab:disc_approx_matrix} compares the Taylor series expansions of the
\gls{discretization} methods for the matrix case. These use a more complex
formula which we won't present here.
\begin{booktable}
  \begin{tabular}{|cll|}
    \hline
    \rowcolor{headingbg}
    \multicolumn{1}{|c}{\textbf{Method}} &
      \multicolumn{1}{c}{\textbf{Conversion to $\mat{A}_d$}} &
      \multicolumn{1}{c|}{\textbf{Taylor series expansion}} \\
    \hline
    Zero-order hold &
      $e^{\mat{A}_c T}$ &
      $\mat{I} + \mat{A}_c T + \frac{1}{2}\mat{A}_c^2T^2 +
        \frac{1}{6}\mat{A}_c^3T^3 + \ldots$ \\
    Bilinear &
      $\left(\mat{I} + \frac{1}{2}\mat{A}_c T\right)
        \left(\mat{I} - \frac{1}{2}\mat{A}_c T\right)^{-1}$ &
      $\mat{I} + \mat{A}_c T + \frac{1}{2}\mat{A}_c^2T^2 +
        \frac{1}{4}\mat{A}_c^3T^3 + \ldots$ \\
    Forward Euler &
      $\mat{I} + \mat{A}_c T$ &
      $\mat{I} + \mat{A}_c T$ \\
    Backward Euler &
      $\left(\mat{I} - \mat{A}_c T\right)^{-1}$ &
      $\mat{I} + \mat{A}_c T + \mat{A}_c^2T^2 + \mat{A}_c^3T^3 + \ldots$ \\
    \hline
  \end{tabular}
  \caption{Taylor series expansions of discretization methods (matrix case).
    The zero-order hold discretization method is exact.}
  \label{tab:disc_approx_matrix}
\end{booktable}

Each of them has different stability properties. The bilinear transform
preserves the (in)stability of the continuous time \gls{system}.

\subsection{Matrix exponential}
\index{discretization!matrix exponential}
\begin{remark}
  Watch the ``How (and why) to raise e to the power of a matrix" video from
  3Blue1Brown's \textit{Essence of linear algebra} series (27 minutes)
  \cite{bib:linalg_matrix_exp} for a visual introduction to the matrix
  exponential.
\end{remark}
\begin{definition}[Matrix exponential]
  Let $\mat{X}$ be an $n \times n$ matrix. The exponential of $\mat{X}$ denoted
  by $e^{\mat{X}}$ is the $n \times n$ matrix given by the following power
  series.
  \begin{equation}
    e^{\mat{X}} = \sum_{k=0}^\infty \frac{1}{k!} \mat{X}^k \label{eq:mat_exp}
  \end{equation}

  where $\mat{X}^0$ is defined to be the identity matrix $\mat{I}$ with the same
  dimensions as $\mat{X}$.
\end{definition}

To understand why the matrix exponential is used in the \gls{discretization}
process, consider the scalar differential equation $\dot{x} = ax$. The solution
to this type of differential equation uses an exponential.
\begin{align*}
  \dot{x} &= ax \\
  \frac{dx}{dt} &= ax(t) \\
  dx &= ax(t) \,dt \\
  \frac{1}{x(t)} \,dx &= a \,dt \\
  \int_0^t \frac{1}{x(t)} \,dx &= \int_0^t a \,dt \\
  \ln(x(t)) \rvert_0^t &= at \rvert_0^t \\
  \ln(x(t)) - \ln(x(0)) &= at - a \cdot 0 \\
  \ln(x(t)) - \ln(x_0) &= at \\
  \ln\left(\frac{x(t)}{x_0}\right) &= at \\
  \frac{x(t)}{x_0} &= e^{at} \\
  x(t) &= e^{at} x_0
\end{align*}

This solution generalizes via the matrix exponential to the set of differential
equations $\dot{\mat{x}} = \mat{A}\mat{x} + \mat{B}\mat{u}$ we use to describe
\glspl{system} (see section \ref{sec:deriv_zoh_ss} for a complete derivation).
\begin{equation*}
  \mat{x}(t) = e^{\mat{A}t} \mat{x}_0 +
    \mat{A}^{-1}(e^{\mat{A}t} - \mat{I})\mat{B} \mat{u}
\end{equation*}

where $\mat{x}_0$ contains the initial conditions and $\mat{u}$ is the constant
input from time $0$ to $t$. If the initial \gls{state} is the current system
\gls{state}, then we can describe the \gls{system}'s \gls{state} over time as
\begin{align*}
  \mat{x}_{k+1} &= e^{\mat{A}T} \mat{x}_k +
    \mat{A}^{-1}(e^{\mat{A}T} - \mat{I})\mat{B} \mat{u}_k
  \intertext{or more compactly,}
  \mat{x}_{k+1} &= \mat{A}_d\mat{x}_k + \mat{B}_d\mat{u}_k
\end{align*}

where $T$ is the time between samples $\mat{x}_k$ and $\mat{x}_{k+1}$. Theorem
\ref{thm:zoh_ss} has more efficient ways to compute $\mat{A}_d$ and $\mat{B}_d$.

\subsection{Definition}

The model can be \glslink{discretization}{discretized} as follows
\begin{align*}
  \mat{x}_{k+1} &= \mat{A}_d \mat{x}_k + \mat{B}_d \mat{u}_k + \mat{w}_k \\
   \mat{y}_k &= \mat{C}_d \mat{x}_k + \mat{D}_d \mat{u}_k + \mat{v}_k
\end{align*}

with covariances
\begin{align*}
  \mat{w}_k &\sim N(0, \mat{Q}_d) \\
  \mat{v}_k &\sim N(0, \mat{R}_d)
\end{align*}
\begin{theorem}[Zero-order hold for state-space]
  \label{thm:zoh_ss}
  \begin{align}
    \mat{A}_d &= e^{\mat{A}_c T} \\
    \mat{B}_d &= \int_0^T e^{\mat{A}_c \tau} d\tau \mat{B}_c =
      \mat{A}_c^{-1} (\mat{A}_d - \mat{I}) \mat{B}_c \\
    \mat{C}_d &= \mat{C}_c \\
    \mat{D}_d &= \mat{D}_c \\
    \mat{Q}_d &= \int_{\tau = 0}^{T} e^{\mat{A}_c\tau} \mat{Q}_c
      e^{\mat{A}_c\T\tau} d\tau \\
    \mat{R}_d &= \frac{1}{T}\mat{R}_c
  \end{align}

  where subscripts $c$ and $d$ denote matrices for the continuous or discrete
  \glspl{system} respectively, $T$ is the sample period of the discrete
  \gls{system}, and $e^{\mat{A}_c T}$ is the matrix exponential of
  $\mat{A}_c T$.

  $\mat{A}_d$ and $\mat{B}_d$ can be computed in one step as
  \begin{equation*}
    e^{
    \begin{bmatrix}
      \mat{A}_c & \mat{B}_c \\
      \mat{0} & \mat{0}
    \end{bmatrix}T} =
    \begin{bmatrix}
      \mat{A}_d & \mat{B}_d \\
      \mat{0} & \mat{I}
    \end{bmatrix}
  \end{equation*}

  and $\mat{Q}_d$ can be computed as
  \begin{equation*}
    \Phi = e^{
    \begin{bmatrix}
      -\mat{A}_c & \mat{Q}_c \\
      \mat{0} & \mat{A}_c\T
    \end{bmatrix}T} =
    \begin{bmatrix}
      -\mat{A}_d & \mat{A}_d^{-1} \mat{Q}_d \\
      \mat{0} & \mat{A}_d\T
    \end{bmatrix}
  \end{equation*}

  where $\mat{Q}_d = \Phi_{2,2}\T \Phi_{1,2}$ \cite{bib:integral_matrix_exp}.
\end{theorem}

See appendix \ref{sec:deriv_zoh_ss} for derivations.

To see why $\mat{R}_c$ is being divided by $T$, consider the discrete white
noise sequence $\mat{v}_k$ and the (non-physically realizable) continuous white
noise process $\mat{v}$. Whereas $\mat{R}_{d,k} = E[\mat{v}_k \mat{v}_k\T]$ is a
covariance matrix, $\mat{R}_c(t)$ defined by
$E[\mat{v}(t) \mat{v}\T(\tau)] = \mat{R}_c(t)\delta(t - \tau)$ is a spectral
density matrix (the Dirac function $\delta(t - \tau)$ has units of
$1/\text{sec}$). The covariance matrix $\mat{R}_c(t)\delta(t - \tau)$ has
infinite-valued elements. The discrete white noise sequence can be made to
approximate the continuous white noise process by shrinking the pulse lengths
($T$) and increasing their amplitude, such that
$\mat{R}_d \rightarrow \frac{1}{T}\mat{R}_c$.

That is, in the limit as $T \rightarrow 0$, the discrete noise sequence tends to
one of infinite-valued pulses of zero duration such that the area under the
``impulse" autocorrelation function is $\mat{R}_d T$. This is equal to the area
$\mat{R}_c$ under the continuous white noise impulse autocorrelation function.

\section{Discrete state-space notation}

Below is the discrete version of state-space notation.
\begin{definition}[Discrete state-space notation]%
  \index{state-space controllers!discrete open-loop}
  \begin{align}
    \mat{x}_{k+1} &= \mat{A}\mat{x}_k + \mat{B}\mat{u}_k \\
    \mat{y}_k &= \mat{C}\mat{x}_k + \mat{D}\mat{u}_k
  \end{align}
  \begin{figurekey}
    \begin{tabular}{llll}
      $\mat{A}$ & system matrix      & $\mat{x}$ & state vector \\
      $\mat{B}$ & input matrix       & $\mat{u}$ & input vector \\
      $\mat{C}$ & output matrix      & $\mat{y}$ & output vector \\
      $\mat{D}$ & feedthrough matrix &  &  \\
    \end{tabular}
  \end{figurekey}
\end{definition}

\section{Closed-loop controller}

With the \gls{control law} $\mtx{u} = \mtx{K}(\mtx{r} - \mtx{x})$, we can derive
the closed-loop state-space equations. We'll discuss where this
\gls{control law} comes from in subsection \ref{sec:lqr}.

First is the \gls{state} update equation. Substitute the \gls{control law} into
equation (\ref{eq:ss_ctrl_x}).

\begin{align}
  \dot{\mtx{x}} &= \mtx{A}\mtx{x} + \mtx{B}\mtx{K}(\mtx{r} - \mtx{x}) \nonumber
    \\
  \dot{\mtx{x}} &= \mtx{A}\mtx{x} + \mtx{B}\mtx{K}\mtx{r} -
    \mtx{B}\mtx{K}\mtx{x} \nonumber \\
  \dot{\mtx{x}} &= (\mtx{A} - \mtx{B}\mtx{K})\mtx{x} + \mtx{B}\mtx{K}\mtx{r}
\end{align}

Now for the \gls{output} equation. Substitute the \gls{control law} into
equation (\ref{eq:ss_ctrl_y}).

\begin{align}
  \mtx{y} &= \mtx{C}\mtx{x} + \mtx{D}(\mtx{K}(\mtx{r} - \mtx{x})) \nonumber \\
  \mtx{y} &= \mtx{C}\mtx{x} + \mtx{D}\mtx{K}\mtx{r} - \mtx{D}\mtx{K}\mtx{x}
    \nonumber \\
  \mtx{y} &= (\mtx{C} - \mtx{D}\mtx{K})\mtx{x} + \mtx{D}\mtx{K}\mtx{r}
\end{align}

Now, we'll do the same for the discrete \gls{system}. We'd like to know whether
the \gls{system} defined by equation (\ref{eq:ssz_ctrl_x}) operating with the
\gls{control law} $\mtx{u}_k = \mtx{K}(\mtx{r}_k - \mtx{x}_k)$ converges to the
\gls{reference} $\mtx{r}_k$.

\begin{align*}
  \mtx{x}_{k+1} &= \mtx{A}\mtx{x}_k + \mtx{B}\mtx{u}_k \\
  \mtx{x}_{k+1} &= \mtx{A}\mtx{x}_k + \mtx{B}(\mtx{K}(\mtx{r}_k - \mtx{x}_k)) \\
  \mtx{x}_{k+1} &= \mtx{A}\mtx{x}_k + \mtx{B}\mtx{K}\mtx{r}_k -
    \mtx{B}\mtx{K}\mtx{x}_k \\
  \mtx{x}_{k+1} &= \mtx{A}\mtx{x}_k - \mtx{B}\mtx{K}\mtx{x}_k +
    \mtx{B}\mtx{K}\mtx{r}_k \\
  \mtx{x}_{k+1} &= (\mtx{A} - \mtx{B}\mtx{K})\mtx{x}_k + \mtx{B}\mtx{K}\mtx{r}_k
\end{align*}

\begin{theorem}[Closed-loop state-space controller]
  \index{State-space controllers!closed-loop}

  \begin{align}
    \dot{\mtx{x}} &= (\mtx{A} - \mtx{B}\mtx{K})\mtx{x} + \mtx{B}\mtx{K}\mtx{r}
      \label{eq:s_ref_ctrl_x} \\
    \mtx{y} &= (\mtx{C} - \mtx{D}\mtx{K})\mtx{x} + \mtx{D}\mtx{K}\mtx{r}
      \label{eq:s_ref_ctrl_y}
  \end{align}

  \begin{align}
    \mtx{x}_{k+1} &= (\mtx{A} - \mtx{B}\mtx{K})\mtx{x}_k +
      \mtx{B}\mtx{K}\mtx{r}_k \label{eq:z_ref_ctrl_x} \\
    \mtx{y}_k &= (\mtx{C} - \mtx{D}\mtx{K})\mtx{x}_k + \mtx{D}\mtx{K}\mtx{r}_k
      \label{eq:z_ref_ctrl_y}
  \end{align}

  \begin{figurekey}
    \begin{tabular}{llll}
      $\mtx{A}$ & system matrix      & $\mtx{K}$ & controller gain matrix \\
      $\mtx{B}$ & input matrix       & $\mtx{x}$ & state vector \\
      $\mtx{C}$ & output matrix      & $\mtx{r}$ & \gls{reference} vector \\
      $\mtx{D}$ & feedthrough matrix & $\mtx{y}$ & output vector \\
    \end{tabular}
  \end{figurekey}
\end{theorem}

\begin{booktable}
  \begin{tabular}{|ll|ll|}
    \hline
    \rowcolor{headingbg}
    \textbf{Matrix} & \textbf{Rows $\times$ Columns} &
    \textbf{Matrix} & \textbf{Rows $\times$ Columns} \\
    \hline
    $\mtx{A}$ & states $\times$ states & $\mtx{x}$ & states $\times$ 1 \\
    $\mtx{B}$ & states $\times$ inputs & $\mtx{u}$ & inputs $\times$ 1 \\
    $\mtx{C}$ & outputs $\times$ states & $\mtx{y}$ & outputs $\times$ 1 \\
    $\mtx{D}$ & outputs $\times$ inputs & $\mtx{r}$ & states $\times$ 1 \\
    $\mtx{K}$ & inputs $\times$ states &  &  \\
    \hline
  \end{tabular}
  \caption{Controller matrix dimensions}
  \label{tab:ctrl_matrix_dims}
\end{booktable}

\index{Stability!eigenvalues}
Instead of commanding the \gls{system} to a \gls{state} using the vector
$\mtx{u}$ directly, we can now specify a vector of desired \glspl{state} through
$\mtx{r}$ and the \gls{controller} will choose values of $\mtx{u}$ for us over
time to make the \gls{system} converge to the \gls{reference}. For equation
(\ref{eq:s_ref_ctrl_x}) to reach steady-state, the eigenvalues of
$\mtx{A} - \mtx{B}\mtx{K}$ must be in the left-half plane. For equation
(\ref{eq:z_ref_ctrl_x}) to have a bounded output, the eigenvalues of
$\mtx{A} - \mtx{B}\mtx{K}$ must be within the unit circle.

The eigenvalues of $\mtx{A} - \mtx{B}\mtx{K}$ are the poles of the closed-loop
\gls{system}. Therefore, the rate of convergence and stability of the
closed-loop \gls{system} can be changed by moving the poles via the eigenvalues
of $\mtx{A} - \mtx{B}\mtx{K}$. $\mtx{A}$ and $\mtx{B}$ are inherent to the
\gls{system}, but $\mtx{K}$ can be chosen arbitrarily by the controller
designer.

\section{Pole placement}
\index{Controller design!pole placement}

This is the practice of placing the poles of a closed-loop \gls{system} directly
to produce a desired response. Python Control offers several pole placement
algorithms for generating controller or observer gains from a set of poles. In
general, pole placement should only be used if you know what you're doing. It's
much easier to let LQR place the poles for you, which we'll discuss next.

\section{Linear-quadratic regulator} \label{sec:lqr}
\index{controller design!linear-quadratic regulator}
\index{optimal control!linear-quadratic regulator}

\subsection{The intuition}

We can demonstrate the basic idea behind the linear-quadratic regulator with the
following flywheel model.
\begin{equation*}
  \dot{x} = ax + bu
\end{equation*}

where $a$ is a negative constant representing the back-EMF of the motor, $x$ is
the angular velocity, $b$ is a positive constant that maps the input voltage to
some change in angular velocity (angular acceleration), $u$ is the voltage
applied to the motor, and $\dot{x}$ is the angular acceleration. Discretized,
this equation would look like
\begin{equation*}
  x_{k+1} = a_d x + b_d u_k
\end{equation*}

If the angular velocity starts from zero and we apply a positive voltage, we'd
see the motor spin up to some constant speed following an exponential decay,
then stay at that speed. If we throw in the control law $u_k = k_p(r_k - x_k)$,
we can make the system converge to a desired state $r_k$ through proportional
feedback. In what manner can we pick the constant $k_p$ that balances getting to
the target angular velocity quickly with getting there efficiently (minimal
oscillations or excessive voltage)?

We can solve this problem with something called the linear-quadratic regulator.
We'll define the following cost function that includes the states and inputs:
\begin{equation*}
  J = \sum_{k=0}^\infty (Q(r_k - x_k)^2 + Ru_k^2)
\end{equation*}

We want to minimize this while obeying the constraint that the system follow our
flywheel dynamics $x_{k+1} = a_d x_k + b_d u_k$.

The cost is the sum of the squares of the error and the input for all time. If
the controller gain $k_p$ we pick in the control law $u_k = k_p(r_k - x_k)$ is
stable, the error $r_k - x_k$ and the input $u_k$ will both go to zero and give
us a finite cost. $Q$ and $R$ let us decide how much the error and input
contribute to the cost (we will require that $Q \geq 0$ and $R > 0$ for reasons
that will be clear shortly\footnote{Lets consider the boundary conditions on the
weights $Q$ and $R$. If we set $Q$ to zero, error doesn't contribute to the
cost, so the optimal solution is to not move. This minimizes the sum of the
inputs over time. If we let $R$ be zero, the input doesn't contribute to the
cost, so infinite inputs are allowed as they minimize the sum of the errors over
time. This isn't useful, so we require that the input be penalized with a
nonzero $R$.}). Penalizing error more will make the controller more aggressive,
while penalizing the input more will make the controller less aggressive. We
want to pick a $k_p$ that minimizes the cost.

There's a common trick for finding the value of a variable that minimizes a
function of that variable. We'll take the derivative (the slope) of the cost
function with respect to the input $u_k$, set the derivative to zero, then solve
for $u_k$. When the slope is zero, the function is at a minimum or maximum. Now,
the cost function we picked is quadratic. All the terms are strictly positive on
account of the squared variables and nonnegative weights, so our cost is
strictly positive and the quadratic function is concave up. The $u_k$ we found
is therefore a minimum.

The actual process of solving for $u_k$ is mathematically intensive and outside
the scope of this explanation (appendix \ref{ch:deriv_lqr} references a
derivation for those curious). The rest of this section will describe the more
general form of the linear-quadratic regulator and how to use it.

\subsection{The mathematical definition}

Instead of placing the poles of a closed-loop \gls{system} manually, the
linear-quadratic regulator (LQR) places the poles for us based on acceptable
relative \gls{error} and \gls{control effort} costs. This method of controller
design uses a quadratic function for the cost-to-go defined as the sum of the
\gls{error} and \gls{control effort} over time for the linear \gls{system}
$\mat{x}_{k+1} = \mat{A}\mat{x}_k + \mat{B}\mat{u}_k$.
\begin{equation*}
  J = \sum_{k=0}^\infty \left(\mat{x}_k\T\mat{Q}\mat{x}_k +
    \mat{u}_k\T\mat{R}\mat{u}_k\right)
\end{equation*}

where $J$ represents a trade-off between \gls{state} excursion and
\gls{control effort} with the weighting factors $\mat{Q}$ and $\mat{R}$. LQR is
a \gls{control law} $\mat{u}$ that minimizes the cost function. $\mat{Q}$ and
$\mat{R}$ slide the cost along a Pareto boundary between state tracking and
\gls{control effort} (see figure \ref{fig:pareto_boundary}). Pareto optimality
for this problem means that an improvement in state \gls{tracking} cannot be
obtained without using more \gls{control effort} to do so. Also, a reduction in
\gls{control effort} cannot be obtained without sacrificing state \gls{tracking}
performance. Pole placement, on the other hand, will have a cost anywhere on,
above, or to the right of the Pareto boundary (no cost can be inside the
boundary).
\begin{svg}{build/\chapterpath/pareto_boundary}
  \caption{Pareto boundary for LQR}
  \label{fig:pareto_boundary}
\end{svg}

The minimum of LQR's cost function is found by setting the derivative of the
cost function to zero and solving for the \gls{control law} $\mat{u}_k$.
However, matrix calculus is used instead of normal calculus to take the
derivative.

The feedback \gls{control law} that minimizes $J$ is shown in theorem
\ref{thm:linear-quadratic_regulator}.
\begin{theorem}[Linear-quadratic regulator]
  \label{thm:linear-quadratic_regulator}
  \begin{align}
    \min_{\mat{u}_k} &\sum\limits_{k=0}^\infty
      \left(\mat{x}_k\T\mat{Q}\mat{x}_k + \mat{u}_k\T\mat{R}\mat{u}_k\right)
      \nonumber \\
    \text{subject to } &\mat{x}_{k+1} = \mat{A}\mat{x}_k + \mat{B}\mat{u}_k
  \end{align}

  If the \gls{system} is controllable, the optimal control policy $\mat{u}_k^*$
  that drives all the \glspl{state} to zero is $-\mat{K}\mat{x}_k$. To converge
  to nonzero \glspl{state}, a \gls{reference} vector $\mat{r}_k$ can be added to
  the \gls{state} $\mat{x}_k$.
  \begin{equation}
    \mat{u}_k = \mat{K}(\mat{r}_k - \mat{x}_k)
  \end{equation}
\end{theorem}
\index{controller design!linear-quadratic regulator!definition}
\index{optimal control!linear-quadratic regulator!definition}

This means that optimal control can be achieved with simply a set of
proportional gains on all the \glspl{state}. To use the \gls{control law}, we
need knowledge of the full \gls{state} of the \gls{system}. That means we either
have to measure all our \glspl{state} directly or estimate those we do not
measure.

See appendix \ref{ch:deriv_lqr} for how $\mat{K}$ is calculated. If the result
is finite, the controller is guaranteed to be stable and
\glslink{robustness}{robust} with a \gls{phase margin} of 60 degrees
\cite{bib:lqr_phase_margin}.
\begin{remark}
  LQR design's $\mat{Q}$ and $\mat{R}$ matrices don't need \gls{discretization},
  but the $\mat{K}$ calculated for continuous time and discrete time
  \glspl{system} will be different. The discrete time gains approach the
  continuous time gains as the sample period tends to zero.
\end{remark}

\subsection{Bryson's rule}
\index{controller design!linear-quadratic regulator!Bryson's rule}
\index{optimal control!linear-quadratic regulator!Bryson's rule}

The next obvious question is what values to choose for $\mat{Q}$ and $\mat{R}$.
While this can be more of an art than a science, Bryson's rule provides a good
starting point. With Bryson's rule, the diagonals of the $\mat{Q}$ and $\mat{R}$
matrices are chosen based on the maximum acceptable value for each \gls{state}
and actuator. The nondiagonal elements are zero. The balance between $\mat{Q}$
and $\mat{R}$ can be slid along the Pareto boundary using a weighting factor
$\rho$.
\begin{equation*}
  J = \sum_0^\infty \left(\rho \left[
    \left(\frac{x_1}{x_{1,max}}\right)^2 + \ldots +
    \left(\frac{x_m}{x_{m,max}}\right)^2\right] + \left[
    \left(\frac{u_1}{u_{1,max}}\right)^2 + \ldots +
    \left(\frac{u_n}{u_{n,max}}\right)^2\right]\right)
\end{equation*}
\begin{equation*}
  \mat{Q} = \begin{bmatrix}
    \frac{\rho}{x_{1,max}^2} & 0 & \ldots & 0 \\
    0 & \frac{\rho}{x_{2,max}^2} & & \vdots \\
    \vdots & & \ddots & 0 \\
    0 & \ldots & 0 & \frac{\rho}{x_{m,max}^2}
  \end{bmatrix}
  \quad
  \mat{R} = \begin{bmatrix}
    \frac{1}{u_{1,max}^2} & 0 & \ldots & 0 \\
    0 & \frac{1}{u_{2,max}^2} & & \vdots \\
    \vdots & & \ddots & 0 \\
    0 & \ldots & 0 & \frac{1}{u_{n,max}^2}
  \end{bmatrix}
\end{equation*}

The index subscript denotes a row of the state or input vector. Small values of
$\rho$ penalize \gls{control effort} while large values of $\rho$ penalize
\gls{state} excursions. Large values would be chosen in applications like
fighter jets where performance is necessary. Spacecrafts would use small values
to conserve their limited fuel supply.

\subsection{Pole placement vs LQR}

This example uses the following continuous second-order \gls{model} for a CIM
motor (a DC brushed motor).
\begin{align*}
  \mat{A} = \begin{bmatrix}
    -\frac{b}{J} & \frac{K_t}{J} \\
    -\frac{K_e}{L} & -\frac{R}{L}
  \end{bmatrix}
  \quad
  \mat{B} = \begin{bmatrix}
    0 \\
    \frac{1}{L}
  \end{bmatrix}
  \quad
  \mat{C} = \begin{bmatrix}
    1 & 0
  \end{bmatrix}
  \quad
  \mat{D} = \begin{bmatrix}
    0
  \end{bmatrix}
\end{align*}

Figure \ref{fig:case_study_pp_lqr} shows the response using various discrete
pole placements and using LQR with the following cost matrices.
\begin{align*}
  \mat{Q} = \begin{bmatrix}
    \frac{1}{20^2} & 0 \\
    0 & 0
  \end{bmatrix}
  \quad
  \mat{R} = \begin{bmatrix}
    \frac{1}{12^2}
  \end{bmatrix}
\end{align*}

With Bryson's rule, this means an angular velocity tolerance of $20$ rad/s, an
infinite current tolerance (in other words, we don't care what the current
does), and a voltage tolerance of $12$ V.
\begin{svg}{build/\chapterpath/case_study_pp_lqr}
  \caption{Second-order CIM motor response with pole placement and LQR}
  \label{fig:case_study_pp_lqr}
\end{svg}

Notice with pole placement that as the current pole moves toward the origin, the
\gls{control effort} becomes more aggressive.

\section{Feedforward}

So far, we've used feedback control for \gls{reference} \gls{tracking} (making a
\gls{system}'s output follow a desired \gls{reference} signal). While this is
effective, it's a reactionary measure; the \gls{system} won't start applying
\gls{control effort} until the \gls{system} is already behind. If we could tell
the \gls{controller} about the desired movement and required input beforehand,
the \gls{system} could react quicker and the feedback \gls{controller} could do
less work. A \gls{controller} that feeds information forward into the
\gls{plant} like this is called a \gls{feedforward controller}.

A \gls{feedforward controller} injects information about the \gls{system}'s
dynamics (like a \gls{model} does) or the desired movement. The feedforward
handles parts of the control actions we already know must be applied to make a
\gls{system} track a \gls{reference}, then feedback compensates for what we do
not or cannot know about the \gls{system}'s behavior at runtime.

There are two types of feedforwards: model-based feedforward and feedforward for
unmodeled dynamics. The first solves a mathematical model of the system for the
inputs required to meet desired velocities and accelerations. The second
compensates for unmodeled forces or behaviors directly so the feedback
controller doesn't have to. Both types can facilitate simpler feedback
controllers; we'll cover examples of each.

\subsection{Plant inversion}
\label{subsec:plant_inversion}

\Gls{plant} inversion is a method of model-based feedforward for \gls{state}
feedback. It solves the \gls{plant} for the input that will make the \gls{plant}
track a desired state. This is called inversion because in a block diagram, the
inverted \gls{plant} feedforward and \gls{plant} cancel out to produce a unity
system from input to output.

While it can be an effective tool, the following should be kept in mind.
\begin{enumerate}
  \item Don't invert an unstable \gls{plant}. If the expected \gls{plant}
    doesn't match the real \gls{plant} exactly, the \gls{plant} inversion will
    still result in an unstable \gls{system}. Stabilize the \gls{plant} first
    with feedback, then inject an inversion.
  \item Don't invert a nonminimum phase system. The advice for pole-zero
    cancellation in subsection \ref{subsec:pole-zero_cancellation} applies here.
\end{enumerate}

\subsubsection{Necessary theorems}

The following theorem will be needed to derive the linear plant inversion
equation.
\begin{theorem}
  \label{thm:partial_xax}

  $\frac{\partial \mtx{x}^T\mtx{A}\mtx{x}}{\partial\mtx{x}} =
    2\mtx{A}\mtx{x}$ where $\mtx{A}$ is symmetric.
\end{theorem}

\subsubsection{Setup}

Let's start with the equation for the \gls{reference} dynamics
\begin{equation*}
  \mtx{r}_{k+1} = \mtx{A}\mtx{r}_k + \mtx{B}\mtx{u}_k
\end{equation*}

where $\mtx{u}_k$ is the feedforward input. Note that this feedforward equation
does not and should not take into account any feedback terms. We want to find
the optimal $\mtx{u}_k$ such that we minimize the \gls{tracking} error between
$\mtx{r}_{k+1}$ and $\mtx{r}_k$.
\begin{equation*}
  \mtx{r}_{k+1} - \mtx{A}\mtx{r}_k = \mtx{B}\mtx{u}_k
\end{equation*}

To solve for $\mtx{u}_k$, we need to take the inverse of the nonsquare matrix
$\mtx{B}$. This isn't possible, but we can find the pseudoinverse given some
constraints on the \gls{state} \gls{tracking} error and \gls{control effort}. To
find the optimal solution for these sorts of trade-offs, one can define a cost
function and attempt to minimize it. To do this, we'll first solve the
expression for $\mtx{0}$.
\begin{equation*}
  \mtx{0} = \mtx{B}\mtx{u}_k - (\mtx{r}_{k+1} - \mtx{A}\mtx{r}_k)
\end{equation*}

This expression will be the \gls{state} \gls{tracking} cost we use in the
following cost function as an $H_2$ norm.
\begin{equation*}
  \mtx{J} = (\mtx{B}\mtx{u}_k - (\mtx{r}_{k+1} - \mtx{A}\mtx{r}_k))^T
    (\mtx{B}\mtx{u}_k - (\mtx{r}_{k+1} - \mtx{A}\mtx{r}_k))
\end{equation*}

\subsubsection{Minimization}

Given theorem \ref{thm:partial_xax}, find the minimum of $\mtx{J}$ by taking the
partial derivative with respect to $\mtx{u}_k$ and setting the result to
$\mtx{0}$.
\begin{align*}
  \frac{\partial\mtx{J}}{\partial\mtx{u}_k} &= 2\mtx{B}^T
    (\mtx{B}\mtx{u}_k - (\mtx{r}_{k+1} - \mtx{A}\mtx{r}_k)) \\
  \mtx{0} &= 2\mtx{B}^T
    (\mtx{B}\mtx{u}_k - (\mtx{r}_{k+1} - \mtx{A}\mtx{r}_k)) \\
  \mtx{0} &= 2\mtx{B}^T\mtx{B}\mtx{u}_k -
    2\mtx{B}^T(\mtx{r}_{k+1} - \mtx{A}\mtx{r}_k) \\
  2\mtx{B}^T\mtx{B}\mtx{u}_k &=
    2\mtx{B}^T(\mtx{r}_{k+1} - \mtx{A}\mtx{r}_k) \\
  \mtx{B}^T\mtx{B}\mtx{u}_k &=
    \mtx{B}^T(\mtx{r}_{k+1} - \mtx{A}\mtx{r}_k) \\
  \mtx{u}_k &=
    (\mtx{B}^T\mtx{B})^{-1} \mtx{B}^T(\mtx{r}_{k+1} - \mtx{A}\mtx{r}_k)
\end{align*}

$(\mtx{B}^T\mtx{B})^{-1} \mtx{B}^T$ is the Moore-Penrose pseudoinverse of
$\mtx{B}$ denoted by $\mtx{B}^\dagger$.
\begin{theorem}[Linear plant inversion]
  \label{thm:linear_plant_inversion}

  Given the discrete model
  $\mtx{x}_{k+1} = \mtx{A}\mtx{x}_k + \mtx{B}\mtx{u}_k$, the plant inversion
  feedforward is
  \begin{equation}
    \mtx{u}_k = \mtx{B}^\dagger (\mtx{r}_{k+1} - \mtx{A}\mtx{r}_k)
  \end{equation}

  where $\mtx{B}^\dagger$ is the Moore-Penrose pseudoinverse of $\mtx{B}$,
  $\mtx{r}_{k+1}$ is the reference at the next timestep, and $\mtx{r}_k$ is the
  reference at the current timestep.
\end{theorem}
\index{feedforward!linear plant inversion}
\index{optimal control!linear plant inversion}

\subsubsection{Discussion}

Linear \gls{plant} inversion in theorem \ref{thm:linear_plant_inversion}
compensates for \gls{reference} dynamics that don't follow how the \gls{model}
inherently behaves. If they do follow the \gls{model}, the feedforward has
nothing to do as the \gls{model} already behaves in the desired manner. When
this occurs, $\mtx{r}_{k+1} - \mtx{A}\mtx{r}_k$ will return a zero vector.

For example, a constant \gls{reference} requires a feedforward that opposes
\gls{system} dynamics that would change the \gls{state} over time. If the
\gls{system} has no dynamics, then $\mtx{A} = \mtx{I}$ and thus
\begin{align*}
  \mtx{u}_k &= \mtx{B}_\dagger (\mtx{r}_{k+1} - \mtx{I}\mtx{r}_k) \\
  \mtx{u}_k &= \mtx{B}_\dagger (\mtx{r}_{k+1} - \mtx{r}_k)
\end{align*}

For a constant \gls{reference}, $\mtx{r}_{k+1} = \mtx{r}_k$.
\begin{align*}
  \mtx{u}_k &= \mtx{B}_\dagger (\mtx{r}_k - \mtx{r}_k) \\
  \mtx{u}_k &= \mtx{B}_\dagger (\mtx{0}) \\
  \mtx{u}_k &= \mtx{0}
\end{align*}

so no feedforward is required to hold a \gls{system} with no dynamics at a
constant \gls{reference}, as expected.

Figure \ref{fig:case_study_ff} shows \gls{plant} inversion applied to a
second-order CIM motor model. \Gls{plant} inversion accounts for the motor
back-EMF and eliminates steady-state error.
\begin{svg}{build/\chapterpath/case_study_ff}
  \caption{Second-order CIM motor response with plant inversion}
  \label{fig:case_study_ff}
\end{svg}

\subsection{Unmodeled dynamics}

In addition to \gls{plant} inversion, one can include feedforwards for unmodeled
dynamics. Consider an elevator model which doesn't include gravity. A constant
voltage offset can be used compensate for this. The feedforward takes the form
of a voltage constant because voltage is proportional to force applied, and the
force is acting in only one direction at all times.
\begin{equation}
  u_k = V_{app}
\end{equation}

where $V_{app}$ is a constant. Another feedforward holds a single-jointed arm
steady in the presence of gravity. It has the following form.
\begin{equation}
  u_k = V_{app} \cos\theta
\end{equation}

where $V_{app}$ is the voltage required to keep the single-jointed arm level
with the ground, and $\theta$ is the angle of the arm relative to the ground.
Therefore, the force applied is greatest when the arm is parallel with the
ground and zero when the arm is perpendicular to the ground (at that point, the
joint supports all the weight).

Note that the elevator model could be augmented easily enough to include gravity
and still be linear, but this wouldn't work for the single-jointed arm since a
trigonometric function is required to model the gravitational force in the arm's
rotating reference frame\footnote{While the applied torque of the motor is
constant throughout the arm's range of motion, the torque caused by gravity in
the opposite direction varies according to the arm's angle.}.

\section{Numerical integration methods}

Most systems don't have linear dynamics and their differential equations can't
be solved analytically. Instead, we'll have to approximate their solutions with
numerical integration.

\subsection{Butcher tableaus}

Butcher tableaus are a more succinct representation for explicit and implicit
Runge-Kutta numerical integration methods. Here's the general structure for
explicit methods.
\begin{equation*}
  \renewcommand\arraystretch{1.2}
  \begin{array}{c|cccc}
    0 \\
    c_2    & a_{2,1} \\
    \vdots & \vdots & \ddots \\
    c_s    & a_{s,1} & \hdots & a_{s,s-1} \\
    \hline
           & b_1    & \hdots & \hdots    & b_s
  \end{array}
\end{equation*}

where $s$ is the number of stages in the method, the matrix $[a_{ij}]$ is the
Runge-Kutta matrix, $b_1, \ldots, b_s$ are the weights, and $c_1, \ldots, c_s$
are the nodes. The top-left quadrant contains the sums of the rows in the
top-right quadrant. Each column in the right half corresponds to a $\mat{k}$
coefficient from $\mat{k}_1$ to $\mat{k}_s$.

The family of solutions to $\dot{\mat{x}} = f(t, \mat{x})$ is given by
\begin{align*}
  \mat{k}_1 &= f(t_k, \mat{x}_k) \\
  \mat{k}_2 &= f(t_k + c_2 h, \mat{x}_k + h (a_{2,1} \mat{k}_1)) \\
  &\ \ \vdots \\
  \mat{k}_s &= f(t_k + c_s h, \mat{x}_k +
    h (a_{s,1} \mat{k}_1 + \ldots + a_{s,s-1} \mat{k}_{s-1})) \\
  \mat{x}_{k+1} &= \mat{x}_k + h \sum_{i=1}^s b_i \mat{k}_i
\end{align*}

where $h$ is the timestep duration.

\subsection{Forward Euler method}
\index{numerical integration!Forward Euler}

The simplest explicit Runge-Kutta integration method is forward Euler
integration. We don't recommend using it because it suffers from numerical
stability issues. We'll demonstrate how to translate its Butcher tableau into
equations that integrate $\dot{\mat{x}} = f(t, \mat{x})$ from $0$ to $h$.
\begin{center}
  \begin{minipage}{0.35\linewidth}
    \centering
    \begin{alignat*}{7}
      \mat{k}_1 &= f(t +
        && {\color{blue}0} h,
        && \mat{x}_k)
        && \\
      \mat{x}_{k+1} &=
        &&
        && \mat{x}_k + h (
        && {\color{deepgreen}1} \mat{k}_1)
    \end{alignat*}
  \end{minipage}
  \quad
  \begin{minipage}{0.35\linewidth}
    \centering
    \begin{equation*}
      \renewcommand\arraystretch{1.2}
      \begin{array}{c|c}
        {\color{blue}0} \\
        \hline
        & {\color{deepgreen}1}
      \end{array}
    \end{equation*}
  \end{minipage}
\end{center}

Remove zeroed out terms.
\begin{align*}
  \mat{k}_1 &= f(t, \mat{x}_k) \\
  \mat{x}_{k+1} &= \mat{x}_k + h \mat{k}_1
  \intertext{Simplify.}
  \mat{x}_{k+1} &= \mat{x}_k + h f(t, \mat{x}_k)
\end{align*}

In FRC, our differential equations are of the form
$\dot{\mat{x}} = f(\mat{x}, \mat{u})$ where $\mat{u}$ is held constant between
timesteps. Since it's time-invariant, we can ignore the time argument of the
integration method. This gives theorem \ref{thm:forward_euler}.
\begin{theorem}[Forward Euler integration]
  \label{thm:forward_euler}

  Given the differential equation $\dot{\mat{x}} = f(\mat{x}_k, \mat{u}_k)$,
  this method solves for $\mat{x}_{k+1}$ at $h$ seconds in the future.
  $\mat{u}$ is assumed to be held constant between timesteps.
  \begin{center}
    \begin{minipage}{0.35\linewidth}
      \centering
      \begin{equation*}
        \mat{x}_{k+1} = \mat{x}_k + h f(\mat{x}_k, \mat{u}_k)
      \end{equation*}
    \end{minipage}
    \quad
    \begin{minipage}{0.35\linewidth}
      \centering
      \begin{equation*}
        \renewcommand\arraystretch{1.2}
        \begin{array}{c|c}
          0 \\
          \hline
          & 1
        \end{array}
      \end{equation*}
    \end{minipage}
  \end{center}
\end{theorem}

\subsection{Runge-Kutta fourth-order method}
\index{numerical integration!Runge-Kutta fourth-order}

The most common method we'll cover is Runge-Kutta fourth-order (RK4). It's
simple and accurate for most systems we'll see in FRC. We'll demonstrate how to
translate its Butcher tableau into equations that integrate
$\dot{\mat{x}} = f(t, \mat{x})$ from $0$ to $h$.
\begin{center}
  \begin{minipage}{0.35\linewidth}
    \centering
    \begin{alignat*}{7}
      \mat{k}_1 &= f(t +
        && {\color{blue}0} h,
        && \mat{x}_k)
        &&
        &&
        &&
        && \\
      \mat{k}_2 &= f(t +
        && {\color{blue}\frac{1}{2}} h,
        && \mat{x}_k + h (
        && {\color{deeporange}\frac{1}{2}} \mat{k}_1))
        &&
        &&
        && \\
      \mat{k}_3 &= f(t +
        && {\color{blue}\frac{1}{2}} h,
        && \mat{x}_k + h (
        && {\color{deeporange}0} \mat{k}_1 +
        && {\color{deeporange}\frac{1}{2}} \mat{k}_2))
        &&
        && \\
      \mat{k}_4 &= f(t +
        && {\color{blue}1} h,
        && \mat{x}_k + h (
        && {\color{deeporange}0} \mat{k}_1 +
        && {\color{deeporange}0} \mat{k}_2 +
        && {\color{deeporange}1} \mat{k}_3))
        && \\
      \mat{x}_{k+1} &=
        &&
        && \mat{x}_k + h (
        && {\color{deepgreen}\frac{1}{6}} \mat{k}_1 +
        && {\color{deepgreen}\frac{1}{3}} \mat{k}_2 +
        && {\color{deepgreen}\frac{1}{3}} \mat{k}_3 +
        && {\color{deepgreen}\frac{1}{6}} \mat{k}_4)
    \end{alignat*}
  \end{minipage}
  \quad
  \begin{minipage}{0.35\linewidth}
    \centering
    \begin{equation*}
      \renewcommand\arraystretch{1.2}
      \begin{array}{c|cccc}
        {\color{blue}0} \\
        {\color{blue}\frac{1}{2}} & {\color{deeporange}\frac{1}{2}} \\
        {\color{blue}\frac{1}{2}} & {\color{deeporange}0}           & {\color{deeporange}\frac{1}{2}} \\
        {\color{blue}1}           & {\color{deeporange}0}           & {\color{deeporange}0}           & {\color{deeporange}1} \\
        \hline
                                  & {\color{deepgreen}\frac{1}{6}}  & {\color{deepgreen}\frac{1}{3}}  & {\color{deepgreen}\frac{1}{3}} & {\color{deepgreen}\frac{1}{6}}
      \end{array}
    \end{equation*}
  \end{minipage}
\end{center}

Remove zeroed out terms.
\begin{align*}
  \mat{k}_1 &= f(t, \mat{x}_k) \\
  \mat{k}_2 &= f(t + \frac{1}{2} h, \mat{x}_k + h \frac{1}{2} \mat{k}_1) \\
  \mat{k}_3 &= f(t + \frac{1}{2} h, \mat{x}_k + h \frac{1}{2} \mat{k}_2) \\
  \mat{k}_4 &= f(t + h, \mat{x}_k + h \mat{k}_3) \\
  \mat{x}_{k+1} &= \mat{x}_k + h \left(
    \frac{1}{6} \mat{k}_1 +
    \frac{1}{3} \mat{k}_2 +
    \frac{1}{3} \mat{k}_3 +
    \frac{1}{6} \mat{k}_4\right)
  \intertext{$\frac{1}{6}$ is usually factored out of the last equation to
    reduce the number of floating point operations.}
  \mat{x}_{k+1} &= \mat{x}_k + h \frac{1}{6} (
    \mat{k}_1 + 2\mat{k}_2 + 2\mat{k}_3 + \mat{k}_4)
\end{align*}

In FRC, our differential equations are of the form
$\dot{\mat{x}} = f(\mat{x}, \mat{u})$ where $\mat{u}$ is held constant between
timesteps. Since it's time-invariant, we can ignore the time argument of the
integration method. This gives theorem \ref{thm:rk4}.
\begin{theorem}[Fourth-order Runge-Kutta integration]
  \label{thm:rk4}

  Given the differential equation $\dot{\mat{x}} = f(\mat{x}_k, \mat{u}_k)$,
  this method solves for $\mat{x}_{k+1}$ at $h$ seconds in the future.
  $\mat{u}$ is assumed to be held constant between timesteps.
  \begin{center}
    \begin{minipage}{0.35\linewidth}
      \centering
      \begin{align*}
        \mat{k}_1 &= f(\mat{x}_k, \mat{u}_k) \\
        \mat{k}_2 &= f(\mat{x}_k + h \frac{1}{2}\mat{k}_1, \mat{u}_k) \\
        \mat{k}_3 &= f(\mat{x}_k + h \frac{1}{2}\mat{k}_2, \mat{u}_k) \\
        \mat{k}_4 &= f(\mat{x}_k + h \mat{k}_3, \mat{u}_k) \\
        \mat{x}_{k+1} &= \mat{x}_k + h \frac{1}{6} (\mat{k}_1 + 2\mat{k}_2 +
          2\mat{k}_3 + \mat{k}_4)
      \end{align*}
    \end{minipage}
    \quad
    \begin{minipage}{0.35\linewidth}
      \centering
      \begin{equation*}
        \renewcommand\arraystretch{1.2}
        \begin{array}{c|cccc}
          0 \\
          \frac{1}{2} & \frac{1}{2} \\
          \frac{1}{2} & 0 & \frac{1}{2} \\
          1 & 0 & 0 & 1 \\
          \hline
          & \frac{1}{6} & \frac{1}{3} & \frac{1}{3} & \frac{1}{6}
        \end{array}
      \end{equation*}
    \end{minipage}
  \end{center}
\end{theorem}

Here's a reference implementation.
\begin{coderemote}{cpp}{snippets/RK4.cpp}
  \caption{RK4 implementation in C++}
\end{coderemote}

Other methods of Runge-Kutta integration exist with various properties
\cite{bib:wiki_rk4}, but the one presented here is popular for its high accuracy
relative to the amount of floating point operations (FLOPs) it requires.

\subsection{Runge-Kutta-Fehlberg method}
\index{numerical integration!Runge-Kutta-Fehlberg}

Runge-Kutta-Fehlberg (RKF45) is a fourth-order method like Runge-Kutta, but it
uses an adaptive stepsize to enforce an upper bound on the integration error.
\begin{theorem}[Runge-Kutta-Fehlberg integration]
  Given the differential equation $\dot{\mat{x}} = f(\mat{x}_k, \mat{u}_k)$,
  this method solves for $\mat{x}_{k+1}$ at $h$ seconds in the future.
  $\mat{u}$ is assumed to be held constant between timesteps. It has the
  following Butcher tableau.
  \begin{equation*}
    \renewcommand\arraystretch{1.2}
    \begin{array}{c|ccccccc}
      0 \\
      \frac{1}{4} & \frac{1}{4} \\
      \frac{3}{8} & \frac{3}{32} & \frac{9}{32} \\
      \frac{12}{13} & \frac{1932}{2197} & -\frac{7200}{2197} & \frac{7296}{2197}
        \\
      1 & \frac{439}{216} & -8 & \frac{3680}{513} & -\frac{845}{4104} \\
      \frac{1}{2} & -\frac{8}{27} & 2 & -\frac{3544}{2565} & \frac{1859}{4104} &
        -\frac{11}{40} \\
      \hline
      & \frac{16}{135} & 0 & \frac{6656}{12825} & \frac{28561}{56430} &
        -\frac{9}{50} & \frac{2}{55} \\
      & \frac{25}{216} & 0 & \frac{1408}{2565} & \frac{2197}{4104} &
        -\frac{1}{5} & 0
    \end{array}
  \end{equation*}
  The first row of coefficients below the table divider gives the fifth-order
  accurate solution. The second row gives an alternative solution which,
  when subtracted from the first solution, gives the error estimate.
\end{theorem}

Here's a reference implementation.
\begin{coderemote}{cpp}{snippets/RKF45.cpp}
  \caption{RKF45 implementation in C++}
\end{coderemote}

\subsection{Dormand-Prince method}
\index{numerical integration!Dormand-Prince}

Dormand-Prince (RKDP) is similar to RKF45, but it has higher precision per unit
work.
\begin{theorem}[Dormand-Prince integration]
  Given the differential equation $\dot{\mat{x}} = f(\mat{x}_k, \mat{u}_k)$,
  this method solves for $\mat{x}_{k+1}$ at $h$ seconds in the future.
  $\mat{u}$ is assumed to be held constant between timesteps. It has the
  following Butcher tableau.
  \begin{equation*}
    \renewcommand\arraystretch{1.2}
    \begin{array}{c|ccccccc}
      0 \\
      \frac{1}{5} & \frac{1}{5} \\
      \frac{3}{10} & \frac{3}{40} & \frac{9}{40} \\
      \frac{4}{5} & \frac{44}{45} & -\frac{56}{15} & \frac{32}{9} \\
      \frac{8}{9} & \frac{19372}{6561} & -\frac{25360}{2187} &
        \frac{64448}{6561} & -\frac{212}{729} \\
      1 & \frac{9017}{3168} & -\frac{355}{33} & \frac{46732}{5247} &
        \frac{49}{176} & -\frac{5103}{18656} \\
      1 & \frac{35}{384} & 0 & \frac{500}{1113} & \frac{125}{192} &
        -\frac{2187}{6784} & \frac{11}{84} \\
      \hline
      & \frac{35}{384} & 0 & \frac{500}{1113} & \frac{125}{192} &
        -\frac{2187}{6784} & \frac{11}{84} & 0 \\
      & \frac{5179}{57600} & 0 & \frac{7571}{16695} & \frac{393}{640} &
        -\frac{92097}{339200} & \frac{187}{2100} & \frac{1}{40}
    \end{array}
  \end{equation*}
  The first row of coefficients below the table divider gives the fifth-order
  accurate solution. The second row gives an alternative solution which,
  when subtracted from the first solution, gives the error estimate.
\end{theorem}

Here's a reference implementation.
\begin{coderemote}{cpp}{snippets/RKDP.cpp}
  \caption{RKDP implementation in C++}
\end{coderemote}

