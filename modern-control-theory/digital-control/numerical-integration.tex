\section{Numerical integration methods}

Most systems don't have linear dynamics and their differential equations can't
be solved analytically. Instead, we'll have to approximate their solutions with
numerical integration.

\subsection{Runge-Kutta method}
\index{numerical integration!Runge-Kutta}

The first and most common method we'll cover is fourth-order Runge-Kutta (RK4).
It's simple and accurate for most systems we'll see in FRC.
\begin{theorem}[Fourth-order Runge-Kutta integration]
  \label{thm:rk4}

  Given the differential equation $\dot{\mtx{x}} = f(\mtx{x}_k, \mtx{u}_k)$,
  this method solves for $\mtx{x}_{k+1}$ at $dt$ seconds in the future.
  $\mtx{u}$ is assumed to be held constant between timesteps.
  \begin{center}
    \begin{minipage}{0.35\linewidth}
      \centering
      \begin{align*}
        \mtx{k}_1 &= f(\mtx{x}_k, \mtx{u}_k) \\
        \mtx{k}_2 &= f(\mtx{x}_k + \frac{1}{2} dt \mtx{k}_1, \mtx{u}_k) \\
        \mtx{k}_3 &= f(\mtx{x}_k + \frac{1}{2} dt \mtx{k}_2, \mtx{u}_k) \\
        \mtx{k}_4 &= f(\mtx{x}_k + dt \mtx{k}_3, \mtx{u}_k) \\
        \mtx{x}_{k+1} &= \mtx{x}_k + \frac{1}{6} dt (\mtx{k}_1 + 2\mtx{k}_2 +
          2\mtx{k}_3 + \mtx{k}_4)
      \end{align*}
    \end{minipage}
    \quad
    \begin{minipage}{0.35\linewidth}
      \centering
      \begin{equation*}
        \renewcommand\arraystretch{1.2}
        \begin{array}{c|cccc}
          0 \\
          \frac{1}{2} & \frac{1}{2} \\
          \frac{1}{2} & 0 & \frac{1}{2} \\
          1 & 0 & 0 & 1 \\
          \hline
          & \frac{1}{6} & \frac{1}{3} & \frac{1}{3} & \frac{1}{6}
        \end{array}
      \end{equation*}
    \end{minipage}
  \end{center}
\end{theorem}

The Butcher tableau on the right in theorem \ref{thm:rk4} is a more succinct
representation of the equations on the left. Each row in the tableau corresponds
to one of the equations, and each column in the right half corresponds to a
$\mtx{k}$ coefficient from $\mtx{k}_1$ to $\mtx{k}_4$.

The top-left quadrant contains the sums of each row in the top-right quadrant.
These sums are the $\mtx{k}$ coefficients in the first argument of
$f(\mtx{x}, \mtx{u})$. The bottom row contains the $\mtx{k}$ coefficients for
the $\mtx{x}_{k+1}$ equation.

Other methods of Runge-Kutta integration exist with various properties
\cite{bib:wiki_rk4}, but the one presented here is popular for its high accuracy
relative to the amount of floating point operations (FLOPs) it requires.

\subsection{Runge-Kutta-Fehlberg method}
\index{numerical integration!Runge-Kutta-Fehlberg}

Runge-Kutta-Fehlberg (RKF45) is a fourth-order method like Runge-Kutta, but it
uses an adaptive stepsize to enforce an upper bound on the integration error.
See the Wikipedia page on RKF45 for an implementation guide
\cite{bib:wiki_rkf45}.

\subsection{Dormand-Prince method}
\index{numerical integration!Dormand-Prince}

Dormand-Prince (RKDP45) is similar to RKF45, but it has higher precision per
unit work.
\begin{theorem}[Dormand-Prince integration]
  Given the differential equation $\dot{\mtx{x}} = f(\mtx{x}_k, \mtx{u}_k)$,
  this method solves for $\mtx{x}_{k+1}$ at $dt$ seconds in the future.
  $\mtx{u}$ is assumed to be held constant between timesteps. It has the
  following Butcher tableau.
  \begin{equation*}
    \renewcommand\arraystretch{1.2}
    \begin{array}{c|ccccccc}
      0 \\
      \frac{1}{5} & \frac{1}{5} \\
      \frac{3}{10} & \frac{3}{40} & \frac{9}{40} \\
      \frac{4}{5} & \frac{44}{45} & -\frac{56}{15} & \frac{32}{9} \\
      \frac{8}{9} & \frac{19372}{6561} & -\frac{25360}{2187} &
        \frac{64448}{6561} & -\frac{212}{729} \\
      1 & \frac{9017}{3168} & -\frac{355}{33} & \frac{46732}{5247} &
        \frac{49}{176} & -\frac{5103}{18656} \\
      1 & \frac{35}{384} & 0 & \frac{500}{1113} & \frac{125}{192} &
        -\frac{2187}{6784} & \frac{11}{84} \\
      \hline
      & \frac{35}{384} & 0 & \frac{500}{1113} & \frac{125}{192} &
        -\frac{2187}{6784} & \frac{11}{84} & 0 \\
      & \frac{5179}{57600} & 0 & \frac{7571}{16695} & \frac{393}{640} &
        -\frac{92097}{339200} & \frac{187}{2100} & \frac{1}{40}
    \end{array}
  \end{equation*}
  The first row of coefficients below the table divider gives the fifth-order
  accurate solution. The second row gives an alternative solution which,
  when subtracted from the first solution, gives the error estimate.
\end{theorem}
