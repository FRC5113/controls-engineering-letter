\section{Taylor series}
\index{discretization!Taylor series}
\begin{remark}
  Watch the ``Taylor series" video from 3Blue1Brown's \textit{Essence of
  calculus} series (22 minutes) \cite{bib:calculus_taylor_series} for an
  explanation of how the Taylor series expansion works.
\end{remark}

The definition for the matrix exponential and the approximations below all use
the \textit{Taylor series expansion}. The Taylor series is a method of
approximating a function like $e^t$ via the summation of weighted polynomial
terms like $t^k$. $e^t$ has the following Taylor series around $t = 0$.
\begin{equation*}
  e^t = \sum_{n = 0}^\infty \frac{t^n}{n!}
\end{equation*}

where a finite upper bound on the number of terms produces an approximation of
$e^t$. As $n$ increases, the polynomial terms increase in power and the weights
by which they are multiplied decrease. For $e^t$ and some other functions, the
Taylor series expansion equals the original function for all values of $t$ as
the number of terms approaches infinity\footnote{Functions for which their
Taylor series expansion converges to and also equals it are called analytic
functions.}. Figure \ref{fig:taylor_series} shows the Taylor series expansion of
$e^t$ around $t = 0$ for a varying number of terms.
\begin{svg}{build/\chapterpath/taylor_series}
  \caption{Taylor series expansions of $e^t$ around $t = 0$ for $n$ terms}
  \label{fig:taylor_series}
\end{svg}

We'll expand the first few terms of the Taylor series expansion in equation
\eqref{eq:mat_exp} for $\mat{X} = \mat{A}T$ so we can compare it with other
methods.
\begin{equation*}
  \sum_{k=0}^3 \frac{1}{k!} (\mat{A}T)^k = \mat{I} + \mat{A}T +
    \frac{1}{2}\mat{A}^2T^2 + \frac{1}{6}\mat{A}^3T^3
\end{equation*}

Table \ref{tab:disc_approx_matrix} compares the Taylor series expansions of the
\gls{discretization} methods for the matrix case. These use a more complex
formula which we won't present here.
\begin{booktable}
  \begin{tabular}{|cll|}
    \hline
    \rowcolor{headingbg}
    \multicolumn{1}{|c}{\textbf{Method}} &
      \multicolumn{1}{c}{\textbf{Conversion}} &
      \multicolumn{1}{c|}{\textbf{Taylor series expansion}} \\
    \hline
    Zero-order hold &
      $\mat{A}_d = e^{\mat{A}_c T}$ &
      $\mat{A}_d = \mat{I} + \mat{A}_c T + \frac{1}{2}\mat{A}_c^2T^2 +
        \frac{1}{6}\mat{A}_c^3T^3 + \ldots$ \\
    Bilinear &
      $\mat{A}_d =
        \left(\mat{I} + \frac{1}{2}\mat{A}_c T\right)
        \left(\mat{I} - \frac{1}{2}\mat{A}_c T\right)^{-1}$ &
      $\mat{A}_d = \mat{I} + \mat{A}_c T + \frac{1}{2}\mat{A}_c^2T^2 +
        \frac{1}{4}\mat{A}_c^3T^3 + \ldots$ \\
    Forward Euler &
      $\mat{A}_d = \mat{I} + \mat{A}_c T$ &
      $\mat{A}_d = \mat{I} + \mat{A}_c T$ \\
    Reverse Euler &
      $\mat{A}_d = \left(\mat{I} - \mat{A}_c T\right)^{-1}$ &
      $\mat{A}_d =
        \mat{I} + \mat{A}_c T + \mat{A}_c^2T^2 + \mat{A}_c^3T^3 + \ldots$ \\
    \hline
  \end{tabular}
  \caption{Taylor series expansions of discretization methods (matrix case).
    The zero-order hold discretization method is exact.}
  \label{tab:disc_approx_matrix}
\end{booktable}

Each of them has different stability properties. The bilinear transform
preserves the (in)stability of the continuous time \gls{system}.
