\section{s-plane to z-plane}
\label{sec:s-plane_to_z-plane}

Transfer functions are converted to impulse responses using the Z-transform. The
s-plane's LHP maps to the inside of a unit circle in the z-plane. Table
\ref{tab:s2z_mapping} contains a few common points and figure
\ref{fig:s2z_mapping} shows the mapping visually.

\begin{booktable}
  \begin{tabular}{|cc|}
    \hline
    \rowcolor{headingbg}
    \textbf{s-plane} & \textbf{z-plane} \\
    \hline
    $(0, 0)$ & $(1, 0)$ \\
    imaginary axis & edge of unit circle \\
    $(-\infty, 0)$ & $(0, 0)$ \\
    \hline
  \end{tabular}
  \caption{Mapping from s-plane to z-plane}
  \label{tab:s2z_mapping}
\end{booktable}

\begin{bookfigure}
  \begin{minisvg}{2}{build/code/s_plane}
  \end{minisvg}
  \hfill
  \begin{minisvg}{2}{build/code/z_plane}
  \end{minisvg}
  \caption{Mapping of axes from s-plane (left) to z-plane (right)}
  \label{fig:s2z_mapping}
\end{bookfigure}

\subsection{z-plane stability}

Eigenvalues of a \gls{system} that are within the unit circle are stable, but
why is that? Let's consider a scalar equation $x_{k + 1} = ax_k$. $a < 1$ makes
$x_{k + 1}$ converge to zero. The same applies to a complex number like
$z = x + yi$ for $x_{k + 1} = zx_k$. If the magnitude of the complex number $z$
is less than one, $x_{k+1}$ will converge to zero. Values with a magnitude of
$1$ oscillate forever because $x_{k+1}$ never decays.

\subsection{z-plane behavior}

As $\omega$ increases in $s = j\omega$, a pole in the z-plane moves around the
perimeter of the unit circle. Once it hits $\frac{\omega_s}{2}$ (half the
sampling frequency) at $(-1, 0)$, the pole wraps around. This is due to poles
faster than the sample frequency folding down to below the sample frequency
(that is, higher frequency signals \textit{alias} to lower frequency ones).

You may notice that poles can be placed at $(0, 0)$ in the z-plane. This is
known as a deadbeat controller. An $\rm N^{th}$-order deadbeat controller decays
to the \gls{reference} in N timesteps. While this sounds great, there are other
considerations like \gls{control effort}, \gls{robustness}, and
\gls{noise immunity}. These will be discussed in more detail with LQR and LQE.

If poles from $(1, 0)$ to $(0, 0)$ on the x-axis approach infinity, then what do
poles from $(-1, 0)$ to $(0, 0)$ represent? Them being faster than infinity
doesn't make sense. Poles in this location exhibit oscillatory behavior similar
to complex conjugate pairs. See figures \ref{fig:z_oscillations_1p} and
\ref{fig:z_oscillations_2p}. The jaggedness of these signals is due to the
frequency of the \gls{system} dynamics being above the Nyquist frequency. The
\glslink{discretization}{discretized} signal doesn't have enough samples to
reconstruct the continuous \gls{system}'s dynamics.

\begin{bookfigure}
  \begin{minisvg}{2}{build/code/z_oscillations_1p}
    \caption{Single poles in various locations in z-plane}
    \label{fig:z_oscillations_1p}
  \end{minisvg}
  \hfill
  \begin{minisvg}{2}{build/code/z_oscillations_2p}
    \caption{Complex conjugate poles in various locations in z-plane}
    \label{fig:z_oscillations_2p}
  \end{minisvg}
\end{bookfigure}

\subsection{Nyquist frequency}
\index{Digital signal processing!Nyquist frequency}
\index{Digital signal processing!aliasing}

To completely reconstruct a signal, the Nyquist-Shannon sampling theorem states
that it must be sampled at a frequency at least twice the maximum frequency it
contains. The highest frequency a given sample rate can capture is called the
Nyquist frequency, which is half the sample frequency. This is why recorded
audio is sampled at $44.1k\,Hz$. The maximum frequency a typical human can hear
is about $20\,kHz$, so the Nyquist frequency is $40\,kHz$. ($44.1kHz$ in
particular was chosen for unrelated historical reasons.)

Frequencies above the Nyquist frequency are folded down across it. The higher
frequency and the folded down lower frequency are said to alias each
other\footnote{The aliases of a frequency $f$ can be expressed as
$f_{alias}(N) \stackrel{def}{=} |f - Nf_s|$. For example, if a $200\,Hz$ sine
wave is sampled at $150\,Hz$, the \gls{observer} will see a $50\,Hz$ signal
instead of a $200\,Hz$ one.}. Figure \ref{fig:nyquist} provides a demonstration
of aliasing.

\begin{svg}{build/code/nyquist}
  \caption{The samples of two sine waves can be identical when at least one of
    them is at a frequency above half the sample rate. In this case, the $2\,Hz$
    sine wave is above the Nyquist frequency $1.5\,Hz$.}
  \label{fig:nyquist}
\end{svg}

The effect of these high-frequency aliases can be reduced with a low-pass filter
(called an anti-aliasing filter in this application).
