\section{Zero-order hold for state-space}
\index{discretization!zero-order hold}

Given the following continuous time state space model
\begin{align*}
  \dot{\mtx{x}} &= \mtx{A}_c\mtx{x} + \mtx{B}_c\mtx{u} + \mtx{w} \\
  \mtx{y} &= \mtx{C}_c\mtx{x} + \mtx{D}_c\mtx{u} + \mtx{v}
\end{align*}

where $\mtx{w}$ is the process noise, $\mtx{v}$ is the measurement noise, and
both are zero-mean white noise sources with covariances of $\mtx{Q}_c$ and
$\mtx{R}_c$ respectively. $\mtx{w}$ and $\mtx{v}$ are defined as normally
distributed random variables.
\begin{align*}
  \mtx{w} &\sim N(0, \mtx{Q}_c) \\
  \mtx{v} &\sim N(0, \mtx{R}_c)
\end{align*}

The model can be \glslink{discretization}{discretized} as follows
\begin{align*}
  \mtx{x}_{k+1} &= \mtx{A}_d \mtx{x}_k + \mtx{B}_d \mtx{u}_k + \mtx{w}_k \\
   \mtx{y}_k &= \mtx{C}_d \mtx{x}_k + \mtx{D}_d \mtx{u}_k + \mtx{v}_k
\end{align*}

with covariances
\begin{align*}
  \mtx{w}_k &\sim N(0, \mtx{Q}_d) \\
  \mtx{v}_k &\sim N(0, \mtx{R}_d)
\end{align*}
\begin{theorem}[Zero-order hold for state-space]
  \label{thm:zoh_ss}

  \begin{align}
    \mtx{A}_d &= e^{\mtx{A}_c T} \\
    \mtx{B}_d &= \int_0^T e^{\mtx{A}_c \tau} d\tau \mtx{B}_c =
      \mtx{A}_c^{-1} (\mtx{A}_d - \mtx{I}) \mtx{B}_c \\
    \mtx{C}_d &= \mtx{C}_c \\
    \mtx{D}_d &= \mtx{D}_c \\
    \mtx{Q}_d &= \int_{\tau = 0}^{T} e^{\mtx{A}_c\tau} \mtx{Q}_c
      e^{\mtx{A}_c^T\tau} d\tau \\
    \mtx{R}_d &= \frac{1}{T}\mtx{R}_c
  \end{align}

  where a subscript of $d$ denotes discrete, a subscript of $c$ denotes the
  continuous version of the corresponding matrix, $T$ is the sample period for
  the discrete \gls{system}, and $e^{\mtx{A}_c T}$ is the matrix exponential of
  $\mtx{A}_c$.
\end{theorem}

See appendix \ref{sec:deriv_zoh_ss} for derivations.

To compute $\mtx{A}_d$ and $\mtx{B}_d$ in one step, one can utilize the
following property.
\begin{equation*}
  e^{
  \begin{bmatrix}
    \mtx{A}_c & \mtx{B}_c \\
    \mtx{0} & \mtx{0}
  \end{bmatrix}T} =
  \begin{bmatrix}
    \mtx{A}_d & \mtx{B}_d \\
    \mtx{0} & \mtx{I}
  \end{bmatrix}
\end{equation*}

$\mtx{Q}_d$ can be computed as
\begin{equation*}
  \Phi = e^{
  \begin{bmatrix}
    -\mtx{A}_c & \mtx{Q}_c \\
    \mtx{0} & \mtx{A}_c^T
  \end{bmatrix}T} =
  \begin{bmatrix}
    -\mtx{A}_d & \mtx{A}_d^{-1} \mtx{Q}_d \\
    \mtx{0} & \mtx{A}_d^T
  \end{bmatrix}
\end{equation*}

where $\mtx{Q}_d = \Phi_{2,2}^T \Phi_{1,2}$ \cite{bib:integral_matrix_exp}.

To see why $\mtx{R}_c$ is being divided by $T$, consider the discrete white
noise sequence $\mtx{v}_k$ and the (non-physically realizable) continuous white
noise process $\mtx{v}$. Whereas $\mtx{R}_{d,k} = E[\mtx{v}_k \mtx{v}_k^T]$ is a
covariance matrix, $\mtx{R}_c(t)$ defined by
$E[\mtx{v}(t) \mtx{v}^T(\tau)] = \mtx{R}_c(t)\delta(t - \tau)$ is a spectral
density matrix (the Dirac function $\delta(t - \tau)$ has units of
$1/\text{sec}$). The covariance matrix $\mtx{R}_c(t)\delta(t - \tau)$ has
infinite-valued elements. The discrete white noise sequence can be made to
approximate the continuous white noise process by shrinking the pulse lengths
($T$) and increasing their amplitude, such that
$\mtx{R}_d \rightarrow \frac{1}{T}\mtx{R}_c$.

That is, in the limit as $T \rightarrow 0$, the discrete noise sequence tends to
one of infinite-valued pulses of zero duration such that the area under the
``impulse" autocorrelation function is $\mtx{R}_d T$. This is equal to the area
$\mtx{R}_c$ under the continuous white noise impulse autocorrelation function.
