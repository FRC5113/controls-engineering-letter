\section{Differential drive kinematics}

A differential drive has two wheels, one on each side, separated by some
distance $2r_b$. The forces they generate when moving forward are shown in
figure \ref{fig:differential_drive_fbd}.
\begin{bookfigure}
  \begin{tikzpicture}[auto, >=latex', circuit ee IEC,
                    set resistor graphic=var resistor IEC graphic]
  % \draw [help lines] (-1,-3) grid (7,4);

  % Right wheel
  \begin{scope}[xshift=5.78cm,yshift=1.83cm]
    \draw[fill=black!50] (0.2,0.68) ellipse (0.13 and 0.67);
    \draw[fill=black!50, color=black!50] (0,0) rectangle (0.2,1.35);
    \draw[fill=white] (0,0.68) ellipse (0.13 and 0.67);
    \draw (0,1.35) -- (0.2,1.35);
    \draw (0,0) -- (0.2,0);
  \end{scope}

  % Right transmission shaft
  \begin{scope}[xshift=5.32cm,yshift=2.42cm]
    \draw[fill=black!50] (0,0) rectangle (0.46,0.1);
  \end{scope}

  % Chassis
  \begin{scope}[xshift=4.44cm,yshift=2.09cm]
    \fill[fill=white] (0,0.8) -- (0,0.2) -- (0.2,0) -- (0.2,0.2)
      -- (0.98,0.2) -- (0.78,0.8) -- cycle;
    \draw (0,0.8) -- (0.78,0.8);
    \draw (0,0.8) -- (0,0.2);
    \draw (0,0.2) -- (0.2,0);
    \draw (0,0.8) -- (0.2,0.6);
    \draw (0.78,0.8) -- (0.98,0.6);
    \draw[fill=white] (0.2,0.6) rectangle (0.98,0);
  \end{scope}

  % Left transmission shaft
  \begin{scope}[xshift=4.09cm,yshift=2.42cm]
    \draw[fill=black!50] (0,0) rectangle (0.46,0.1);
  \end{scope}

  % Left wheel
  \begin{scope}[xshift=3.75cm,yshift=1.83cm]
    \draw[fill=black!50] (0.2,0.68) ellipse (0.13 and 0.67);
    \draw[fill=black!50, color=black!50] (0,0) rectangle (0.2,1.35);
    \draw[fill=white] (0,0.68) ellipse (0.13 and 0.67);
    \draw (0,1.35) -- (0.2,1.35);
    \draw (0,0) -- (0.2,0);
  \end{scope}

  % Angular velocity arrow for left wheel
  \draw[line width=0.7pt,->] (4.24,1.97) arc (-30:30:1) node[above]
    {$\omega_l$};

  % Angular velocity arrow for right wheel
  \draw[line width=0.7pt,->] (5.44,1.97) arc (-30:30:1) node[above]
    {$\omega_r$};

  % Wheel radius arrow
  \begin{scope}[xshift=3.5cm,yshift=1.83cm]
    \draw[line width=0.7pt,<->] (0,0) -- node[left] {$r$} (0,0.67);
  \end{scope}

  % Robot radius arrow
  \begin{scope}[xshift=4.65cm,yshift=1.83cm]
    \draw[line width=0.7pt,<->] (0,0) -- node[below] {$r_b$} (0.39,0);
  \end{scope}

  % Descriptions inside graphic
  \draw (4.99,2.42) node {$J$};
\end{tikzpicture}

  \caption{Differential drive free body diagram}
  \label{fig:differential_drive_fbd}
\end{bookfigure}

\subsection{Inverse kinematics}

The mapping from $v$ and $\omega$ to the left and right wheel velocities $v_l$
and $v_r$ is derived as follows. Let $\vec{v}_c$ be the velocity vector of the
center of rotation, $\vec{v}_l$ be the velocity vector of the left wheel,
$\vec{v}_r$ be the velocity vector of the right wheel, $r_b$ is the distance
from the center of rotation to each wheel, and $\omega$ is the counterclockwise
turning rate around the center of rotation.

Once we have the vector equation representing the wheel's velocity, we'll
project it onto the wheel direction vector using the dot product.

First, we'll derive $v_l$.
\begin{align}
  \vec{v}_l &= v_c \hat{i} + \omega \hat{k} \times r_b \hat{j} \nonumber \\
  \vec{v}_l &= v_c \hat{i} - \omega r_b \hat{i} \nonumber \\
  \vec{v}_l &= (v_c - \omega r_b) \hat{i} \nonumber
  \intertext{Now, project this vector onto the left wheel, which is pointed in
    the $\hat{i}$ direction.}
  v_l &= (v_c - \omega r_b) \hat{i} \cdot \frac{\hat{i}}{\norm{\hat{i}}}
    \nonumber
  \intertext{The magnitude of $\hat{i}$ is $1$, so the denominator cancels.}
  v_l &= (v_c - \omega r_b) \hat{i} \cdot \hat{i} \nonumber \\
  v_l &= v_c - \omega r_b \label{eq:diff_vl}
\end{align}

Next, we'll derive $v_r$.
\begin{align}
  \vec{v}_r &= v_c \hat{i} + \omega \hat{k} \times r_b \hat{j} \nonumber \\
  \vec{v}_r &= v_c \hat{i} + \omega r_b \hat{i} \nonumber \\
  \vec{v}_r &= (v_c + \omega r_b) \hat{i} \nonumber
  \intertext{Now, project this vector onto the right wheel, which is pointed in
    the $\hat{i}$ direction.}
  v_r &= (v_c + \omega r_b) \hat{i} \cdot \frac{\hat{i}}{\norm{\hat{i}}}
    \nonumber
  \intertext{The magnitude of $\hat{i}$ is $1$, so the denominator cancels.}
  v_r &= (v_c + \omega r_b) \hat{i} \cdot \hat{i} \nonumber \\
  v_r &= v_c + \omega r_b \label{eq:diff_vr}
\end{align}

So the two inverse kinematic equations are as follows.
\begin{align}
  v_l &= v_c - \omega r_b \\
  v_r &= v_c + \omega r_b
\end{align}

\subsection{Forward kinematics}

The mapping from the left and right wheel velocities $v_l$ and $v_r$ to $v$ and
$\omega$ is derived as follows.
\begin{align}
  v_r &= v_c + \omega r_b \nonumber \\
  v_c &= v_r - \omega r_b \label{eq:diff_drive_v_c}
\end{align}

Substitute equation \eqref{eq:diff_drive_v_c} equation for $v_l$.
\begin{align*}
  v_l &= v_c - \omega r_b \\
  v_l &= (v_r - \omega r_b) - \omega r_b \\
  v_l &= v_r - 2\omega r_b \\
  2\omega r_b &= v_r - v_l \\
  \omega &= \frac{v_r - v_l}{2 r_b}
\end{align*}

Substitute this back into equation \eqref{eq:diff_drive_v_c}.
\begin{align*}
  v_c &= v_r - \omega r_b \\
  v_c &= v_r - \left(\frac{v_r - v_l}{2 r_b}\right) r_b \\
  v_c &= v_r - \frac{v_r - v_l}{2} \\
  v_c &= v_r - \frac{v_r}{2} + \frac{v_l}{2} \\
  v_c &= \frac{v_r + v_l}{2}
\end{align*}

So the two forward kinematic equations are as follows.
\begin{align}
  v_c &= \frac{v_r + v_l}{2} \\
  \omega &= \frac{v_r - v_l}{2 r_b}
\end{align}
