\section{Angular motion}
\begin{equation*}
  \sum \tau = I\alpha
\end{equation*}

where $\sum \tau$ is the sum of all torques applied to an object in
Newton-meters, $I$ is the moment of inertia of the object in $kg\mbox{-}m^2$
(also called the rotational mass), and $\alpha$ is the net angular acceleration
of the object in $\frac{rad}{s^2}$.
\begin{equation*}
  \theta(t) = \theta_0 + \omega_0 t + \frac{1}{2}\alpha t^2
\end{equation*}

where $\theta(t)$ is an object's angle at time $t$, $\theta_0$ is the initial
angle, $\omega_0$ is the initial angular velocity, and $\alpha$ is the angular
acceleration.
