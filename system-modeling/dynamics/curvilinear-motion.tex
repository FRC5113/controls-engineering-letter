\section{Curvilinear motion}

For these derivations, we'll assume positive $x$ ($\hat{i}$) is forward,
positive $y$ ($\hat{j}$) is to the left, positive $z$ ($\hat{k}$) is up, and
the robot is facing in the $x$ direction. This axes convention is known as
North-West-Up (NWU), and is shown in figure \ref{fig:nwu_axes_convention}.
\begin{bookfigure}
  \begin{tikzpicture}[auto, >=latex']
    %\draw [help lines] (-4,-3) grid (4,3);

    % Main axes
    \draw[<->] (4,0) -- (-4,0) node[below] {\small +y};
    \draw[<->] (0,-3) -- (0,3) node[right] {\small +x};
  \end{tikzpicture}

  \caption{2D projection of North-West-Up (NWU) axes convention. The positive
    $z$ axis is pointed out of the page toward the reader.}
  \label{fig:nwu_axes_convention}
\end{bookfigure}

The main equation we'll need is the following.
\begin{equation*}
  \vec{v}_B = \vec{v}_A + \omega_A \times \vec{r}_{B|A}
\end{equation*}

where $\vec{v}_B$ is the velocity vector at point B, $\vec{v}_A$ is the velocity
vector at point A, $\omega_A$ is the angular velocity vector at point A, and
$\vec{r}_{B|A}$ is the distance vector from point A to point B (also described
as the ``distance to B relative to A").

\subsection{Differential drive}

A differential drive has two wheels, one on each side, separated by some
distance $2r_b$. The forces they generate when moving forward are shown in
figure \ref{fig:differential_drive_fbd}.
\begin{bookfigure}
  \begin{tikzpicture}[auto, >=latex']
    %\draw [help lines] (-4,-3) grid (4,3);

    % Coordinate axes
    \begin{scope}[xshift=2cm,yshift=-2cm]
      \draw[->] (0,0) -- (-1,0) node[below] {\small +$y$};
      \draw[->] (0,0) -- (0,1) node[right] {\small +$x$};
    \end{scope}

    % Shared axle
    \fill[black] (-1.2,-0.1) rectangle (1.2,0.1);

    % Chassis
    \filldraw[draw=black,fill=white] (-1,-1) rectangle (1,1);

    % Left wheel
    \begin{scope}[xshift=-1.2cm,yshift=0cm]
      % Wheel vector
      \draw[->,thick] (0,0) -- (0,1) node[left] {\small $v_l$};

      % Wheel
      \filldraw[draw=black,fill=lightgray] (-0.1,-0.5) rectangle (0.1,0.5);
    \end{scope}

    % Right wheel
    \begin{scope}[xshift=1.2cm,yshift=0cm]
      % Wheel vector
      \draw[->,thick] (0,0) -- (0,1) node[right] {\small $v_r$};

      % Wheel
      \filldraw[draw=black,fill=lightgray] (-0.1,-0.5) rectangle (0.1,0.5);
    \end{scope}
  \end{tikzpicture}

  \caption{Differential drive free body diagram}
  \label{fig:differential_drive_fbd}
\end{bookfigure}

\subsubsection{Inverse kinematics}

The mapping from $v$ and $\omega$ to the left and right wheel velocities $v_l$
and $v_r$ is derived as follows. Let $\vec{v}_c$ be the velocity vector of the
center of rotation, $\vec{v}_l$ be the velocity vector of the left wheel,
$\vec{v}_r$ be the velocity vector of the right wheel, $r_b$ is the distance
from the center of rotation to each wheel, and $\omega$ is the counterclockwise
turning rate around the center of rotation.

Once we have the vector equation representing the wheel's velocity, we'll
project it onto the wheel direction vector using the dot product.

First, we'll derive $v_l$.
\begin{align*}
  \vec{v}_l &= v_c \hat{i} + \omega \hat{k} \times r_b \hat{j} \\
  \vec{v}_l &= v_c \hat{i} - \omega r_b \hat{i} \\
  \vec{v}_l &= (v_c - \omega r_b) \hat{i}
\end{align*}

Now, project this vector onto the left wheel, which is pointed in the $\hat{i}$
direction.
\begin{equation*}
  v_l = (v_c - \omega r_b) \hat{i} \cdot \frac{\hat{i}}{\norm{\hat{i}}}
\end{equation*}

The magnitude of $\hat{i}$ is $1$, so the denominator cancels.
\begin{align}
  v_l &= (v_c - \omega r_b) \hat{i} \cdot \hat{i} \nonumber \\
  v_l &= v_c - \omega r_b \label{eq:diff_vl}
\end{align}

Next, we'll derive $v_r$.
\begin{align*}
  \vec{v}_r &= v_c \hat{i} + \omega \hat{k} \times r_b \hat{j} \\
  \vec{v}_r &= v_c \hat{i} + \omega r_b \hat{i} \\
  \vec{v}_r &= (v_c + \omega r_b) \hat{i}
\end{align*}

Now, project this vector onto the right wheel, which is pointed in the $\hat{i}$
direction.
\begin{equation*}
  v_r = (v_c + \omega r_b) \hat{i} \cdot \frac{\hat{i}}{\norm{\hat{i}}}
\end{equation*}

The magnitude of $\hat{i}$ is $1$, so the denominator cancels.
\begin{align}
  v_r &= (v_c + \omega r_b) \hat{i} \cdot \hat{i} \nonumber \\
  v_r &= v_c + \omega r_b \label{eq:diff_vr}
\end{align}

So the two inverse kinematic equations are as follows.
\begin{align}
  v_l &= v_c - \omega r_b \\
  v_r &= v_c + \omega r_b
\end{align}

\subsubsection{Forward kinematics}

The mapping from the left and right wheel velocities $v_l$ and $v_r$ to $v$ and
$\omega$ is derived as follows.
\begin{align}
  v_r &= v_c + \omega r_b \nonumber \\
  v_c &= v_r - \omega r_b \label{eq:diff_drive_v_c}
\end{align}

Substitute equation \eqref{eq:diff_drive_v_c} equation for $v_l$.
\begin{align*}
  v_l &= v_c - \omega r_b \\
  v_l &= (v_r - \omega r_b) - \omega r_b \\
  v_l &= v_r - 2\omega r_b \\
  2\omega r_b &= v_r - v_l \\
  \omega &= \frac{v_r - v_l}{2 r_b}
\end{align*}

Substitute this back into equation \eqref{eq:diff_drive_v_c}.
\begin{align*}
  v_c &= v_r - \omega r_b \\
  v_c &= v_r - \left(\frac{v_r - v_l}{2 r_b}\right) r_b \\
  v_c &= v_r - \frac{v_r - v_l}{2} \\
  v_c &= v_r - \frac{v_r}{2} + \frac{v_l}{2} \\
  v_c &= \frac{v_r + v_l}{2}
\end{align*}

So the two forward kinematic equations are as follows.
\begin{align}
  v_c &= \frac{v_r + v_l}{2} \\
  \omega &= \frac{v_r - v_l}{2 r_b}
\end{align}

\subsection{Mecanum drive}

A mecanum drive has four wheels, one on each corner of a rectangular chassis.
The wheels have rollers offset at $45$ degrees (whether it's clockwise or not
varies per wheel). The forces they generate when moving forward are shown in
figure \ref{fig:mecanum_drive_fbd}.
\begin{bookfigure}
  \begin{tikzpicture}[auto, >=latex']
    %\draw [help lines] (-4,-3) grid (4,3);

    % Coordinate axes
    \begin{scope}[xshift=2cm,yshift=-2cm]
      \draw[->] (0,0) -- (-1,0) node[below] {\small +$y$};
      \draw[->] (0,0) -- (0,1) node[right] {\small +$x$};
    \end{scope}

    % Shared front axle
    \begin{scope}[xshift=0cm,yshift=0.7cm]
      \fill[black] (-1.2,-0.1) rectangle (1.2,0.1);
    \end{scope}

    % Shared rear axle
    \begin{scope}[xshift=0cm,yshift=-0.7cm]
      \fill[black] (-1.2,-0.1) rectangle (1.2,0.1);
    \end{scope}

    % Chassis
    \filldraw[draw=black,fill=white] (-1,-1) rectangle (1,1);

    % Front-left wheel
    \begin{scope}[xshift=-1.3cm,yshift=0.7cm]
      % Wheel vector
      \draw[->,thick,rotate around={-45:(0,0)}] (0,0) -- (0,1)
        node[left] {\small $v_{fl}$};

      % Wheel
      \filldraw[draw=black,fill=lightgray] (-0.2,-0.45) rectangle (0.2,0.45);
      \begin{scope}[xshift=0cm,yshift=-0.3cm]
        \filldraw[draw=black,fill=lightgray,rotate around={45:(0,0)}]
          (-0.1,-0.2) rectangle (0.1,0.2);
      \end{scope}
      \filldraw[draw=black,fill=lightgray,rotate around={45:(0,0)}]
        (-0.1,-0.2) rectangle (0.1,0.2);
      \begin{scope}[xshift=0cm,yshift=0.3cm]
        \filldraw[draw=black,fill=lightgray,rotate around={45:(0,0)}]
          (-0.1,-0.2) rectangle (0.1,0.2);
      \end{scope}
    \end{scope}

    % Front-right wheel
    \begin{scope}[xshift=1.3cm,yshift=0.7cm]
      % Wheel vector
      \draw[->,thick,rotate around={45:(0,0)}] (0,0) -- (0,1)
        node[right] {\small $v_{fr}$};

      % Wheel
      \filldraw[draw=black,fill=lightgray] (-0.2,-0.45) rectangle (0.2,0.45);
      \begin{scope}[xshift=0cm,yshift=-0.3cm]
        \filldraw[draw=black,fill=lightgray,rotate around={-45:(0,0)}]
          (-0.1,-0.2) rectangle (0.1,0.2);
      \end{scope}
      \filldraw[draw=black,fill=lightgray,rotate around={-45:(0,0)}]
        (-0.1,-0.2) rectangle (0.1,0.2);
      \begin{scope}[xshift=0cm,yshift=0.3cm]
        \filldraw[draw=black,fill=lightgray,rotate around={-45:(0,0)}]
          (-0.1,-0.2) rectangle (0.1,0.2);
      \end{scope}
    \end{scope}

    % Rear-left wheel
    \begin{scope}[xshift=-1.3cm,yshift=-0.7cm]
      % Wheel vector
      \draw[->,thick,rotate around={45:(0,0)}] (0,0) -- (0,1)
        node[left] {\small $v_{rl}$};

      % Wheel
      \filldraw[draw=black,fill=lightgray] (-0.2,-0.45) rectangle (0.2,0.45);
      \begin{scope}[xshift=0cm,yshift=-0.3cm]
        \filldraw[draw=black,fill=lightgray,rotate around={-45:(0,0)}]
          (-0.1,-0.2) rectangle (0.1,0.2);
      \end{scope}
      \filldraw[draw=black,fill=lightgray,rotate around={-45:(0,0)}]
        (-0.1,-0.2) rectangle (0.1,0.2);
      \begin{scope}[xshift=0cm,yshift=0.3cm]
        \filldraw[draw=black,fill=lightgray,rotate around={-45:(0,0)}]
          (-0.1,-0.2) rectangle (0.1,0.2);
      \end{scope}
    \end{scope}

    % Rear-right wheel
    \begin{scope}[xshift=1.3cm,yshift=-0.7cm]
      % Wheel vector
      \draw[->,thick,rotate around={-45:(0,0)}] (0,0) -- (0,1)
        node[right] {\small $v_{rr}$};

      % Wheel
      \filldraw[draw=black,fill=lightgray] (-0.2,-0.45) rectangle (0.2,0.45);
      \begin{scope}[xshift=0cm,yshift=-0.3cm]
        \filldraw[draw=black,fill=lightgray,rotate around={45:(0,0)}]
          (-0.1,-0.2) rectangle (0.1,0.2);
      \end{scope}
      \filldraw[draw=black,fill=lightgray,rotate around={45:(0,0)}]
        (-0.1,-0.2) rectangle (0.1,0.2);
      \begin{scope}[xshift=0cm,yshift=0.3cm]
        \filldraw[draw=black,fill=lightgray,rotate around={45:(0,0)}]
          (-0.1,-0.2) rectangle (0.1,0.2);
      \end{scope}
    \end{scope}
  \end{tikzpicture}

  \caption{Mecanum drive free body diagram}
  \label{fig:mecanum_drive_fbd}
\end{bookfigure}

Note that the velocity of the wheel is the same as the velocity in the diagram
for the purposes of feedback control. The rollers on the wheel redirect the
velocity vector.

\subsubsection{Inverse kinematics}

First, we'll derive the front-left wheel kinematics.
\begin{align*}
  \vec{v}_{fl} &= v_x \hat{i} + v_y \hat{j} +
    \omega \hat{k} \times (r_{fl_x} \hat{i} + r_{fl_y} \hat{j}) \\
  \vec{v}_{fl} &= v_x \hat{i} + v_y \hat{j} +
    \omega r_{fl_x} \hat{j} - \omega r_{fl_y} \hat{i} \\
  \vec{v}_{fl} &= (v_x - \omega r_{fl_y}) \hat{i} +
    (v_y + \omega r_{fl_x}) \hat{j}
\end{align*}

Project the front-left wheel onto its wheel vector.
\begin{align}
  v_{fl} &= ((v_x - \omega r_{fl_y}) \hat{i} + (v_y + \omega r_{fl_x}) \hat{j})
    \cdot \frac{\hat{i} - \hat{j}}{\sqrt{2}} \nonumber \\
  v_{fl} &= ((v_x - \omega r_{fl_y}) - (v_y + \omega r_{fl_x}))
    \frac{1}{\sqrt{2}} \nonumber \\
  v_{fl} &= (v_x - \omega r_{fl_y} - v_y - \omega r_{fl_x})
    \frac{1}{\sqrt{2}} \nonumber \\
  v_{fl} &= (v_x - v_y - \omega r_{fl_y} - \omega r_{fl_x})
    \frac{1}{\sqrt{2}} \nonumber \\
  v_{fl} &= (v_x - v_y - \omega (r_{fl_x} + r_{fl_y}))
    \frac{1}{\sqrt{2}} \nonumber \\
  v_{fl} &= \frac{1}{\sqrt{2}} v_x - \frac{1}{\sqrt{2}} v_y -
    \frac{1}{\sqrt{2}} \omega (r_{fl_x} + r_{fl_y})
\end{align}

Next, we'll derive the front-right wheel kinematics.
\begin{align*}
  \vec{v}_{fr} &= v_x \hat{i} + v_y \hat{j} +
    \omega \hat{k} \times (r_{fr_x} \hat{i} + r_{fr_y} \hat{j}) \\
  \vec{v}_{fr} &= v_x \hat{i} + v_y \hat{j} +
    \omega r_{fr_x} \hat{j} - \omega r_{fr_y} \hat{i} \\
  \vec{v}_{fr} &= (v_x - \omega r_{fr_y}) \hat{i} +
    (v_y + \omega r_{fr_x}) \hat{j}
\end{align*}

Project the front-right wheel onto its wheel vector.
\begin{align}
  v_{fr} &= ((v_x - \omega r_{fr_y}) \hat{i} + (v_y + \omega r_{fr_x}) \hat{j})
    \cdot (\hat{i} + \hat{j}) \frac{1}{\sqrt{2}} \nonumber \\
  v_{fr} &= ((v_x - \omega r_{fr_y}) + (v_y + \omega r_{fr_x}))
    \frac{1}{\sqrt{2}} \nonumber \\
  v_{fr} &= (v_x - \omega r_{fr_y} + v_y + \omega r_{fr_x})
    \frac{1}{\sqrt{2}} \nonumber \\
  v_{fr} &= (v_x + v_y - \omega r_{fr_y} + \omega r_{fr_x})
    \frac{1}{\sqrt{2}} \nonumber \\
  v_{fr} &= (v_x + v_y + \omega (r_{fr_x} - r_{fr_y}))
    \frac{1}{\sqrt{2}} \nonumber \\
  v_{fr} &= \frac{1}{\sqrt{2}} v_x + \frac{1}{\sqrt{2}} v_y +
    \frac{1}{\sqrt{2}} \omega (r_{fr_x} - r_{fr_y})
\end{align}

Next, we'll derive the rear-left wheel kinematics.
\begin{align*}
  \vec{v}_{rl} &= v_x \hat{i} + v_y \hat{j} +
    \omega \hat{k} \times (r_{rl_x} \hat{i} + r_{rl_y} \hat{j}) \\
  \vec{v}_{rl} &= v_x \hat{i} + v_y \hat{j} +
    \omega r_{rl_x} \hat{j} - \omega r_{rl_y} \hat{i} \\
  \vec{v}_{rl} &= (v_x - \omega r_{rl_y}) \hat{i} +
    (v_y + \omega r_{rl_x}) \hat{j}
\end{align*}

Project the rear-left wheel onto its wheel vector.
\begin{align}
  v_{rl} &= ((v_x - \omega r_{rl_y}) \hat{i} + (v_y + \omega r_{rl_x}) \hat{j})
    \cdot (\hat{i} + \hat{j}) \frac{1}{\sqrt{2}} \nonumber \\
  v_{rl} &= ((v_x - \omega r_{rl_y}) + (v_y + \omega r_{rl_x}))
    \frac{1}{\sqrt{2}} \nonumber \\
  v_{rl} &= (v_x - \omega r_{rl_y} + v_y + \omega r_{rl_x})
    \frac{1}{\sqrt{2}} \nonumber \\
  v_{rl} &= (v_x + v_y - \omega r_{rl_y} + \omega r_{rl_x})
    \frac{1}{\sqrt{2}} \nonumber \\
  v_{rl} &= (v_x + v_y + \omega (r_{rl_x} - r_{rl_y}))
    \frac{1}{\sqrt{2}} \nonumber \\
  v_{rl} &= \frac{1}{\sqrt{2}} v_x + \frac{1}{\sqrt{2}} v_y +
    \frac{1}{\sqrt{2}} \omega (r_{rl_x} - r_{rl_y})
\end{align}

Next, we'll derive the rear-right wheel kinematics.
\begin{align*}
  \vec{v}_{rr} &= v_x \hat{i} + v_y \hat{j} +
    \omega \hat{k} \times (r_{rr_x} \hat{i} + r_{rr_y} \hat{j}) \\
  \vec{v}_{rr} &= v_x \hat{i} + v_y \hat{j} +
    \omega r_{rr_x} \hat{j} - \omega r_{rr_y} \hat{i} \\
  \vec{v}_{rr} &= (v_x - \omega r_{rr_y}) \hat{i} +
    (v_y + \omega r_{rr_x}) \hat{j}
\end{align*}

Project the rear-right wheel onto its wheel vector.
\begin{align}
  v_{rr} &= ((v_x - \omega r_{rr_y}) \hat{i} + (v_y + \omega r_{rr_x}) \hat{j})
    \cdot \frac{\hat{i} - \hat{j}}{\sqrt{2}} \nonumber \\
  v_{rr} &= ((v_x - \omega r_{rr_y}) - (v_y + \omega r_{rr_x}))
    \frac{1}{\sqrt{2}} \nonumber \\
  v_{rr} &= (v_x - \omega r_{rr_y} - v_y - \omega r_{rr_x})
    \frac{1}{\sqrt{2}} \nonumber \\
  v_{rr} &= (v_x - v_y - \omega r_{rr_y} - \omega r_{rr_x})
    \frac{1}{\sqrt{2}} \nonumber \\
  v_{rr} &= (v_x - v_y - \omega (r_{rr_x} + r_{rr_y}))
    \frac{1}{\sqrt{2}} \nonumber \\
  v_{rr} &= \frac{1}{\sqrt{2}} v_x - \frac{1}{\sqrt{2}} v_y -
    \frac{1}{\sqrt{2}} \omega (r_{rr_x} + r_{rr_y})
\end{align}

This gives the following inverse kinematic equations.
\begin{align*}
  v_{fl} &= \frac{1}{\sqrt{2}} v_x - \frac{1}{\sqrt{2}} v_y -
    \frac{1}{\sqrt{2}} \omega (r_{fl_x} + r_{fl_y}) \\
  v_{fr} &= \frac{1}{\sqrt{2}} v_x + \frac{1}{\sqrt{2}} v_y +
    \frac{1}{\sqrt{2}} \omega (r_{fr_x} - r_{fr_y}) \\
  v_{rl} &= \frac{1}{\sqrt{2}} v_x + \frac{1}{\sqrt{2}} v_y +
    \frac{1}{\sqrt{2}} \omega (r_{rl_x} - r_{rl_y}) \\
  v_{rr} &= \frac{1}{\sqrt{2}} v_x - \frac{1}{\sqrt{2}} v_y -
    \frac{1}{\sqrt{2}} \omega (r_{rr_x} + r_{rr_y})
\end{align*}

Now we'll factor them out into matrices.
\begin{align}
  \begin{bmatrix}
    v_{fl} \\
    v_{fr} \\
    v_{rl} \\
    v_{rr}
  \end{bmatrix} &=
  \begin{bmatrix}
    \frac{1}{\sqrt{2}} & -\frac{1}{\sqrt{2}} &
      -\frac{1}{\sqrt{2}}(r_{fl_x} + r_{fl_y}) \\
    \frac{1}{\sqrt{2}} &  \frac{1}{\sqrt{2}} &
      \frac{1}{\sqrt{2}}(r_{fr_x} - r_{fr_y}) \\
    \frac{1}{\sqrt{2}} &  \frac{1}{\sqrt{2}} &
      \frac{1}{\sqrt{2}}(r_{rl_x} - r_{rl_y}) \\
    \frac{1}{\sqrt{2}} & -\frac{1}{\sqrt{2}} &
      -\frac{1}{\sqrt{2}}(r_{rr_x} + r_{rr_y})
  \end{bmatrix}
  \begin{bmatrix}
    v_x \\
    v_y \\
    \omega
  \end{bmatrix} \nonumber \\
  \begin{bmatrix}
    v_{fl} \\
    v_{fr} \\
    v_{rl} \\
    v_{rr}
  \end{bmatrix} &= \frac{1}{\sqrt{2}}
  \begin{bmatrix}
    1 & -1  & -(r_{fl_x} + r_{fl_y}) \\
    1 &  1  &  (r_{fr_x} - r_{fr_y}) \\
    1 &  1  &  (r_{rl_x} - r_{rl_y}) \\
    1 & -1  & -(r_{rr_x} + r_{rr_y})
  \end{bmatrix}
  \begin{bmatrix}
    v_x \\
    v_y \\
    \omega
  \end{bmatrix}
\end{align}

\subsubsection{Forward kinematics}

Let $\mat{M}$ be the $4 \times 3$ inverse kinematics matrix above including the
$\frac{1}{\sqrt{2}}$ factor. The forward kinematics are
\begin{equation}
  \begin{bmatrix}
    v_x \\
    v_y \\
    \omega
  \end{bmatrix} =
  \mat{M}^\dagger
  \begin{bmatrix}
    v_{fl} \\
    v_{fr} \\
    v_{rl} \\
    v_{rr}
  \end{bmatrix}
\end{equation}

where $\mat{M}^\dagger$ is the pseudoinverse of $\mat{M}$.

\subsection{Swerve drive}

A swerve drive has an arbitrary number of wheels which can rotate in place
independent of the chassis. The forces they generate are shown in figure
\ref{fig:swerve_drive_fbd}.
\begin{bookfigure}
  \begin{tikzpicture}[auto, >=latex']
    %\draw [help lines] (-4,-3) grid (4,3);

    % Coordinate axes
    \begin{scope}[xshift=2cm,yshift=-2cm]
      \draw[->] (0,0) -- (-1,0) node[below] {\small +$y$};
      \draw[->] (0,0) -- (0,1) node[right] {\small +$x$};
    \end{scope}

    % Chassis
    \filldraw[draw=black,fill=white] (-1,-1) rectangle (1,1);

    % Front-left wheel
    \begin{scope}[xshift=-0.75cm,yshift=0.75cm]
      % Wheel vector
      \draw[->,thick,rotate around={-45:(0,0)}] (0,0) -- (0,1)
        node[left] {\small $v_1$};

      % Wheel
      \draw[black] (0,0) circle (0.25);
      \filldraw[draw=black,fill=lightgray,rotate around={-45:(0,0)}]
        (-0.1,-0.2) rectangle (0.1,0.2);
    \end{scope}

    % Front-right wheel
    \begin{scope}[xshift=0.75cm,yshift=0.75cm]
      % Wheel vector
      \draw[->,thick,rotate around={-135:(0,0)}] (0,0) -- (0,1)
        node[right] {\small $v_2$};

      % Wheel
      \draw[black] (0,0) circle (0.25);
      \filldraw[draw=black,fill=lightgray,rotate around={-135:(0,0)}]
        (-0.1,-0.2) rectangle (0.1,0.2);
    \end{scope}

    % Rear-left wheel
    \begin{scope}[xshift=-0.75cm,yshift=-0.75cm]
      % Wheel vector
      \draw[->,thick,rotate around={45:(0,0)}] (0,0) -- (0,1)
        node[left] {\small $v_3$};

      % Wheel
      \draw[black] (0,0) circle (0.25);
      \filldraw[draw=black,fill=lightgray,rotate around={45:(0,0)}]
        (-0.1,-0.2) rectangle (0.1,0.2);
    \end{scope}

    % Rear-right wheel
    \begin{scope}[xshift=0.75cm,yshift=-0.75cm]
      % Wheel vector
      \draw[->,thick,rotate around={135:(0,0)}] (0,0) -- (0,1)
        node[right] {\small $v_4$};

      % Wheel
      \draw[black] (0,0) circle (0.25);
      \filldraw[draw=black,fill=lightgray,rotate around={135:(0,0)}]
        (-0.1,-0.2) rectangle (0.1,0.2);
    \end{scope}
  \end{tikzpicture}

  \caption{Swerve drive free body diagram}
  \label{fig:swerve_drive_fbd}
\end{bookfigure}

\subsubsection{Inverse kinematics}
\begin{align*}
  \vec{v}_{wheel1} &= \vec{v}_{robot} + \vec{\omega}_{robot} \times
    \vec{r}_{robot2wheel1} \\
  \vec{v}_{wheel2} &= \vec{v}_{robot} + \vec{\omega}_{robot} \times
    \vec{r}_{robot2wheel2} \\
  \vec{v}_{wheel3} &= \vec{v}_{robot} + \vec{\omega}_{robot} \times
    \vec{r}_{robot2wheel3} \\
  \vec{v}_{wheel4} &= \vec{v}_{robot} + \vec{\omega}_{robot} \times
    \vec{r}_{robot2wheel4}
\end{align*}

where $\vec{v}_{wheel}$ is the wheel velocity vector, $\vec{v}_{robot}$ is the
robot velocity vector, $\vec{\omega}_{robot}$ is the robot angular velocity
vector, $\vec{r}_{robot2wheel}$ is the displacement vector from the robot's
center of rotation to the wheel\footnote{The robot's center of rotation need not
coincide with the robot's geometric center.},
$\vec{v}_{robot} = v_x \hat{i} + v_y \hat{j}$, and
$\vec{r}_{robot2wheel} = r_x \hat{i} + r_y \hat{j}$. The number suffixes denote
a specific wheel in figure \ref{fig:swerve_drive_fbd}.
\begin{align*}
  \vec{v}_1 &= v_x \hat{i} + v_y \hat{j} + \omega \hat{k} \times
    (r_{1x} \hat{i} + r_{1y} \hat{j}) \\
  \vec{v}_2 &= v_x \hat{i} + v_y \hat{j} + \omega \hat{k} \times
    (r_{2x} \hat{i} + r_{2y} \hat{j}) \\
  \vec{v}_3 &= v_x \hat{i} + v_y \hat{j} + \omega \hat{k} \times
    (r_{3x} \hat{i} + r_{3y} \hat{j}) \\
  \vec{v}_4 &= v_x \hat{i} + v_y \hat{j} + \omega \hat{k} \times
    (r_{4x} \hat{i} + r_{4y} \hat{j})
\end{align*}
\begin{align*}
  \vec{v}_1 &= v_x \hat{i} + v_y \hat{j} +
    (\omega r_{1x} \hat{j} - \omega r_{1y} \hat{i}) \\
  \vec{v}_2 &= v_x \hat{i} + v_y \hat{j} +
    (\omega r_{2x} \hat{j} - \omega r_{2y} \hat{i}) \\
  \vec{v}_3 &= v_x \hat{i} + v_y \hat{j} +
    (\omega r_{3x} \hat{j} - \omega r_{3y} \hat{i}) \\
  \vec{v}_4 &= v_x \hat{i} + v_y \hat{j} +
    (\omega r_{4x} \hat{j} - \omega r_{4y} \hat{i})
\end{align*}
\begin{align*}
  \vec{v}_1 &= v_x \hat{i} + v_y \hat{j} +
    \omega r_{1x} \hat{j} - \omega r_{1y} \hat{i} \\
  \vec{v}_2 &= v_x \hat{i} + v_y \hat{j} +
    \omega r_{2x} \hat{j} - \omega r_{2y} \hat{i} \\
  \vec{v}_3 &= v_x \hat{i} + v_y \hat{j} +
    \omega r_{3x} \hat{j} - \omega r_{3y} \hat{i} \\
  \vec{v}_4 &= v_x \hat{i} + v_y \hat{j} +
    \omega r_{4x} \hat{j} - \omega r_{4y} \hat{i}
\end{align*}
\begin{align*}
  \vec{v}_1 &= v_x \hat{i} - \omega r_{1y} \hat{i} +
    v_y \hat{j} + \omega r_{1x} \hat{j} \\
  \vec{v}_2 &= v_x \hat{i} - \omega r_{2y} \hat{i} +
    v_y \hat{j} + \omega r_{2x} \hat{j} \\
  \vec{v}_3 &= v_x \hat{i} - \omega r_{3y} \hat{i} +
    v_y \hat{j} + \omega r_{3x} \hat{j} \\
  \vec{v}_4 &= v_x \hat{i} - \omega r_{4y} \hat{i} +
    v_y \hat{j} + \omega r_{4x} \hat{j}
\end{align*}
\begin{align*}
  \vec{v}_1 &= (v_x - \omega r_{1y}) \hat{i} + (v_y + \omega r_{1x}) \hat{j} \\
  \vec{v}_2 &= (v_x - \omega r_{2y}) \hat{i} + (v_y + \omega r_{2x}) \hat{j} \\
  \vec{v}_3 &= (v_x - \omega r_{3y}) \hat{i} + (v_y + \omega r_{3x}) \hat{j} \\
  \vec{v}_4 &= (v_x - \omega r_{4y}) \hat{i} + (v_y + \omega r_{4x}) \hat{j}
\end{align*}

Separate the i-hat components into independent equations.
\begin{align*}
  v_{1x} &= v_x - \omega r_{1y} \\
  v_{2x} &= v_x - \omega r_{2y} \\
  v_{3x} &= v_x - \omega r_{3y} \\
  v_{4x} &= v_x - \omega r_{4y}
\end{align*}

Separate the j-hat components into independent equations.
\begin{align*}
  v_{1y} &= v_y + \omega r_{1x} \\
  v_{2y} &= v_y + \omega r_{2x} \\
  v_{3y} &= v_y + \omega r_{3x} \\
  v_{4y} &= v_y + \omega r_{4x}
\end{align*}

Now we'll factor them out into matrices.
\begin{equation*}
  \begin{bmatrix}
    v_{1x} \\
    v_{2x} \\
    v_{3x} \\
    v_{4x} \\
    v_{1y} \\
    v_{2y} \\
    v_{3y} \\
    v_{4y}
  \end{bmatrix} =
  \begin{bmatrix}
    1 & 0 & -r_{1y} \\
    1 & 0 & -r_{2y} \\
    1 & 0 & -r_{3y} \\
    1 & 0 & -r_{4y} \\
    0 & 1 &  r_{1x} \\
    0 & 1 &  r_{2x} \\
    0 & 1 &  r_{3x} \\
    0 & 1 &  r_{4x}
  \end{bmatrix}
  \begin{bmatrix}
    v_x \\
    v_y \\
    \omega
  \end{bmatrix}
\end{equation*}

Rearrange the rows so the $x$ and $y$ components are in adjacent rows.
\begin{equation}
  \label{eq:swerve_ik}
  \begin{bmatrix}
    v_{1x} \\
    v_{1y} \\
    v_{2x} \\
    v_{2y} \\
    v_{3x} \\
    v_{3y} \\
    v_{4x} \\
    v_{4y}
  \end{bmatrix} =
  \begin{bmatrix}
    1 & 0 & -r_{1y} \\
    0 & 1 &  r_{1x} \\
    1 & 0 & -r_{2y} \\
    0 & 1 &  r_{2x} \\
    1 & 0 & -r_{3y} \\
    0 & 1 &  r_{3x} \\
    1 & 0 & -r_{4y} \\
    0 & 1 &  r_{4x}
  \end{bmatrix}
  \begin{bmatrix}
    v_x \\
    v_y \\
    \omega
  \end{bmatrix}
\end{equation}

To convert the swerve module $x$ and $y$ velocity components to a velocity and
heading, use the Pythagorean theorem and arctangent respectively. Here's an
example for module 1.
\begin{align}
  v_1 &= \sqrt{v_{1x}^2 + v_{1y}^2} \\
  \theta_1 &= \tan^{-1}\left(\frac{v_{1y}}{v_{1x}}\right)
\end{align}

\subsubsection{Forward kinematics}

Let $\mat{M}$ be the $8 \times 3$ inverse kinematics matrix from equation
\eqref{eq:swerve_ik}. The forward kinematics are
\begin{equation}
  \begin{bmatrix}
    v_x \\
    v_y \\
    \omega
  \end{bmatrix} =
  \mat{M}^\dagger
  \begin{bmatrix}
    v_{1x} \\
    v_{1y} \\
    v_{2x} \\
    v_{2y} \\
    v_{3x} \\
    v_{3y} \\
    v_{4x} \\
    v_{4y}
  \end{bmatrix}
\end{equation}

where $\mat{M}^\dagger$ is the pseudoinverse of $\mat{M}$.
