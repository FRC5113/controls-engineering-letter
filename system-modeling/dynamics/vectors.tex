\section{Vectors}

Vectors are quantities with a magnitude and a direction. Vectors in
three-dimensional space have a coordinate for each spatial direction $x$, $y$,
and $z$. Let's take the vector $\vec{a} = \langle 1, 2, 3 \rangle$. $\vec{a}$ is
a three-dimensional vector that describes a movement 1 unit in the $x$
direction, 2 units in the $y$ direction, and 3 units in the $z$ direction.

We define $\hat{i}$, $\hat{j}$, and $\hat{k}$ as vectors that represent the
fundamental movements one can make the three-dimensional space: 1 unit of
movement in the $x$ direction, 1 unit of movement $y$ direction, and 1 unit of
movement in the $z$ direction respectively. These three vectors form a
\textit{basis} of three-dimensional space because copies of them can be added
together to reach any point in three-dimensional space.
\begin{align*}
  \hat{i} = \langle 1, 0, 0 \rangle \\
  \hat{j} = \langle 0, 1, 0 \rangle \\
  \hat{k} = \langle 0, 0, 1 \rangle
\end{align*}

We can also write the vector $\vec{a}$ in terms of these basis vectors.
\begin{equation*}
  \vec{a} = 1\hat{i} + 2\hat{j} + 3\hat{k}
\end{equation*}

\subsection{Basic vector operations}

We will now show this is equivalent to the original notation through some vector
mathematics. First, we'll substitute in the values for the basis vectors.
\begin{align*}
  \vec{a} &= 1\langle 1, 0, 0 \rangle + 2\langle 0, 1, 0 \rangle +
    3\langle 0, 0, 1 \rangle
  \intertext{Scalars are multiplied component-wise with vectors.}
  \vec{a} &= \langle 1, 0, 0 \rangle + \langle 0, 2, 0 \rangle +
    \langle 0, 0, 3 \rangle \\
  \intertext{Vectors are added by summing each of their components.}
  \vec{a} &= \langle 1, 2, 3 \rangle
\end{align*}

\subsection{Cross product}

The cross product is denoted by $\times$. The cross product of the basis vectors
are computed as follows.
\begin{align*}
  \hat{i} \times \hat{j} &= \hat{k} \\
  \hat{j} \times \hat{k} &= \hat{i} \\
  \hat{k} \times \hat{i} &= \hat{j}
\end{align*}

They proceed in a cyclic fashion through i, j, and k. If a vector is crossed
with itself, it produces the zero vector (a scalar zero for each coordinate).
The cross products of the basis vectors in the opposite order progress backwards
and include a negative sign.
\begin{align*}
  \hat{i} \times \hat{k} &= -\hat{j} \\
  \hat{k} \times \hat{j} &= -\hat{i} \\
  \hat{j} \times \hat{i} &= -\hat{k}
\end{align*}

Given vectors
$\vec{u} = a\hat{i} + b\hat{j} + c\hat{k}$ and
$\vec{v} = d\hat{i} + e\hat{j} + f\hat{k}$, $\vec{u} \times \vec{v}$ is computed
using the distributive property.
\begin{align*}
  \vec{u} \times \vec{v} &= (a\hat{i} + b\hat{j} + c\hat{k}) \times
    (d\hat{i} + e\hat{j} + f\hat{k}) \\
  \vec{u} \times \vec{v} &= ad(\hat{i} \times \hat{i}) + ae(\hat{i} \times
    \hat{j}) + af(\hat{i} \times \hat{k}) \\
      &\qquad + bd(\hat{j} \times \hat{i}) + be(\hat{j} \times \hat{j}) \\
      &\qquad + bf(\hat{j} \times \hat{k}) \\
      &\qquad + cd(\hat{k} \times \hat{i}) + ce(\hat{k} \times \hat{j}) +
    cf(\hat{k} \times \hat{k}) \\
  \vec{u} \times \vec{v} &= ae\hat{k} + af(-\hat{j}) + bd(-\hat{k}) +
    bf\hat{i} + cd\hat{j} + ce(-\hat{i}) \\
  \vec{u} \times \vec{v} &= ae\hat{k} - af\hat{j} - bd\hat{k} + bf\hat{i} +
    cd\hat{j} - ce\hat{i} \\
  \vec{u} \times \vec{v} &= (bf - ce)\hat{i} + (cd - af)\hat{j} +
    (ae - bd)\hat{k}
\end{align*}
