\section{Pendulum}

Kinematics and dynamics are a rather large topics, so for now, we'll just focus
on the basics required for working with the \glspl{model} in this book. We'll
derive the same \gls{model}, a pendulum, using three approaches: sum of forces,
sum of torques, and conservation of energy.
\begin{bookfigure}
  \begin{subfigure}{0.5\textwidth}
    \centering
    \begin{tikzpicture}
      % Save length of g-vector and theta to macros
      \pgfmathsetmacro{\Gvec}{1.5}
      \pgfmathsetmacro{\myAngle}{30}
      % Calculate lengths of vector components
      \pgfmathsetmacro{\Gcos}{\Gvec*cos(\myAngle)}
      \pgfmathsetmacro{\Gsin}{\Gvec*sin(\myAngle)}

      \coordinate (centro) at (0,0);
      \draw[dashed,gray,-] (centro) -- ++ (0,-3.5)
        node (mary) [black,below] {$ $};
      \draw[thick] (centro) -- ++(270+\myAngle:3) coordinate (bob);
      \path pic [draw,->,"$\theta$",angle eccentricity=1.5]
        {angle=mary--centro--bob};
      \draw [draw=violet,-stealth] (bob) -- ($(bob)!-\Gcos cm!(centro)$)
        coordinate (gcos)
        node[midway,above right] {$mg\cos\theta$};
      \draw [dashed,draw=red,-stealth] (bob) -- ($(bob)!2*\Gsin cm!90:(centro)$)
        coordinate node[midway,above right] {};
      \draw [draw=violet,-stealth] (bob) -- ($(bob)!\Gsin cm!90:(centro)$)
        coordinate (gsin)
        node[midway,above left] {$mg\sin\theta$};
      \draw [draw=blue,-stealth] (bob) -- ++(0,-\Gvec)
        coordinate (g)
        node[near end,left] {$mg$};
      \pic [draw,->,"$\theta$",angle eccentricity=1.5] {angle=g--bob--gcos};
      \filldraw [fill=black!40,draw=black] (bob) circle[radius=0.2];
    \end{tikzpicture}
    \caption{Force diagram of a pendulum}
    \label{subfig:force_pendulum}
  \end{subfigure}%
  \begin{subfigure}{0.5\textwidth}
    \centering
    \begin{tikzpicture}
      % Save length of g-vector and theta to macros
      \pgfmathsetmacro{\Gvec}{1.5}
      \pgfmathsetmacro{\myAngle}{30}
      % Calculate lengths of vector components
      \pgfmathsetmacro{\Gcos}{\Gvec*cos(\myAngle)}
      \pgfmathsetmacro{\Gsin}{\Gvec*sin(\myAngle)}

      \coordinate (centro) at (0,0);
      \coordinate (heightmes_lo) at (-1,0);
      \coordinate (heightmes_hi) at (-0.25,0);
      \coordinate (h) at (\Gcos/2,\Gsin);

      \draw[thick] (centro) -- ++(270+\myAngle:3) coordinate (bob_lo);
      \draw[dashed,gray,-] (centro) -- ++ (0,0 |- bob_lo)
        node (mary) [black,below]{$ $};
      \draw[dashed,gray,-] (heightmes_lo |- bob_lo) -- (bob_lo)
        node [black,below]{$ $};
      \draw[<->] (heightmes_lo) -- ++ (0,0 |- bob_lo)
        node [black,pos=0.5,left]{$y_1$};
      \pic [draw,->, "$\theta$",angle eccentricity=1.5]
        {angle=mary--centro--bob_lo};

      % Save length of g-vector and theta to macros
      \pgfmathsetmacro{\Gvec}{1.5cm}
      \pgfmathsetmacro{\myAngle}{45}
      % Calculate lengths of vector components
      \pgfmathsetmacro{\Gcos}{\Gvec*cos(\myAngle)}
      \pgfmathsetmacro{\Gsin}{\Gvec*sin(\myAngle)}

      \draw[gray,thick] (centro) -- ++(270+\myAngle:3) coordinate (bob_hi);
      \pic [draw,->,"$\theta_0$",angle eccentricity=1.5,angle radius=1cm]
        {angle=mary--centro--bob_hi};

      \draw[dashed,gray,-] (0,0 |- bob_hi) -- (bob_hi)
        node (mary) [black,below]{$ $};
      \draw[<->] (heightmes_hi) -- ++ (0,0 |- bob_hi)
        node (mary) [black,pos=0.5,left]{$y_0$};
      \draw[<->] (h |- bob_hi) -- (h |- bob_lo)
        node [black,pos=0.5,left]{$h$};

      % Path of pendulum
      \pic [draw,dashed,gray,<-,angle eccentricity=1.5,angle radius=2*\Gvec]
        {angle=mary--centro--bob_hi};

      % Pendulum balls
      \filldraw [fill=black!40,draw=black] (bob_lo) circle[radius=0.2];
      \filldraw [fill=black!20,draw=gray] (bob_hi) circle[radius=0.2];
    \end{tikzpicture}
    \caption{Trigonometry of a pendulum}
    \label{subfig:trig_pendulum}
  \end{subfigure}
  \caption{Pendulum force diagrams}
\end{bookfigure}

\subsection{Force derivation}
\index{physics!sum of forces}

Consider figure \ref{subfig:force_pendulum}, which shows the forces acting on a
pendulum.

Note that the path of the pendulum sweeps out an arc of a circle. The angle
$\theta$ is measured in radians. The blue arrow is the gravitational force
acting on the bob, and the violet arrows are that same force resolved into
components parallel and perpendicular to the bob's instantaneous motion. The
direction of the bob's instantaneous velocity always points along the red axis,
which is considered the tangential axis because its direction is always tangent
to the circle. Consider Newton's second law
\begin{equation*}
  F = ma
\end{equation*}

where $F$ is the sum of forces on the object, $m$ is mass, and $a$ is the
acceleration. Because we are only concerned with changes in speed, and because
the bob is forced to stay in a circular path, we apply Newton's equation to the
tangential axis only. The short violet arrow represents the component of the
gravitational force in the tangential axis, and trigonometry can be used to
determine its magnitude. Therefore
\begin{align*}
  -mg\sin\theta &= ma \\
  a &= -g\sin\theta
\end{align*}

where $g$ is the acceleration due to gravity near the surface of the earth. The
negative sign on the right hand side implies that $\theta$ and a always point in
opposite directions. This makes sense because when a pendulum swings further to
the left, we would expect it to accelerate back toward the right.

This linear acceleration $a$ along the red axis can be related to the change in
angle $\theta$ by the arc length formulas; $s$ is arc length and $l$ is the
length of the pendulum.
\begin{align}
  s &= l\theta \label{eq:arc_length} \\
  v &= \frac{ds}{dt} = l\frac{d\theta}{dt} \nonumber \\
  a &= \frac{d^2s}{dt^2} = l\frac{d^2\theta}{dt^2} \nonumber
\end{align}

Therefore
\begin{align*}
  l\frac{d^2\theta}{dt^2} &= -g\sin\theta \\
  \frac{d^2\theta}{dt^2} &= -\frac{g}{l}\sin\theta \\
  \ddot{\theta} &= -\frac{g}{l}\sin\theta
\end{align*}

\subsection{Torque derivation}
\index{physics!sum of torques}

The equation can be obtained using two definitions for torque.
\begin{equation*}
  \mtx{\tau} = \mtx{r} \times \mtx{F}
\end{equation*}

First start by defining the torque on the pendulum bob using the force due to
gravity.
\begin{equation*}
  \mtx{\tau} = \mtx{l} \times \mtx{F}_g
\end{equation*}

where $\mtx{l}$ is the length vector of the pendulum and $\mtx{F}_g$ is the
force due to gravity.

For now just consider the magnitude of the torque on the pendulum.
\begin{equation*}
  \lvert\tau\rvert = -mgl\sin\theta
\end{equation*}

where $m$ is the mass of the pendulum, $g$ is the acceleration due to gravity,
$l$ is the length of the pendulum and $\theta$ is the angle between the length
vector and the force due to gravity.

Next rewrite the angular momentum.
\begin{equation*}
  \mtx{L} = \mtx{r} \times \mtx{p} =
    m\mtx{r} \times (\mtx{\omega} \times \mtx{r})
\end{equation*}

Again just consider the magnitude of the angular momentum.
\begin{align*}
  \lvert\mtx{L}\rvert &= mr^2\omega \\
  \lvert\mtx{L}\rvert &= ml^2 \frac{d\theta}{dt} \\
  \frac{d}{dt}\lvert\mtx{L}\rvert &= ml^2 \frac{d^2\theta}{dt^2}
\end{align*}

According to $\tau = \frac{d\mtx{L}}{dt}$, we can just compare the magnitudes.
\begin{align*}
  -mgl\sin\theta &= ml^2\frac{d^2\theta}{dt^2} \\
  -\frac{g}{l}\sin\theta &= \frac{d^2\theta}{dt^2} \\
  \ddot{\theta} &= -\frac{g}{l}\sin\theta
\end{align*}

which is the same result from force analysis.

\subsection{Energy derivation}
\index{physics!conservation of energy}

The equation can also be obtained via the conservation of mechanical energy
principle: any object falling a vertical distance $h$ would acquire kinetic
energy equal to that which it lost to the fall. In other words, gravitational
potential energy is converted into kinetic energy. Change in potential energy is
given by
\begin{equation*}
  \Delta U = mgh
\end{equation*}

The change in kinetic energy (body started from rest) is given by
\begin{equation*}
  \Delta K = \frac{1}{2}mv^2
\end{equation*}

Since no energy is lost, the gain in one must be equal to the loss in the other
\begin{equation*}
  \frac{1}{2}mv^2 = mgh
\end{equation*}

The change in velocity for a given change in height can be expressed as
\begin{equation*}
  v = \sqrt{2gh}
\end{equation*}

Using equation \eqref{eq:arc_length}, this equation can be rewritten in terms of
$\frac{d\theta}{dt}$.
\begin{align}
  v = l\frac{d\theta}{dt} &= \sqrt{2gh} \nonumber \\
  \frac{d\theta}{dt} &= \frac{2gh}{l} \label{eq:energy_dtheta}
\end{align}

where $h$ is the vertical distance the pendulum fell. Look at figure \ref{subfig:trig_pendulum}, which presents the trigonometry of a pendulum. If the pendulum
starts its swing from some initial angle $\theta_0$, then $y_0$, the vertical
distance from the pivot point, is given by
\begin{equation*}
  y_0 = l\cos\theta_0
\end{equation*}

Similarly for $y_1$, we have
\begin{equation*}
  y_1 = l\cos\theta
\end{equation*}

Then $h$ is the difference of the two
\begin{equation*}
  h = l(\cos\theta - \cos\theta_0)
\end{equation*}

Substituting this into equation \eqref{eq:energy_dtheta} gives
\begin{equation*}
  \frac{d\theta}{dt} = \sqrt{\frac{2g}{l}(\cos\theta - \cos\theta_0)}
\end{equation*}

This equation is known as the first integral of motion. It gives the velocity in
terms of the location and includes an integration constant related to the
initial displacement ($\theta_0$). We can differentiate by applying the chain
rule with respect to time. Doing so gives the acceleration.
\begin{align*}
  \frac{d}{dt}\frac{d\theta}{dt} &=
    \frac{d}{dt}\sqrt{\frac{2g}{l}(\cos\theta - \cos\theta_0)} \\
  \frac{d^2\theta}{dt^2} &= \frac{1}{2}\frac
    {-\frac{2g}{l}\sin\theta}
    {\sqrt{\frac{2g}{l}(\cos\theta - \cos\theta_0)}}\frac{d\theta}{dt} \\
  \frac{d^2\theta}{dt^2} &= \frac{1}{2}\frac
    {-\frac{2g}{l}\sin\theta}
    {\sqrt{\frac{2g}{l}(\cos\theta - \cos\theta_0)}}
    \sqrt{\frac{2g}{l}(\cos\theta - \cos\theta_0)} \\
  \frac{d^2\theta}{dt^2} &= -\frac{g}{l}\sin\theta \\
  \ddot{\theta} &= -\frac{g}{l}\sin\theta
\end{align*}

which is the same result from force analysis.
