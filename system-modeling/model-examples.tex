\chapterimage{system-modeling.jpg}{Hills by northbound freeway between Santa Maria and Ventura}

\chapter{Model examples}

A \gls{model} is a set of differential equations describing how the \gls{system}
behaves over time. There are two common approaches for developing them.

\begin{enumerate}
  \item Collecting data on the physical system's behavior and performing
    \gls{system} identification with it.
  \item Using physics to derive the \gls{system}'s model from first principles.
\end{enumerate}

We'll use the second approach in this book.

The \glspl{model} derived here should cover most types of motion seen on an FRC
robot. Furthermore, they can be easily tweaked to describe many types of
mechanisms just by pattern-matching. There's only so many ways to hook up a mass
to a motor in FRC. The flywheel \gls{model} can be used for spinning mechanisms,
the elevator \gls{model} can be used for spinning mechanisms transformed to
linear motion, and the single-jointed arm \gls{model} can be used for rotating
servo mechanisms (it's just the flywheel \gls{model} augmented with a position
\gls{state}).

These \glspl{model} assume all motor controllers driving DC brushed motors are
set to brake mode instead of coast mode. Brake mode behaves the same as coast
mode except where the applied voltage is zero. In brake mode, the motor leads
are shorted together to prevent movement. In coast mode, the motor leads are an
open circuit.

\renewcommand*{\chapterpath}{\partpath/model-examples}
\section{Pendulum}

\subsection{State-space model}

Below is the \gls{model} for a pendulum
\begin{equation*}
  \ddot{\theta} = -\frac{g}{l}\sin\theta
\end{equation*}

where $\theta$ is the angle of the pendulum and $l$ is the length of the
pendulum.

Since state-space representation requires that only single derivatives be used,
they should be broken up as separate \glspl{state}. We'll reassign
$\dot{\theta}$ to be $\omega$ so the derivatives are easier to keep straight for
state-space representation.
\begin{align*}
  \dot{\omega} &= -\frac{g}{l}\sin\theta
  \intertext{Now separate the \glspl{state}.}
  \dot{\theta} &= \omega \\
  \dot{\omega} &= -\frac{g}{l} \sin\theta
  \intertext{This makes our state vector
    $\begin{bmatrix}\theta & \omega\end{bmatrix}\T$ and our nonlinear model the
    following.}
  f(\mat{x}, \mat{u}) &=
  \begin{bmatrix}
    \omega \\
    -\frac{g}{l}\sin\theta
  \end{bmatrix}
\end{align*}

\subsubsection{Linearization around $\theta = 0$}

To apply our tools for linear control theory, the \gls{model} must be a linear
combination of the \glspl{state} and \glspl{input} (addition and multiplication
by constants). Since this \gls{model} is nonlinear on account of the sine
function, we should \glslink{linearization}{linearize}
\index{nonlinear control!linearization} it.

Linearization finds a tangent line to the nonlinear dynamics at a desired point
in the state-space. The Taylor series is a way to approximate arbitrary
functions with polynomials, so we can use it for linearization.

The taylor series expansion for $\sin\theta$ around $\theta = 0$ is
$\theta - \frac{1}{6} \theta^3 + \frac{1}{120} \theta^5 - \ldots$. We'll take
just the first-order term $\theta$ to obtain a linear function.
\begin{align}
  \dot{\theta} &= \omega \nonumber \\
  \dot{\omega} &= -\frac{g}{l} \theta \nonumber
  \intertext{Now write the \gls{model} in state-space representation. We'll
    write out the system of equations with the zeroed variables included to
    assist with this.}
  \dot{\theta} &= \;\;\;\,0 \theta + 1 \omega \nonumber \\
  \dot{\omega} &= -\frac{g}{l} \theta + 0 \omega \nonumber
  \intertext{Factor out $\theta$ and $\omega$ into a column vector.}
  \dot{
  \begin{bmatrix}
    \theta \\
    \omega
  \end{bmatrix}} &=
  \begin{bmatrix}
    0 & 1 \\
    -\frac{g}{l} & 0
  \end{bmatrix}
  \begin{bmatrix}
    \theta \\
    \omega
  \end{bmatrix}
\end{align}

\subsubsection{Linearization with the Jacobian}

Here's the original nonlinear model in state-space representation.
\begin{align*}
  f(\mat{x}, \mat{u}) &=
  \begin{bmatrix}
    \omega \\
    -\frac{g}{l}\sin\theta
  \end{bmatrix}
  \intertext{If we want to linearize around an arbitrary point, we can take the
    Jacobian with respect to $\mat{x}$.}
  \frac{\partial f(\mat{x}, \mat{u})}{\partial\mat{x}} &=
  \begin{bmatrix}
    0 & 1 \\
    -\frac{g}{l}\cos\theta & 0
  \end{bmatrix}
\end{align*}

For full \gls{state} feedback, knowledge of all \glspl{state} is required. If
not all \glspl{state} are measured directly, an estimator can be used to
supplement them.

We may only be measuring $\theta$ in the pendulum example, not $\dot{\theta}$,
so we'll need to estimate the latter. The $\mat{C}$ matrix the \gls{observer}
would use in this case is
\begin{align*}
  \mat{C} &= \begin{bmatrix}
    1 & 0 \\
  \end{bmatrix}
  \intertext{Therefore, the output vector is}
  \mat{y} &= \mat{C}\mat{x} \\
  \mat{y} &= \begin{bmatrix}
    1 & 0
  \end{bmatrix}
  \begin{bmatrix}
    \theta \\
    \omega
  \end{bmatrix} \\
  \mat{y} &= 1\theta + 0\omega \\
  \mat{y} &= \theta
\end{align*}
\begin{theorem}[Linear time-varying pendulum state-space model with no input]
  \begin{align*}
    \dot{\mat{x}} &= \mat{A} \mat{x} \\
    \mat{y} &= \mat{C} \mat{x}
  \end{align*}
  \begin{equation*}
    \mat{x} =
    \begin{bmatrix}
      \theta \\
      \omega
    \end{bmatrix}
    \quad
    \mat{y} = \theta
  \end{equation*}
  \begin{align}
    \mat{A} &=
    \begin{bmatrix}
      0 & 1 \\
      -\frac{g}{l}\cos\theta & 0
    \end{bmatrix} \\
    \mat{C} &=
    \begin{bmatrix}
      1 & 0
    \end{bmatrix}
  \end{align}
\end{theorem}

\section{DC brushed motor}
\label{sec:dc_brushed_motor}

We will be deriving a first-order \gls{model} for a DC brushed motor. A
second-order \gls{model} would include the inductance of the motor windings as
well, but we're assuming the time constant of the inductor is small enough that
its affect on the \gls{model} behavior is negligible for FRC use cases (see
subsection \ref{subsec:do_flywheels_need_pd_control} for a demonstration of this
for a real DC brushed motor).

The first-order \gls{model} will only require numbers from the motor's
datasheet. The second-order \gls{model} would require measuring the motor
inductance as well, which generally isn't in the datasheet. It can be difficult
to measure accurately without the right equipment.

\subsection{Equations of motion}

The circuit for a DC brushed motor is shown in figure
\ref{fig:dc_motor_circuit}.
\begin{bookfigure}
  \begin{tikzpicture}[auto, >=latex', circuit ee IEC,
                      set resistor graphic=var resistor IEC graphic]
    \node [opencircuit] (start) at (0,0) {};
    \node [] (V+) at (-0.5,0) { $+$ };
    \node [opencircuit] (end) at (0,-3.5) {};
    \node [] (V-) at (-0.5,-3.5) { $-$ };
    \node [coordinate] (topright) at (2.5,0) {};
    \node [coordinate] (bottomright) at (2.5,-3.5) {};
    \node [] at (0, -1.75) { $V$ };
    \draw (start) to (topright)
                  to [resistor={near start, info'={ $R$ }},
                      voltage source={near end, direction info'={<-},
                      info={ $V_{emf}=\frac{\omega}{K_v}$ }}] (bottomright)
                  to (end);
  \end{tikzpicture}

  \caption{DC brushed motor circuit}
  \label{fig:dc_motor_circuit}
\end{bookfigure}

$V$ is the voltage applied to the motor, $I$ is the current through the motor in
Amps, $R$ is the resistance across the motor in Ohms, $\omega$ is the angular
velocity of the motor in radians per second, and $K_v$ is the angular velocity
constant in radians per second per Volt. This circuit reflects the following
relation.
\begin{equation}
  V = IR + \frac{\omega}{K_v} \label{eq:motor_V}
\end{equation}

The mechanical relation for a DC brushed motor is
\begin{equation}
  \tau = K_t I
\end{equation}

where $\tau$ is the torque produced by the motor in Newton-meters and $K_t$ is
the torque constant in Newton-meters per Amp. Therefore
\begin{equation*}
  I = \frac{\tau}{K_t}
\end{equation*}

Substitute this into equation \eqref{eq:motor_V}.

\index{FRC models!DC brushed motor equations}
\begin{equation}
  V = \frac{\tau}{K_t} R + \frac{\omega}{K_v} \label{eq:motor_tau_V}
\end{equation}

\subsection{Calculating constants}

A typical motor's datasheet should include graphs of the motor's measured torque
and current for different angular velocities for a given voltage applied to the
motor. Figure \ref{fig:motor_data} is an example. Data for the most common
motors in FRC can be found at \url{https://motors.vex.com/}.
\begin{svg}{build/\chapterpath/motor_data}
  \caption{Example motor datasheet for 775pro}
  \label{fig:motor_data}
\end{svg}

\subsubsection{Torque constant $K_t$}
\begin{align}
  \tau &= K_t I \nonumber \\
  K_t &= \frac{\tau}{I} \nonumber \\
  K_t &= \frac{\tau_{stall}}{I_{stall}}
\end{align}

where $\tau_{stall}$ is the stall torque and $I_{stall}$ is the stall current of
the motor from its datasheet.

\subsubsection{Resistance $R$}

Recall equation \eqref{eq:motor_V}.
\begin{align}
  V &= IR + \frac{\omega}{K_v} \nonumber
  \intertext{When the motor is stalled, $\omega = 0$.}
  V &= I_{stall} R \nonumber \\
  R &= \frac{V}{I_{stall}}
\end{align}

where $I_{stall}$ is the stall current of the motor and $V$ is the voltage
applied to the motor at stall.

\subsubsection{Angular velocity constant $K_v$}

Recall equation \eqref{eq:motor_V}.
\begin{align}
  V &= IR + \frac{\omega}{K_v} \nonumber \\
  V - IR &= \frac{\omega}{K_v} \nonumber \\
  K_v &= \frac{\omega}{V - IR} \nonumber
  \intertext{When the motor is spinning under no load,}
  K_v &= \frac{\omega_{free}}{V - I_{free}R}
\end{align}

where $\omega_{free}$ is the angular velocity of the motor under no load (also
known as the free speed), and $V$ is the voltage applied to the motor when it's
spinning at $\omega_{free}$, and $I_{free}$ is the current drawn by the motor
under no load.

If several identical motors are being used in one gearbox for a mechanism,
multiply the stall torque by the number of motors.

\subsection{Current limiting}

Current limiting of a DC brushed motor reduces the maximum input voltage to
avoid exceeding a current threshold. Predictive current limiting uses a
projected estimate of the current, so the voltage is reduced before the current
threshold is exceeded. Reactive current limiting uses an actual current
measurement, so the voltage is reduced after the current threshold is exceeded.

The following pseudocode demonstrates each type of current limiting.
\begin{code}
  \begin{lstlisting}[style=customPython]
# Normal feedback control
V = K @ (r - x)

# Calculations for predictive current limiting
omega = angular_velocity_measurement
I = V / R - omega / (Kv * R)

# Calculations for reactive current limiting
I = current_measurement
omega = Kv * V - I * R * Kv  # or can be angular velocity measurement

# If predicted/actual current above max, limit current by reducing voltage
if I > I_max:
    V = I_max * R + omega / Kv
  \end{lstlisting}
  \caption{Limits current of DC motor to $I_{max}$}
\end{code}

\section{Elevator}
\label{sec:ss_model_elevator}

\subsection{Continuous state-space model}
\index{FRC models!elevator equations}

The position and velocity derivatives of the elevator can be written as

\begin{align}
  \dot{x}_m &= v_m \label{eq:elevator_cont_ss_pos} \\
  \dot{v}_m &= a_m \label{eq:elevator_cont_ss_vel}
\end{align}

where by equation (\ref{eq:elevator_accel}),

\begin{equation*}
  a_m = \frac{GK_t}{Rrm} V - \frac{G^2 K_t}{Rr^2 m K_v} v_m
\end{equation*}

Substitute this into equation (\ref{eq:elevator_cont_ss_vel}).

\begin{align}
  \dot{v}_m &= \frac{GK_t}{Rrm} V - \frac{G^2 K_t}{Rr^2 m K_v} v_m \nonumber \\
  \dot{v}_m &= -\frac{G^2 K_t}{Rr^2 m K_v} v_m + \frac{GK_t}{Rrm} V
\end{align}

Factor out $v_m$ and $V$ into column vectors.

\begin{align*}
  \dot{\begin{bmatrix}
    v_m
  \end{bmatrix}} &=
  \begin{bmatrix}
    -\frac{G^2 K_t}{Rr^2 m K_v}
  \end{bmatrix}
  \begin{bmatrix}
    v_m
  \end{bmatrix} +
  \begin{bmatrix}
    \frac{GK_t}{Rrm}
  \end{bmatrix}
  \begin{bmatrix}
    V
  \end{bmatrix}
\end{align*}

Augment the matrix equation with the position state $x$, which has the model
equation $\dot{x} = v_m$. The matrix elements corresponding to $v_m$ will be
$1$, and the others will be $0$ since they don't appear, so
$\dot{x} = 0x + 1v_m + 0V$. The existing rows will have zeroes inserted where
$x$ is multiplied in.

\begin{align*}
  \dot{\begin{bmatrix}
    x \\
    v_m
  \end{bmatrix}} &=
  \begin{bmatrix}
    0 & 1 \\
    0 & -\frac{G^2 K_t}{Rr^2 m K_v}
  \end{bmatrix}
  \begin{bmatrix}
    x \\
    v_m
  \end{bmatrix} +
  \begin{bmatrix}
    0 \\
    \frac{GK_t}{Rrm}
  \end{bmatrix}
  \begin{bmatrix}
    V
  \end{bmatrix}
\end{align*}

\begin{theorem}[Elevator state-space model]
  \begin{align*}
    \dot{\mtx{x}} &= \mtx{A} \mtx{x} + \mtx{B} \mtx{u} \\
    \mtx{y} &= \mtx{C} \mtx{x} + \mtx{D} \mtx{u}
  \end{align*}
  \begin{equation*}
    \begin{array}{ccc}
      \mtx{x} =
      \begin{bmatrix}
        x \\
        v_m
      \end{bmatrix} &
      \mtx{y} = x &
      \mtx{u} = V
    \end{array}
  \end{equation*}
  \begin{equation}
    \begin{array}{cccc}
      \mtx{A} =
      \begin{bmatrix}
        0 & 1 \\
        0 & -\frac{G^2 K_t}{Rr^2 mK_v}
      \end{bmatrix} &
      \mtx{B} =
      \begin{bmatrix}
        0 \\
        \frac{GK_t}{Rrm}
      \end{bmatrix} &
      \mtx{C} =
      \begin{bmatrix}
        1 & 0
      \end{bmatrix} &
      \mtx{D} = 0
    \end{array}
  \end{equation}
\end{theorem}

\subsection{Model augmentation}

As per subsection \ref{subsec:u_error_estimation}, we will now augment the
\gls{model} so a $u_{error}$ term is added to the \gls{control input}.

The \gls{plant} and \gls{observer} augmentations should be performed before the
\gls{model} is \glslink{discretization}{discretized}. After the \gls{controller}
gain is computed with the unaugmented discrete \gls{model}, the controller may
be augmented. Therefore, the \gls{plant} and \gls{observer} augmentations assume
a continuous \gls{model} and the \gls{controller} augmentation assumes a
discrete \gls{controller}.

\begin{equation*}
  \begin{array}{ccc}
    \mtx{x}_{aug} =
    \begin{bmatrix}
      x \\
      v_m \\
      u_{error}
    \end{bmatrix} &
    \mtx{y} = x &
    \mtx{u} = V
  \end{array}
\end{equation*}

\begin{equation}
  \begin{array}{cccc}
    \mtx{A}_{aug} =
    \begin{bmatrix}
      \mtx{A} & \mtx{B} \\
      \mtx{0}_{1 \times 2} & 0
    \end{bmatrix} &
    \mtx{B}_{aug} =
    \begin{bmatrix}
      \mtx{B} \\
      0
    \end{bmatrix} &
    \mtx{C}_{aug} = \begin{bmatrix}
      \mtx{C} & 0
    \end{bmatrix} &
    \mtx{D}_{aug} = \mtx{D}
  \end{array}
\end{equation}

\begin{equation}
  \begin{array}{cc}
    \mtx{K}_{aug} = \begin{bmatrix}
      \mtx{K} & 1
    \end{bmatrix} &
    \mtx{r}_{aug} = \begin{bmatrix}
      \mtx{r} \\
      0
    \end{bmatrix}
  \end{array}
\end{equation}

This will compensate for unmodeled dynamics such as gravity. However, using a
constant voltage feedforward to counteract gravity is preferred over $u_{error}$
estimation in this case because it results in a simpler controller with similar
performance.

\subsection{Gravity feedforward}

Input voltage is proportional to force and gravity is a constant force, so a
constant voltage feedforward can compensate for gravity. We'll model gravity as
a disturbance described by $-mg$. To compensate for it, we want to find a
voltage that is equal and opposite to it. The bottom row of the continuous
elevator model contains the acceleration terms.

\begin{equation*}
  Bu_{ff} = -(\text{unmodeled dynamics})
\end{equation*}

where $B$ is the motor acceleration term from $\mtx{B}$ and $u_{ff}$ is the
voltage feedforward.

\begin{align*}
  Bu_{ff} &= -(-mg) \\
  Bu_{ff} &= mg \\
  \frac{G K_t}{Rrm} u_{ff} &= mg \\
  u_{ff} &= \frac{Rrm^2 g}{G K_t}
\end{align*}

\subsection{Simulation}

Python Control will be used to \glslink{discretization}{discretize} the
\gls{model} and simulate it. One of the frccontrol
examples\footnote{\url{https://github.com/calcmogul/frccontrol/blob/master/examples/elevator.py}}
creates and tests a controller for it.

\begin{remark}
  Python Control currently doesn't support finding the transmission zeroes of
  MIMO \glspl{system} with a different number of \glspl{input} than
  \glspl{output}, so \texttt{control.pzmap()} and
  \texttt{frccontrol.System.plot\_pzmaps()} fail with an error if Slycot isn't
  installed.
\end{remark}

Figure \ref{fig:elevator_pzmaps} shows the pole-zero maps for the open-loop
\gls{system}, closed-loop \gls{system}, and \gls{observer}. Figure
\ref{fig:elevator_response} shows the \gls{system} response with them.

\begin{svg}{build/frccontrol/examples/elevator_pzmaps}
  \caption{Elevator pole-zero maps}
  \label{fig:elevator_pzmaps}
\end{svg}

\begin{svg}{build/frccontrol/examples/elevator_response}
  \caption{Elevator response}
  \label{fig:elevator_response}
\end{svg}

\subsection{Implementation}

The script linked above also generates two files: ElevatorCoeffs.h and
ElevatorCoeffs.cpp. These can be used with the WPILib StateSpacePlant,
StateSpaceController, and StateSpaceObserver classes in C++ and Java. A C++
implementation of this elevator controller is available online\footnote{
\url{https://github.com/calcmogul/allwpilib/tree/state-space/wpilibcExamples/src/main/cpp/examples/StateSpaceElevator}}.

\section{Flywheel}
\label{sec:ss_model_flywheel}

\subsection{Continuous state-space model}
\index{FRC models!flywheel equations}

By equation \eqref{eq:dot_omega_flywheel}
\begin{equation*}
  \dot{\omega} = -\frac{G^2 K_t}{K_v RJ} \omega + \frac{G K_t}{RJ} V
\end{equation*}

Factor out $\omega$ and $V$ into column vectors.
\begin{align*}
  \dot{\begin{bmatrix}
    \omega
  \end{bmatrix}} &=
  \begin{bmatrix}
    -\frac{G^2 K_t}{K_v RJ}
  \end{bmatrix}
  \begin{bmatrix}
    \omega
  \end{bmatrix} +
  \begin{bmatrix}
    \frac{GK_t}{RJ}
  \end{bmatrix}
  \begin{bmatrix}
    V
  \end{bmatrix}
\end{align*}
\begin{theorem}[Flywheel state-space model]
  \begin{align*}
    \dot{\mat{x}} &= \mat{A} \mat{x} + \mat{B} \mat{u} \\
    \mat{y} &= \mat{C} \mat{x} + \mat{D} \mat{u}
  \end{align*}
  \begin{equation*}
    \begin{array}{ccc}
      \mat{x} = \omega &
      \mat{y} = \omega &
      \mat{u} = V
    \end{array}
  \end{equation*}
  \begin{equation}
    \begin{array}{cccc}
      \mat{A} = -\frac{G^2 K_t}{K_v RJ} &
      \mat{B} = \frac{G K_t}{RJ} &
      \mat{C} = 1 &
      \mat{D} = 0
    \end{array}
  \end{equation}
\end{theorem}

\subsection{Model augmentation}

As per subsection \ref{subsec:input_error_estimation}, we will now augment the
\gls{model} so a $u_{error}$ state is added to the \gls{control input}.

The \gls{plant} and \gls{observer} augmentations should be performed before the
\gls{model} is \glslink{discretization}{discretized}. After the \gls{controller}
gain is computed with the unaugmented discrete \gls{model}, the controller may
be augmented. Therefore, the \gls{plant} and \gls{observer} augmentations assume
a continuous \gls{model} and the \gls{controller} augmentation assumes a
discrete \gls{controller}.
\begin{equation*}
  \begin{array}{ccc}
    \mat{x} =
    \begin{bmatrix}
      \omega \\
      u_{error}
    \end{bmatrix} &
    \mat{y} = \omega &
    \mat{u} = V
  \end{array}
\end{equation*}
\begin{equation}
  \begin{array}{cccc}
    \mat{A}_{aug} =
    \begin{bmatrix}
      \mat{A} & \mat{B} \\
      0 & 0
    \end{bmatrix} &
    \mat{B}_{aug} =
    \begin{bmatrix}
      \mat{B} \\
      0
    \end{bmatrix} &
    \mat{C}_{aug} = \begin{bmatrix}
      \mat{C} & 0
    \end{bmatrix} &
    \mat{D}_{aug} = \mat{D}
  \end{array}
\end{equation}
\begin{equation}
  \begin{array}{cc}
    \mat{K}_{aug} = \begin{bmatrix}
      \mat{K} & 1
    \end{bmatrix} &
    \mat{r}_{aug} = \begin{bmatrix}
      \mat{r} \\
      0
    \end{bmatrix}
  \end{array}
\end{equation}

This will compensate for unmodeled dynamics such as projectiles slowing down the
flywheel.

\subsection{Simulation}

Python Control will be used to \glslink{discretization}{discretize} the
\gls{model} and simulate it. One of the frccontrol
examples\footnote{\url{https://github.com/calcmogul/frccontrol/blob/main/examples/flywheel.py}}
creates and tests a controller for it. Figure \ref{fig:flywheel_response} shows
the closed-loop \gls{system} response.
\begin{svg}{build/frccontrol/examples/flywheel_response}
  \caption{Flywheel response}
  \label{fig:flywheel_response}
\end{svg}

Notice how the \gls{control effort} when the \gls{reference} is reached is
nonzero. This is a plant inversion feedforward compensating for the \gls{system}
dynamics attempting to slow the flywheel down when no voltage is applied.

\subsection{Implementation}

C++ and Java implementations of this flywheel controller are available
online\footnote{\url{https://github.com/wpilibsuite/allwpilib/blob/main/wpilibcExamples/src/main/cpp/examples/StateSpaceFlywheel/cpp/Robot.cpp}}
\footnote{\url{https://github.com/wpilibsuite/allwpilib/blob/main/wpilibjExamples/src/main/java/edu/wpi/first/wpilibj/examples/statespaceflywheel/Robot.java}}.

\subsection{Flywheel model without encoder}

In the FIRST Robotics Competition, we can get the current drawn for specific
channels on the power distribution panel. We can theoretically use this to
estimate the angular velocity of a DC motor without an encoder. We'll start with
the flywheel model derived earlier as equation \eqref{eq:dot_omega_flywheel}.
\begin{align*}
  \dot{\omega} &= \frac{G K_t}{RJ} V - \frac{G^2 K_t}{K_v RJ} \omega \\
  \dot{\omega} &= -\frac{G^2 K_t}{K_v RJ} \omega + \frac{G K_t}{RJ} V
\end{align*}

Next, we'll derive the current $I$ as an output.
\begin{align*}
  V &= IR + \frac{\omega}{K_v} \\
  IR &= V - \frac{\omega}{K_v} \\
  I &= -\frac{1}{K_v R} \omega + \frac{1}{R} V
\end{align*}

Therefore,
\begin{theorem}[Flywheel state-space model without encoder]
  \begin{align*}
    \dot{\mat{x}} &= \mat{A} \mat{x} + \mat{B} \mat{u} \\
    \mat{y} &= \mat{C} \mat{x} + \mat{D} \mat{u}
  \end{align*}
  \begin{equation*}
    \begin{array}{ccc}
      \mat{x} = \omega &
      \mat{y} = I &
      \mat{u} = V
    \end{array}
  \end{equation*}
  \begin{equation}
    \begin{array}{cccc}
      \mat{A} = -\frac{G^2 K_t}{K_v RJ} &
      \mat{B} = \frac{G K_t}{RJ} &
      \mat{C} = -\frac{1}{K_v R} &
      \mat{D} = \frac{1}{R}
    \end{array}
  \end{equation}
\end{theorem}

Notice that in this \gls{model}, the \gls{output} doesn't provide any direct
measurements of the \gls{state}. To estimate the full \gls{state} (also known as
full observability), we only need the \glspl{output} to collectively include
linear combinations of every \gls{state}\footnote{While the flywheel model's
outputs are a linear combination of both the states and inputs, \glspl{input}
don't provide new information about the \glspl{state}. Therefore, they don't
affect whether the system is observable.}. We'll revisit this in chapter
\ref{ch:stochastic_control_theory} with an example that uses range measurements
to estimate an object's orientation.

The effectiveness of this \gls{model}'s \gls{observer} is heavily dependent on
the quality of the current sensor used. If the sensor's noise isn't zero-mean,
the \gls{observer} won't converge to the true \gls{state}.

\subsection{Voltage compensation}

To improve controller \gls{tracking}, one may want to use the voltage
renormalized to the power rail voltage to compensate for voltage drop when
current spikes occur. This can be done as follows.
\begin{equation}
  V = V_{cmd} \frac{V_{nominal}}{V_{rail}}
\end{equation}

where $V$ is the \gls{controller}'s new input voltage, $V_{cmd}$ is the old
input voltage, $V_{nominal}$ is the rail voltage when effects like voltage drop
due to current draw are ignored, and $V_{rail}$ is the real rail voltage.

To drive the \gls{model} with a more accurate voltage that includes voltage
drop, the reciprocal can be used.
\begin{equation}
  V = V_{cmd} \frac{V_{rail}}{V_{nominal}}
\end{equation}

where $V$ is the \gls{model}'s new input voltage. Note that if both the
\gls{controller} compensation and \gls{model} compensation equations are
applied, the original voltage is obtained. The \gls{model} input only drops from
ideal if the compensated \gls{controller} voltage saturates.

\section{Single-jointed arm}

\subsection{Equations of motion}

This single-jointed arm consists of a DC brushed motor attached to a pulley that
spins a straight bar in pitch.
\begin{bookfigure}
  \begin{tikzpicture}[auto, >=latex', circuit ee IEC,
                      set resistor graphic=var resistor IEC graphic]
    % \draw [help lines] (-1,-3) grid (7,4);

    % Electrical equivalent circuit
    \draw (0,2) to [voltage source={direction info'={->}, info'=$V$}] (0,0);
    \draw (0,2) to [current direction={info=$I$}] (0,3);
    \draw (0,3) -- (0.5,3);
    \draw (0.5,3) to [resistor={info={$R$}}] (2,3);

    \draw (2,3) -- (2.5,3);
    \draw (2.5,3) to [voltage source={direction info'={->}, info'=$V_{emf}$}]
      (2.5,0);
    \draw (0,0) -- (2.5,0);

    % Motor
    \begin{scope}[xshift=2.4cm,yshift=1.05cm]
      \draw[fill=black] (0,0) rectangle (0.2,0.9);
      \draw[fill=white] (0.1,0.45) ellipse (0.3 and 0.3);
    \end{scope}

    % Transmission gear one
    \begin{scope}[xshift=3.75cm,yshift=1.17cm]
      \draw[fill=black!50] (0.2,0.33) ellipse (0.08 and 0.33);
      \draw[fill=black!50, color=black!50] (0,0) rectangle (0.2,0.66);
      \draw[fill=white] (0,0.33) ellipse (0.08 and 0.33);
      \draw (0,0.66) -- (0.2,0.66);
      \draw (0,0) -- (0.2,0) node[pos=0.5,below] {$G$};
    \end{scope}

    % Output shaft of motor
    \begin{scope}[xshift=2.8cm,yshift=1.45cm]
      \draw[fill=black!50] (0,0) rectangle (0.95,0.1);
    \end{scope}

    % Angular velocity arrow of drive -> transmission
    \draw[line width=0.7pt,<-] (3.2,1) arc (-30:30:1) node[above] {$\omega_m$};

    % Transmission gear two
    \begin{scope}[xshift=3.75cm,yshift=1.83cm]
      \draw[fill=black!50] (0.2,0.68) ellipse (0.13 and 0.67);
      \draw[fill=black!50, color=black!50] (0,0) rectangle (0.2,1.35);
      \draw[fill=white] (0,0.68) ellipse (0.13 and 0.67);
      \draw (0,1.35) -- (0.2,1.35);
      \draw (0,0) -- (0.2,0);
    \end{scope}
    \begin{scope}[xshift=5.075cm,yshift=2.4cm]
      % Single-jointed arm
      \draw[fill=white] (0,0) -- (0.1,-0.05) -- (0.35,1.45) -- (0.25,1.5)
        -- cycle;
      \draw[fill=black!50] (0.1,-0.05) -- (0.3,-0.05) -- (0.55,1.45) --
        (0.35,1.45) -- cycle;
      \draw[fill=white] (0.25,1.5) -- (0.35,1.45) -- (0.55,1.45) -- (0.45,1.5)
        -- cycle;

      % Arm length arrow
      \draw[line width=0.7pt,<->] (0.55,-0.05) -- node[right] {$l$} (0.8,1.45);

      % Mass label
      \draw (-0.05,1.2) node {$m$};
    \end{scope}

    % Transmission shaft from gear two to arm
    \begin{scope}[xshift=4.09cm,yshift=2.42cm]
      \draw[fill=black!50] (0,0) rectangle (1.06,0.1);
    \end{scope}

    % Angular velocity arrow between transmission and arm
    \draw[line width=0.7pt,->] (4.54,1.97) arc (-30:30:1) node[above]
      {$\omega_{arm}$};

    % Descriptions of subsystems under graphic
    \begin{scope}[xshift=-0.5cm,yshift=-0.28cm]
      \draw[decorate,decoration={brace,amplitude=10pt}]
        (3.5,0) -- (0,0) node[midway,yshift=-20pt] {circuit};
      \draw[decorate,decoration={brace,amplitude=10pt}]
        (6.55,0) -- (3.75,0) node[midway,yshift=-20pt] {mechanics};
    \end{scope}
  \end{tikzpicture}

  \caption{Single-jointed arm system diagram}
  \label{fig:single_jointed_arm}
\end{bookfigure}

Gear ratios are written as output over input, so $G$ is greater than one in
figure \ref{fig:single_jointed_arm}.

We want to derive an equation for the arm angular acceleration
$\dot{\omega}_{arm}$ given an input voltage $V$, which we can integrate to get
arm angular velocity and angle.

We will start with the equation derived earlier for a DC brushed motor, equation
\eqref{eq:motor_tau_V}.
\begin{align}
  V &= \frac{\tau_m}{K_t} R + \frac{\omega_m}{K_v} \nonumber
  \intertext{Solve for the angular acceleration. First, we'll rearrange the
    terms because from inspection, $V$ is the \gls{model} \gls{input},
    $\omega_m$ is the \gls{state}, and $\tau_m$ contains the angular
    acceleration.}
  V &= \frac{R}{K_t} \tau_m + \frac{1}{K_v} \omega_m \nonumber
  \intertext{Solve for $\tau_m$.}
  V &= \frac{R}{K_t} \tau_m + \frac{1}{K_v} \omega_m \nonumber \\
  \frac{R}{K_t} \tau_m &= V - \frac{1}{K_v} \omega_m \nonumber \\
  \tau_m &= \frac{K_t}{R} V - \frac{K_t}{K_v R} \omega_m
  \intertext{Since $\tau_m G = \tau_{arm}$ and $\omega_m = G \omega_{arm}$,}
  \left(\frac{\tau_{arm}}{G}\right) &= \frac{K_t}{R} V -
    \frac{K_t}{K_v R} (G \omega_{arm}) \nonumber \\
  \frac{\tau_{arm}}{G} &= \frac{K_t}{R} V - \frac{G K_t}{K_v R} \omega_{arm}
    \nonumber \\
  \tau_{arm} &= \frac{G K_t}{R} V - \frac{G^2 K_t}{K_v R} \omega_{arm}
    \label{eq:tau_arm}
  \intertext{The torque applied to the arm is defined as}
  \tau_{arm} &= J \dot{\omega}_{arm} \label{eq:tau_arm_def}
  \intertext{where $J$ is the moment of inertia of the arm and
    $\dot{\omega}_{arm}$ is the angular acceleration. Substitute equation
    \eqref{eq:tau_arm_def} into equation \eqref{eq:tau_arm}.}
  (J \dot{\omega}_{arm}) &= \frac{G K_t}{R} V - \frac{G^2 K_t}{K_v R}
    \omega_{arm} \nonumber \\
  \dot{\omega}_{arm} &= -\frac{G^2 K_t}{K_v RJ} \omega_{arm} +
    \frac{G K_t}{RJ} V \nonumber
  \intertext{We'll relabel $\omega_{arm}$ as $\omega$ for convenience.}
  \dot{\omega} &= -\frac{G^2 K_t}{K_v RJ} \omega + \frac{G K_t}{RJ} V
    \label{eq:dot_omega_arm}
\end{align}

This model will be converted to state-space notation in section
\ref{sec:ss_model_single-jointed_arm}.

\subsection{Calculating constants}

\subsubsection{Moment of inertia J}

Given the simplicity of this mechanism, it may be easier to compute this value
theoretically using material properties in CAD. $J$ can also be approximated as
the moment of inertia of a thin rod rotating around one end. Therefore
\begin{equation}
  J = \frac{1}{3}ml^2
\end{equation}

where $m$ is the mass of the arm and $l$ is the length of the arm. Otherwise, a
procedure for measuring it experimentally is presented below.

First, rearrange equation \eqref{eq:dot_omega_arm} into the form $y = mx + b$
such that $J$ is in the numerator of $m$.
\begin{align}
  \dot{\omega} &= -\frac{G^2 K_t}{K_v RJ} \omega + \frac{G K_t}{RJ} V \nonumber
    \\
  J\dot{\omega} &= -\frac{G^2 K_t}{K_v R} \omega + \frac{G K_t}{R} V \nonumber
  \intertext{Multiply by $\frac{K_v R}{G^2 K_t}$ on both sides.}
  \frac{J K_v R}{G^2 K_t} \dot{\omega} &= -\omega + \frac{G K_t}{R} \cdot
    \frac{K_v R}{G^2 K_t} V \nonumber \\
  \frac{J K_v R}{G^2 K_t} \dot{\omega} &= -\omega + \frac{K_v}{G} V \nonumber \\
  \omega &= -\frac{J K_v R}{G^2 K_t} \dot{\omega} + \frac{K_v}{G} V
    \label{eq:arm_J_regression}
\end{align}

The test procedure is as follows.
\begin{enumerate}
  \item Orient the arm such that its axis of rotation is aligned with gravity
    (i.e., the arm is on its side). This avoids gravity affecting the
    measurements.
  \item Run the arm forward at a constant voltage. Record the angular velocity
    over time.
  \item Compute the angular acceleration from the angular velocity data as the
    difference between each sample divided by the time between them.
  \item Perform a linear regression of angular velocity versus angular
    acceleration. The slope of this line has the form $-\frac{J K_v R}{G^2 K_t}$
    as per equation \eqref{eq:arm_J_regression}.
  \item Multiply the slope by $-\frac{G^2 K_t}{K_v R}$ to obtain a least squares
    estimate of $J$.
\end{enumerate}

\section{Differential drive}
\label{sec:ss_model_differential_drive}

\subsection{Velocity subspace state-space model}
\index{FRC models!differential drive equations}

By equations \eqref{eq:diff_drive_model_right} and
\eqref{eq:diff_drive_model_left}
\begin{align*}
  \dot{v}_l &= \left(\frac{1}{m} + \frac{r_b^2}{J}\right)
    \left(C_1 v_l + C_2 V_l\right) +
    \left(\frac{1}{m} - \frac{r_b^2}{J}\right) \left(C_3 v_r + C_4 V_r\right) \\
  \dot{v}_r &= \left(\frac{1}{m} - \frac{r_b^2}{J}\right)
    \left(C_1 v_l + C_2 V_l\right) +
    \left(\frac{1}{m} + \frac{r_b^2}{J}\right) \left(C_3 v_r + C_4 V_r\right)
\end{align*}

Regroup the terms into states $v_l$ and $v_r$ and inputs $V_l$ and $V_r$.
\begin{align*}
  \dot{v}_l &= \left(\frac{1}{m} + \frac{r_b^2}{J}\right) C_1 v_l +
    \left(\frac{1}{m} + \frac{r_b^2}{J}\right) C_2 V_l +
    \left(\frac{1}{m} - \frac{r_b^2}{J}\right) C_3 v_r +
    \left(\frac{1}{m} - \frac{r_b^2}{J}\right) C_4 V_r \\
  \dot{v}_r &= \left(\frac{1}{m} - \frac{r_b^2}{J}\right) C_1 v_l +
    \left(\frac{1}{m} - \frac{r_b^2}{J}\right) C_2 V_l +
    \left(\frac{1}{m} + \frac{r_b^2}{J}\right) C_3 v_r +
    \left(\frac{1}{m} + \frac{r_b^2}{J}\right) C_4 V_r
\end{align*}
\begin{align*}
  \dot{v}_l &= \left(\frac{1}{m} + \frac{r_b^2}{J}\right) C_1 v_l +
    \left(\frac{1}{m} - \frac{r_b^2}{J}\right) C_3 v_r +
    \left(\frac{1}{m} + \frac{r_b^2}{J}\right) C_2 V_l +
    \left(\frac{1}{m} - \frac{r_b^2}{J}\right) C_4 V_r \\
  \dot{v}_r &= \left(\frac{1}{m} - \frac{r_b^2}{J}\right) C_1 v_l +
    \left(\frac{1}{m} + \frac{r_b^2}{J}\right) C_3 v_r +
    \left(\frac{1}{m} - \frac{r_b^2}{J}\right) C_2 V_l +
    \left(\frac{1}{m} + \frac{r_b^2}{J}\right) C_4 V_r
\end{align*}

Factor out $v_l$ and $v_r$ into a column vector and $V_l$ and $V_r$ into a
column vector.
\begin{align*}
  \dot{\begin{bmatrix}
    v_l \\
    v_r
  \end{bmatrix}} &=
  \begin{bmatrix}
    \left(\frac{1}{m} + \frac{r_b^2}{J}\right) C_1 &
    \left(\frac{1}{m} - \frac{r_b^2}{J}\right) C_3 \\
    \left(\frac{1}{m} - \frac{r_b^2}{J}\right) C_1 &
    \left(\frac{1}{m} + \frac{r_b^2}{J}\right) C_3
  \end{bmatrix}
  \begin{bmatrix}
    v_l \\
    v_r
  \end{bmatrix} +
  \begin{bmatrix}
    \left(\frac{1}{m} + \frac{r_b^2}{J}\right) C_2 &
    \left(\frac{1}{m} - \frac{r_b^2}{J}\right) C_4 \\
    \left(\frac{1}{m} - \frac{r_b^2}{J}\right) C_2 &
    \left(\frac{1}{m} + \frac{r_b^2}{J}\right) C_4
  \end{bmatrix}
  \begin{bmatrix}
    V_l \\
    V_r
  \end{bmatrix}
\end{align*}
\begin{theorem}[Differential drive velocity state-space model]
  \label{thm:diff_drive_velocity_ss_model}

  \begin{align*}
    \dot{\mtx{x}} &= \mtx{A} \mtx{x} + \mtx{B} \mtx{u} \\
    \mtx{y} &= \mtx{C} \mtx{x} + \mtx{D} \mtx{u}
  \end{align*}
  \begin{equation*}
    \begin{array}{ccc}
      \mtx{x} =
      \begin{bmatrix}
        v_l \\
        v_r
      \end{bmatrix} &
      \mtx{y} =
      \begin{bmatrix}
        v_l \\
        v_r
      \end{bmatrix} &
      \mtx{u} =
      \begin{bmatrix}
        V_l \\
        V_r
      \end{bmatrix}
    \end{array}
  \end{equation*}
  \begin{equation}
    \label{eq:diff_drive_ss_model}
    \begin{array}{ll}
      \mtx{A} =
      \begin{bmatrix}
        \left(\frac{1}{m} + \frac{r_b^2}{J}\right) C_1 & \left(\frac{1}{m} - \frac{r_b^2}{J}\right) C_3 \\
        \left(\frac{1}{m} - \frac{r_b^2}{J}\right) C_1 & \left(\frac{1}{m} + \frac{r_b^2}{J}\right) C_3
      \end{bmatrix} &
      \mtx{B} =
      \begin{bmatrix}
        \left(\frac{1}{m} + \frac{r_b^2}{J}\right) C_2 & \left(\frac{1}{m} - \frac{r_b^2}{J}\right) C_4 \\
        \left(\frac{1}{m} - \frac{r_b^2}{J}\right) C_2 & \left(\frac{1}{m} + \frac{r_b^2}{J}\right) C_4
      \end{bmatrix} \\
      \mtx{C} =
      \begin{bmatrix}
        1 & 0 \\
        0 & 1 \\
      \end{bmatrix} &
      \mtx{D} = \mtx{0}_{2 \times 2}
    \end{array}
  \end{equation}

  where $C_1 = -\frac{G_l^2 K_t}{K_v R r^2}$, $C_2 = \frac{G_l K_t}{Rr}$,
  $C_3 = -\frac{G_r^2 K_t}{K_v R r^2}$, and $C_4 = \frac{G_r K_t}{Rr}$.
\end{theorem}

\subsubsection{Simulation}

Python Control will be used to \glslink{discretization}{discretize} the
\gls{model} and simulate it. One of the frccontrol
examples\footnote{\url{https://github.com/calcmogul/frccontrol/blob/main/examples/differential_drive.py}}
creates and tests a controller for it. Figure \ref{fig:diff_drive_response}
shows the closed-loop \gls{system} response.
\begin{svg}{build/frccontrol/examples/differential_drive_response}
  \caption{Drivetrain response}
  \label{fig:diff_drive_response}
\end{svg}

Given the high inertia in drivetrains, it's better to drive the \gls{reference}
with a motion profile instead of a \gls{step input} for reproducibility.

\subsection{Linear time-varying model}
\index{controller design!linear time-varying control}
\index{nonlinear control!linear time-varying control}
\index{optimal control!linear time-varying control}

The model in theorem \ref{thm:diff_drive_velocity_ss_model} is linear, but only
includes the velocity dynamics, not the dynamics of the drivetrain's global
pose. The change in global pose is defined by these three equations.
\begin{align*}
  \dot{x} &= \frac{v_l + v_r}{2}\cos\theta = \frac{v_r}{2}\cos\theta +
    \frac{v_l}{2}\cos\theta \\
  \dot{y} &= \frac{v_l + v_r}{2}\sin\theta = \frac{v_r}{2}\sin\theta +
    \frac{v_l}{2}\sin\theta \\
  \dot{\theta} &= \frac{v_r - v_l}{2r_b} = \frac{v_r}{2r_b} - \frac{v_l}{2r_b}
\end{align*}

Next, we'll augment the linear subspace's state with the global pose $x$, $y$,
and $\theta$. Here's the model as a vector function where
$\mtx{x} = \begin{bmatrix} x & y & \theta & v_l & v_r \end{bmatrix}^T$ and
$\mtx{u} = \begin{bmatrix} V_l & V_r \end{bmatrix}^T$.
\begin{equation}
  f(\mtx{x}, \mtx{u}) =
  \begin{bmatrix}
    \frac{v_r}{2}\cos\theta + \frac{v_l}{2}\cos\theta \\
    \frac{v_r}{2}\sin\theta + \frac{v_l}{2}\sin\theta \\
    \frac{v_r}{2r_b} - \frac{v_l}{2r_b} \\
    \left(\frac{1}{m} + \frac{r_b^2}{J}\right) C_1 v_l +
      \left(\frac{1}{m} - \frac{r_b^2}{J}\right) C_3 v_r +
      \left(\frac{1}{m} + \frac{r_b^2}{J}\right) C_2 V_l +
      \left(\frac{1}{m} - \frac{r_b^2}{J}\right) C_4 V_r \\
    \left(\frac{1}{m} - \frac{r_b^2}{J}\right) C_1 v_l +
      \left(\frac{1}{m} + \frac{r_b^2}{J}\right) C_3 v_r +
      \left(\frac{1}{m} - \frac{r_b^2}{J}\right) C_2 V_l +
      \left(\frac{1}{m} + \frac{r_b^2}{J}\right) C_4 V_r
  \end{bmatrix}
  \label{eq:ltv_diff_drive_f}
\end{equation}

As mentioned in chapter \ref{ch:nonlinear_control}, one can approximate a
nonlinear system via linearizations around points of interest in the state-space
and design controllers for those linearized subspaces. If we sample
linearization points progressively closer together, we converge on a control
policy for the original nonlinear system. Since the linear \gls{plant} being
controlled varies with time, its controller is called a linear time-varying
(LTV) controller.

If we use LQRs for the linearized subspaces, the nonlinear control policy will
also be locally optimal. We'll be taking this approach with a differential
drive. To create an LQR, we need to linearize equation
\eqref{eq:ltv_diff_drive_f}.
\begin{align*}
  \frac{\partial f(\mtx{x}, \mtx{u})}{\partial\mtx{x}} &=
  \begin{bmatrix}
    0 & 0 & -\frac{v_l + v_r}{2}\sin\theta & \frac{1}{2}\cos\theta &
      \frac{1}{2}\cos\theta \\
    0 & 0 & \frac{v_l + v_r}{2}\cos\theta & \frac{1}{2}\sin\theta &
      \frac{1}{2}\sin\theta \\
    0 & 0 & 0 & -\frac{1}{2r_b} & \frac{1}{2r_b} \\
    0 & 0 & 0 & \left(\frac{1}{m} + \frac{r_b^2}{J}\right) C_1 &
      \left(\frac{1}{m} - \frac{r_b^2}{J}\right) C_3 \\
    0 & 0 & 0 & \left(\frac{1}{m} - \frac{r_b^2}{J}\right) C_1 &
      \left(\frac{1}{m} + \frac{r_b^2}{J}\right) C_3
  \end{bmatrix} \\
  \frac{\partial f(\mtx{x}, \mtx{u})}{\partial\mtx{u}} &=
  \begin{bmatrix}
    0 & 0 \\
    0 & 0 \\
    0 & 0 \\
    \left(\frac{1}{m} + \frac{r_b^2}{J}\right) C_2 &
    \left(\frac{1}{m} - \frac{r_b^2}{J}\right) C_4 \\
    \left(\frac{1}{m} - \frac{r_b^2}{J}\right) C_2 &
    \left(\frac{1}{m} + \frac{r_b^2}{J}\right) C_4
  \end{bmatrix}
\end{align*}

Therefore,
\begin{theorem}[Linear time-varying differential drive state-space model]
  \begin{align*}
    \dot{\mtx{x}} &= \mtx{A}\mtx{x} + \mtx{B}\mtx{u} \\
    \mtx{y} &= \mtx{C}\mtx{x} + \mtx{D}\mtx{u}
  \end{align*}
  \begin{equation*}
    \begin{array}{ccc}
      \mtx{x} =
      \begin{bmatrix}
        x & y & \theta & v_l & v_r
      \end{bmatrix}^T &
      \mtx{y} =
      \begin{bmatrix}
        \theta & v_l & v_r
      \end{bmatrix}^T &
      \mtx{u} =
      \begin{bmatrix}
        V_l & V_r
      \end{bmatrix}^T
    \end{array}
  \end{equation*}

  \begin{equation}
    \begin{array}{ll}
      \mtx{A} =
      \begin{bmatrix}
        0 & 0 & -vs & \frac{1}{2}c & \frac{1}{2}c \\
        0 & 0 & vc & \frac{1}{2}s & \frac{1}{2}s \\
        0 & 0 & 0 & -\frac{1}{2r_b} & \frac{1}{2r_b} \\
        0 & 0 & 0 & \left(\frac{1}{m} + \frac{r_b^2}{J}\right) C_1 &
          \left(\frac{1}{m} - \frac{r_b^2}{J}\right) C_3 \\
        0 & 0 & 0 & \left(\frac{1}{m} - \frac{r_b^2}{J}\right) C_1 &
          \left(\frac{1}{m} + \frac{r_b^2}{J}\right) C_3
      \end{bmatrix} &
      \mtx{B} =
      \begin{bmatrix}
        0 & 0 \\
        0 & 0 \\
        0 & 0 \\
        \left(\frac{1}{m} + \frac{r_b^2}{J}\right) C_2 &
        \left(\frac{1}{m} - \frac{r_b^2}{J}\right) C_4 \\
        \left(\frac{1}{m} - \frac{r_b^2}{J}\right) C_2 &
        \left(\frac{1}{m} + \frac{r_b^2}{J}\right) C_4
      \end{bmatrix} \\
      \mtx{C} =
      \begin{bmatrix}
        0 & 0 & 1 & 0 & 0 \\
        0 & 0 & 0 & 1 & 0 \\
        0 & 0 & 0 & 0 & 1
      \end{bmatrix} &
      \mtx{D} = \mtx{0}_{3 \times 2}
    \end{array}
  \end{equation}

  where $v = \frac{v_l + v_r}{2}$, $c = \cos\theta$, $s = \sin\theta$,
  $C_1 = -\frac{G_l^2 K_t}{K_v R r^2}$, $C_2 = \frac{G_l K_t}{Rr}$,
  $C_3 = -\frac{G_r^2 K_t}{K_v R r^2}$, and $C_4 = \frac{G_r K_t}{Rr}$. The
  constants $C_1$ through $C_4$ are from the derivation in section
  \ref{sec:differential_drive}.
\end{theorem}

We can also use this in an extended Kalman filter as is since the measurement
model ($\mtx{y} = \mtx{C}\mtx{x} + \mtx{D}\mtx{u}$) is linear.

\subsection{Improving model accuracy}

Figures \ref{fig:ltv_diff_drive_nonrotated_firstorder_xy} and
\ref{fig:ltv_diff_drive_nonrotated_firstorder_response} demonstrate the
tracking behavior of the linearized differential drive controller.
\begin{bookfigure}
  \begin{minisvg}{2}{build/\chapterpath/ltv_diff_drive_nonrotated_firstorder_xy}
    \caption{Linear time-varying differential drive controller x-y plot
      (first-order)}
    \label{fig:ltv_diff_drive_nonrotated_firstorder_xy}
  \end{minisvg}
  \hfill
  \begin{minisvg}{2}{build/\chapterpath/ltv_diff_drive_nonrotated_firstorder_response}
    \caption{Linear time-varying differential drive controller response
      (first-order)}
    \label{fig:ltv_diff_drive_nonrotated_firstorder_response}
  \end{minisvg}
\end{bookfigure}

The linearized differential drive model doesn't track well because the
first-order linearization of $\mtx{A}$ doesn't capture the full heading
dynamics, making the \gls{model} update inaccurate. This linearization
inaccuracy is evident in the Hessian matrix (second partial derivative with
respect to the state vector) being nonzero.
\begin{equation*}
  \frac{\partial^2 f(\mtx{x}, \mtx{u})}{\partial\mtx{x}^2} =
  \begin{bmatrix}
    0 & 0 & -\frac{v_l + v_r}{2}\cos\theta & 0 & 0 \\
    0 & 0 & -\frac{v_l + v_r}{2}\sin\theta & 0 & 0 \\
    0 & 0 & 0 & 0 & 0 \\
    0 & 0 & 0 & 0 & 0 \\
    0 & 0 & 0 & 0 & 0
  \end{bmatrix}
\end{equation*}

The second-order Taylor series expansion of the \gls{model} around $\mtx{x}_0$
would be
\begin{equation*}
  f(\mtx{x}, \mtx{u}_0) \approx f(\mtx{x}_0, \mtx{u}_0) +
    \frac{\partial f(\mtx{x}, \mtx{u})}{\partial\mtx{x}}(\mtx{x} - \mtx{x}_0) +
    \frac{1}{2}\frac{\partial^2 f(\mtx{x}, \mtx{u})}{\partial\mtx{x}^2}
    (\mtx{x} - \mtx{x}_0)^2
\end{equation*}

To include higher-order dynamics in the linearized differential drive model
integration, we'll apply the fourth-order Runge-Kutta integration method (RK4)
from theorem \ref{thm:rk4} to equation \eqref{eq:ltv_diff_drive_f}.

Figures \ref{fig:ltv_diff_drive_nonrotated_xy} and
\ref{fig:ltv_diff_drive_nonrotated_response} show a simulation using RK4
instead of the first-order \gls{model}.
\begin{bookfigure}
  \begin{minisvg}{2}{build/\chapterpath/ltv_diff_drive_nonrotated_xy}
    \caption{Linear time-varying differential drive controller (global reference
        frame formulation) x-y plot}
    \label{fig:ltv_diff_drive_nonrotated_xy}
  \end{minisvg}
  \hfill
  \begin{minisvg}{2}{build/\chapterpath/ltv_diff_drive_nonrotated_response}
    \caption{Linear time-varying differential drive controller (global reference
        frame formulation) response}
    \label{fig:ltv_diff_drive_nonrotated_response}
  \end{minisvg}
\end{bookfigure}

\subsection{Cross track error controller}

Figures \ref{fig:ltv_diff_drive_nonrotated_xy} and
\ref{fig:ltv_diff_drive_nonrotated_response} show the tracking performance of
the linearized differential drive controller for a given trajectory. The
performance-effort trade-off can be tuned rather intuitively via the Q and R
gains. However, if the $x$ and $y$ error cost are too high, the $x$ and $y$
components of the controller will fight each other, and it will take longer to
converge to the path. This can be fixed by applying a clockwise rotation matrix
to the global tracking error to transform it into the robot's coordinate frame.
\begin{equation*}
  \crdfrm{R}{\begin{bmatrix}
    e_x \\
    e_y \\
    e_\theta
  \end{bmatrix}} =
  \begin{bmatrix}
    \cos\theta & \sin\theta & 0 \\
    -\sin\theta & \cos\theta & 0 \\
    0 & 0 & 1
  \end{bmatrix}
  \crdfrm{G}{\begin{bmatrix}
    e_x \\
    e_y \\
    e_\theta
  \end{bmatrix}}
\end{equation*}

where the the superscript $R$ represents the robot's coordinate frame and the
superscript $G$ represents the global coordinate frame.

With this transformation, the $x$ and $y$ error cost in LQR penalize the error
ahead of the robot and cross-track error respectively instead of global pose
error. Since the cross-track error is always measured from the robot's
coordinate frame, the \gls{model} used to compute the LQR should be linearized
around $\theta = 0$ at all times.
\begin{align*}
  \mtx{A} &=
  \begin{bmatrix}
    0 & 0 & -\frac{v_l + v_r}{2}\sin 0 & \frac{1}{2}\cos 0 &
      \frac{1}{2}\cos 0 \\
    0 & 0 & \frac{v_l + v_r}{2}\cos 0 & \frac{1}{2}\sin 0 &
      \frac{1}{2}\sin 0 \\
    0 & 0 & 0 & -\frac{1}{2r_b} & \frac{1}{2r_b} \\
    0 & 0 & 0 & \left(\frac{1}{m} + \frac{r_b^2}{J}\right) C_1 &
      \left(\frac{1}{m} - \frac{r_b^2}{J}\right) C_3 \\
    0 & 0 & 0 & \left(\frac{1}{m} - \frac{r_b^2}{J}\right) C_1 &
      \left(\frac{1}{m} + \frac{r_b^2}{J}\right) C_3
  \end{bmatrix} \\
  \mtx{A} &=
  \begin{bmatrix}
    0 & 0 & 0 & \frac{1}{2} & \frac{1}{2} \\
    0 & 0 & \frac{v_l + v_r}{2} & 0 & 0 \\
    0 & 0 & 0 & -\frac{1}{2r_b} & \frac{1}{2r_b} \\
    0 & 0 & 0 & \left(\frac{1}{m} + \frac{r_b^2}{J}\right) C_1 &
      \left(\frac{1}{m} - \frac{r_b^2}{J}\right) C_3 \\
    0 & 0 & 0 & \left(\frac{1}{m} - \frac{r_b^2}{J}\right) C_1 &
      \left(\frac{1}{m} + \frac{r_b^2}{J}\right) C_3
  \end{bmatrix}
\end{align*}
\begin{theorem}[Linear time-varying differential drive controller]
  \label{thm:linear_time-varying_diff_drive_controller}

  \begin{equation*}
    \begin{array}{ccc}
      \mtx{x} =
      \begin{bmatrix}
        x & y & \theta & v_l & v_r
      \end{bmatrix}^T &
      \mtx{y} =
      \begin{bmatrix}
        \theta & v_l & v_r
      \end{bmatrix}^T &
      \mtx{u} =
      \begin{bmatrix}
        V_l & V_r
      \end{bmatrix}^T
    \end{array}
  \end{equation*}

  The following $\mtx{A}$ and $\mtx{B}$ matrices of a continuous system are used
  to compute the LQR.

  \begin{equation}
    \begin{array}{ll}
      \mtx{A} =
      \begin{bmatrix}
        0 & 0 & 0 & \frac{1}{2} & \frac{1}{2} \\
        0 & 0 & \frac{v_l + v_r}{2} & 0 & 0 \\
        0 & 0 & 0 & -\frac{1}{2r_b} & \frac{1}{2r_b} \\
        0 & 0 & 0 & \left(\frac{1}{m} + \frac{r_b^2}{J}\right) C_1 &
          \left(\frac{1}{m} - \frac{r_b^2}{J}\right) C_3 \\
        0 & 0 & 0 & \left(\frac{1}{m} - \frac{r_b^2}{J}\right) C_1 &
          \left(\frac{1}{m} + \frac{r_b^2}{J}\right) C_3
      \end{bmatrix} &
      \mtx{B} =
      \begin{bmatrix}
        0 & 0 \\
        0 & 0 \\
        0 & 0 \\
        \left(\frac{1}{m} + \frac{r_b^2}{J}\right) C_2 &
        \left(\frac{1}{m} - \frac{r_b^2}{J}\right) C_4 \\
        \left(\frac{1}{m} - \frac{r_b^2}{J}\right) C_2 &
        \left(\frac{1}{m} + \frac{r_b^2}{J}\right) C_4
      \end{bmatrix}
    \end{array}
  \end{equation}

  where $v = \frac{v_l + v_r}{2}$, $C_1 = -\frac{G_l^2 K_t}{K_v R r^2}$,
  $C_2 = \frac{G_l K_t}{Rr}$, $C_3 = -\frac{G_r^2 K_t}{K_v R r^2}$, and
  $C_4 = \frac{G_r K_t}{Rr}$. The constants $C_1$ through $C_4$ are from the
  derivation in section \ref{sec:differential_drive}.
  \begin{equation}
    \mtx{u} = \mtx{K}
    \left[
      \begin{array}{c|c}
        \begin{array}{cc}
          \cos\theta & \sin\theta \\
          -\sin\theta & \cos\theta
        \end{array} & \mtx{0}_{2 \times 3} \\
        \hline
        \mtx{0}_{3 \times 2} & \mtx{I}_{3 \times 3}
      \end{array}
    \right]
    (\mtx{r} - \mtx{x})
  \end{equation}

  The controller gain $\mtx{K}$ from LQR should be recomputed from the model
  linearized around the current state during every timestep.
\end{theorem}

With the \gls{model} in theorem
\ref{thm:linear_time-varying_diff_drive_controller}, $y$ is uncontrollable at
$v = 0$ because nonholonomic drivetrains are unable to move sideways. Some DARE
solvers throw errors in this case, but one can avoid it by linearizing the model
around a slightly nonzero velocity instead.

The controller in theorem \ref{thm:linear_time-varying_diff_drive_controller}
results in figures \ref{fig:ltv_diff_drive_traj_xy} and
\ref{fig:ltv_diff_drive_traj_response}, which show slightly better tracking
performance than the previous formulation.
\begin{bookfigure}
  \begin{minisvg}{2}{build/\chapterpath/ltv_diff_drive_traj_xy}
    \caption{Linear time-varying differential drive controller x-y plot}
    \label{fig:ltv_diff_drive_traj_xy}
  \end{minisvg}
  \hfill
  \begin{minisvg}{2}{build/\chapterpath/ltv_diff_drive_traj_response}
    \caption{Linear time-varying differential drive controller response}
    \label{fig:ltv_diff_drive_traj_response}
  \end{minisvg}
\end{bookfigure}

\subsection{Nonlinear observer design}

\subsubsection{Encoder position augmentation}

Estimation of the global pose can be significantly improved if encoder position
measurements are used instead of velocity measurements. By augmenting the plant
with the line integral of each wheel's velocity over time, we can provide a
mapping from model states to position measurements. We can augment the linear
subspace of the model as follows.

Augment the matrix equation with position states $x_l$ and $x_r$, which have the
model equations $\dot{x}_l = v_l$ and $\dot{x}_r = v_r$. The matrix elements
corresponding to $v_l$ in the first equation and $v_r$ in the second equation
will be $1$, and the others will be $0$ since they don't appear, so
$\dot{x}_l = 1v_l + 0v_r + 0x_l + 0x_r + 0V_l + 0V_r$ and
$\dot{x}_r = 0v_l + 1v_r + 0x_l + 0x_r + 0V_l + 0V_r$. The existing rows will
have zeroes inserted where $x_l$ and $x_r$ are multiplied in.
\begin{align*}
  \dot{\begin{bmatrix}
    x_l \\
    x_r
  \end{bmatrix}} &=
  \begin{bmatrix}
    1 & 0 \\
    0 & 1
  \end{bmatrix}
  \begin{bmatrix}
    v_l \\
    v_r
  \end{bmatrix} +
  \begin{bmatrix}
    0 & 0 \\
    0 & 0
  \end{bmatrix}
  \begin{bmatrix}
    V_l \\
    V_r
  \end{bmatrix}
\end{align*}

This produces the following linear subspace over
$\mtx{x} = \begin{bmatrix}v_l & v_r & x_l & x_r\end{bmatrix}^T$.
\begin{equation}
  \begin{array}{ll}
    \mtx{A} =
    \begin{bmatrix}
      \left(\frac{1}{m} + \frac{r_b^2}{J}\right) C_1 &
        \left(\frac{1}{m} - \frac{r_b^2}{J}\right) C_3 & 0 & 0 \\
      \left(\frac{1}{m} - \frac{r_b^2}{J}\right) C_1 &
        \left(\frac{1}{m} + \frac{r_b^2}{J}\right) C_3 & 0 & 0 \\
      1 & 0 & 0 & 0 \\
      0 & 1 & 0 & 0
    \end{bmatrix} &
    \mtx{B} =
    \begin{bmatrix}
      \left(\frac{1}{m} + \frac{r_b^2}{J}\right) C_2 &
        \left(\frac{1}{m} - \frac{r_b^2}{J}\right) C_4 \\
      \left(\frac{1}{m} - \frac{r_b^2}{J}\right) C_2 &
        \left(\frac{1}{m} + \frac{r_b^2}{J}\right) C_4 \\
      0 & 0 \\
      0 & 0
    \end{bmatrix}
    \label{eq:diff_drive_linear_subspace}
  \end{array}
\end{equation}

The measurement model for the complete nonlinear model is now
$\mtx{y} = \begin{bmatrix}\theta & x_l & x_r\end{bmatrix}^T$ instead of
$\mtx{y} = \begin{bmatrix}\theta & v_l & v_r\end{bmatrix}^T$.

\subsubsection{U error estimation}

As per subsection \ref{subsec:input_error_estimation}, we will now augment the
\gls{model} so $u_{error}$ states are added to the \glspl{control input}.

The \gls{plant} and \gls{observer} augmentations should be performed before the
\gls{model} is \glslink{discretization}{discretized}. After the \gls{controller}
gain is computed with the unaugmented discrete \gls{model}, the controller may
be augmented. Therefore, the \gls{plant} and \gls{observer} augmentations assume
a continuous \gls{model} and the \gls{controller} augmentation assumes a
discrete \gls{controller}.

The three $u_{error}$ states we'll be adding are $u_{error,l}$, $u_{error,r}$,
and $u_{error,heading}$ for left voltage error, right voltage error, and heading
error respectively. The left and right wheel positions are filtered encoder
positions and are not adjusted for heading error. The turning angle computed
from the left and right wheel positions is adjusted by the gyroscope heading.
The heading $u_{error}$ state is the heading error between what the wheel
positions imply and the gyroscope measurement.

The full state is thus
$\mtx{x} = \begin{bmatrix}x & y & \theta & v_l & v_r & x_l & x_r & u_{error,l} &
  u_{error,r} & u_{error,heading}\end{bmatrix}^T$.

The complete nonlinear model is as follows. Let $v = \frac{v_l + v_r}{2}$. The
three $u_{error}$ states augment the linear subspace, so the nonlinear pose
dynamics are the same.
\begin{align}
  \dot{\begin{bmatrix}
    x \\
    y \\
    \theta
  \end{bmatrix}} &=
    \begin{bmatrix}
      v\cos\theta \\
      v\sin\theta \\
      \frac{v_r}{2r_b} - \frac{v_l}{2r_b}
    \end{bmatrix}
\end{align}

The left and right voltage error states should be mapped to the corresponding
velocity states, so the system matrix should be augmented with $\mtx{B}$.

The heading $u_{error}$ is measuring counterclockwise encoder understeer
relative to the gyroscope heading, so it should add to the left position and
subtract from the right position for clockwise correction of encoder positions.
That corresponds to the following input mapping vector.
\begin{equation*}
  \mtx{B}_{\theta} = \begin{bmatrix}
    0 \\
    0 \\
    1 \\
    -1
  \end{bmatrix}
\end{equation*}

Now we'll augment the linear system matrix horizontally to accomodate the
$u_{error}$ states.
\begin{equation}
  \dot{\begin{bmatrix}
    v_l \\
    v_r \\
    x_l \\
    x_r
  \end{bmatrix}} =
    \begin{bmatrix}
      \mtx{A} & \mtx{B} & \mtx{B}_{\theta}
    \end{bmatrix}
    \begin{bmatrix}
      v_l \\
      v_r \\
      x_l \\
      x_r \\
      u_{error,l} \\
      u_{error,r} \\
      u_{error,heading}
    \end{bmatrix} + \mtx{B}\mtx{u}
\end{equation}

$\mtx{A}$ and $\mtx{B}$ are the linear subspace from equation
\eqref{eq:diff_drive_linear_subspace}.

The $u_{error}$ states have no dynamics. The observer selects them to minimize
the difference between the expected and actual measurements.
\begin{equation}
  \dot{\begin{bmatrix}
    u_{error,l} \\
    u_{error,r} \\
    u_{error,heading}
  \end{bmatrix}} = \mtx{0}_{3 \times 1}
\end{equation}

The controller is augmented as follows.
\begin{equation}
  \begin{array}{ccc}
    \mtx{K}_{error} =
    \begin{bmatrix}
      1 & 0 & 0 \\
      0 & 1 & 0
    \end{bmatrix} &
    \mtx{K}_{aug} = \begin{bmatrix}
      \mtx{K} & \mtx{K}_{error}
    \end{bmatrix} &
    \mtx{r}_{aug} = \begin{bmatrix}
      \mtx{r} \\
      0 \\
      0 \\
      0
    \end{bmatrix}
  \end{array}
\end{equation}

This controller augmentation compensates for unmodeled dynamics like:
\begin{enumerate}
  \item Understeer caused by wheel friction inherent in skid-steer robots
  \item Battery voltage drop under load, which reduces the available control
    authority
\end{enumerate}
\begin{remark}
  The process noise for the voltage error states should be how much the voltage
  can be expected to drop. The heading error state should be the encoder
  \gls{model} uncertainty.
\end{remark}

