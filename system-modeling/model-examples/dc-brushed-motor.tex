\section{DC brushed motor}
\label{sec:dc_brushed_motor}

We will be deriving a first-order \gls{model} for a DC brushed motor. A
second-order \gls{model} would include the inductance of the motor windings as
well, but we're assuming the time constant of the inductor is small enough that
its affect on the \gls{model} behavior is negligible for FRC use cases (see
subsection \ref{subsec:root_locus} for a demonstration of this for a real DC
brushed motor).

The first-order \gls{model} will only require numbers from the motor's
datasheet. The second-order \gls{model} would require measuring the motor
inductance as well, which generally isn't in the datasheet. It can be difficult
to measure accurately without the right equipment.

\subsection{Equations of motion}

The circuit for a DC brushed motor is shown in figure
\ref{fig:dc_motor_circuit}.
\begin{bookfigure}
  \begin{tikzpicture}[auto, >=latex', circuit ee IEC,
                      set resistor graphic=var resistor IEC graphic]
    \node [opencircuit] (start) at (0,0) {};
    \node [] (V+) at (-0.5,0) { $+$ };
    \node [opencircuit] (end) at (0,-3.5) {};
    \node [] (V-) at (-0.5,-3.5) { $-$ };
    \node [coordinate] (topright) at (2.5,0) {};
    \node [coordinate] (bottomright) at (2.5,-3.5) {};
    \node [] at (0, -1.75) { $V$ };
    \draw (start) to (topright)
                  to [resistor={near start, info'={ $R$ }},
                      voltage source={near end, direction info'={<-},
                      info={ $V_{emf}=\frac{\omega}{K_v}$ }}] (bottomright)
                  to (end);
  \end{tikzpicture}

  \caption{DC brushed motor circuit}
  \label{fig:dc_motor_circuit}
\end{bookfigure}

$V$ is the voltage applied to the motor, $I$ is the current through the motor in
Amps, $R$ is the resistance across the motor in Ohms, $\omega$ is the angular
velocity of the motor in radians per second, and $K_v$ is the angular velocity
constant in radians per second per Volt. This circuit reflects the following
relation.
\begin{equation}
  V = IR + \frac{\omega}{K_v} \label{eq:motor_V}
\end{equation}

The mechanical relation for a DC brushed motor is
\begin{equation}
  \tau = K_t I
\end{equation}

where $\tau$ is the torque produced by the motor in Newton-meters and $K_t$ is
the torque constant in Newton-meters per Amp. Therefore
\begin{equation*}
  I = \frac{\tau}{K_t}
\end{equation*}

Substitute this into equation \eqref{eq:motor_V}.

\index{FRC models!DC brushed motor equations}
\begin{equation}
  V = \frac{\tau}{K_t} R + \frac{\omega}{K_v} \label{eq:motor_tau_V}
\end{equation}

\subsection{Calculating constants}

A typical motor's datasheet should include graphs of the motor's measured torque
and current for different angular velocities for a given voltage applied to the
motor. Figure \ref{fig:motor_data} is an example. Data for the most common
motors in FRC can be found at \url{https://motors.vex.com}.
\begin{svg}{build/\chapterpath/motor_data}
  \caption{Example motor datasheet for 775pro}
  \label{fig:motor_data}
\end{svg}

\subsubsection{Torque constant $K_t$}
\begin{align}
  \tau &= K_t I \nonumber \\
  K_t &= \frac{\tau}{I} \nonumber \\
  K_t &= \frac{\tau_{stall}}{I_{stall}}
\end{align}

where $\tau_{stall}$ is the stall torque and $I_{stall}$ is the stall current of
the motor from its datasheet.

\subsubsection{Resistance $R$}

Recall equation \eqref{eq:motor_V}.
\begin{equation*}
  V = IR + \frac{\omega}{K_v}
\end{equation*}

When the motor is stalled, $\omega = 0$.
\begin{align}
  V &= I_{stall} R \nonumber \\
  R &= \frac{V}{I_{stall}}
\end{align}

where $I_{stall}$ is the stall current of the motor and $V$ is the voltage
applied to the motor at stall.

\subsubsection{Angular velocity constant $K_v$}

Recall equation \eqref{eq:motor_V}.
\begin{align*}
  V &= IR + \frac{\omega}{K_v} \\
  V - IR &= \frac{\omega}{K_v} \\
  K_v &= \frac{\omega}{V - IR}
\end{align*}

When the motor is spinning under no load
\begin{align}
  K_v &= \frac{\omega_{free}}{V - I_{free}R}
\end{align}

where $\omega_{free}$ is the angular velocity of the motor under no load (also
known as the free speed), and $V$ is the voltage applied to the motor when it's
spinning at $\omega_{free}$, and $I_{free}$ is the current drawn by the motor
under no load.

If several identical motors are being used in one gearbox for a mechanism,
multiply the stall torque by the number of motors.

\subsection{Current limiting}

Current limiting of a DC brushed motor reduces the maximum input voltage to
avoid exceeding a current threshold. Predictive current limiting uses a
projected estimate of the current, so the voltage is reduced before the current
threshold is exceeded. Reactive current limiting uses an actual current
measurement, so the voltage is reduced after the current threshold is exceeded.

The following pseudocode demonstrates each type of current limiting.
\begin{code}{Python}{snippets/current_limit.py}
  \caption{Limits current of DC motor to $I_{max}$}
\end{code}
