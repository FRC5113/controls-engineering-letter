\section{Single-jointed arm}

\subsection{Equations of motion}

This single-jointed arm consists of a DC brushed motor attached to a pulley that
spins a straight bar in pitch.

\begin{bookfigure}
  \begin{tikzpicture}[auto, >=latex', circuit ee IEC,
                      set resistor graphic=var resistor IEC graphic]
    % \draw [help lines] (-1,-3) grid (7,4);

    % Electrical equivalent circuit
    \draw (0,2) to [voltage source={direction info'={->}, info'=$V$}] (0,0);
    \draw (0,2) to [current direction={info=$I$}] (0,3);
    \draw (0,3) -- (0.5,3);
    \draw (0.5,3) to [resistor={info={$R$}}] (2,3);

    \draw (2,3) -- (2.5,3);
    \draw (2.5,3) to [voltage source={direction info'={->}, info'=$V_{emf}$}]
      (2.5,0);
    \draw (0,0) -- (2.5,0);

    % Motor
    \begin{scope}[xshift=2.4cm,yshift=1.05cm]
      \draw[fill=black] (0,0) rectangle (0.2,0.9);
      \draw[fill=white] (0.1,0.45) ellipse (0.3 and 0.3);
    \end{scope}

    % Transmission gear one
    \begin{scope}[xshift=3.75cm,yshift=1.17cm]
      \draw[fill=black!50] (0.2,0.33) ellipse (0.08 and 0.33);
      \draw[fill=black!50, color=black!50] (0,0) rectangle (0.2,0.66);
      \draw[fill=white] (0,0.33) ellipse (0.08 and 0.33);
      \draw (0,0.66) -- (0.2,0.66);
      \draw (0,0) -- (0.2,0) node[pos=0.5,below] {$G$};
    \end{scope}

    % Output shaft of motor
    \begin{scope}[xshift=2.8cm,yshift=1.45cm]
      \draw[fill=black!50] (0,0) rectangle (0.95,0.1);
    \end{scope}

    % Angular velocity arrow of drive -> transmission
    \draw[line width=0.7pt,<-] (3.2,1) arc (-30:30:1) node[above] {$\omega_m$};

    % Transmission gear two
    \begin{scope}[xshift=3.75cm,yshift=1.83cm]
      \draw[fill=black!50] (0.2,0.68) ellipse (0.13 and 0.67);
      \draw[fill=black!50, color=black!50] (0,0) rectangle (0.2,1.35);
      \draw[fill=white] (0,0.68) ellipse (0.13 and 0.67);
      \draw (0,1.35) -- (0.2,1.35);
      \draw (0,0) -- (0.2,0);
    \end{scope}

    \begin{scope}[xshift=5.075cm,yshift=2.4cm]
      % Single-jointed arm
      \draw[fill=white] (0,0) -- (0.1,-0.05) -- (0.35,1.45) -- (0.25,1.5)
        -- cycle;
      \draw[fill=black!50] (0.1,-0.05) -- (0.3,-0.05) -- (0.55,1.45) --
        (0.35,1.45) -- cycle;
      \draw[fill=white] (0.25,1.5) -- (0.35,1.45) -- (0.55,1.45) -- (0.45,1.5)
        -- cycle;

      % Arm length arrow
      \draw[line width=0.7pt,<->] (0.55,-0.05) -- node[right] {$l$} (0.8,1.45);

      % Mass label
      \draw (-0.05,1.2) node {$m$};
    \end{scope}

    % Transmission shaft from gear two to arm
    \begin{scope}[xshift=4.09cm,yshift=2.42cm]
      \draw[fill=black!50] (0,0) rectangle (1.06,0.1);
    \end{scope}

    % Angular velocity arrow between transmission and arm
    \draw[line width=0.7pt,->] (4.54,1.97) arc (-30:30:1) node[above]
      {$\omega_{arm}$};

    % Descriptions of subsystems under graphic
    \begin{scope}[xshift=-0.5cm,yshift=-0.28cm]
      \draw[decorate,decoration={brace,amplitude=10pt}]
        (3.5,0) -- (0,0) node[midway,yshift=-20pt] {circuit};
      \draw[decorate,decoration={brace,amplitude=10pt}]
        (6.55,0) -- (3.75,0) node[midway,yshift=-20pt] {mechanics};
    \end{scope}
  \end{tikzpicture}

  \caption{Single-jointed arm system diagram}
  \label{fig:single_jointed_arm}
\end{bookfigure}

Gear ratios are written as output over input, so $G$ is greater than one in
figure \ref{fig:single_jointed_arm}.

We want to derive an equation for the arm angular acceleration $\dot{\omega}_m$
given an input voltage $V$, which we can integrate to get arm angular velocity
and angle.

We will start with the equation derived earlier for a DC brushed motor, equation
(\ref{eq:motor_tau_V}).

\begin{equation*}
  V = \frac{\tau_m}{K_t} R + \frac{\omega_m}{K_v}
\end{equation*}

Solve for the angular acceleration. First, we'll rearrange the terms because
from inspection, $V$ is the \gls{model} \gls{input}, $\omega_m$ is the
\gls{state}, and $\tau_m$ contains the angular acceleration.

\begin{equation*}
  V = \frac{R}{K_t} \tau_m + \frac{1}{K_v} \omega_m
\end{equation*}

Solve for $\tau_m$.

\begin{align*}
  V &= \frac{R}{K_t} \tau_m + \frac{1}{K_v} \omega_m \\
  \frac{R}{K_t} \tau_m &= V - \frac{1}{K_v} \omega_m \\
  \tau_m &= \frac{K_t}{R} V - \frac{K_t}{K_v R} \omega_m
\end{align*}

Since $\tau_m G = \tau_{arm}$ and $\omega_m = G \omega_{arm}$

\begin{align}
  \left(\frac{\tau_{arm}}{G}\right) &= \frac{K_t}{R} V -
    \frac{K_t}{K_v R} (G \omega_f) \nonumber \\
  \frac{\tau_{arm}}{G} &= \frac{K_t}{R} V - \frac{G K_t}{K_v R} \omega_{arm}
    \nonumber \\
  \tau_{arm} &= \frac{G K_t}{R} V - \frac{G^2 K_t}{K_v R} \omega_{arm}
    \label{eq:tau_arm}
\end{align}

The torque applied to the arm is defined as

\begin{equation}
  \tau_{arm} = J \dot{\omega}_{arm} \label{eq:tau_arm_def}
\end{equation}

where $J$ is the moment of inertia of the arm and $\dot{\omega}_{arm}$ is the
angular acceleration. Substitute equation (\ref{eq:tau_arm_def}) into equation
(\ref{eq:tau_arm}).

\begin{align}
  (J \dot{\omega}_{arm}) &= \frac{G K_t}{R} V - \frac{G^2 K_t}{K_v R}
    \omega_{arm} \nonumber \\
  \dot{\omega}_{arm} &= \frac{G K_t}{RJ} V - \frac{G^2 K_t}{K_v RJ} \omega_{arm}
    \label{eq:dot_omega_arm}
\end{align}

This model will be converted to state-space notation in section
\ref{sec:ss_model_single-jointed_arm}.

\subsection{Calculating constants}

\subsubsection{Moment of inertia J}

Given the simplicity of this mechanism, it may be easier to compute this value
theoretically using material properties in CAD. $J$ can also be approximated as
the moment of inertia of a thin rod rotating around one end. Therefore

\begin{equation}
  J = \frac{1}{3}ml^2
\end{equation}

where $m$ is the mass of the arm and $l$ is the length of the arm. Otherwise, a
procedure for measuring it experimentally is presented below.

First, rearrange equation (\ref{eq:dot_omega_arm}) into the form $y = mx + b$
such that $J$ is in the numerator of $m$.

\begin{align*}
  \dot{\omega}_{arm} &= \frac{G K_t}{RJ} V - \frac{G^2 K_t}{K_v RJ} \omega_{arm}
    \\
  J\dot{\omega}_{arm} &= \frac{G K_t}{R} V - \frac{G^2 K_t}{K_v R} \omega_{arm}
\end{align*}

Multiply by $\frac{K_v R}{G^2 K_t}$ on both sides.

\begin{align}
  \frac{J K_v R}{G^2 K_t} \dot{\omega}_{arm} &= \frac{G K_t}{R} \cdot
    \frac{K_v R}{G^2 K_t} V - \omega_{arm} \nonumber \\
  \frac{J K_v R}{G^2 K_t} \dot{\omega}_{arm} &= \frac{K_v}{G} V - \omega_{arm}
    \nonumber \\
  \omega_{arm} &= \frac{K_v}{G} V - \frac{J K_v R}{G^2 K_t} \dot{\omega}_{arm}
    \label{eq:arm_J_regression}
\end{align}

The test procedure is as follows.

\begin{enumerate}
  \item Orient the arm such that its axis of rotation is aligned with gravity
    (i.e., the arm is on its side). This avoids gravity affecting the
    measurements.
  \item Run the arm forward at a constant voltage. Record the angular velocity
    over time.
  \item Compute the angular acceleration from the angular velocity data as the
    difference between each sample divided by the time between them.
  \item Perform a linear regression of angular velocity versus angular
    acceleration. The slope of this line has the form $-\frac{J K_v R}{G^2 K_t}$
    as per equation (\ref{eq:arm_J_regression}).
  \item Multiply the slope by $-\frac{G^2 K_t}{K_v R}$ to obtain a least squares
    estimate of $J$.
\end{enumerate}
