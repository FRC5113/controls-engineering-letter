\chapterimage{system-modeling.jpg}{Hills by northbound freeway between Santa Maria and Ventura}

\chapter{Lagrangian mechanics examples}

A \gls{model} is a set of differential equations describing how the \gls{system}
behaves over time. There are two common approaches for developing them.
\begin{enumerate}
  \item Collecting data on the physical system's behavior and performing
    \gls{system} identification with it.
  \item Using physics to derive the \gls{system}'s model from first principles.
\end{enumerate}

This chapter covers the second approach using Lagrangian mechanics.

We suggest reading \textit{An introduction to Lagrangian and Hamiltonian
Mechanics} by Simon J.A. Malham for the basics
\cite{bib:an_intro_to_lagrangian_and_hamiltonian_mechanics}. Then, we'll apply
Lagrangian mechanics to FRC mechanisms in particular.

The \glspl{model} derived here should cover most types of motion seen on an FRC
robot. Furthermore, they can be easily tweaked to describe many types of
mechanisms just by pattern-matching. There's only so many ways to hook up a mass
to a motor in FRC. The flywheel \gls{model} can be used for spinning mechanisms,
the elevator \gls{model} can be used for spinning mechanisms transformed to
linear motion, and the single-jointed arm \gls{model} can be used for rotating
servo mechanisms (it's just the flywheel \gls{model} augmented with a position
\gls{state}).

These \glspl{model} assume all motor controllers driving DC brushed motors are
set to brake mode instead of coast mode. Brake mode behaves the same as coast
mode except where the applied voltage is zero. In brake mode, the motor leads
are shorted together to prevent movement. In coast mode, the motor leads are an
open circuit.

\renewcommand*{\chapterpath}{\partpath/lagrangian-mechanics-examples}
\section{Single-jointed arm}

\subsection{Equations of motion}

This single-jointed arm consists of a DC brushed motor attached to a pulley that
spins a straight bar in pitch.
\begin{bookfigure}
  \begin{tikzpicture}[auto, >=latex', circuit ee IEC,
                      set resistor graphic=var resistor IEC graphic]
    % \draw [help lines] (-1,-3) grid (7,4);

    % Electrical equivalent circuit
    \draw (0,2) to [voltage source={direction info'={->}, info'=$V$}] (0,0);
    \draw (0,2) to [current direction={info=$I$}] (0,3);
    \draw (0,3) -- (0.5,3);
    \draw (0.5,3) to [resistor={info={$R$}}] (2,3);

    \draw (2,3) -- (2.5,3);
    \draw (2.5,3) to [voltage source={direction info'={->}, info'=$V_{emf}$}]
      (2.5,0);
    \draw (0,0) -- (2.5,0);

    % Motor
    \begin{scope}[xshift=2.4cm,yshift=1.05cm]
      \draw[fill=black] (0,0) rectangle (0.2,0.9);
      \draw[fill=white] (0.1,0.45) ellipse (0.3 and 0.3);
    \end{scope}

    % Transmission gear one
    \begin{scope}[xshift=3.75cm,yshift=1.17cm]
      \draw[fill=black!50] (0.2,0.33) ellipse (0.08 and 0.33);
      \draw[fill=black!50, color=black!50] (0,0) rectangle (0.2,0.66);
      \draw[fill=white] (0,0.33) ellipse (0.08 and 0.33);
      \draw (0,0.66) -- (0.2,0.66);
      \draw (0,0) -- (0.2,0) node[pos=0.5,below] {$G$};
    \end{scope}

    % Output shaft of motor
    \begin{scope}[xshift=2.8cm,yshift=1.45cm]
      \draw[fill=black!50] (0,0) rectangle (0.95,0.1);
    \end{scope}

    % Angular velocity arrow of drive -> transmission
    \draw[line width=0.7pt,<-] (3.2,1) arc (-30:30:1) node[above] {$\omega_m$};

    % Transmission gear two
    \begin{scope}[xshift=3.75cm,yshift=1.83cm]
      \draw[fill=black!50] (0.2,0.68) ellipse (0.13 and 0.67);
      \draw[fill=black!50, color=black!50] (0,0) rectangle (0.2,1.35);
      \draw[fill=white] (0,0.68) ellipse (0.13 and 0.67);
      \draw (0,1.35) -- (0.2,1.35);
      \draw (0,0) -- (0.2,0);
    \end{scope}
    \begin{scope}[xshift=5.075cm,yshift=2.4cm]
      % Single-jointed arm
      \draw[fill=white] (0,0) -- (0.1,-0.05) -- (0.35,1.45) -- (0.25,1.5)
        -- cycle;
      \draw[fill=black!50] (0.1,-0.05) -- (0.3,-0.05) -- (0.55,1.45) --
        (0.35,1.45) -- cycle;
      \draw[fill=white] (0.25,1.5) -- (0.35,1.45) -- (0.55,1.45) -- (0.45,1.5)
        -- cycle;

      % Arm length arrow
      \draw[line width=0.7pt,<->] (0.55,-0.05) -- node[right] {$l$} (0.8,1.45);

      % Mass label
      \draw (-0.05,1.2) node {$m$};
    \end{scope}

    % Transmission shaft from gear two to arm
    \begin{scope}[xshift=4.09cm,yshift=2.42cm]
      \draw[fill=black!50] (0,0) rectangle (1.06,0.1);
    \end{scope}

    % Angular velocity arrow between transmission and arm
    \draw[line width=0.7pt,->] (4.54,1.97) arc (-30:30:1) node[above]
      {$\omega_{arm}$};

    % Descriptions of subsystems under graphic
    \begin{scope}[xshift=-0.5cm,yshift=-0.28cm]
      \draw[decorate,decoration={brace,amplitude=10pt}]
        (3.5,0) -- (0,0) node[midway,yshift=-20pt] {circuit};
      \draw[decorate,decoration={brace,amplitude=10pt}]
        (6.55,0) -- (3.75,0) node[midway,yshift=-20pt] {mechanics};
    \end{scope}
  \end{tikzpicture}

  \caption{Single-jointed arm system diagram}
  \label{fig:single_jointed_arm}
\end{bookfigure}

Gear ratios are written as output over input, so $G$ is greater than one in
figure \ref{fig:single_jointed_arm}.

We want to derive an equation for the arm angular acceleration
$\dot{\omega}_{arm}$ given an input voltage $V$, which we can integrate to get
arm angular velocity and angle.

We will start with the equation derived earlier for a DC brushed motor, equation
\eqref{eq:motor_tau_V}.
\begin{align}
  V &= \frac{\tau_m}{K_t} R + \frac{\omega_m}{K_v} \nonumber
  \intertext{Solve for the angular acceleration. First, we'll rearrange the
    terms because from inspection, $V$ is the \gls{model} \gls{input},
    $\omega_m$ is the \gls{state}, and $\tau_m$ contains the angular
    acceleration.}
  V &= \frac{R}{K_t} \tau_m + \frac{1}{K_v} \omega_m \nonumber
  \intertext{Solve for $\tau_m$.}
  V &= \frac{R}{K_t} \tau_m + \frac{1}{K_v} \omega_m \nonumber \\
  \frac{R}{K_t} \tau_m &= V - \frac{1}{K_v} \omega_m \nonumber \\
  \tau_m &= \frac{K_t}{R} V - \frac{K_t}{K_v R} \omega_m
  \intertext{Since $\tau_m G = \tau_{arm}$ and $\omega_m = G \omega_{arm}$,}
  \left(\frac{\tau_{arm}}{G}\right) &= \frac{K_t}{R} V -
    \frac{K_t}{K_v R} (G \omega_{arm}) \nonumber \\
  \frac{\tau_{arm}}{G} &= \frac{K_t}{R} V - \frac{G K_t}{K_v R} \omega_{arm}
    \nonumber \\
  \tau_{arm} &= \frac{G K_t}{R} V - \frac{G^2 K_t}{K_v R} \omega_{arm}
    \label{eq:tau_arm}
  \intertext{The torque applied to the arm is defined as}
  \tau_{arm} &= J \dot{\omega}_{arm} \label{eq:tau_arm_def}
  \intertext{where $J$ is the moment of inertia of the arm and
    $\dot{\omega}_{arm}$ is the angular acceleration. Substitute equation
    \eqref{eq:tau_arm_def} into equation \eqref{eq:tau_arm}.}
  (J \dot{\omega}_{arm}) &= \frac{G K_t}{R} V - \frac{G^2 K_t}{K_v R}
    \omega_{arm} \nonumber \\
  \dot{\omega}_{arm} &= -\frac{G^2 K_t}{K_v RJ} \omega_{arm} +
    \frac{G K_t}{RJ} V \nonumber
  \intertext{We'll relabel $\omega_{arm}$ as $\omega$ for convenience.}
  \dot{\omega} &= -\frac{G^2 K_t}{K_v RJ} \omega + \frac{G K_t}{RJ} V
    \label{eq:dot_omega_arm}
\end{align}

This model will be converted to state-space notation in section
\ref{sec:ss_model_single-jointed_arm}.

\subsection{Calculating constants}

\subsubsection{Moment of inertia J}

Given the simplicity of this mechanism, it may be easier to compute this value
theoretically using material properties in CAD. $J$ can also be approximated as
the moment of inertia of a thin rod rotating around one end. Therefore
\begin{equation}
  J = \frac{1}{3}ml^2
\end{equation}

where $m$ is the mass of the arm and $l$ is the length of the arm. Otherwise, a
procedure for measuring it experimentally is presented below.

First, rearrange equation \eqref{eq:dot_omega_arm} into the form $y = mx + b$
such that $J$ is in the numerator of $m$.
\begin{align}
  \dot{\omega} &= -\frac{G^2 K_t}{K_v RJ} \omega + \frac{G K_t}{RJ} V \nonumber
    \\
  J\dot{\omega} &= -\frac{G^2 K_t}{K_v R} \omega + \frac{G K_t}{R} V \nonumber
  \intertext{Multiply by $\frac{K_v R}{G^2 K_t}$ on both sides.}
  \frac{J K_v R}{G^2 K_t} \dot{\omega} &= -\omega + \frac{G K_t}{R} \cdot
    \frac{K_v R}{G^2 K_t} V \nonumber \\
  \frac{J K_v R}{G^2 K_t} \dot{\omega} &= -\omega + \frac{K_v}{G} V \nonumber \\
  \omega &= -\frac{J K_v R}{G^2 K_t} \dot{\omega} + \frac{K_v}{G} V
    \label{eq:arm_J_regression}
\end{align}

The test procedure is as follows.
\begin{enumerate}
  \item Orient the arm such that its axis of rotation is aligned with gravity
    (i.e., the arm is on its side). This avoids gravity affecting the
    measurements.
  \item Run the arm forward at a constant voltage. Record the angular velocity
    over time.
  \item Compute the angular acceleration from the angular velocity data as the
    difference between each sample divided by the time between them.
  \item Perform a linear regression of angular velocity versus angular
    acceleration. The slope of this line has the form $-\frac{J K_v R}{G^2 K_t}$
    as per equation \eqref{eq:arm_J_regression}.
  \item Multiply the slope by $-\frac{G^2 K_t}{K_v R}$ to obtain a least squares
    estimate of $J$.
\end{enumerate}

\section{Pendulum}

\subsection{State-space model}

Below is the \gls{model} for a pendulum
\begin{equation*}
  \ddot{\theta} = -\frac{g}{l}\sin\theta
\end{equation*}

where $\theta$ is the angle of the pendulum and $l$ is the length of the
pendulum.

Since state-space representation requires that only single derivatives be used,
they should be broken up as separate \glspl{state}. We'll reassign
$\dot{\theta}$ to be $\omega$ so the derivatives are easier to keep straight for
state-space representation.
\begin{align*}
  \dot{\omega} &= -\frac{g}{l}\sin\theta
  \intertext{Now separate the \glspl{state}.}
  \dot{\theta} &= \omega \\
  \dot{\omega} &= -\frac{g}{l} \sin\theta
  \intertext{This makes our state vector
    $\begin{bmatrix}\theta & \omega\end{bmatrix}\T$ and our nonlinear model the
    following.}
  f(\mat{x}, \mat{u}) &=
  \begin{bmatrix}
    \omega \\
    -\frac{g}{l}\sin\theta
  \end{bmatrix}
\end{align*}

\subsubsection{Linearization around $\theta = 0$}

To apply our tools for linear control theory, the \gls{model} must be a linear
combination of the \glspl{state} and \glspl{input} (addition and multiplication
by constants). Since this \gls{model} is nonlinear on account of the sine
function, we should \glslink{linearization}{linearize}
\index{nonlinear control!linearization} it.

Linearization finds a tangent line to the nonlinear dynamics at a desired point
in the state-space. The Taylor series is a way to approximate arbitrary
functions with polynomials, so we can use it for linearization.

The taylor series expansion for $\sin\theta$ around $\theta = 0$ is
$\theta - \frac{1}{6} \theta^3 + \frac{1}{120} \theta^5 - \ldots$. We'll take
just the first-order term $\theta$ to obtain a linear function.
\begin{align}
  \dot{\theta} &= \omega \nonumber \\
  \dot{\omega} &= -\frac{g}{l} \theta \nonumber
  \intertext{Now write the \gls{model} in state-space representation. We'll
    write out the system of equations with the zeroed variables included to
    assist with this.}
  \dot{\theta} &= \;\;\;\,0 \theta + 1 \omega \nonumber \\
  \dot{\omega} &= -\frac{g}{l} \theta + 0 \omega \nonumber
  \intertext{Factor out $\theta$ and $\omega$ into a column vector.}
  \dot{
  \begin{bmatrix}
    \theta \\
    \omega
  \end{bmatrix}} &=
  \begin{bmatrix}
    0 & 1 \\
    -\frac{g}{l} & 0
  \end{bmatrix}
  \begin{bmatrix}
    \theta \\
    \omega
  \end{bmatrix}
\end{align}

\subsubsection{Linearization with the Jacobian}

Here's the original nonlinear model in state-space representation.
\begin{align*}
  f(\mat{x}, \mat{u}) &=
  \begin{bmatrix}
    \omega \\
    -\frac{g}{l}\sin\theta
  \end{bmatrix}
  \intertext{If we want to linearize around an arbitrary point, we can take the
    Jacobian with respect to $\mat{x}$.}
  \frac{\partial f(\mat{x}, \mat{u})}{\partial\mat{x}} &=
  \begin{bmatrix}
    0 & 1 \\
    -\frac{g}{l}\cos\theta & 0
  \end{bmatrix}
\end{align*}

For full \gls{state} feedback, knowledge of all \glspl{state} is required. If
not all \glspl{state} are measured directly, an estimator can be used to
supplement them.

We may only be measuring $\theta$ in the pendulum example, not $\dot{\theta}$,
so we'll need to estimate the latter. The $\mat{C}$ matrix the \gls{observer}
would use in this case is
\begin{align*}
  \mat{C} &= \begin{bmatrix}
    1 & 0 \\
  \end{bmatrix}
  \intertext{Therefore, the output vector is}
  \mat{y} &= \mat{C}\mat{x} \\
  \mat{y} &= \begin{bmatrix}
    1 & 0
  \end{bmatrix}
  \begin{bmatrix}
    \theta \\
    \omega
  \end{bmatrix} \\
  \mat{y} &= 1\theta + 0\omega \\
  \mat{y} &= \theta
\end{align*}
\begin{theorem}[Linear time-varying pendulum state-space model with no input]
  \begin{align*}
    \dot{\mat{x}} &= \mat{A} \mat{x} \\
    \mat{y} &= \mat{C} \mat{x}
  \end{align*}
  \begin{equation*}
    \mat{x} =
    \begin{bmatrix}
      \theta \\
      \omega
    \end{bmatrix}
    \quad
    \mat{y} = \theta
  \end{equation*}
  \begin{align}
    \mat{A} &=
    \begin{bmatrix}
      0 & 1 \\
      -\frac{g}{l}\cos\theta & 0
    \end{bmatrix} \\
    \mat{C} &=
    \begin{bmatrix}
      1 & 0
    \end{bmatrix}
  \end{align}
\end{theorem}

