\section{1-DOF mechanism feedforward model}

\subsection{Introduction}

The feedforward model we'll be using is
\begin{equation}
  u = K_s\sgn(v) + K_v v + K_a a \label{eq:ols_u_ff}
\end{equation}

where $u$ is input voltage, $v$ is velocity, $a$ is acceleration, and $K_s$,
$K_v$, and $K_a$ are constants reflecting how much each variable contributes to
the required voltage.

Since this is a linear multiple regression, the most obvious approach for
obtaining the feedforward gains $K_s$, $K_v$, and $K_a$ is to use ordinary least
squares (OLS) on 4-tuples of recorded voltage, velocity sign, velocity, and
acceleration. The problem is we can't cleanly measure acceleration directly, and
numerical differentiation of velocity is noisy. Ideally, we could find the
feedforward gains with samples of current velocity, current input, next
velocity, and the nominal sample period instead.

\subsection{Overview}

First, we'll put equation \eqref{eq:ols_u_ff} into the form
$\frac{dx}{dt} = Ax + Bu + c$. Then, we'll discretize it to obtain an equation
of the form $x_{k+1} = \alpha x_k + \beta u_k + \gamma\sgn(x_k)$. Then, we'll
perform OLS on $x_{k+1} = \alpha x_k + \beta u_k + \gamma\sgn(x_k)$ using
4-tuples of current velocity, next velocity, input voltage, and velocity sign to
obtain $\alpha$, $\beta$, and $\gamma$. Since we solved for those in the
discretization step earlier, we have a system of three equations we can solve
for $K_s$, $K_v$, and $K_a$.

\subsection{Derivation}

First, solve $u = K_s\sgn(v) + K_v v + K_a a$ for $a$.
\begin{align*}
  K_a a &= u - K_s\sgn(v) - K_v v \\
  a &= \frac{1}{K_a} u - \frac{K_s}{K_a}\sgn(v) - \frac{K_v}{K_a} v \\
  a &= -\frac{K_v}{K_a} v + \frac{1}{K_a} u - \frac{K_s}{K_a}\sgn(v)
  \intertext{Let $x = v$, $\frac{dx}{dt} = a$, $A = -\frac{K_v}{K_a}$,
    $B = \frac{1}{K_a}$, and $c = -\frac{K_s}{K_a}\sgn(x)$.}
  \frac{dx}{dt} &= Ax + Bu + c
\end{align*}

Since $Bu + c$ is a constant over the time interval $[0, T)$ where $T$ is the
sample period, this equation can be integrated according to appendix
\ref{sec:deriv_linear_system_zoh}, which gives
\begin{align*}
  x_{k+1} &= e^{AT} x_k + A^{-1} (e^{AT} - 1) (Bu_k + c) \\
  x_{k+1} &= e^{AT} x_k + A^{-1} (e^{AT} - 1)Bu_k + A^{-1} (e^{AT} - 1)c
\end{align*}

Substitute $c$ back in.
\begin{align*}
  x_{k+1} &= e^{AT} x_k + A^{-1} \left(e^{AT} - 1\right) B u_k
    + A^{-1} \left(e^{AT} - 1\right) \left(-\frac{K_s}{K_a} \sgn(x_k)\right) \\
  x_{k+1} &= e^{AT} x_k +  A^{-1} \left(e^{AT} - 1\right) B u_k
    - \frac{K_s}{K_a} A^{-1} \left(e^{AT} - 1\right) \sgn(x_k)
\end{align*}

This equation has the form
$x_{k+1} = \alpha x_k + \beta u_k + \gamma \sgn(x_k)$. Running OLS with
$(x_{k+1}, x_k, u_k, \sgn(x_k))$ 4-tuples will give $\alpha$, $\beta$, and
$\gamma$. To obtain $K_s$, $K_v$, and $K_a$, solve the following system.
\begin{equation*}
  \begin{cases}
    \alpha = e^{AT} \\
    \beta = A^{-1} \left(e^{AT} - 1\right)B \\
    \gamma = -\frac{K_s}{K_a} A^{-1} \left(e^{AT} - 1\right)
  \end{cases}
\end{equation*}

Substitute $\alpha$ into the second and third equation.
\begin{equation}
  \begin{cases}
    \alpha = e^{AT} \\
    \beta = A^{-1} (\alpha - 1) B \\
    \gamma = -\frac{K_s}{K_a} A^{-1} (\alpha - 1)
  \end{cases} \label{eq:ols_system}
\end{equation}

Divide the second equation by the third equation and solve for $K_s$.
\begin{align}
  \frac{\beta}{\gamma} &= \frac{A^{-1} (\alpha - 1) B}
    {-\frac{K_s}{K_a} A^{-1} (\alpha - 1)} \nonumber \\
  \frac{\beta}{\gamma} &= -\frac{K_a B}{K_s} \nonumber \\
  \frac{\beta}{\gamma} &= -\frac{K_a \left(\frac{1}{K_a}\right)}{K_s} \nonumber
    \\
  \frac{\beta}{\gamma} &= -\frac{1}{K_s} \nonumber \\
  K_s &= -\frac{\gamma}{\beta}
  \intertext{Solve the second equation in \eqref{eq:ols_system} for $K_v$.}
  \beta &= A^{-1} (\alpha - 1) B \nonumber \\
  \beta &= \left(-\frac{K_v}{K_a}\right)^{-1} (\alpha - 1)
    \left(\frac{1}{K_a}\right) \nonumber \\
  \beta &= -\frac{K_a}{K_v} (\alpha - 1) \frac{1}{K_a} \nonumber \\
  \beta &= \frac{1}{K_v} (1 - \alpha) \nonumber \\
  K_v \beta &= 1 - \alpha \nonumber \\
  K_v &= \frac{1 - \alpha}{\beta} \label{eq:ols_kv}
  \intertext{Solve the first equation in \eqref{eq:ols_system} for $K_a$.}
  \alpha &= e^{-\frac{K_v}{K_a} T} \nonumber \\
  \ln\alpha &= -\frac{K_v}{K_a} T \nonumber \\
  K_a \ln\alpha &= -K_v T \nonumber \\
  K_a &= -\frac{K_v T}{\ln\alpha} \nonumber
  \intertext{Substitute in $K_v$ from equation \eqref{eq:ols_kv}.}
  K_a &= -\frac{\left(\frac{1 - \alpha}{\beta}\right) T}{\ln\alpha} \nonumber \\
  K_a &= -\frac{(1 - \alpha) T}{\beta \ln\alpha} \nonumber \\
  K_a &= \frac{(\alpha - 1) T}{\beta \ln\alpha}
\end{align}

\subsection{Results}

The new system identification process is as follows.
\begin{enumerate}
  \item Gather input voltage and velocity from a dynamic acceleration event
  \item Given an equation of the form
    $x_{k+1} = \alpha x_k + \beta u_k + \gamma\sgn(x_k)$, perform OLS on
    $(x_{k+1}, x_k, u_k, \sgn(x_k))$ 4-tuples to obtain $\alpha$, $\beta$, and
    $\gamma$
\end{enumerate}

The feedforward gains are
\begin{align}
  K_s &= -\frac{\gamma}{\beta} \\
  K_v &= \frac{1 - \alpha}{\beta} \\
  K_a &= \frac{(\alpha - 1) T}{\beta \ln\alpha} \\
      &\alpha > 0, \beta \neq 0 \nonumber
\end{align}

where $T$ is the sample period of the velocity data.

For $K_s$, $K_v$, and $K_a$ to be defined, the dataset must include robot
movement, and there must be a discernible relationship between a change in the
input voltage and a change in the velocity (in other words, the robot must have
accelerated).
