\section{1-DOF mechanism model}
\begin{equation*}
  \dot{\mat{x}} = \mat{A}\mat{x} + \mat{B}\mat{u}
\end{equation*}
\begin{equation*}
  \mat{x} = \text{velocity}
  \quad
  \dot{\mat{x}} = \text{acceleration}
  \quad
  \mat{u} = \text{voltage}
\end{equation*}

We want to derive what $\mat{A}$ and $\mat{B}$ are from the following
feedforward model
\begin{equation*}
  \mat{u} = K_v \mat{x} + K_a \dot{\mat{x}}
\end{equation*}

Since this equation is a linear multiple regression, the constants $K_v$ and
$K_a$ can be determined by applying ordinary least squares (OLS) to vectors of
recorded input voltage, velocity, and acceleration data from quasistatic
velocity tests and acceleration tests.

$K_v$ is a proportional constant that describes how much voltage is required to
maintain a given constant velocity by offsetting the electromagnetic resistance
of the motor and any friction that increases linearly with speed (viscous drag).
The relationship between speed and voltage (for a given initial acceleration) is
linear for permanent-magnet DC motors in the FRC operating regime.

$K_a$ is a proportional constant that describes how much voltage is required to
induce a given acceleration in the motor shaft. As with $K_v$, the relationship
between voltage and acceleration (for a given initial velocity) is linear.

To convert $\mat{u} = K_v \mat{x} + K_a \dot{\mat{x}}$ to state-space form,
simply solve for $\dot{\mat{x}}$.
\begin{align*}
  \mat{u} &= K_v \mat{x} + K_a \dot{\mat{x}} \\
  K_a \dot{\mat{x}} &= \mat{u} - K_v \mat{x} \\
  K_a \dot{\mat{x}} &= -K_v \mat{x} + \mat{u} \\
  \dot{\mat{x}} &= -\frac{K_v}{K_a} \mat{x} + \frac{1}{K_a} \mat{u}
\end{align*}

By inspection, $\mat{A} = -\frac{K_v}{K_a}$ and $\mat{B} = \frac{1}{K_a}$. A
model with position and velocity states would be
\begin{theorem}[1-DOF mechanism position model]
  \begin{equation*}
    \dot{\mat{x}} = \mat{A}\mat{x} + \mat{B}\mat{u}
  \end{equation*}
  \begin{equation*}
    \mat{x} =
    \begin{bmatrix}
      \text{position} \\
      \text{velocity}
    \end{bmatrix}
    \quad
    \mat{u} =
    \begin{bmatrix}
      \text{voltage}
    \end{bmatrix}
  \end{equation*}
  \begin{equation}
    \dot{\mat{x}} =
    \begin{bmatrix}
      0 & 1 \\
      0 & -\frac{K_v}{K_a}
    \end{bmatrix}
    \mat{x} +
    \begin{bmatrix}
      0 \\
      \frac{1}{K_a}
    \end{bmatrix}
    \mat{u}
  \end{equation}
\end{theorem}
