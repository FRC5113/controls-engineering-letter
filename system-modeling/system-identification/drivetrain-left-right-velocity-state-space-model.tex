\section{Drivetrain left/right velocity state-space model}
\begin{equation*}
  \dot{\mat{x}} = \mat{A}\mat{x} + \mat{B}\mat{u}
\end{equation*}
\begin{equation*}
  \mat{x} = \begin{bmatrix}
    \text{left velocity} \\
    \text{right velocity}
  \end{bmatrix}
  \quad
  \dot{\mat{x}} = \begin{bmatrix}
    \text{left acceleration} \\
    \text{right acceleration}
  \end{bmatrix}
  \quad
  \mat{u} = \begin{bmatrix}
    \text{left voltage} \\
    \text{right voltage}
  \end{bmatrix}
\end{equation*}

We want to derive what $\mat{A}$ and $\mat{B}$ are from linear and angular
feedforward models. Since the left and right dynamics are symmetric, we'll guess
the model has the form
\begin{equation*}
  \mat{A} = \begin{bmatrix}
    A_1 & A_2 \\
    A_2 & A_1
  \end{bmatrix}
  \quad
  \mat{B} = \begin{bmatrix}
    B_1 & B_2 \\
    B_2 & B_1
  \end{bmatrix}
\end{equation*}

Let $K_{v,lin}$ be the linear velocity gain, $K_{a,lin}$ be the linear
acceleration gain, $K_{v,ang}$ be the angular velocity gain, and $K_{a,ang}$ be
the angular acceleration gain. Let $\mat{u} =
\begin{bmatrix}
  K_{v,lin} v & K_{v,lin} v
\end{bmatrix}\T$ be the input that makes both sides of the drivetrain move at a
constant velocity $v$. Therefore, $\mat{x} =
\begin{bmatrix}
  v & v
\end{bmatrix}\T$ and $\dot{\mat{x}} =
\begin{bmatrix}
  0 & 0
\end{bmatrix}\T$. Substitute these into the state-space model.
\begin{align*}
  \begin{bmatrix}
    0 \\
    0
  \end{bmatrix} &=
  \begin{bmatrix}
    A_1 & A_2 \\
    A_2 & A_1
  \end{bmatrix}
  \begin{bmatrix}
    v \\
    v
  \end{bmatrix} +
  \begin{bmatrix}
    B_1 & B_2 \\
    B_2 & B_1
  \end{bmatrix}
  \begin{bmatrix}
    K_{v,lin} v \\
    K_{v,lin} v
  \end{bmatrix}
  \intertext{Since the column vectors contain the same element, the elements in
    the second row can be rearranged.}
  \begin{bmatrix}
    0 \\
    0
  \end{bmatrix} &=
  \begin{bmatrix}
    A_1 & A_2 \\
    A_1 & A_2
  \end{bmatrix}
  \begin{bmatrix}
    v \\
    v
  \end{bmatrix} +
  \begin{bmatrix}
    B_1 & B_2 \\
    B_1 & B_2
  \end{bmatrix}
  \begin{bmatrix}
    K_{v,lin} v \\
    K_{v,lin} v
  \end{bmatrix}
  \intertext{Since the rows are linearly dependent, we can use just one of
    them.}
  0 &=
    \begin{bmatrix}
      A_1 & A_2
    \end{bmatrix} v +
    \begin{bmatrix}
      B_1 & B_2
    \end{bmatrix} K_{v,lin} v \\
  0 &=
    \begin{bmatrix}
      v & v & K_{v,lin} v & K_{v,lin} v
    \end{bmatrix}
    \begin{bmatrix}
      A_1 \\
      A_2 \\
      B_1 \\
      B_2
    \end{bmatrix} \\
  0 &=
    \begin{bmatrix}
      1 & 1 & K_{v,lin} & K_{v,lin}
    \end{bmatrix}
    \begin{bmatrix}
      A_1 \\
      A_2 \\
      B_1 \\
      B_2
    \end{bmatrix}
\end{align*}

Let $\mat{u} =
\begin{bmatrix}
  K_{v,lin} v + K_{a,lin} a & K_{v,lin} v + K_{a,lin} a
\end{bmatrix}\T$ be the input that accelerates both sides of the drivetrain by
$a$ from an initial velocity of $v$. Therefore, $\mat{x} =
\begin{bmatrix}
  v & v
\end{bmatrix}\T$ and $\dot{\mat{x}} =
\begin{bmatrix}
  a & a
\end{bmatrix}\T$. Substitute these into the state-space model.
\begin{align*}
  \begin{bmatrix}
    a \\
    a
  \end{bmatrix} &=
  \begin{bmatrix}
    A_1 & A_2 \\
    A_2 & A_1
  \end{bmatrix}
  \begin{bmatrix}
    v \\
    v
  \end{bmatrix} +
  \begin{bmatrix}
    B_1 & B_2 \\
    B_2 & B_1
  \end{bmatrix}
  \begin{bmatrix}
    K_{v,lin} v + K_{a,lin} a \\
    K_{v,lin} v + K_{a,lin} a
  \end{bmatrix}
  \intertext{Since the column vectors contain the same element, the elements in
    the second row can be rearranged.}
  \begin{bmatrix}
    a \\
    a
  \end{bmatrix} &=
  \begin{bmatrix}
    A_1 & A_2 \\
    A_1 & A_2
  \end{bmatrix}
  \begin{bmatrix}
    v \\
    v
  \end{bmatrix} +
  \begin{bmatrix}
    B_1 & B_2 \\
    B_1 & B_2
  \end{bmatrix}
  \begin{bmatrix}
    K_{v,lin} v + K_{a,lin} a \\
    K_{v,lin} v + K_{a,lin} a
  \end{bmatrix}
  \intertext{Since the rows are linearly dependent, we can use just one of
    them.}
  a &=
    \begin{bmatrix}
      A_1 & A_2
    \end{bmatrix} v +
    \begin{bmatrix}
      B_1 & B_2
    \end{bmatrix}
    \begin{bmatrix}
      K_{v,lin} v + K_{a,lin} a
    \end{bmatrix} \\
  a &=
    \begin{bmatrix}
      v & v & K_{v,lin} v + K_{a,lin} a & K_{v,lin} + K_{a,lin} a
    \end{bmatrix}
    \begin{bmatrix}
      A_1 \\
      A_2 \\
      B_1 \\
      B_2
    \end{bmatrix}
\end{align*}

Let $\mat{u} =
\begin{bmatrix}
  -K_{v,ang} v & K_{v,ang} v
\end{bmatrix}\T$ be the input that rotates the drivetrain in place where each
wheel has a constant velocity $v$. Therefore, $\mat{x} =
\begin{bmatrix}
  -v & v
\end{bmatrix}\T$ and $\dot{\mat{x}} =
\begin{bmatrix}
  0 & 0
\end{bmatrix}\T$.
\begin{align*}
  \begin{bmatrix}
    0 \\
    0
  \end{bmatrix} &=
    \begin{bmatrix}
      A_1 & A_2 \\
      A_2 & A_1
    \end{bmatrix}
    \begin{bmatrix}
      -v \\
      v
    \end{bmatrix} +
    \begin{bmatrix}
      B_1 & B_2 \\
      B_2 & B_1
    \end{bmatrix}
    \begin{bmatrix}
      -K_{v,ang} v \\
      K_{v,ang} v
    \end{bmatrix} \\
  \begin{bmatrix}
    0 \\
    0
  \end{bmatrix} &=
    \begin{bmatrix}
      -A_1 & A_2 \\
      -A_2 & A_1
    \end{bmatrix}
    \begin{bmatrix}
      v \\
      v
    \end{bmatrix} +
    \begin{bmatrix}
      -B_1 & B_2 \\
      -B_2 & B_1
    \end{bmatrix}
    \begin{bmatrix}
      K_{v,ang} v \\
      K_{v,ang} v
    \end{bmatrix} \\
  \begin{bmatrix}
    0 \\
    0
  \end{bmatrix} &=
    \begin{bmatrix}
      -A_1 & A_2 \\
      A_1 & -A_2
    \end{bmatrix}
    \begin{bmatrix}
      v \\
      v
    \end{bmatrix} +
    \begin{bmatrix}
      -B_1 & B_2 \\
      B_1 & -B_2
    \end{bmatrix}
    \begin{bmatrix}
      K_{v,ang} v \\
      K_{v,ang} v
    \end{bmatrix}
  \intertext{Since the column vectors contain the same element, the elements in
    the second row can be rearranged.}
  \begin{bmatrix}
    0 \\
    0
  \end{bmatrix} &=
  \begin{bmatrix}
    -A_1 & A_2 \\
    -A_1 & A_2
  \end{bmatrix}
  \begin{bmatrix}
    v \\
    v
  \end{bmatrix} +
  \begin{bmatrix}
    -B_1 & B_2 \\
    -B_1 & B_2
  \end{bmatrix}
  \begin{bmatrix}
    K_{v,ang} v \\
    K_{v,ang} v
  \end{bmatrix}
  \intertext{Since the rows are linearly dependent, we can use just one of
    them.}
  0 &=
    \begin{bmatrix}
      -A_1 & A_2
    \end{bmatrix} v +
    \begin{bmatrix}
      -B_1 & B_2
    \end{bmatrix} K_{v,ang} v \\
  0 &= -v A_1 + v A_2 - K_{v,ang} v B_1 + K_{v,ang} v B_2 \\
  0 &=
    \begin{bmatrix}
      -v & v & -K_{v,ang} v & K_{v,ang} v
    \end{bmatrix}
    \begin{bmatrix}
      A_1 \\
      A_2 \\
      B_1 \\
      B_2
    \end{bmatrix} \\
  0 &=
    \begin{bmatrix}
      -1 & 1 & -K_{v,ang} & K_{v,ang}
    \end{bmatrix}
    \begin{bmatrix}
      A_1 \\
      A_2 \\
      B_1 \\
      B_2
    \end{bmatrix}
\end{align*}

Let $\mat{u} =
\begin{bmatrix}
  -K_{v,ang} v - K_{a,ang} a & K_{v,ang} v + K_{a,ang} a
\end{bmatrix}\T$ be the input that rotates the drivetrain in place where each
wheel has an initial speed of $v$ and accelerates by $a$. Therefore, $\mat{x} =
\begin{bmatrix}
  -v & v
\end{bmatrix}\T$ and $\dot{\mat{x}} =
\begin{bmatrix}
  -a & a
\end{bmatrix}\T$.
\begin{align*}
  \begin{bmatrix}
    -a \\
    a
  \end{bmatrix} &=
    \begin{bmatrix}
      A_1 & A_2 \\
      A_2 & A_1
    \end{bmatrix}
    \begin{bmatrix}
      -v \\
      v
    \end{bmatrix} +
    \begin{bmatrix}
      B_1 & B_2 \\
      B_2 & B_1
    \end{bmatrix}
    \begin{bmatrix}
      -K_{v,ang} v - K_{a,ang} a \\
      K_{v,ang} v + K_{a,ang} a
    \end{bmatrix} \\
  \begin{bmatrix}
    -a \\
    a
  \end{bmatrix} &=
    \begin{bmatrix}
      -A_1 & A_2 \\
      -A_2 & A_1
    \end{bmatrix}
    \begin{bmatrix}
      v \\
      v
    \end{bmatrix} +
    \begin{bmatrix}
      -B_1 & B_2 \\
      -B_2 & B_1
    \end{bmatrix}
    \begin{bmatrix}
      K_{v,ang} v + K_{a,ang} a \\
      K_{v,ang} v + K_{a,ang} a
    \end{bmatrix} \\
  \begin{bmatrix}
    -a \\
    a
  \end{bmatrix} &=
    \begin{bmatrix}
      -A_1 & A_2 \\
      A1 & -A_2
    \end{bmatrix}
    \begin{bmatrix}
      v \\
      v
    \end{bmatrix} +
    \begin{bmatrix}
      -B_1 & B_2 \\
      B_1 & -B_2
    \end{bmatrix}
    \begin{bmatrix}
      K_{v,ang} v + K_{a,ang} a \\
      K_{v,ang} v + K_{a,ang} a
    \end{bmatrix}
  \intertext{Since the column vectors contain the same element, the elements in
    the second row can be rearranged.}
  \begin{bmatrix}
    -a \\
    -a
  \end{bmatrix} &=
  \begin{bmatrix}
    -A_1 & A_2 \\
    -A_1 & A_2
  \end{bmatrix}
  \begin{bmatrix}
    v \\
    v
  \end{bmatrix} +
  \begin{bmatrix}
    -B_1 & B_2 \\
    -B_1 & B_2
  \end{bmatrix}
  \begin{bmatrix}
    K_{v,ang} v + K_{a,ang} a \\
    K_{v,ang} v + K_{a,ang} a
  \end{bmatrix}
\end{align*}

Since the rows are linearly dependent, we can use just one of them.
\begin{align*}
  -a &=
    \begin{bmatrix}
      -A_1 & A_2
    \end{bmatrix} v +
    \begin{bmatrix}
      -B_1 & B_2
    \end{bmatrix}
    \begin{bmatrix}
      K_{v,ang} v + K_{a,ang} a
    \end{bmatrix} \\
  -a &= -v A_1 + v A_2 - (K_{v,ang} v + K_{a,ang} a) B_1 + (K_{v,ang} v + K_{a,ang} a) B_2 \\
  -a &=
    \begin{bmatrix}
      -v & v & -(K_{v,ang} v + K_{a,ang} a) & K_{v,ang} v+ K_{a,ang} a
    \end{bmatrix}
    \begin{bmatrix}
      A_1 \\
      A_2 \\
      B_1 \\
      B_2
    \end{bmatrix} \\
  a &=
    \begin{bmatrix}
      v & -v & K_{v,ang} v + K_{a,ang} a & -(K_{v,ang} v + K_{a,ang} a)
    \end{bmatrix}
    \begin{bmatrix}
      A_1 \\
      A_2 \\
      B_1 \\
      B_2
    \end{bmatrix}
\end{align*}

Now stack the rows.
\begin{equation*}
  \begin{bmatrix}
    0 \\
    a \\
    0 \\
    a
  \end{bmatrix} =
  \begin{bmatrix}
    1 & 1 & K_{v,lin} & K_{v,lin} \\
    v & v & K_{v,lin} v + K_{a,lin} a & K_{v,lin} v + K_{a,lin} a \\
    -1 & 1 & -K_{v,ang} & K_{v,ang} \\
    v & -v & K_{v,ang} v + K_{a,ang} a & -(K_{v,ang} v + K_{a,ang} a)
  \end{bmatrix}
  \begin{bmatrix}
    A_1 \\
    A_2 \\
    B_1 \\
    B_2
  \end{bmatrix}
\end{equation*}

Solve for matrix elements with Wolfram Alpha. Let
$b = K_{v,lin}$, $c = K_{a,lin}$, $d = K_{v,ang}$, and
$f = K_{a,ang}$.
\begin{verbatim}
inverse of {{1, 1, b, b},
            {v, v, b v + c a, b v + c a},
            {-1, 1, -d, d},
            {v, -v, d v + f a, -(d v + f a)}}
           * {{0}, {a}, {0}, {a}}
\end{verbatim}
\begin{align*}
  \begin{bmatrix}
    A_1 \\
    A_2 \\
    B_1 \\
    B_2
  \end{bmatrix} &= \frac{1}{2cf}
  \begin{bmatrix}
    -cd - bf \\
    cd - bf \\
    c + f \\
    f - c
  \end{bmatrix} \\
  \begin{bmatrix}
    A_1 \\
    A_2 \\
    B_1 \\
    B_2
  \end{bmatrix} &= \frac{1}{2 K_{a,lin} K_{a,ang}}
  \begin{bmatrix}
    -K_{a,lin} K_{v,ang} - K_{v,lin} K_{a,ang} \\
    K_{a,lin} K_{v,ang} - K_{v,lin} K_{a,ang} \\
    K_{a,lin} + K_{a,ang} \\
    K_{a,ang} - K_{a,lin}
  \end{bmatrix} \\
  \begin{bmatrix}
    A_1 \\
    A_2 \\
    B_1 \\
    B_2
  \end{bmatrix} &= \frac{1}{2}
  \begin{bmatrix}
    -\frac{K_{a,lin} K_{v,ang}}{K_{a,lin} K_{a,ang}} -
      \frac{K_{v,lin} K_{a,ang}}{K_{a,lin} K_{a,ang}} \\
    \frac{K_{a,lin} K_{v,ang}}{K_{a,lin} K_{a,ang}} -
      \frac{K_{v,lin} K_{a,ang}}{K_{a,lin} K_{a,ang}} \\
    \frac{K_{a,lin}}{K_{a,lin} K_{a,ang}} +
      \frac{K_{a,ang}}{K_{a,lin} K_{a,ang}} \\
    \frac{K_{a,ang}}{K_{a,lin} K_{a,ang}} -
      \frac{K_{a,lin}}{K_{a,lin} K_{a,ang}}
  \end{bmatrix} \\
  \begin{bmatrix}
    A_1 \\
    A_2 \\
    B_1 \\
    B_2
  \end{bmatrix} &= \frac{1}{2}
  \begin{bmatrix}
    -\frac{K_{v,ang}}{K_{a,ang}} -
      \frac{K_{v,lin}}{K_{a,lin}} \\
    \frac{K_{v,ang}}{K_{a,ang}} -
      \frac{K_{v,lin}}{K_{a,lin}} \\
    \frac{1}{K_{a,ang}} + \frac{1}{K_{a,lin}} \\
    \frac{1}{K_{a,lin}} - \frac{1}{K_{a,ang}}
  \end{bmatrix} \\
  \begin{bmatrix}
    A_1 \\
    A_2 \\
    B_1 \\
    B_2
  \end{bmatrix} &= \frac{1}{2}
  \begin{bmatrix}
    -\left(\frac{K_{v,lin}}{K_{a,lin}} +
      \frac{K_{v,ang}}{K_{a,ang}}\right) \\
    -\left(\frac{K_{v,lin}}{K_{a,lin}} -
      \frac{K_{v,ang}}{K_{a,ang}}\right) \\
    \frac{1}{K_{a,lin}} + \frac{1}{K_{a,ang}} \\
    \frac{1}{K_{a,lin}} - \frac{1}{K_{a,ang}}
  \end{bmatrix}
\end{align*}

To summarize,
\begin{theorem}[Drivetrain left/right velocity model]
  \begin{equation*}
    \dot{\mat{x}} = \mat{A}\mat{x} + \mat{B}\mat{u}
  \end{equation*}
  \begin{equation*}
    \mat{x} =
    \begin{bmatrix}
      \text{left velocity} \\
      \text{right velocity}
    \end{bmatrix}
    \quad
    \mat{u} =
    \begin{bmatrix}
      \text{left voltage} \\
      \text{right voltage}
    \end{bmatrix}
  \end{equation*}
  \begin{equation*}
    \dot{\mat{x}} =
    \begin{bmatrix}
      A_1 & A_2 \\
      A_2 & A_1
    \end{bmatrix} \mat{x} +
    \begin{bmatrix}
      B_1 & B_2 \\
      B_2 & B_1
    \end{bmatrix} \mat{u}
  \end{equation*}
  \begin{equation}
    \begin{bmatrix}
      A_1 \\
      A_2 \\
      B_1 \\
      B_2
    \end{bmatrix} = \frac{1}{2}
    \begin{bmatrix}
      -\left(\frac{K_{v,lin}}{K_{a,lin}} +
        \frac{K_{v,ang}}{K_{a,ang}}\right) \\
      -\left(\frac{K_{v,lin}}{K_{a,lin}} -
        \frac{K_{v,ang}}{K_{a,ang}}\right) \\
      \frac{1}{K_{a,lin}} + \frac{1}{K_{a,ang}} \\
      \frac{1}{K_{a,lin}} - \frac{1}{K_{a,ang}}
    \end{bmatrix}
  \end{equation}

  $K_{v,ang}$ and $K_{a,ang}$ have units of $\frac{V}{L/T}$ and
  $\frac{V}{L/T^2}$ respectively where $V$ means voltage units, $L$ means length
  units, and $T$ means time units.
\end{theorem}

If $K_v$ and $K_a$ are the same for both the linear and angular cases, it devolves to the one-dimensional case. This means the left and right sides are decoupled.
\begin{align*}
  \begin{bmatrix}
    A_1 \\
    A_2 \\
    B_1 \\
    B_2
  \end{bmatrix} &= \frac{1}{2}
  \begin{bmatrix}
    -\left(\frac{K_v}{K_a} + \frac{K_v}{K_a}\right) \\
    -\left(\frac{K_v}{K_a} - \frac{K_v}{K_a}\right) \\
    \frac{1}{K_a} + \frac{1}{K_a} \\
    \frac{1}{K_a} - \frac{1}{K_a}
  \end{bmatrix} \\
  \begin{bmatrix}
    A_1 \\
    A_2 \\
    B_1 \\
    B_2
  \end{bmatrix} &= \frac{1}{2}
  \begin{bmatrix}
    -2\frac{K_v}{K_a} \\
    0 \\
    \frac{2}{K_a} \\
    0
  \end{bmatrix} \\
  \begin{bmatrix}
    A_1 \\
    A_2 \\
    B_1 \\
    B_2
  \end{bmatrix} &=
  \begin{bmatrix}
    -\frac{K_v}{K_a} \\
    0 \\
    \frac{1}{K_a} \\
    0
  \end{bmatrix}
\end{align*}
