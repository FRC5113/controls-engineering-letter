\section{Curvilinear motion}

\subsection{Differential drive kinematics}

A differential drive has two wheels, one on each side, separated by some
distance $2r_b$.

For this derivation, we'll assume positive $\hat{i}$ is forward, positive
$\hat{j}$ is to the right, and positive $\hat{k}$ is up and that the robot is
facing in the $\hat{i}$ direction.

The mapping from $v$ and $\omega$ to the left and right wheel velocities $v_l$
and $v_r$ are derived as follows. Let $\vec{v}_c$ be the velocity vector of the
center of rotation, $\vec{v}_l$ be the velocity vector of the left wheel,
$\vec{v}_r$ be the velocity vector of the right wheel, $r_b$ is the distance
from the center of rotation to each wheel, and $\omega$ is the counterclockwise
turning rate around the center of rotation.

The main equation we'll need is the following.

\begin{equation*}
  \vec{v}_B = \vec{v}_A + \omega_A \times \vec{r}_{B|A}
\end{equation*}

where $\vec{v}_B$ is the velocity vector at point B, $\vec{v}_A$ is the velocity
vector at point A, $\omega_A$ is the angular velocity vector at point A, and
$\vec{r}_{B|A}$ is the distance vector from point A to point B (also described
as the ``distance to B relative to A").

Once we have the vector equation representing the wheel's velocity, we'll
project it onto the wheel direction vector using the dot product.

First, we'll derive $v_l$.

\begin{align*}
  \vec{v}_l &= v_c \hat{i} + \omega \hat{k} \times r_b \hat{j} \\
  \vec{v}_l &= v_c \hat{i} - \omega r_b \hat{i} \\
  \vec{v}_l &= (v_c - \omega r_b) \hat{i}
\end{align*}

Now, project this vector onto the left wheel, which is pointed in the $\hat{i}$
direction.

\begin{equation*}
  v_l = (v_c - \omega r_b) \hat{i} \cdot \frac{\hat{i}}{\norm{\hat{i}}}
\end{equation*}

The magnitude of $\hat{i}$ is $1$, so the denominator cancels.

\begin{align}
  v_l &= (v_c - \omega r_b) \hat{i} \cdot \hat{i} \nonumber \\
  v_l &= v_c - \omega r_b \label{eq:diff_vl}
\end{align}

Next, we'll derive $v_r$.

\begin{align*}
  \vec{v}_r &= v_c \hat{i} + \omega \hat{k} \times r_b \hat{j} \\
  \vec{v}_r &= v_c \hat{i} + \omega r_b \hat{i} \\
  \vec{v}_r &= (v_c + \omega r_b) \hat{i}
\end{align*}

Now, project this vector onto the right wheel, which is pointed in the $\hat{i}$
direction.

\begin{equation*}
  v_r = (v_c + \omega r_b) \hat{i} \cdot \frac{\hat{i}}{\norm{\hat{i}}}
\end{equation*}

The magnitude of $\hat{i}$ is $1$, so the denominator cancels.

\begin{align}
  v_r &= (v_c + \omega r_b) \hat{i} \cdot \hat{i} \nonumber \\
  v_r &= v_c + \omega r_b \label{eq:diff_vr}
\end{align}
