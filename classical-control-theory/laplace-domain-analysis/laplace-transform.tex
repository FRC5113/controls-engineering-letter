\section{Laplace transform}

The Laplace domain is a generalization of the frequency domain that has the
frequency ($j\omega$) on the imaginary y-axis and a real number on the x-axis,
yielding a two-dimensional coordinate system. We represent coordinates in this
space as a complex number $s = \sigma + j\omega$. The real part $\sigma$
corresponds to the x-axis and the imaginary part $j\omega$ corresponds to the
y-axis (see figure \ref{fig:laplace_domain}).
\begin{bookfigure}
  \begin{tikzpicture}[auto, >=latex']
    %\draw [help lines] (-4,-2) grid (4,4);

    % Draw main axes
    \draw[<->] (-4,0) -- (4,0) node[below] {\small Re($\sigma$)};
    \draw[<->] (0,-3) -- (0,3) node[right] {\small Im($j\omega$)};
  \end{tikzpicture}

  \caption{Laplace domain}
  \label{fig:laplace_domain}
\end{bookfigure}

To extend our analogy of each coordinate being represented by some basis, we now
have the y coordinate representing the oscillation frequency of the
\gls{system response} (the frequency domain) and also the x coordinate
representing the speed at which that oscillation decays and the \gls{system}
converges to zero (i.e., a decaying exponential). Figure
\ref{fig:impulse_response_poles} shows this for various points.

If we move the component frequencies in the Fmajor4 chord example parallel to
the real axis to $\sigma = -25$, the resulting time domain response attenuates
according to the decaying exponential $e^{-25t}$ (see figure
\ref{fig:laplace_chord_attenuating}).
\begin{svg}{build/\chapterpath/laplace_chord_attenuating}
  \caption{Fmajor4 chord at $\sigma = 0$ and $\sigma = -25$}
  \label{fig:laplace_chord_attenuating}
\end{svg}

Note that this explanation as a basis isn't exact because the Laplace basis
isn't orthogonal (that is, the x and y coordinates affect each other and have
cross-talk). In the frequency domain, we had a basis of sine waves that we
represented as delta functions in the frequency domain. Each frequency
contribution was independent of the others. In the Laplace domain, this is not
the case; a pure exponential is $\frac{1}{s - a}$ (a rational function where $a$
is a real number) instead of a delta function. This function is nonzero at
points that aren't actually frequencies present in the time domain. Figure
\ref{fig:laplace_chord_3d} demonstrates this, which shows the Laplace transform
of the Fmajor4 chord plotted in 3D.
\begin{svg}{build/\chapterpath/laplace_chord_3d}
  \caption{Laplace transform of Fmajor4 chord plotted in 3D}
  \label{fig:laplace_chord_3d}
\end{svg}

Notice how the values of the function around each component frequency decrease
according to $\frac{1}{\sqrt{x^2 + y^2}}$ in the $x$ and $y$ directions (in just
the $x$ direction, it would be $\frac{1}{x}$).

