\section{Definition}
\index{PID control}

Negative feedback loops drive the difference between the \gls{reference} and
\gls{output} (usually defined as the \gls{error}) to zero. PID controllers are
commonly used for this.

\begin{definition}[PID controller]
  \begin{equation}
    u(t) = K_p e(t) + K_i \int_0^t e(\tau) \,d\tau + K_d \frac{de}{dt}
    \label{eq:pos_pid}
  \end{equation}

  where $e(t)$ is the error at the current time $t$, $\tau$ is the integration
  variable, $K_p$ is the proportional gain, $K_i$ is the integral gain, and
  $K_d$ is the derivative gain.
\end{definition}

The integral integrates from time $0$ to the current time $t$. We use $\tau$ for
the integration because we need a variable to take on multiple values throughout
the integral, but we can't use $t$ because we already defined that as the
current time.

Figure \ref{fig:pid_ctrl_diag} shows a block diagram for a \gls{system}
controlled by a PID controller.

\begin{bookfigure}
  \begin{tikzpicture}[auto, >=latex']
    \fontsize{9pt}{10pt}

    % Place the blocks
    \node [name=input] {$r(t)$};
    \node [sum, right=0.5cm of input] (errorsum) {};
    \node [coordinate, right=0.75cm of errorsum] (branch) {};
    \node [block, right=0.5cm of branch] (I) { $K_i \int_0^t e(\tau) \,d\tau$ };
    \node [block, above=0.5cm of I] (P) { $K_p e(t)$ };
    \node [block, below=0.5cm of I] (D) { $K_d \frac{de(t)}{dt}$ };
    \node [sum, right=0.5cm of I] (ctrlsum) {};
    \node [block, right=0.75cm of ctrlsum] (plant) {Plant};
    \node [right=0.75cm of plant] (output) {};
    \node [coordinate, below=0.5cm of D] (measurements) {};

    % Connect the nodes
    \draw [arrow] (input) -- node[pos=0.9] {$+$} (errorsum);
    \draw [-] (errorsum) -- node {$e(t)$} (branch);
    \draw [arrow] (branch) |- (P);
    \draw [arrow] (branch) -- (I);
    \draw [arrow] (branch) |- (D);
    \draw [arrow] (P) -| node[pos=0.95, left] {$+$} (ctrlsum);
    \draw [arrow] (I) -- node[pos=0.9, below] {$+$} (ctrlsum);
    \draw [arrow] (D) -| node[pos=0.95, right] {$+$} (ctrlsum);
    \draw [arrow] (ctrlsum) -- node {$u(t)$} (plant);
    \draw [arrow] (plant) -- node [name=y] {$y(t)$} (output);
    \draw [-] (y) |- (measurements);
    \draw [arrow] (measurements) -| node[pos=0.99, right] {$-$} (errorsum);
  \end{tikzpicture}

  \caption{PID controller diagram}
  \label{fig:pid_ctrl_diag}

  \begin{figurekey}
    \begin{tabular}{llll}
      $r(t)$ & \gls{reference} & $u(t)$ & \gls{control input} \\
      $e(t)$ & \gls{error} & $y(t)$ & \gls{output} \\
    \end{tabular}
  \end{figurekey}
\end{bookfigure}

\subsection{Proportional gain $K_p$}

The \textit{Proportional} gain compensates for the current \gls{error}. This
gain acts like a ``software-defined spring". Recall from physics that we model
springs as $F = -kx$ where $F$ is the force applied, $k$ is a proportional
constant, and $x$ is the displacement from the equilibrium point ($0$ in this
case). For P control of a DC motor toward zero, $F$ is proportional to the
voltage applied, $k$ is the proportional constant from the P term, and $x$ is
the measured position. In other words, the ``force" with which the proportional
gain pulls the \gls{system}'s \gls{output} toward the \gls{reference} is
proportional to the \gls{error}.

\subsection{Integral gain $K_i$}

The \textit{Integral} gain compensates for past \gls{error} (i.e.,
\gls{steady-state error}). It integrates the \gls{error} over time and adds the
current total times $K_i$ to the \gls{control input}. When the \gls{system} is
close to the \gls{reference} in steady-state, the proportional term is too small
to pull the \gls{output} to the \gls{reference} and the derivative term is zero.
The integral gain is commonly used to address this problem.

There are better approaches to fix \gls{steady-state error} like using
feedforwards or constraining when the integral control acts using other
knowledge of the \gls{system}. We will discuss these in more detail when we get
to modern control theory.

\subsection{Derivative gain $K_d$}

The \textit{Derivative} gain compensates for future \gls{error} by slowing the
controller down if the \gls{error} is changing over time. This gain acts like a
``software-defined damper". These are commonly seen on door closers, and their
damping force increases linearly with velocity.
