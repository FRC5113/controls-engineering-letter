\section{Classical vs modern control}

State-space notation provides a more convenient and compact way to model and
analyze \glspl{system} with multiple \glspl{input} and \glspl{output}. For a
\gls{system} with $p$ \glspl{input} and $q$ \glspl{output}, we would have to
write $q \times p$ transfer functions to represent it. Not only is the resulting
algebra unwieldy, but it only works for linear \glspl{system}. Including nonzero
initial conditions complicates the algebra even more. State-space representation
uses the time domain instead of the Laplace domain, so it can model nonlinear
\glspl{system}\footnote{This book focuses on analysis and control of linear
\glspl{system}. See chapter \ref{ch:nonlinear_control} for more on nonlinear
control.} and trivially supports nonzero initial conditions.

If modern control theory is so great and classical control theory isn't needed
to use it, why learn classical control theory at all? We teach classical control
theory because it provides a framework within which to understand results from
the mathematical machinery of modern control as well as vocabulary with which to
communicate that understanding. For example, faster poles (poles moved to the
left in the s-plane) mean faster decay, and oscillation means there is at least
one pair of complex conjugate poles. Not only can you describe what happened
succinctly, but you know why it happened from a theoretical perspective.
