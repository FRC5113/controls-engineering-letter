\section{Nonminimum phase zeroes}

While poles in the RHP are unstable, the same is not true for zeroes. They can
be characterized by the \gls{system} initially moving in the wrong direction
before heading toward the \gls{reference}. Since the poles always move toward
the zeroes, zeroes impose a ``speed limit" on the \gls{system response} because
it takes a finite amount of time to move the wrong direction, then change
directions.

One example of this type of \gls{system} is bicycle steering. Try riding a
bicycle without holding the handle bars, then poke the right handle; the bicycle
turns right. Furthermore, if one is holding the handlebars and wants to turn
left, rotating the handlebars counterclockwise will make the bicycle fall toward
the right. The rider has to lean into the turn and overpower the nonminimum
phase dymamics to go the desired direction.

Another example is a segway. To move forward by some distance, the segway must
first roll backward to rotate the segway forward. Once the segway starts falling
in that direction, it begins rolling forward to avoid falling over until
it reaches the target distance. At that point, the segway increases its forward
speed to pitch backward and slow itself down. To come to a stop, the segway
rolls backward again to level itself out.
