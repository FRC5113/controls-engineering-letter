\section{Parts of a transfer function}

A transfer function maps an input coordinate to an output coordinate in the
Laplace domain. These can be obtained by applying the Laplace transform to a
differential equation and rearranging the terms to obtain a ratio of the output
variable to the input variable. Equation (\ref{eq:transfer_func}) is an example
of a transfer function.

\begin{equation} \label{eq:transfer_func}
  H(s) = \frac{\overbrace{(s-9+9i)(s-9-9i)}^{zeroes}}
    {\underbrace{s(s+10)}_{poles}}
\end{equation}

\subsection{Poles and zeroes}
\label{subsec:poles_and_zeroes}

The roots of factors in the numerator of a transfer function are called
\textit{zeroes} because they make the transfer function approach zero. Likewise,
the roots of factors in the denominator of a transfer function are called
\textit{poles} because they make the transfer function approach infinity; on a
3D graph, these look like the poles of a circus tent (see figure
\ref{fig:tf_3d}).

When the factors of the denominator are broken apart using partial fraction
expansion into something like $\frac{A}{s + a} + \frac{B}{s + b}$, the constants
$A$ and $B$ are called residues, which determine how much each pole contributes
to the \gls{system response}.

You may notice that the factors representing poles look a lot like the Laplace
transform for a decaying exponential. This is why the time domain responses of
\glspl{system} are typically decaying exponentials. One differential equation
which can produce this kind of response is $\dot{x} = ax$ where $\dot{x}$ is the
derivative of $x$ and $a < 0$.

\begin{svg}{build/code/tf_3d}
  \caption{Equation \ref{eq:transfer_func} plotted in 3D}
  \label{fig:tf_3d}
\end{svg}

\begin{remark}
  Imaginary poles and zeroes always come in complex conjugate pairs (e.g.,
  $-2 + 3i$, $-2 - 3i$).
\end{remark}

\index{Stability!poles and zeroes}
The locations of the closed-loop poles in the complex plane determine the
stability of the \gls{system}. Each pole represents a frequency mode of the
\gls{system}, and their location determines how much of each response is induced
for a given input frequency. Figure \ref{fig:impulse_response_poles} shows the
\glspl{impulse response} in the time domain for transfer functions with various
pole locations. They all have an initial condition of $1$.

\begin{bookfigure}
  \begin{tikzpicture}[auto, >=latex']
    % \draw [help lines] (-4,-2) grid (4,4);

    % Draw main axes
    \draw[->] (-4.2,0) -- (4.2,0) node[below] {\small Re($\sigma$)};
    \draw[->] (0,-2) -- (0,4.2) node[right] {\small Im($j\omega$)};

    % Stable: e^-1.75t * cos(1.75wt) (80/3*w for readability)
    \drawtimeplot{-2.125cm}{2.5cm}{0.125cm}{0.44375cm}{
      exp(-1.75 * \x) * cos(80/3 * 1.75 * deg(\x))}
    \drawpole{-1.75cm}{1.75cm}

    % Stable: e^-2.5t
    \drawtimeplot{-2.25cm}{0.75cm}{0.125cm}{0.125cm}{exp(-2 * \x)}
    \drawpole{-2cm}{0cm}

    % Stable: e^-t
    \drawtimeplot{-1.125cm}{-0.75cm}{0.125cm}{0.125cm}{exp(-\x)}
    \drawpole{-1cm}{0cm}

    % Marginally stable: cos(wt) (80/3*w for readability)
    \drawtimeplot{-0.75cm}{1.125cm}{0.125cm}{0.44375cm}{cos(80/3 * deg(\x))}
    \drawpole{0cm}{1cm}

    % Marginally stable cos(2wt) (80/3*w for readability)
    \drawtimeplot{0cm}{2.75cm}{0.125cm}{0.44375cm}{cos(80/3 * 2 * deg(\x))}
    \drawpole{0cm}{2cm}

    % Integrator
    \drawtimeplot{0.25cm}{-0.75cm}{0.125cm}{0.125cm}{1}
    \drawpole{0cm}{0cm}

    % Unstable: e^t
    \drawtimeplot{1.125cm}{0.75cm}{0.125cm}{0.125cm}{exp(\x)}
    \drawpole{1cm}{0cm}

    % Unstable: e^2t
    \drawtimeplot{2.25cm}{-0.75cm}{0.125cm}{0.125cm}{exp(2 * \x)}
    \drawpole{2cm}{0cm}

    % Unstable: e^0.75t * cos(1.75wt) (80/3*w for readability)
    \drawtimeplot{1.5cm}{2.25cm}{0.125cm}{0.44375cm}{
      exp(0.75 * \x) * cos(80/3 * 1.75 * deg(\x))}
    \drawpole{0.75cm}{1.75cm}

    % LHP and RHP labels
    \draw (-3.5,1.5) node {LHP};
    \draw (3.5,1.5) node {RHP};

    % Stable and unstable labels
    \draw (-2.5,3.5) node {\small Stable};
    \draw (2.5,3.5) node {\small Unstable};
  \end{tikzpicture}

  \caption{Impulse response vs pole location}
  \label{fig:impulse_response_poles}
\end{bookfigure}

\begin{booktable}
  \begin{tabular}{|ll|}
    \hline
    \rowcolor{headingbg}
    \textbf{Location} & \textbf{Stability} \\
    \hline
    Left Half-plane (LHP) & Stable \\
    Imaginary axis & Marginally stable \\
    Right Half-plane (RHP) & Unstable \\
    \hline
  \end{tabular}

  \caption{Pole location and stability}
  \label{tab:pole_locations}
\end{booktable}

When a \gls{system} is stable, its output may oscillate but it converges to
steady-state. When a \gls{system} is marginally stable, its output oscillates at
a constant amplitude forever. When a \gls{system} is unstable, its output grows
without bound.

\subsection{Nonminimum phase zeroes}

While poles in the RHP are unstable, the same is not true for zeroes. They can
be characterized by the \gls{system} initially moving in the wrong direction
before heading toward the \gls{reference}. Since the poles always move toward
the zeroes, zeroes impose a ``speed limit" on the \gls{system response} because
it takes a finite amount of time to move the wrong direction, then change
directions.

One example of this type of \gls{system} is bicycle steering. Try riding a
bicycle without holding the handle bars, then poke the right handle; the bicycle
turns right. Furthermore, if one is holding the handlebars and wants to turn
left, rotating the handlebars counterclockwise will make the bicycle fall toward
the right. The rider has to lean into the turn and overpower the nonminimum
phase dynamics to go the desired direction.

Another example is a Segway. To move forward by some distance, the Segway must
first roll backward to rotate the Segway forward. Once the Segway starts falling
in that direction, it begins rolling forward to avoid falling over until
it reaches the target distance. At that point, the Segway increases its forward
speed to pitch backward and slow itself down. To come to a stop, the Segway
rolls backward again to level itself out.

\subsection{Pole-zero cancellation}
\label{subsec:pole-zero_cancellation}

Pole-zero cancellation occurs when a pole and zero are located at the same place
in the s-plane. This effectively eliminates the contribution of each to the
\gls{system} dynamics. By placing poles and zeroes at various locations (this is
done by placing transfer functions in series), we can eliminate undesired
\gls{system} dynamics. While this may appear to be a useful design tool at
first, there are major caveats. Most of these are due to \gls{model} uncertainty
resulting in poles which aren't in the locations the controls designer expected.

Notch filters are typically used to dampen a specific range of frequencies in
the \gls{system response}. If its band is made too narrow, it can still leave the
undesirable dynamics, but now you can no longer measure them in the response.
They are still happening, but they are what's called \textit{unobservable}.

Never pole-zero cancel unstable or nonminimum phase dynamics. If the \gls{model}
doesn't quite reflect reality, an attempted pole cancellation by placing a
nonminimum phase zero results in the pole still moving to the zero placed next
to it. You have the same dynamics as before, but the pole is also stuck where it
is no matter how much \gls{feedback gain} is applied. For an attempted
nonminimum phase zero cancellation, you have effectively placed an unstable pole
that's unobservable. This means the \gls{system} will be going unstable and
blowing up, but you won't be able to detect this and react to it.

Keep in mind when making design decisions that the \gls{model} likely isn't
perfect. The whole point of feedback control is to be robust to this kind of
uncertainty.
