\section{Gain margin and phase margin}
\label{sec:gain_phase_margin}
\index{Stability!gain margin}
\index{Stability!phase margin}

One generally needs to learn about Bode plots and Nyquist plots to truly
understand gain and phase margin and their origins, but those plots are large
topics unto themselves. Since we won't be using either of them for controller
design, we'll just cover what gain and phase margin are in a general sense and
how they are used. We'll be discussing how \gls{discretization} affects phase
margin in section \ref{sec:phase_loss}.

Gain margin and phase margin are two metrics for measuring a \gls{system}'s
relative stability. Gain and phase margin are the amounts by which the
closed-loop gain and phase can be varied respectively before the \gls{system}
becomes unstable. In a sense, they are safety margins for when unmodeled
dynamics affect the \gls{system response}.

For a more thorough explanation of gain and phase margin, watch Brian Douglas's
video on them \cite{bib:gain_phase_margin}. He has other videos too on classical
control methods like Bode and Nyquist plots that we recommend.
