\section{Pole-zero cancellation}
\label{sec:pole-zero_cancellation}

Pole-zero cancellation occurs when a pole and zero are located at the same place
in the s-plane. This effectively eliminates the contribution of each to the
\gls{system} dynamics. By placing poles and zeroes at various locations (this is
done by placing transfer functions in series), we can eliminate undesired
\gls{system} dynamics. While this may appear to be a useful design tool at
first, there are major caveats. Most of these are due to \gls{model} uncertainty
resulting in poles which aren't in the locations the controls designer expected.

Notch filters are typically used to dampen a specific range of frequencies in
the \gls{system response}. If its band is made too narrow, it can still leave the
undesirable dynamics, but now you can no longer measure them in the response.
They are still happening, but they are what's called \textit{unobservable}.

Never pole-zero cancel unstable or nonminimum phase dynamics. If the \gls{model}
doesn't quite reflect reality, an attempted pole cancellation by placing a
nonminimum phase zero results in the pole still moving to the zero placed next
to it. You have the same dynamics as before, but the pole is also stuck where it
is no matter how much \gls{feedback gain} is applied. For an attempted
nonminimum phase zero cancellation, you have effectively placed an unstable pole
that's unobservable. This means the \gls{system} will be going unstable and
blowing up, but you won't be able to detect this and react to it.

Keep in mind when making design decisions that the \gls{model} likely isn't
perfect. The whole point of feedback control is to be robust to this kind of
uncertainty.
