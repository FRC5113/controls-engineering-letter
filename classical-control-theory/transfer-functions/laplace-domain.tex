\section{Laplace domain}

The Laplace domain has the frequency ($j\omega$) on the imaginary y-axis and a
real number on the x-axis, yielding a two-dimensional coordinate system. We
represent coordinates in this space as a complex number $s = \sigma + j\omega$.
The real part $\sigma$ corresponds to the x-axis and the imaginary part
$j\omega$ corresponds to the y-axis (see figure \ref{fig:laplace_domain}).

\begin{bookfigure}
  \begin{tikzpicture}[auto, >=latex']
    %\draw [help lines] (-4,-2) grid (4,4);

    % Draw main axes
    \draw[<->] (-4.2,1) -- (4.2,1) node[below] {\small Re($\sigma$)};
    \draw[<->] (0,-2) -- (0,4.2) node[right] {\small Im($j\omega$)};
  \end{tikzpicture}

  \caption{Laplace domain}
  \label{fig:laplace_domain}
\end{bookfigure}

A transfer function maps an input coordinate to an output coordinate in the
Laplace domain. These can be obtained by applying the Laplace transform to a
differential equation and rearranging the terms to obtain a ratio of the output
variable to the input variable. Equation (\ref{eq:transfer_func}) is an example
of a transfer function.

\begin{equation} \label{eq:transfer_func}
  H(s) = \frac{\overbrace{(s-9+9i)(s-9-9i)}^{zeroes}}
    {\underbrace{s(s+10)}_{poles}}
\end{equation}
