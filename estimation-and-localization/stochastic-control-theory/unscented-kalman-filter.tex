\section{Unscented Kalman filter}
\label{sec:ukf}
\index{nonlinear control!Unscented Kalman filter}
\index{state-space observers!Unscented Kalman filter}

In this book, we have covered the Kalman filter, which is the optimal unbiased
estimator for linear \glspl{system}. It isn't optimal for nonlinear
\glspl{system}, but several extensions to it have been developed to make it more
accurate.

The unscented Kalman filter (UKF) propagates carefully chosen points called
sigma points through the nonlinear state and measurement models to obtain
estimates of the true covariances (as opposed to linearized versions of them).
We recommend reading Roger Labbe's book \textit{Kalman and Bayesian Filters in
Python} for more on
UKFs\footnote{\url{https://github.com/rlabbe/Kalman-and-Bayesian-Filters-in-Python/blob/master/10-Unscented-Kalman-Filter.ipynb}}.

The original paper on the UKF is also an option \cite{bib:ukf}. See also the
equations for van der Merwe's sigma point algorithm \cite{bib:ukf_sigma_points}.
Here's a paper on a quaternion-based Unscented Kalman filter for orientation
tracking \cite{bib:ukf_state_tracking}.

Here's an interview about the origin of the UKF with its creator
\cite{bib:first-hand_the_ut}.

\subsection{Square-root UKF}

The UKF uses a matrix square root (Cholesky decomposition) to compute the
scaling for the sigma points. This operation can introduce numerical
instability. The square-root UKF avoids this by propagating the square-root of
the error covariance directly instead of the error covariance; the Cholesky
decomposition is replaced with a QR decomposition and a Cholesky rank-1 update.

See the original paper \cite{bib:ukf_square_root} and this implementation
tutorial \cite{bib:ukf_square_root_tutorial}.
