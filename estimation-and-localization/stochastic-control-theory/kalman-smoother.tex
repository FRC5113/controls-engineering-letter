\section{Kalman smoother}
\index{state-space observers!Kalman smoother}

The Kalman filter uses the data up to the current time to produce an optimal
estimate of the system \gls{state}. If data beyond the current time is
available, it can be ran through a Kalman smoother to produce a better estimate.
This is done by recording measurements, then applying the smoother to it
offline.

The Kalman smoother does a forward pass on the available data, then a backward
pass through the system dynamics so it takes into account the data before and
after the current time. This produces \gls{state} variances that are lower than
that of a Kalman filter.

\subsection{Predict and update equations}

One first does a forward pass with the typical Kalman filter equations and
stores the results. Then one can use the Rauch-Tung-Striebel (RTS) algorithm to
do the backward pass. Theorem \ref{thm:kalman_smoother} shows the predict and
and update steps for the forward and backward passes for a Kalman smoother at
the $k^{th}$ timestep.

See section 3 of
\url{https://users.aalto.fi/~ssarkka/course_k2011/pdf/handout7.pdf} for a
derivation of the Rauch-Tung-Striebel smoother.

\index{state-space observers!Kalman smoother!equations}
\begin{theorem}[Kalman smoother]
  \label{thm:kalman_smoother}
  \begin{align}
    \text{Forward predict step} \nonumber \\
    \hat{\mat{x}}_{k+1}^- &= \mat{A}\hat{\mat{x}}_k + \mat{B} \mat{u}_k \\
    \mat{P}_{k+1}^- &= \mat{A} \mat{P}_k^- \mat{A}\T + \mat{Q} \\
    \text{Forward update step} \nonumber \\
    \mat{K}_{k+1} &=
      \mat{P}_{k+1}^- \mat{C}\T (\mat{C}\mat{P}_{k+1}^- \mat{C}\T +
      \mat{R})^{-1} \\
    \hat{\mat{x}}_{k+1}^+ &=
      \hat{\mat{x}}_{k+1}^- + \mat{K}_{k+1}(\mat{y}_{k+1} -
      \mat{C} \hat{\mat{x}}_{k+1}^- - \mat{D}\mat{u}_{k+1}) \\
    \mat{P}_{k+1}^+ &= (\mat{I} - \mat{K}_{k+1}\mat{C})\mat{P}_{k+1}^- \\
    \text{Backward update step} \nonumber \\
    \mat{K}_k &= \mat{P}_k^+ \mat{A}_k\T (\mat{P}_{k+1}^-)^{-1} \\
    \hat{\mat{x}}_{k|N} &= \hat{\mat{x}}_k^+ +
      \mat{K}_k(\hat{\mat{x}}_{k+1|N} - \hat{\mat{x}}_{k+1}^-) \\
    \mat{P}_{k|N} &=
      \mat{P}_k^+ + \mat{K}_k(\mat{P}_{k+1|N} - \mat{P}_{k+1}^-)\mat{K}_k\T \\
    \text{Backward initial conditions} \nonumber \\
    \hat{\mat{x}}_{N|N} &= \hat{\mat{x}}_N^+ \\
    \mat{P}_{N|N} &= \mat{P}_N^+
  \end{align}
\end{theorem}

\subsection{Example}

We will modify the robot model so that instead of a velocity of $0.8$ cm/s with
random noise, the velocity is modeled as a random walk from the current
velocity.
\begin{equation}
  \mat{x}_k =
  \begin{bmatrix}
    x_k \\
    v_k \\
    x_k^w
  \end{bmatrix}
\end{equation}
\begin{equation}
  \mat{x}_{k+1} =
  \begin{bmatrix}
    1 & 1 & 0 \\
    0 & 1 & 0 \\
    0 & 0 & 1
  \end{bmatrix} \mat{x}_k +
  \begin{bmatrix}
    0 \\
    0.1 \\
    0
  \end{bmatrix} w_k
\end{equation}

We will use the same observation model as before.

Using the same data from subsection \ref{subsec:filter_simulation}, figures
\ref{fig:smoother_robot_pos}, \ref{fig:smoother_robot_vel}, and
\ref{fig:smoother_wall_pos} show the improved \gls{state} estimates and figure
\ref{fig:smoother_robot_pos_variance} shows the improved robot position
covariance with a Kalman smoother.

Notice how the wall position produced by the smoother is a constant. This is
because that \gls{state} has no dynamics, so the final estimate from the Kalman
filter is already the best estimate.
\begin{svg}{build/\chapterpath/kalman_smoother_robot_pos}
  \caption{Robot position with Kalman smoother}
  \label{fig:smoother_robot_pos}
\end{svg}
\begin{svg}{build/\chapterpath/kalman_smoother_robot_vel}
  \caption{Robot velocity with Kalman smoother}
  \label{fig:smoother_robot_vel}
\end{svg}
\begin{svg}{build/\chapterpath/kalman_smoother_wall_pos}
  \caption{Wall position with Kalman smoother}
  \label{fig:smoother_wall_pos}
\end{svg}
\begin{svg}{build/\chapterpath/kalman_smoother_robot_pos_variance}
  \caption{Robot position variance with Kalman smoother}
  \label{fig:smoother_robot_pos_variance}
\end{svg}

See Roger Labbe's book \textit{Kalman and Bayesian Filters in Python} for more
on
smoothing.\footnote{\url{https://github.com/rlabbe/Kalman-and-Bayesian-Filters-in-Python/blob/master/13-Smoothing.ipynb}}
