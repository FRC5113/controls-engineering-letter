\section{Pose exponential}

We can obtain a more accurate approximation of the pose by including first-order
dynamics for the heading $\theta$. To provide a rationale for the math we're
about to do, we need to cover some aspects of group theory.

\subsection{What is a group?}

In mathematics, a group is a set equipped with a binary operation (an operation
with two arguments) that combines any two elements (of the set) to form a third
element in such a way that four conditions called \textit{group axioms} are
satisfied: closure, associativity, identity, and invertibility.

\textit{Closure} means that the result is in the same set as the arguments.

\textit{Associativity} means that within an expression containing two or more
occurrences in a row of the same associative operator, the order in which the
operations are performed does not matter as long as the sequence of the operands
is not changed. In other words, different groupings of the operations produces
the same result.

\textit{Identity}, or an identity element, is a special type of element of a set
with respect to a binary operation on that set, which leaves any element of the
set unchanged when combined with it. For example, the additive identity of the
set of integers is zero, which means that any integer summed with zero produces
that integer.

\textit{Invertibility} means there is an element that can ``undo" the effect of
combination with another given element. For integers and the addition operator,
the inverse element would be the negation.

\subsection{What is a pose?}

To develop what a pose is in group theory, we need to define a few key groups.
SO(2) is the special orthogonal group in dimension 2. They represent a 2D
rotation.

SE(2) is the special euclidean group in dimension 2. They represent a 2D
rotation and a 2D translation, which we call a 2D pose. In other words, pose is
an element of SE(2).

\subsection{What is a twist?}

A 2D twist is an element of the tangent space of SE(2) (like the tangential
distance traveled by the robot along an arc in SE(2)). We use the ``pose
exponential" to map a twist (an element of the tangent space) to an element of
SE(2). In other words, we map a twist to a pose.

We call it a pose exponential because it's an exponential map onto a pose. The
term exponential is used because the solution for integrating a change in
something (like a slope or tangent) over time is usually an exponential. For
example, $\frac{dx}{dt} = ax$ has the solution $x = x_0 e^{at}$.

We use the pose exponential to take encoder measurement deltas and gyro angle
deltas (which are in the tangent space and are thus a twist) and turn them into
a change in pose. This gets added to the pose from the last update.

\subsection{Derivation}

We can obtain a more accurate approximation of the pose than Euler integration
by including first-order dynamics for the heading $\theta$.
\begin{equation*}
  \mat{x} = \begin{bmatrix}
    x \\
    y \\
    \theta
  \end{bmatrix}
\end{equation*}

$v_x$, $v_y$, and $\omega$ are the $x$ and $y$ velocities of the robot within
its local coordinate frame, which will be treated as constants.
\begin{remark}
  There are two coordinate frames used here: robot and global. A superscript on
  the left side of a matrix denotes the coordinate frame in which that matrix is
  represented. The robot's coordinate frame is denoted by $R$ and the global
  coordinate frame is denoted by $G$.
\end{remark}

In the robot frame (the tangent space)
\begin{align*}
  \crdfrm{R}{dx} &= \crdfrm{R}{v_x} \,dt \\
  \crdfrm{R}{dy} &= \crdfrm{R}{v_y} \,dt \\
  \crdfrm{R}{d\theta} &= \crdfrm{R}{\omega} \,dt
\end{align*}

To transform this into the global frame SE(2), we apply a counterclockwise
rotation matrix where $\theta$ changes over time.
\begin{align*}
  \crdfrm{G}{\begin{bmatrix}
    dx \\
    dy \\
    d\theta
  \end{bmatrix}} &=
  \begin{bmatrix}
    \cos\theta(t) & -\sin\theta(t) & 0 \\
    \sin\theta(t) &  \cos\theta(t) & 0 \\
                0 &              0 & 1
  \end{bmatrix}
  \crdfrm{R}{\begin{bmatrix}
    v_x \\
    v_y \\
    \omega
  \end{bmatrix}} dt \\
  \crdfrm{G}{\begin{bmatrix}
    dx \\
    dy \\
    d\theta
  \end{bmatrix}} &=
  \begin{bmatrix}
    \cos\omega t & -\sin\omega t & 0 \\
    \sin\omega t &  \cos\omega t & 0 \\
               0 &             0 & 1
  \end{bmatrix}
  \crdfrm{R}{\begin{bmatrix}
    v_x \\
    v_y \\
    \omega
  \end{bmatrix}} dt
\end{align*}

Now, integrate the matrix equation (matrices are integrated element-wise). This
derivation heavily utilizes the integration method described in section
\ref{subsec:calculus_change_of_vars}.
\begin{align*}
  \crdfrm{G}{\begin{bmatrix}
    \Delta x \\
    \Delta y \\
    \Delta \theta
  \end{bmatrix}} &=
  \left.\begin{bmatrix}
     \frac{\sin\omega t}{\omega} & \frac{\cos\omega t}{\omega} & 0 \\
    -\frac{\cos\omega t}{\omega} & \frac{\sin\omega t}{\omega} & 0 \\
    0 & 0 & t
  \end{bmatrix}
  \crdfrm{R}{\begin{bmatrix}
    v_x \\
    v_y \\
    \omega
  \end{bmatrix}} \right|_0^t \\
  \crdfrm{G}{\begin{bmatrix}
    \Delta x \\
    \Delta y \\
    \Delta \theta
  \end{bmatrix}} &=
  \begin{bmatrix}
    \frac{\sin\omega t}{\omega} & \frac{\cos\omega t - 1}{\omega} & 0 \\
    \frac{1 - \cos\omega t}{\omega} & \frac{\sin\omega t}{\omega} & 0 \\
    0 & 0 & t
  \end{bmatrix}
  \crdfrm{R}{\begin{bmatrix}
    v_x \\
    v_y \\
    \omega
  \end{bmatrix}}
\end{align*}

This equation assumes a starting orientation of $\theta = 0$. For nonzero
starting orientations, we can apply a counterclockwise rotation by $\theta$.
\begin{equation}
  \crdfrm{G}{\begin{bmatrix}
    \Delta x \\
    \Delta y \\
    \Delta \theta
  \end{bmatrix}} =
  \begin{bmatrix}
    \cos\theta & -\sin\theta & 0 \\
    \sin\theta &  \cos\theta & 0 \\
             0 &           0 & 1
  \end{bmatrix}
  \begin{bmatrix}
    \frac{\sin\omega t}{\omega} & \frac{\cos\omega t - 1}{\omega} & 0 \\
    \frac{1 - \cos\omega t}{\omega} & \frac{\sin\omega t}{\omega} & 0 \\
    0 & 0 & t
  \end{bmatrix}
  \crdfrm{R}{\begin{bmatrix}
    v_x \\
    v_y \\
    \omega
  \end{bmatrix}}
  \label{eq:se2_exp_vel}
\end{equation}
\begin{remark}
  Control system implementations will generally have a model update and a
  controller update in a given iteration. Equation \eqref{eq:se2_exp_vel} (the
  model update) uses the current velocity to advance the state to the next
  timestep (into the future). Since controllers use the current state, the
  controller update should be run before the model update.
\end{remark}

If we factor out a $t$, we can use change in pose between updates instead of
velocities.
\begin{align}
  \crdfrm{G}{\begin{bmatrix}
    \Delta x \\
    \Delta y \\
    \Delta \theta
  \end{bmatrix}} &=
  \begin{bmatrix}
    \cos\theta & -\sin\theta & 0 \\
    \sin\theta &  \cos\theta & 0 \\
             0 &           0 & 1
  \end{bmatrix}
  \begin{bmatrix}
    \frac{\sin\omega t}{\omega} & \frac{\cos\omega t - 1}{\omega} & 0 \\
    \frac{1 - \cos\omega t}{\omega} & \frac{\sin\omega t}{\omega} & 0 \\
    0 & 0 & t
  \end{bmatrix}
  \crdfrm{R}{\begin{bmatrix}
    v_x \\
    v_y \\
    \omega
  \end{bmatrix}} \nonumber \\
  \crdfrm{G}{\begin{bmatrix}
    \Delta x \\
    \Delta y \\
    \Delta \theta
  \end{bmatrix}} &=
  \begin{bmatrix}
    \cos\theta & -\sin\theta & 0 \\
    \sin\theta &  \cos\theta & 0 \\
             0 &           0 & 1
  \end{bmatrix}
  \begin{bmatrix}
    \frac{\sin\omega t}{\omega t} & \frac{\cos\omega t - 1}{\omega t} & 0 \\
    \frac{1 - \cos\omega t}{\omega t} & \frac{\sin\omega t}{\omega t} & 0 \\
    0 & 0 & 1
  \end{bmatrix}
  \crdfrm{R}{\begin{bmatrix}
    v_x \\
    v_y \\
    \omega
  \end{bmatrix}} t \nonumber \\
  \crdfrm{G}{\begin{bmatrix}
    \Delta x \\
    \Delta y \\
    \Delta \theta
  \end{bmatrix}} &=
  \begin{bmatrix}
    \cos\theta & -\sin\theta & 0 \\
    \sin\theta &  \cos\theta & 0 \\
             0 &           0 & 1
  \end{bmatrix}
  \begin{bmatrix}
    \frac{\sin\omega t}{\omega t} & \frac{\cos\omega t - 1}{\omega t} & 0 \\
    \frac{1 - \cos\omega t}{\omega t} & \frac{\sin\omega t}{\omega t} & 0 \\
    0 & 0 & 1
  \end{bmatrix}
  \crdfrm{R}{\begin{bmatrix}
    v_x t \\
    v_y t \\
    \omega t
  \end{bmatrix}} \nonumber \\
  \crdfrm{G}{\begin{bmatrix}
    \Delta x \\
    \Delta y \\
    \Delta \theta
  \end{bmatrix}} &=
  \begin{bmatrix}
    \cos\theta & -\sin\theta & 0 \\
    \sin\theta &  \cos\theta & 0 \\
             0 &           0 & 1
  \end{bmatrix}
  \crdfrm{R}{\begin{bmatrix}
    \frac{\sin\Delta\theta}{\Delta\theta} &
      \frac{\cos\Delta\theta - 1}{\Delta\theta} & 0 \\
    \frac{1 - \cos\Delta\theta}{\Delta\theta} &
      \frac{\sin\Delta\theta}{\Delta\theta} & 0 \\
    0 & 0 & 1
  \end{bmatrix}}
  \crdfrm{R}{\begin{bmatrix}
    \Delta x \\
    \Delta y \\
    \Delta\theta
  \end{bmatrix}}
  \label{eq:se2_exp_twist}
\end{align}

The vector
$\crdfrm{R}{\begin{bmatrix}\Delta x & \Delta y & \Delta\theta \end{bmatrix}}\T$
is a twist because it's an element of the tangent space (the robot's local
coordinate frame).
\begin{remark}
  Control system implementations will generally have a model update and a
  controller update in a given iteration. Equation \eqref{eq:se2_exp_twist} (the
  model update) uses local distance and heading deltas between the previous and
  current timestep, so it advances the state to the current timestep. Since
  controllers use the current state, the controller update should be run after
  the model update.
\end{remark}

When the robot is traveling on a straight trajectory ($\omega = 0$), some
expressions within the equation above are indeterminate. We can approximate
these with Taylor series expansions.
\begin{align*}
  \frac{\sin\omega t}{\omega} &= t - \frac{t^3 \omega^2}{6} + \ldots \\
  \frac{\sin\omega t}{\omega} &\sim t - \frac{t^3 \omega^2}{6} \\
  \frac{\cos\omega t - 1}{\omega} &= -\frac{t^2 \omega}{2} + \frac{t^4 \omega^3}{4} - \ldots \\
  \frac{\cos\omega t - 1}{\omega} &\sim -\frac{t^2 \omega}{2} \\
  \frac{1 - \cos\omega t}{\omega} &= \frac{t^2 \omega}{2} - \frac{t^4 \omega^3}{4} + \ldots \\
  \frac{1 - \cos\omega t}{\omega} &\sim \frac{t^2 \omega}{2}
\end{align*}

If we let $\omega = 0$, we should get the standard kinematic equations like
$x = vt$ with a rotation applied to them.
\begin{align*}
  \frac{\sin\omega t}{\omega} &\sim t - \frac{t^3 \cdot 0^2}{6} = t \\
  \frac{\cos\omega t - 1}{\omega} &\sim -\frac{t^2 \cdot 0}{2} = 0 \\
  \frac{1 - \cos\omega t}{\omega} &\sim \frac{t^2 \cdot 0}{2} = 0
\end{align*}

Now substitute these into equation \eqref{eq:pose_exponential}.
\begin{align*}
  \crdfrm{G}{\begin{bmatrix}
    \Delta x \\
    \Delta y \\
    \Delta \theta
  \end{bmatrix}} &=
  \begin{bmatrix}
    \cos\theta & -\sin\theta & 0 \\
    \sin\theta &  \cos\theta & 0 \\
             0 &           0 & 1
  \end{bmatrix}
  \begin{bmatrix}
    t & 0 & 0 \\
    0 & t & 0 \\
    0 & 0 & t
  \end{bmatrix}
  \crdfrm{R}{\begin{bmatrix}
    v_x \\
    v_y \\
    \omega
  \end{bmatrix}} \\
  \crdfrm{G}{\begin{bmatrix}
    \Delta x \\
    \Delta y \\
    \Delta \theta
  \end{bmatrix}} &=
  \begin{bmatrix}
    \cos\theta & -\sin\theta & 0 \\
    \sin\theta &  \cos\theta & 0 \\
             0 &           0 & 1
  \end{bmatrix}
  \begin{bmatrix}
    v_x t \\
    v_y t \\
    \omega t
  \end{bmatrix}
\end{align*}

As expected, the equations simplify to the first-order case with a rotation
matrix applied to the velocities in the robot's local coordinate frame.

Differential drive robots have $v_y = 0$ since they only move in the direction
of the current heading, which is along the x-axis. Therefore,
\begin{align*}
  \crdfrm{G}{\begin{bmatrix}
    \Delta x \\
    \Delta y \\
    \Delta \theta
  \end{bmatrix}} &=
  \begin{bmatrix}
    \cos\theta & -\sin\theta & 0 \\
    \sin\theta &  \cos\theta & 0 \\
             0 &           0 & 1
  \end{bmatrix}
  \begin{bmatrix}
    v_x t \\
    0 \\
    \omega t
  \end{bmatrix} \\
  \crdfrm{G}{\begin{bmatrix}
    \Delta x \\
    \Delta y \\
    \Delta \theta
  \end{bmatrix}} &=
  \begin{bmatrix}
    v_x t \cos\theta \\
    v_x t \sin\theta \\
    \omega t
  \end{bmatrix}
\end{align*}
\begin{theorem}[Pose exponential]
  \begin{equation}
    \crdfrm{G}{\begin{bmatrix}
      \Delta x \\
      \Delta y \\
      \Delta \theta
    \end{bmatrix}} =
    \begin{bmatrix}
      \cos\theta & -\sin\theta & 0 \\
      \sin\theta &  \cos\theta & 0 \\
               0 &           0 & 1
    \end{bmatrix}
    \begin{bmatrix}
      \frac{\sin\omega t}{\omega} & \frac{\cos\omega t - 1}{\omega} & 0 \\
      \frac{1 - \cos\omega t}{\omega} & \frac{\sin\omega t}{\omega} & 0 \\
      0 & 0 & t
    \end{bmatrix}
    \crdfrm{R}{\begin{bmatrix}
      v_x \\
      v_y \\
      \omega
    \end{bmatrix}}
    \label{eq:pose_exponential}
  \end{equation}

  or
  \begin{equation}
    \crdfrm{G}{\begin{bmatrix}
      \Delta x \\
      \Delta y \\
      \Delta \theta
    \end{bmatrix}} =
    \begin{bmatrix}
      \cos\theta & -\sin\theta & 0 \\
      \sin\theta &  \cos\theta & 0 \\
               0 &           0 & 1
    \end{bmatrix}
    \crdfrm{R}{\begin{bmatrix}
      \frac{\sin\Delta\theta}{\Delta\theta} &
        \frac{\cos\Delta\theta - 1}{\Delta\theta} & 0 \\
      \frac{1 - \cos\Delta\theta}{\Delta\theta} &
        \frac{\sin\Delta\theta}{\Delta\theta} & 0 \\
      0 & 0 & 1
    \end{bmatrix}}
    \crdfrm{R}{\begin{bmatrix}
      \Delta x \\
      \Delta y \\
      \Delta\theta
    \end{bmatrix}}
  \end{equation}

  where $G$ denotes global coordinate frame and $R$ denotes robot's coordinate
  frame.

  For sufficiently small $\omega$:
  \begin{align}
    \frac{\sin\omega t}{\omega} &= t - \frac{t^3 \omega^2}{6} &
    \frac{\cos\omega t - 1}{\omega} &= -\frac{t^2 \omega}{2} &
    \frac{1 - \cos\omega t}{\omega} &= \frac{t^2 \omega}{2} \\
    \frac{\sin\omega t}{\omega t} &= 1 - \frac{t^2 \omega^2}{6} &
    \frac{\cos\omega t - 1}{\omega t} &= -\frac{t \omega}{2} &
    \frac{1 - \cos\omega t}{\omega t} &= \frac{t \omega}{2}
  \end{align}
  \begin{figurekey}
    \begin{tabular}{llll}
      $\Delta x$ & change in pose's $x$ & $v_x$ & velocity along $x$-axis \\
      $\Delta y$ & change in pose's $y$ & $v_y$ & velocity along $y$-axis \\
      $\Delta \theta$ & change in pose's $\theta$ & $\omega$ & angular velocity
        \\
      $t$ & Time since last pose update & $\theta$ & starting angle in global
        coordinate frame
    \end{tabular}
  \end{figurekey}

  This change in pose can be added directly to the previous pose estimate to
  update it.
\end{theorem}

Figure \ref{fig:ramsete_twist_odometry_error} shows the error in the global pose
coordinates over time for the simpler odometry method compared to the method
using twists (uses the Ramsete controller from subsection
\ref{sec:ramsete_unicycle_controller}).
\begin{svg}{build/\chapterpath/ramsete_twist_odometry_error}
  \caption{Odometry error compared to method using twists ($dt = 0.05$ s)}
  \label{fig:ramsete_twist_odometry_error}
\end{svg}

The highest error is $4$ cm in $x$ over a $10$ m path starting with a $2$ m
displacement. The difference would be even more noticeable for paths with higher
curvatures and longer durations. One mitigation is using a smaller update
period. However, there are bigger sources of error like turning scrub on skid
steer robots that should be dealt with before odometry numerical accuracy.

\subsection{Lie groups}

While we avoided the topic in our explanation, pose is what's known as a Lie
group (a group that is also a differentiable manifold). There's a lot of
mathematical and controls results  developed around Lie groups, so we're
mentioning the connection here in case you want to search the Internet for more
information.
