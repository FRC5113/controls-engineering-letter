\section{Nonlinear pose estimation}
\label{sec:nonlinear_pose_estimation}

The simplest way to perform pose estimation is to integrate the velocity in each
orthogonal direction over time. In two dimensions, one could use

\begin{align*}
  x_{k+1} &= x_k + v_k\cos\theta_k\,T \\
  y_{k+1} &= y_k + v_k\sin\theta_k\,T
\end{align*}

where $T$ is the sample period. $x_d$, $y_d$, and $\theta_d$ are obtained from a
desired time-based trajectory.

This odometry approach assumes that the robot follows a straight path between
samples (that is, $\omega = 0$). We can obtain a more accurate approximation by
including first order dynamics for the heading $\theta$.

\begin{equation*}
  \mtx{x} = \begin{bmatrix}
    x \\
    y \\
    \theta
  \end{bmatrix}
\end{equation*}

$v_x$, $v_y$, and $\omega$ are the $x$ and $y$ velocities of the robot within
its local coordinate frame, which will be treated as constants. These values in
a column vector are called a "twist".

\begin{remark}
  There are two coordinate frames used here: robot and global. A superscript on
  the left side of a matrix denotes the coordinate frame in which that matrix is
  represented. The robot's coordinate frame is denoted by $R$ and the global
  coordinate frame is denoted by $G$.
\end{remark}

In the robot frame

\begin{align*}
  \crdfrm{R}{dx} &= \crdfrm{R}{v_x} \,dt \\
  \crdfrm{R}{dy} &= \crdfrm{R}{v_y} \,dt \\
  \crdfrm{R}{d\theta} &= \crdfrm{R}{\omega} \,dt
\end{align*}

To transform this into the global frame, we apply a counterclockwise rotation
matrix where $\theta$ changes over time.

\begin{align*}
  \crdfrm{G}{\begin{bmatrix}
    dx \\
    dy \\
    d\theta
  \end{bmatrix}} &=
  \begin{bmatrix}
    \cos\theta(t) & -\sin\theta(t) & 0 \\
    \sin\theta(t) &  \cos\theta(t) & 0 \\
                0 &              0 & 1
  \end{bmatrix}
  \crdfrm{R}{\begin{bmatrix}
    v_x \\
    v_y \\
    \omega
  \end{bmatrix}} dt \\
  \crdfrm{G}{\begin{bmatrix}
    dx \\
    dy \\
    d\theta
  \end{bmatrix}} &=
  \begin{bmatrix}
    \cos\omega t & -\sin\omega t & 0 \\
    \sin\omega t &  \cos\omega t & 0 \\
               0 &             0 & 1
  \end{bmatrix}
  \crdfrm{R}{\begin{bmatrix}
    v_x \\
    v_y \\
    \omega
  \end{bmatrix}} dt
\end{align*}

Now, integrate the matrix equation (matrices are integrated element-wise). This
derivation heavily utilizes the integration method described in section
\ref{subsec:calculus_change_of_vars}.

\begin{align*}
  \crdfrm{G}{\begin{bmatrix}
    \Delta x \\
    \Delta y \\
    \Delta \theta
  \end{bmatrix}} &=
  \left.\begin{bmatrix}
     \frac{\sin\omega t}{\omega} & \frac{\cos\omega t}{\omega} & 0 \\
    -\frac{\cos\omega t}{\omega} & \frac{\sin\omega t}{\omega} & 0 \\
    0 & 0 & t
  \end{bmatrix}
  \crdfrm{R}{\begin{bmatrix}
    v_x \\
    v_y \\
    \omega
  \end{bmatrix}} \right|_0^t \\
  \crdfrm{G}{\begin{bmatrix}
    \Delta x \\
    \Delta y \\
    \Delta \theta
  \end{bmatrix}} &=
  \begin{bmatrix}
    \frac{\sin\omega t}{\omega} & \frac{\cos\omega t - 1}{\omega} & 0 \\
    \frac{1 - \cos\omega t}{\omega} & \frac{\sin\omega t}{\omega} & 0 \\
    0 & 0 & t
  \end{bmatrix}
  \crdfrm{R}{\begin{bmatrix}
    v_x \\
    v_y \\
    \omega
  \end{bmatrix}}
\end{align*}

This equation assumes a starting orientation of $\theta = 0$. For nonzero
starting orientations, we can apply a counterclockwise rotation by $\theta$.

\begin{equation*}
  \crdfrm{G}{\begin{bmatrix}
    \Delta x \\
    \Delta y \\
    \Delta \theta
  \end{bmatrix}} =
  \begin{bmatrix}
    \cos\theta & -\sin\theta & 0 \\
    \sin\theta &  \cos\theta & 0 \\
             0 &           0 & 1
  \end{bmatrix}
  \begin{bmatrix}
    \frac{\sin\omega t}{\omega} & \frac{\cos\omega t - 1}{\omega} & 0 \\
    \frac{1 - \cos\omega t}{\omega} & \frac{\sin\omega t}{\omega} & 0 \\
    0 & 0 & t
  \end{bmatrix}
  \crdfrm{R}{\begin{bmatrix}
    v_x \\
    v_y \\
    \omega
  \end{bmatrix}}
\end{equation*}

If we factor out a $t$, we can use change in pose between updates instead of
velocities.

\begin{align*}
  \crdfrm{G}{\begin{bmatrix}
    \Delta x \\
    \Delta y \\
    \Delta \theta
  \end{bmatrix}} &=
  \begin{bmatrix}
    \cos\theta & -\sin\theta & 0 \\
    \sin\theta &  \cos\theta & 0 \\
             0 &           0 & 1
  \end{bmatrix}
  \begin{bmatrix}
    \frac{\sin\omega t}{\omega} & \frac{\cos\omega t - 1}{\omega} & 0 \\
    \frac{1 - \cos\omega t}{\omega} & \frac{\sin\omega t}{\omega} & 0 \\
    0 & 0 & t
  \end{bmatrix}
  \crdfrm{R}{\begin{bmatrix}
    v_x \\
    v_y \\
    \omega
  \end{bmatrix}} \\
  \crdfrm{G}{\begin{bmatrix}
    \Delta x \\
    \Delta y \\
    \Delta \theta
  \end{bmatrix}} &=
  \begin{bmatrix}
    \cos\theta & -\sin\theta & 0 \\
    \sin\theta &  \cos\theta & 0 \\
             0 &           0 & 1
  \end{bmatrix}
  \begin{bmatrix}
    \frac{\sin\omega t}{\omega t} & \frac{\cos\omega t - 1}{\omega t} & 0 \\
    \frac{1 - \cos\omega t}{\omega t} & \frac{\sin\omega t}{\omega t} & 0 \\
    0 & 0 & 1
  \end{bmatrix}
  \crdfrm{R}{\begin{bmatrix}
    v_x \\
    v_y \\
    \omega
  \end{bmatrix}} t \\
  \crdfrm{G}{\begin{bmatrix}
    \Delta x \\
    \Delta y \\
    \Delta \theta
  \end{bmatrix}} &=
  \begin{bmatrix}
    \cos\theta & -\sin\theta & 0 \\
    \sin\theta &  \cos\theta & 0 \\
             0 &           0 & 1
  \end{bmatrix}
  \begin{bmatrix}
    \frac{\sin\omega t}{\omega t} & \frac{\cos\omega t - 1}{\omega t} & 0 \\
    \frac{1 - \cos\omega t}{\omega t} & \frac{\sin\omega t}{\omega t} & 0 \\
    0 & 0 & 1
  \end{bmatrix}
  \crdfrm{R}{\begin{bmatrix}
    v_x t \\
    v_y t \\
    \omega t
  \end{bmatrix}} \\
  \crdfrm{G}{\begin{bmatrix}
    \Delta x \\
    \Delta y \\
    \Delta \theta
  \end{bmatrix}} &=
  \begin{bmatrix}
    \cos\theta & -\sin\theta & 0 \\
    \sin\theta &  \cos\theta & 0 \\
             0 &           0 & 1
  \end{bmatrix}
  \crdfrm{R}{\begin{bmatrix}
    \frac{\sin\Delta\theta}{\Delta\theta} &
      \frac{\cos\Delta\theta - 1}{\Delta\theta} & 0 \\
    \frac{1 - \cos\Delta\theta}{\Delta\theta} &
      \frac{\sin\Delta\theta}{\Delta\theta} & 0 \\
    0 & 0 & 1
  \end{bmatrix}}
  \crdfrm{R}{\begin{bmatrix}
    \Delta x \\
    \Delta y \\
    \Delta\theta
  \end{bmatrix}}
\end{align*}

The previous version used the current velocity and projected the model forward
to the next timestep (into the future). As such, the prediction must be done
after any controller calculations are performed. With the second version, the
locally measured pose deltas can only be measured using past samples, so the
model update must be performed before any controller calculations to advance the
model to the current timestep.

When the robot is traveling on a straight trajectory ($\omega = 0$), some
expressions within the equation above are indeterminate. We can approximate
these with Taylor series expansions.

\begin{align*}
  \frac{\sin\omega t}{\omega} &= t - \frac{t^3 \omega^2}{6} + \ldots \\
  \frac{\sin\omega t}{\omega} &\sim t - \frac{t^3 \omega^2}{6} \\
  \frac{\cos\omega t - 1}{\omega} &= -\frac{t^2 \omega}{2} + \frac{t^4 \omega^3}{4} - \ldots \\
  \frac{\cos\omega t - 1}{\omega} &\sim -\frac{t^2 \omega}{2} \\
  \frac{1 - \cos\omega t}{\omega} &= \frac{t^2 \omega}{2} - \frac{t^4 \omega^3}{4} + \ldots \\
  \frac{1 - \cos\omega t}{\omega} &\sim \frac{t^2 \omega}{2}
\end{align*}

If we let $\omega = 0$, we should get the standard kinematic equations like
$x = vt$ with a rotation applied to them.

\begin{align*}
  \frac{\sin\omega t}{\omega} &\sim t - \frac{t^3 \cdot 0^2}{6} = t \\
  \frac{\cos\omega t - 1}{\omega} &\sim -\frac{t^2 \cdot 0}{2} = 0 \\
  \frac{1 - \cos\omega t}{\omega} &\sim \frac{t^2 \cdot 0}{2} = 0
\end{align*}

Now substitute these into equation (\ref{eq:nonlinear_pose_estimator}).

\begin{align*}
  \crdfrm{G}{\begin{bmatrix}
    \Delta x \\
    \Delta y \\
    \Delta \theta
  \end{bmatrix}} &=
  \begin{bmatrix}
    \cos\theta & -\sin\theta & 0 \\
    \sin\theta &  \cos\theta & 0 \\
             0 &           0 & 1
  \end{bmatrix}
  \begin{bmatrix}
    t & 0 & 0 \\
    0 & t & 0 \\
    0 & 0 & t
  \end{bmatrix}
  \crdfrm{R}{\begin{bmatrix}
    v_x \\
    v_y \\
    \omega
  \end{bmatrix}} \\
  \crdfrm{G}{\begin{bmatrix}
    \Delta x \\
    \Delta y \\
    \Delta \theta
  \end{bmatrix}} &=
  \begin{bmatrix}
    \cos\theta & -\sin\theta & 0 \\
    \sin\theta &  \cos\theta & 0 \\
             0 &           0 & 1
  \end{bmatrix}
  \begin{bmatrix}
    v_x t \\
    v_y t \\
    \omega t
  \end{bmatrix}
\end{align*}

As expected, the equations simplify to the first order case with a rotation
matrix applied to the velocities in the robot's local coordinate frame.

Differential drive robots have $v_y = 0$ since they only move in the direction
of the current heading, which is along the x-axis. Therefore,

\begin{align*}
  \crdfrm{G}{\begin{bmatrix}
    \Delta x \\
    \Delta y \\
    \Delta \theta
  \end{bmatrix}} &=
  \begin{bmatrix}
    \cos\theta & -\sin\theta & 0 \\
    \sin\theta &  \cos\theta & 0 \\
             0 &           0 & 1
  \end{bmatrix}
  \begin{bmatrix}
    v_x t \\
    0 \\
    \omega t
  \end{bmatrix} \\
  \crdfrm{G}{\begin{bmatrix}
    \Delta x \\
    \Delta y \\
    \Delta \theta
  \end{bmatrix}} &=
  \begin{bmatrix}
    v_x t \cos\theta \\
    v_y t \sin\theta \\
    \omega t
  \end{bmatrix}
\end{align*}

\begin{theorem}[Nonlinear pose estimator with constant curvature]
  \begin{equation}
    \crdfrm{G}{\begin{bmatrix}
      \Delta x \\
      \Delta y \\
      \Delta \theta
    \end{bmatrix}} =
    \begin{bmatrix}
      \cos\theta & -\sin\theta & 0 \\
      \sin\theta &  \cos\theta & 0 \\
               0 &           0 & 1
    \end{bmatrix}
    \begin{bmatrix}
      \frac{\sin\omega t}{\omega} & \frac{\cos\omega t - 1}{\omega} & 0 \\
      \frac{1 - \cos\omega t}{\omega} & \frac{\sin\omega t}{\omega} & 0 \\
      0 & 0 & t
    \end{bmatrix}
    \crdfrm{R}{\begin{bmatrix}
      v_x \\
      v_y \\
      \omega
    \end{bmatrix}}
    \label{eq:nonlinear_pose_estimator}
  \end{equation}

  or

  \begin{equation}
    \crdfrm{G}{\begin{bmatrix}
      \Delta x \\
      \Delta y \\
      \Delta \theta
    \end{bmatrix}} =
    \begin{bmatrix}
      \cos\theta & -\sin\theta & 0 \\
      \sin\theta &  \cos\theta & 0 \\
               0 &           0 & 1
    \end{bmatrix}
    \crdfrm{R}{\begin{bmatrix}
      \frac{\sin\Delta\theta}{\Delta\theta} &
        \frac{\cos\Delta\theta - 1}{\Delta\theta} & 0 \\
      \frac{1 - \cos\Delta\theta}{\Delta\theta} &
        \frac{\sin\Delta\theta}{\Delta\theta} & 0 \\
      0 & 0 & 1
    \end{bmatrix}}
    \crdfrm{R}{\begin{bmatrix}
      \Delta x \\
      \Delta y \\
      \Delta\theta
    \end{bmatrix}}
  \end{equation}

  where $G$ denotes global coordinate frame and $R$ denotes robot's coordinate
  frame.

  For sufficiently small $\omega$:

  \begin{align}
    \frac{\sin\omega t}{\omega} &= t - \frac{t^3 \omega^2}{6} &
    \frac{\cos\omega t - 1}{\omega} &= -\frac{t^2 \omega}{2} &
    \frac{1 - \cos\omega t}{\omega} &= \frac{t^2 \omega}{2} \\
    \frac{\sin\omega t}{\omega t} &= 1 - \frac{t^2 \omega^2}{6} &
    \frac{\cos\omega t - 1}{\omega t} &= -\frac{t \omega}{2} &
    \frac{1 - \cos\omega t}{\omega t} &= \frac{t \omega}{2}
  \end{align}

  \begin{figurekey}
    \begin{tabulary}{\linewidth}{LLLL}
      $\Delta x$ & change in pose's $x$ & $v_x$ & velocity along $x$-axis \\
      $\Delta y$ & change in pose's $y$ & $v_y$ & velocity along $y$-axis \\
      $\Delta \theta$ & change in pose's $\theta$ & $\omega$ & angular velocity
        \\
      $t$ & Time since last pose update & $\theta$ & starting angle in global
        coordinate frame
    \end{tabulary}
  \end{figurekey}

  This change in pose can be added directly to the previous pose estimate to
  update it.
\end{theorem}

Figure \ref{fig:ramsete_twist_odometry_error} shows the error in the global pose
coordinates over time for the simpler odometry method compared to the method
using twists (uses the Ramsete controller from subsection
\ref{sec:ramsete_nonlinear_controller}).

\begin{svg}{build/code/ramsete_twist_odometry_error}
  \caption{Odometry error compared to method using twists ($dt = 0.05s$)}
  \label{fig:ramsete_twist_odometry_error}
\end{svg}

The highest error is $4cm$ in $x$ over a $10m$ path starting with a $2m$
displacement. The difference would be even more noticeable for paths with higher
curvatures and longer durations. One mitigation is using a smaller update
period. However, there are bigger sources of error like turning scrub on skid
steer robots that should be dealt with before odometry numerical accuracy.
