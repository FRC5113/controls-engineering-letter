\newglossaryentry{agent}{
  name={agent},
  description={An independent actor being controlled through autonomy or
    human-in-the-loop (e.g., a robot, aircraft, etc.).}}
\newglossaryentry{control effort}{
  name={control effort},
  description={A term describing how much force, pressure, etc. an actuator is
    exerting.}}
\newglossaryentry{control input}{
  name={control input},
  description={The input of a \gls{plant} used for the purpose of controlling
    it.}}
\newglossaryentry{control law}{
  name={control law},
  description={Also known as control policy, is a mathematical formula used by
    the \gls{controller} to determine the \gls{input} u that is sent to the
    \gls{plant}. This control law is designed to drive the \gls{system} from its
    current \gls{state} to some other desired \gls{state}.}}
\newglossaryentry{control system}{
  name={control system},
  description={Monitors and controls the behavior of a \gls{system}.}}
\newglossaryentry{controller}{
  name={controller},
  description={Used in positive or negative feedback with a \gls{plant} to bring
    about a desired \gls{system} \gls{state} by driving the difference between a
    \gls{reference} signal and the \gls{output} to zero.}}
\newglossaryentry{discretization}{
  name={discretization},
  description={The process by which a continuous (e.g., analog) \gls{system} or
    \gls{controller} design is converted to discrete (e.g., digital).}}
\newglossaryentry{disturbance}{
  name={disturbance},
  description={An external force acting on a \gls{system} that isn't included in
    the \gls{system}'s \gls{model}.}}
\newglossaryentry{disturbance rejection}{
  name={disturbance rejection},
  description={The quality of a feedback control \gls{system} to compensate for
    external forces to reach a desired \gls{reference}.}}
\newglossaryentry{error}{
  name={error},
  description={\Gls{reference} minus an \gls{output} or \gls{state}.}}
\newglossaryentry{feedback gain}{
  name={feedback gain},
  description={The gain from the \gls{output} to an earlier point in a
    \gls{control system} diagram.}}
\newglossaryentry{gain}{
  name={gain},
  description={A proportional value that shows the relationship between the
    magnitude of an input signal to the magnitude of an output signal at
    steady-state.}}
\newglossaryentry{gain margin}{
  name={gain margin},
  description={See section \ref{sec:gain_phase_margin} on gain and phase
    margin.}}
\newglossaryentry{impulse response}{
  name={impulse response},
  description={The response of a \gls{system} to the Dirac delta function.}}
\newglossaryentry{input}{
  name={input},
  description={An input to the \gls{plant} (hence the name) that can be used to
  change the \gls{plant}'s \gls{state}.}}
\newglossaryentry{linearization}{
  name={linearization},
  description={A method by which a nonlinear \gls{system}'s dynamics are
    approximated by a linear \gls{system}.}}
\newglossaryentry{localization}{
  name={localization},
  description={The process of using measurements of the environment to determine
    an \gls{agent}'s \gls{pose}.}}
\newglossaryentry{model}{
  name={model},
  description={A set of mathematical equations that reflects some aspect of a
    physical \gls{system}'s behavior.}}
\newglossaryentry{noise immunity}{
  name={noise immunity},
  description={The quality of a \gls{system} to have its performance or
    stability unaffected by noise in the \glspl{output} (see also:
    \gls{robustness}).}}
\newglossaryentry{observer}{
  name={observer},
  description={In control theory, a \gls{system} that provides an estimate of
    the internal \gls{state} of a given real \gls{system} from measurements of
    the \gls{input} and \gls{output} of the real \gls{system}.}}
\newglossaryentry{open-loop gain}{
  name={open-loop gain},
  description={The gain directly from the \gls{input} to the \gls{output},
    ignoring loops.}}
\newglossaryentry{output}{
  name={output},
  description={Measurements from sensors.}}
\newglossaryentry{output-based control}{
  name={output-based control},
  description={Controls the \gls{system}'s \gls{state} via the \glspl{output}.}}
\newglossaryentry{overshoot}{
  name={overshoot},
  description={The amount by which a \gls{system}'s \gls{state} surpasses the
    \gls{reference} after rising toward it.}}
\newglossaryentry{phase margin}{
  name={phase margin},
  description={See section \ref{sec:gain_phase_margin} on gain and phase
    margin.}}
\newglossaryentry{plant}{
  name={plant},
  description={The \gls{system} or collection of actuators being controlled.}}
\newglossaryentry{pose}{
  name={pose},
  description={The orientation of an \gls{agent} in the world, which is
    represented by all or part of the \gls{agent}'s \gls{state}.}}
\newglossaryentry{process variable}{
  name={process variable},
  description={The term used to describe the \gls{output} of a \gls{plant} in
    the context of PID control.}}
\newglossaryentry{realization}{
  name={realization},
  description={In control theory, this is an implementation of a given
    input-output behavior as a state-space \gls{model}.}}
\newglossaryentry{reference}{
  name={reference},
  description={The desired state. This value is used as the reference point for
    a controller's error calculation.}}
\newglossaryentry{regulator}{
  name={regulator},
  description={A \gls{controller} that attempts to minimize the \gls{error} from
    a constant \gls{reference} in the presence of disturbances.}}
\newglossaryentry{rise time}{
  name={rise time},
  description={The time a \gls{system} takes to initially reach the
    \gls{reference} after applying a \gls{step input}.}}
\newglossaryentry{robustness}{
  name={robustness},
  description={The quality of a feedback \gls{control system} to remain stable
    in response to disturbances and uncertainty.}}
\newglossaryentry{setpoint}{
  name={setpoint},
  description={The term used to describe the \gls{reference} of a PID
    controller.}}
\newglossaryentry{settling time}{
  name={settling time},
  description={The time a \gls{system} takes to settle at the \gls{reference}
    after a \gls{step input} is applied.}}
\newglossaryentry{state}{
  name={state},
  description={A characteristic of a \gls{system} (e.g., velocity) that can be
    used to determine the \gls{system}'s future behavior.}}
\newglossaryentry{state feedback}{
  name={state feedback},
  description={Uses \gls{state} instead of \gls{output} in feedback.}}
\newglossaryentry{steady-state error}{
  name={steady-state error},
  description={\Gls{error} after \gls{system} reaches equilibrium.}}
\newglossaryentry{step input}{
  name={step input},
  description={A \gls{system} \gls{input} that is $0$ for $t < 0$ and a constant
  greater than $0$ for $t \geq 0$. A step input that is $1$ for $t \geq 0$ is
  called a unit step input.}}
\newglossaryentry{step response}{
  name={step response},
  description={The response of a \gls{system} to a \gls{step input}.}}
\newglossaryentry{stochastic process}{
  name={stochastic process},
  description={A process whose \gls{model} is partially or completely defined by
    random variables.}}
\newglossaryentry{system}{
  name={system},
  description={A term encompassing a \gls{plant} and its interaction with a
    \gls{controller} and \gls{observer}, which is treated as a single entity.
    Mathematically speaking, a \gls{system} maps \glspl{input} to \glspl{output}
    through a linear combination of \glspl{state}.}}
\newglossaryentry{system response}{
  name={system response},
  description={The behavior of a \gls{system} over time for a given
    \gls{input}.}}
\newglossaryentry{time-invariant}{
  name={time-invariant},
  description={The \gls{system}'s fundamental response does not change over
    time.}}
\newglossaryentry{tracking}{
  name={tracking},
  description={In control theory, the process of making the output of a
    \gls{control system} follow the \gls{reference}.}}
\newglossaryentry{unity feedback}{
  name={unity feedback},
  description={A feedback network in a \gls{control system} diagram with a
    feedback gain of 1.}}
